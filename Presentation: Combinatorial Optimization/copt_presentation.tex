% !TEX program = xelatex
\documentclass[aspectratio=1610]{beamer}
\usepackage[T1]{fontenc}
\usetheme{wildcat}
\usepackage{xcolor, mathtools, optidef}
\usepackage{tikz}
\usetikzlibrary{decorations.pathreplacing, arrows.meta, shapes, calc, positioning}
\DeclareMathOperator{\poa}{\mathsf{PoA}}
\input{../letterfonts}
\input{macros}
\title{Super-Polynomial Lower Bound of TSP Extended Formula}
\date{May 2025}
\author{Soham Chatterjee}

% You change the titlegraphic to whatever you want, or comment it out to remove it.
% \titlegraphic{\includegraphics[scale=0.25]{logo-northwestern.pdf}}

% % You can directly change the colors using the following macros.
% % You must redefine colors AFTER the theme is loaded.
% % For example, these provide shades of Yale Blue (#00356b)
% \definecolor{wcprimary}{RGB}{0,53,107}      % Main color
% \definecolor{wcprimary140}{RGB}{0, 34, 70}
% \definecolor{wcprimary130}{RGB}{0, 40, 80}
% \definecolor{wcprimary120}{RGB}{0, 45, 91}
% \definecolor{wcprimary110}{RGB}{0, 50, 102}
% \definecolor{wcprimary40}{RGB}{153, 174, 196}
% \definecolor{wcprimary30}{RGB}{179, 194, 211}
% \definecolor{wcprimary20}{RGB}{204, 215, 225}
% \definecolor{wcprimary10}{RGB}{230, 235, 240}

% % Now for the alerted orange (#bd5319) and example green (#5f712d)
% \definecolor{wcalerted}{RGB}{189,83,25}
% \definecolor{wcexample}{RGB}{95,113,45}

% % If you want to change the slide background color, 
% % you can use the following command:
%\setbeamercolor{background canvas}{bg=nupurple10!30}

% % Turn off section slides
% \AtBeginSection{}

% Change the font theme
%\usefonttheme{wildcat-overleaf}

% Change the bg pattern manually: Simple Single Color
% \renewcommand{\bgpattern}{
%     \draw[color=wcprimary,fill=wcprimary] (0,0) rectangle (\paperwidth,\paperheight);
% }

\definecolor{doc}{HTML}{DCBCD0}
\definecolor{myg}{RGB}{56, 140, 70}
\definecolor{myb}{RGB}{45, 111, 177}
\definecolor{myr}{RGB}{199, 68, 64}
\definecolor{mybg}{HTML}{F2F2F9}
\definecolor{mytheorembg}{HTML}{F2F2F9}
\definecolor{mytheoremfr}{HTML}{00007B}
\definecolor{myexamplebg}{HTML}{F2FBF8}
\definecolor{myexamplefr}{HTML}{88D6D1}
\definecolor{myexampleti}{HTML}{2A7F7F}
\definecolor{mydefinitbg}{HTML}{E5E5FF}
\definecolor{mydefinitfr}{HTML}{3F3FA3}
\definecolor{notesgreen}{RGB}{0,162,0}
\definecolor{myp}{RGB}{197, 92, 212}
\definecolor{mygr}{HTML}{2C3338}
\definecolor{myred}{RGB}{127,0,0}
\definecolor{myyellow}{RGB}{169,121,69}
\definecolor{OrangeRed}{HTML}{ED135A}
\definecolor{Dandelion}{HTML}{FDBC42}
\definecolor{light-gray}{gray}{0.95}
\definecolor{Emerald}{HTML}{00A99D}
\definecolor{RoyalBlue}{HTML}{0071BC}
\definecolor{mytoccolor}{HTML}{886830}


\renewcommand{\P}{\ensuremath{\mathsf{P}}}
\newcommand{\PSPACE}{\ensuremath{\mathsf{PSPACE}}}
\newcommand{\TQBF}{\ensuremath{\mathsf{TQBF}}}
\newcommand{\CONP}{\ensuremath{\mathsf{coNP}}}
\newcommand{\NP}{\ensuremath{\mathsf{NP}}}
\newcommand{\ONP}{\ensuremath{\mathsf{ONP}}}
\newcommand{\EXP}{\ensuremath{\mathsf{EXP}}}
\newcommand{\RP}{\ensuremath{\mathsf{RP}}}
\newcommand{\RE}{\ensuremath{\mathsf{RE}}}
\newcommand{\RL}{\ensuremath{\mathsf{RL}}}
\newcommand{\R}[1]{\ensuremath{\mathsf{R#1}}}
\newcommand{\BPP}{\ensuremath{\mathsf{BPP}}}
\newcommand{\SIZE}{\ensuremath{\mathsf{SIZE}}}
\newcommand{\DTIME}{\ensuremath{\mathsf{DTIME}}}
\newcommand{\DSPACE}{\ensuremath{\mathsf{DSPACE}}}
\newcommand{\NL}{\ensuremath{\mathsf{NL}}}
\newcommand{\NSIZE}{\ensuremath{\mathsf{NSIZE}}}
\newcommand{\poly}{\ensuremath{\mathsf{poly}}}
\newcommand{\AC}{\ensuremath{\mathsf{AC}}}
\newcommand{\NC}{\ensuremath{\mathsf{NC}}}
\newcommand{\PH}{\ensuremath{\mathsf{PH}}}
\newcommand{\SAT}{\ensuremath{\mathsf{SAT}}}
\newcommand{\CONE}{\ensuremath{\mathsf{coNEXP}}}
\newcommand{\NE}{\ensuremath{\mathsf{NEXP}}}

% ibliographystyle{alpha}
% \bibliography{refs}\b

\begin{document}

\begin{frame}
	\titlepage
\end{frame}

% \begin{frame}{Table of Contents}
%     \tableofcontents
% \end{frame}

\begin{frame}{Introduction}
	\begin{definition}[Travelling Salesman]
		Given a graph $G=(V,E)$, $S\subseteq V$ and weights $w:E\to \bbR$ find minimum weight cycle which visits every vertex of $S$ exactly once.
	\end{definition}\pause

	We will focus on $S=V$. 
	\begin{itemize}
        \item We know Traveling Salesman Problem is $\NP$-complete. 
        \item In [Yannkakis, 1988, STOC] he proved every symmetric LP for the TSP has expnential size.
        \item Here we will show TSP admits no polynomial-size LP.
        \item This proof also shows unconditional super-polynomial lower bound on the number of inequalities.
        \item Therefore it is impossible to prove $\P=\NP$ by means of a polynomial size LP.
    \end{itemize}
\end{frame}
\begin{frame}{Definitions}
	Let $P=\{x\in\bbR^n\mid Ax\leq b\}=conv(V)$ is a polytope  with $A\in\bbR^{m\times d}, b\in \bbR^m$ and $V\subseteq \bbR^d$. We will consider $V$ as the characteristic vector for all hamiltonian paths.\pause

	\begin{definition}[Extension Polytope]
		An extension of $P$ is a polytope $Q\subseteq \bbR^{d+e}$ such that there is a linear map $\pi:\bbR^{d+e}\to\bbR^{d}$ such that $\pi(Q)=P$.
	\end{definition}\pause

	\begin{definition}[Extended Formula]
			An EF $Q$ is an extension of $P$ is a linear system in variable s $(x,y)$ such that $$x\in P\iff \exs\ y\ (x,y)\in Q$$\pause

			Extension complexity of $P$ is the minimum size  EF of $P$ where size of a polytope is the number inequalities. We denote by $xc(P)$.
	\end{definition}
\end{frame}
\begin{frame}{Some Polytopes}

	\begin{itemize}
		\item $TSP(n)$ is the traveling salesman polytope for $K_n=(V_n,E_n)$. Let $C\subseteq E_n$ denotes a tour of $K_n$. Then $\chi^C$ denotes the characteristic vector of $C$. Then \pause
		
		$$TSP(n)\coloneqq conv\{\chi^C\mid C\subseteq E_n\text{ is a tour of $K_n$}\}$$\pause

		\item Given $G=(V,E)$, for any $S\subseteq V$, $\chi^S$ denote characteristic vector of $S$. Then\pause
		
		$$IND(G)\coloneqq conv\{\chi^S\mid S\text{ is independent set of $G$}\}$$
		\item The correlation polytope $COR(n)$ is $$COR(n)\coloneqq conv\{bb^T\mid b\in\{0,1\}^n\}$$
	\end{itemize}


\end{frame}
    
\end{document}