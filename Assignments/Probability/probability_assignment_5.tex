\documentclass[a4paper, 11pt]{article}
\usepackage{comment} % enables the use of multi-line comments (\ifx \fi) 
\usepackage{fullpage} % changes the margin
\usepackage[a4paper, total={7in, 10in}]{geometry}
\usepackage{amsmath,mathtools,mathdots,adjustbox,wrapfig}
\usepackage{amssymb,amsthm}  % assumes amsmath package installed
\usepackage{float}
\usepackage{xcolor}
\usepackage{mdframed}
%\usepackage{subfig}
\usepackage[shortlabels]{enumitem}
\usepackage{indentfirst}
\usepackage{hyperref}
\hypersetup{
    colorlinks=true,
    linkcolor=doc!80,
    citecolor=myr,
    filecolor=myr,      
    urlcolor=doc!80,
    pdftitle={Assignment}, %%%%%%%%%%%%%%%%   WRITE ASSIGNMENT PDF NAME  %%%%%%%%%%%%%%%%%%%%
}
\usepackage[most,many,breakable]{tcolorbox}
\usepackage{tikz}
%\usepackage{caption}
\usepackage{mathpazo}
\usepackage[skip=0.333\baselineskip]{caption}
\usepackage{subcaption}
% Use the libertine package for the Libertine font
\usepackage{libertine}
\usepackage[libertine]{newtxmath}
\usepackage{libertine}
\usepackage{physics}
\usepackage[ruled,vlined,linesnumbered]{algorithm2e}
\usepackage{mathrsfs}
\usepackage{tikz-cd}
\usepackage{float}
\usepackage{pgfplots}
\definecolor{mytheorembg}{HTML}{F2F2F9}
\definecolor{mytheoremfr}{HTML}{00007B}
\definecolor{doc}{RGB}{0,60,110}
\definecolor{myg}{RGB}{56, 140, 70}
\definecolor{myb}{RGB}{45, 111, 177}
\definecolor{myr}{RGB}{199, 68, 64}

\usetikzlibrary{decorations.pathreplacing,angles,quotes,patterns}
\definecolor{mytheorembg}{HTML}{F2F2F9}
\definecolor{mytheoremfr}{HTML}{00007B}
\definecolor{doc}{RGB}{0,60,110}
\definecolor{myg}{RGB}{56, 140, 70}
\definecolor{myb}{RGB}{45, 111, 177}
\definecolor{myr}{RGB}{199, 68, 64}

\tcbuselibrary{theorems,skins,hooks}
\newtcbtheorem{problem}{Problem}
{%
	enhanced,
	breakable,
	colback = mytheorembg,
	frame hidden,
	boxrule = 0sp,
	borderline west = {2pt}{0pt}{mytheoremfr},
	arc=5pt,
	detach title,
	before upper = \tcbtitle\par\smallskip,
	coltitle = mytheoremfr,
	fonttitle = \bfseries\sffamily,
	description font = \mdseries,
	separator sign none,
	segmentation style={solid, mytheoremfr},
}
{p}

\newtheorem{lemma}{Lemma}
\renewenvironment{proof}{\noindent{\it \textbf{Proof:}}\hspace*{1em}}{\qed\bigskip\\}
% To give references for any problem use like this
% suppose the problem number is p3 then 2 options either 
% \hyperref[p:p3]{<text you want to use to hyperlink> \ref{p:p3}}
%                  or directly 
%                   \ref{p:p3}



\input{../../letterfonts}

\input{../../macros}

\setlength{\parindent}{0pt}

%%%%%%%%%%%%%%%%%%%%%%%%%%%%%%%%%%%%%%%%%%%%%%%%%%%%%%%%%%%%%%%%%%%%%%%%%%%%%%%%%%%%%%%%%%%%%%%%%%%%%%%%%%%%%%%%%%%%%%%%%%%%%%%%%%%%%%%%

\begin{document}

%%%%%%%%%%%%%%%%%%%%%%%%%%%%%%%%%%%%%%%%%%%%%%%%%%%%%%%%%%%%%%%%%%%%%%%%%%%%%%%%%%%%%%%%%%%%%%%%%%%%%%%%%%%%%%%%%%%%%%%%%%%%%%%%%%%%%%%%

\textsf{\noindent \large\textbf{Soham Chatterjee} \hfill \textbf{Assignment - 5}\\
    Email: \href{soham.chatterjee@tifr.res.in}{soham.chatterjee@tifr.res.in} \hfill Dept: STCS\\
    \normalsize Course: Probability Theory \hfill Date: \today
    \\  \noindent\rule{7in}{2.8pt}
}

[All the problems I discussed with Spandan, Soumyadeep]
%%%%%%%%%%%%%%%%%%%%%%%%%%%%%%%%%%%%%%%%%%%%%%%%%%%%%%%%%%%%%%%%%%%%%%%%%%%%%%%%%%%%%%%%%%%%%%%%%%%%%%%%%%%%%%%%%%%%%%%%%%%%%%%%%%%%%%%%
% Problem 1
%%%%%%%%%%%%%%%%%%%%%%%%%%%%%%%%%%%%%%%%%%%%%%%%%%%%%%%%%%%%%%%%%%%%%%%%%%%%%%%%%%%%%%%%%%%%%%%%%%%%%%%%%%%%%%%%%%%%%%%%%%%%%%%%%%%%%%%%

\begin{problem}{%problem statement
    }{p1% problem reference text
    }
Let $X,Y_1,Y_2$ be three random variables with joint density $f_{X,Y_1,Y_2}$. For a fixed $y_1$, consider two random variables $\tilde{X}$, $\tilde{Y_2}$ with joint distribution $g_{\tdX,\tilde{Y_2}}$ defined as $g_{\tdX,\tilde{Y_2}}(x,y_2)=f_{X,Y_2\mid Y_1}(x,y_2\mid y_1)$. Show that $$\bbE[\tdX\mid \tdY=y_2]=\bbE[X\mid Y_1=y_1,Y_2=y_2]$$What is the relevance of this fact in our derivation of recursive estimation in the lecture?
\end{problem}
\solve{
We have $$g_{\tdX\mid \tdY_2}(x\mid y_2)=\frac{g_{\tdX,\tdY_2}(x,y_2)}{g_{\tdY_2}(y_2)}=\frac{f_{X,Y_2\mid Y_1}(x,y_2\mid y_1)}{f_{Y_2\mid Y_1}(y_2\mid y_1)}=\frac{\frac{f_{X,Y_1,Y_2}(x,y_1,y_2)}{f_{Y_1}(y_1)}}{\frac{f_{Y_1,Y_2}(y_1,y_2)}{f_{Y_1}(y_1)}}=f_{X\mid Y_1,Y_2}(x\mid y_1,y_2)$$Therefore $\bbE[\tdX\mid \tdY=y_2]=\bbE[X\mid Y_1=y_1,Y_2=y_2]$. This is used to derive the iterative estimator which is used in the recurrence relation for Kalman Filter.
}
%%%%%%%%%%%%%%%%%%%%%%%%%%%%%%%%%%%%%%%%%%%%%%%%%%%%%%%%%%%%%%%%%%%%%%%%%%%%%%%%%%%%%%%%%%%%%%%%%%%%%%%%%%%%%%%%%%%%%%%%%%%%%%%%%%%%%%%%
% Problem 2
%%%%%%%%%%%%%%%%%%%%%%%%%%%%%%%%%%%%%%%%%%%%%%%%%%%%%%%%%%%%%%%%%%%%%%%%%%%%%%%%%%%%%%%%%%%%%%%%%%%%%%%%%%%%%%%%%%%%%%%%%%%%%%%%%%%%%%%%

\begin{problem}{%problem statement
    }{p2% problem reference text
    }
Consider the Kalman filtering problem for the scalar system:

\begin{align*}
	X_k  =\alpha X_{k-1}+W_k && 	Y_k =h X_k+Z_k
\end{align*}
as described in class (i.e., $W_k \sim N\left(0, \sigma_W^2\right)$ i.i.d, $Z_k \sim N\left(0, \sigma_Z^2\right)$, and $X_1$ are independent). The initial condition is $X_1 \sim N\left(0, \sigma_{X_1}^2\right)$. For the numerical exercises below you can assume $\sigma_{X_1}^2=\sigma_Z^2=\sigma_W^2=h=1$.\begin{enumerate}[label=(\alph*)]
	\item Plot sample paths of the process $\left\{X_k\right\}$ for different values of $\alpha$. Pick a representative set of values of $\alpha$ to show the effect of $\alpha$ on how the sample paths look like. Can you explain qualitatively the effect?
\item Let $\hat{X}_k=E\left[X_k \mid Y_1, \ldots, Y_k\right]$. For those sample paths of $\left\{X_k\right\}$ plotted in part (a), plot in the same figure the sample paths of the estimates $\left\{\hat{X}_k\right\}$. What is the qualitative effect of $\alpha$ on the estimation errors?
\item Let $\tilde{X}_k=E\left[X_k \mid Y_k\right]$. This is the state estimate based only on the current observation. For the sample paths in (a) and (b), plot the sample paths of $\left\{\tilde{X}_k\right\}$ in the same figure as well. How does the difference in the accuracy of the estimators $\hat{X}_k$ and $\tilde{X}_k$ depend on the value of $\alpha$ ? Explain qualitatively.
\item Let $f_k$ be the conditional distribution of $X_k$ give the observations up to time $k$. For your favorite value of $\alpha$, plot $f_k$ for several values of $k$ to get a feel of how the distribution evolves in time. Do these distributions depend on the random outcome of the experiment? How?
\item What happens to the distribution of $X_k$ as $k \rightarrow \infty$ ? Give a quantitative answer. Does your answer depend on $\alpha$ ? Does your answer depend on $\sigma_{X_1}^2$ ?
\item What happens to the MMSE estimation error $\sigma_k^2$ of $\hat{X}_k$ as $k \rightarrow \infty$ ? Does it converge to zero, a finite non-zero value or infinity? How does your answer depend on $\alpha$ ? An answer supported by numerical evidence together with some analysis would be fine; it doesn't have to be totally rigorous.
\end{enumerate}
\end{problem}
\solve{
\begin{enumerate}[label=(\alph*)]
	\item Here we have taken the values of $\alpha$ to be $\{-1,0.8,1,1.2\}$ in \hyperref[fig:fig1]{Figure 1}. Here we can see that when the value of $\alpha$ is $1.2$  then the sample value increases. And when the value of $\alpha$ is $-1$ it oscillates around $0$. But for $\alpha=0.8$ the sample values remains close to $0$. Therefore the sample values converges when $|\alpha|<1$ and otherwise diverges.

	

\begin{figure}[hbt!]%
	\centering
	\includegraphics[width=8cm]{./ass5.2a.pdf}\label{fig:fig1}%
	\caption{Plot of $X_k$ for different $\alpha\in \{-1,0.8,1,1.2\}$}%
%	\label{fig:example}%
\end{figure}

\item In the  following plot we can see that the predicted values $\bbE[X\mid Y_1,\dots, Y_k]$ matches almost correctly with the sample values $X_k$. From the plots we conclude that as $|\alpha$ becomes larger it has lesser effect on the estimation which we also showed in part (f) where we showed if $|\alpha|$ becomes larger then the MMSE estimation is independent of $\alpha$. 

\begin{figure}[hbt!]
	\begin{subfigure}{.475\linewidth}
		\includegraphics[width=\linewidth]{./ass5.2b-0.8.pdf}
		\caption{Plot of $X_k$ vs $\hat{X}_k$ for $\alpha=0.8$}
		\label{MLEDdet}
	\end{subfigure}\hfill % <-- "\hfill"
	\begin{subfigure}{.475\linewidth}
		\includegraphics[width=\linewidth]{./ass5.2b-1.pdf}
		\caption{Plot of $X_k$ vs $\hat{X}_k$ for $\alpha=1$}
		\label{energydetPSK}
	\end{subfigure}
	
	\medskip % create some *vertical* separation between the graphs
	\begin{subfigure}{.475\linewidth}
		\includegraphics[width=\linewidth]{./ass5.2b-1.2.pdf}
		\caption{Plot of $X_k$ vs $\hat{X}_k$ for $\alpha=1.2$}
		\label{velcomp}
	\end{subfigure}\hfill % <-- "\hfill"
	\begin{subfigure}{.475\linewidth}
		\includegraphics[width=\linewidth]{./ass5.2b--1.pdf}
		\caption{Plot of $X_k$ vs $\hat{X}_k$ for $\alpha=-1$}
		\label{estcomp}
	\end{subfigure}
	
	\caption{Compared $X_k$ and predicted $\hat{X_k}$}
	\label{fig:roc}
\end{figure}



\item Here we compare $X_k$, $\hat{S}_k$ and $\tdX_k=\bbE[X_k\mid Y_k]$ for all values of $\alpha$. Now we have $\cov(X_k,Y_k)=h\rho_k^2$ and $\Var[Y_k]=h^2\rho_k^2+\sg_Z^2$. Hence we have $$\bbE[X_k\mid Y_k]=\frac{h\rho_k^2 Y_k}{h^2\rho_k^2+\sg_Z^2}$$So we have\begin{align*}
	\bbE[X_k-\tdX_k]^2 & = \bbE\lt[X_k-\frac{h\rho_k^2 Y_k}{h^2\rho_k^2+\sg_Z^2}\rt] ^2\\
	& = \frac{1}{\lt(h^2\rho_k^2+\sg_Z^2\rt)^2}\bbE\lt[(h^2\rho_k^2+\sg_Z^2)X_k-h\rho_k^2(hX_k+Z_k)\rt]^2\\
	& = \frac{1}{\lt(h^2\rho_k^2+\sg_Z^2\rt)^2}\bbE[\sg_Z^2X_k-h\rho_k^2Z_k]^2\\
	& = \frac{\sg_Z^4\rho_k^2+h^2\rho_k^4\sg_Z^2}{\lt(h^2\rho_k^2+\sg_Z^2\rt)^2}= \frac{\sg_Z^2\rho_k^2}{h^2\rho_k^2+\sg_Z^2}
\end{align*}Now this is the MMSE estimation of $\sg_k^2$ of $X_k$ which comparing with part (f) we can see that we obtained the same estimation value. Therefore both $\hat{X}_k$ and $\tdX_k$ are equally good estimating sample values. 

\begin{figure}[hbt!]
\begin{subfigure}{.475\linewidth}
	\includegraphics[width=\linewidth]{./ass5.2c-0.8.pdf}
	\caption{Plot of $X_k$ vs $\hat{X}_k$ vs $\tilde{X}_k$ for $\alpha=0.8$}
	\label{MLEDdet}
\end{subfigure}\hfill % <-- "\hfill"
\begin{subfigure}{.475\linewidth}
	\includegraphics[width=\linewidth]{./ass5.2c-1.pdf}
	\caption{Plot of $X_k$ vs $\hat{X}_k$ vs $\tilde{X}_k$ for $\alpha=1$}
	\label{energydetPSK}
\end{subfigure}

\medskip % create some *vertical* separation between the graphs
\begin{subfigure}{.475\linewidth}
	\includegraphics[width=\linewidth]{./ass5.2c-1.2.pdf}
	\caption{Plot of $X_k$ vs $\hat{X}_k$ vs $\tilde{X}_k$ for $\alpha=1.2$}
	\label{velcomp}
\end{subfigure}\hfill % <-- "\hfill"
\begin{subfigure}{.475\linewidth}
	\includegraphics[width=\linewidth]{./ass5.2c--1.pdf}
	\caption{Plot of $X_k$ vs $\hat{X}_k$ vs $\tilde{X}_k$ for $\alpha=-1$}
	\label{estcomp}
\end{subfigure}

\caption{Compared $X_k$ and predicted $\hat{X_k}$ and $\tdX_k$ }
\label{fig:roc}
\end{figure}
\item Here we plot $f_k$ for values $k\in[10]$ with $\alpha=0.8$. Now the conditional distribution $X\mid Y_1,\dots, Y_k$ approaches the distribution $N\lt(0,\frac{\sg2}{1-\alpha^2}\rt)$ for large $k$ and also we notice from the plot that this doesn't depend on the $Y_k$.\newpage

\begin{figure}[hbt!]
	\centering
	\includegraphics[width=16cm]{./ass5.2d-0.8.pdf}
	\caption{Plot of density $f_k$ of $X_k\mid Y_1,...,Y_k$ for $\alpha=0.8$, $k\in[10]$}
\end{figure}

\item We will induct on $k$. Since $X_1,W_2$ are independent and we have $X_2=\alpha X_1+W_2$ hence $X_2\sim N\lt(0, \alpha^2\sg_{X_1}^2+\sg_W^2\rt)$. Now $X_{k-1}$ and $W_k$ are independent and we have $X_k=\alpha X_{k-1}+W_k$. By inductive hypothesis $X_{k-1}$ follows Gaussian Distribution. Hence $X_k$ also follows Gaussian Distribution. Hence $\bbE[X_k]=0$. Now we have to calculate $\Var[X_k]$.$$\Var[X_k]=\alpha^2\Var[X_{k-1}]+\sg_W^2=\alpha^2(\alpha^2\Var[X_{k-2}]+\sg_W^2)+\sg_W^2=\cdots = \alpha^{2k-2}\sg_{X_1}^2+\sg_W^2\sum_{i=0}^{k-1}\alpha^{2i}$$Therefore $X_k\sim N\lt(0,\alpha^{2k-2}\sg_{X_1}^2+\sg_W^2\sum\limits_{i=0}^{k-1}\alpha^{2i}\rt)$. Hence if $|\alpha|<1$, $\lim\limits_{k\to\infty}\Var[X_k]=\frac{\sg_W^2}{1-\alpha^2}$. Hence as $k\to\infty$, $X_k\to N\lt(0,\frac{\sg_W^2}{1-\alpha^2}\rt)$. Now if $|\alpha|\geq 1$, then as $k\to \infty$, $\alpha^{2k-2}$ diverges. Therefore $\Var[X_k]$ diverges to $+\infty$.\parinn

If $|\alpha|<1$ then $\Var[X_k]=\frac{\sg_W^2}{1-\alpha^2}$. Hence it doesn't depend on $\sg_{X_1}^2$. 
\item Let $\rho_k$ denote the variance of $X_k$. Then we have the formula $$\rho_n=\alpha^2\rho_{n-1}^2+\sg_W^2\quad\text{for } n\geq 2$$Hence we know the behavior of the conditional variance $\sg_k^2$ of ${X_k}$. Hence we know the MMSE of ${X_k\mid Y_1,\dots, Y_k}$ as $k\to \infty$. Now we have $$\sg_k^2=\frac{\rho_k^2\sg_Z^2}{h^2\rho_k^2+\sg_Z^2}=\frac{\sg_Z^2}{h^2+\frac{\sg_Z^2}{\rho_k^2}}$$From the previous part  if $|\alpha|<1$ then $\lim\limits_{k\to \infty}\rho_k=\frac{\sg_W^2}{1-\alpha^2}$ and if $|\alpha|\geq 1$ then as $k\to \infty$, $\rho_k$ diverges to $+\infty$.  Therefore when $|\alpha|<1$, $\lim\limits_{k\to\infty}\sg_k^2=\frac{\sg_Z^2\sg_W^2}{h^2\sg_W^2+(1-\alpha^2)\sg_Z^2}$ and when $|\alpha|\geq 1$ we have $\lim\limits_{k\to\infty}\sg_k^2=\frac{\sg_Z^2}{h^2}$.
\end{enumerate}
}

%%%%%%%%%%%%%%%%%%%%%%%%%%%%%%%%%%%%%%%%%%%%%%%%%%%%%%%%%%%%%%%%%%%%%%%%%%%%%%%%%%%%%%%%%%%%%%%%%%%%%%%%%%%%%%%%%%%%%%%%%%%%%%%%%%%%%%%%
% Problem 3
%%%%%%%%%%%%%%%%%%%%%%%%%%%%%%%%%%%%%%%%%%%%%%%%%%%%%%%%%%%%%%%%%%%%%%%%%%%%%%%%%%%%%%%%%%%%%%%%%%%%%%%%%%%%%%%%%%%%%%%%%%%%%%%%%%%%%%%%


\begin{problem}{%problem statement
    }{p3% problem reference text
    }
For the system in \hyperref[p:p2]{Problem \ref{p:p2}} derive a recursive algorithm for computing the one-step ahead estimator: $\bbE[{X_k\mid Y_1,Y_2,\dots, Y_k,Y_{k+1}}]$. This means we can look ahead one step to estimate the state.
\end{problem}
\solve{
We have $Y_{k+1}=hX_{k+1}+W_{k+1}$. and $X_{k+1}=\alpha X_k+Z_k$. Therefore combining these two we have $$Y_{k+1}=\alpha hX_k+(hW_k+Z_{k+1})$$Now take $T_k=hW_k+Z_{k+1}$. Then $T_k\sim N(0,h^2\sg_W^2+\sg_Z^2)$. Now denote $Y_1^k=(Y_1,\dots, Y_k)$ and denote $y_1^k=(y_1,\dots, y_k)$. Then we have  $$f\st_{X_k\mid Y^{K+1}_1}(x\mid y_1^{k+1})=\frac{f\st_{Y_{k+1}\mid X_k,Y^{K}_1}(y_{k+1}\mid x_k,y_1^{k})}{f\st_{Y_{k+1}\mid Y^{K}_1}(y_{k+1}\mid y_1^{k})}f_{X_k\mid Y_1^k}(x_k\mid y_1^k)$$Combining this with $\alpha hX_k+(hW_k+Z_{k+1})$ we can write $$\bbE[X_k\mid Y_1^{k+1}]=h\alpha \bbE[X_{k-1}\mid Y_1^k]$$
}



\end{document}
