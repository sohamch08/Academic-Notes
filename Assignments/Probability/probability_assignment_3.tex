\documentclass[a4paper, 11pt]{article}
\usepackage{comment} % enables the use of multi-line comments (\ifx \fi) 
\usepackage{fullpage} % changes the margin
\usepackage[a4paper, total={7in, 10in}]{geometry}
\usepackage{amsmath,mathtools,mathdots}
\usepackage{amssymb,amsthm}  % assumes amsmath package installed
\usepackage{float}
\usepackage{xcolor}
\usepackage{mdframed}
\usepackage[shortlabels]{enumitem}
\usepackage{indentfirst}
\usepackage{hyperref}
\hypersetup{
	colorlinks=true,
	linkcolor=doc!80,
	citecolor=myr,
	filecolor=myr,      
	urlcolor=doc!80,
	pdftitle={Assignment}, %%%%%%%%%%%%%%%%   WRITE ASSIGNMENT PDF NAME  %%%%%%%%%%%%%%%%%%%%
}
\usepackage[most,many,breakable]{tcolorbox}
\usepackage{tikz}
\usepackage{caption}
\usepackage{mathpazo}
% Use the libertine package for the Libertine font
\usepackage{libertine}
\usepackage[libertine]{newtxmath}
\usepackage{libertine}
\usepackage{physics}
\usepackage[ruled,vlined,linesnumbered]{algorithm2e}
\usepackage{mathrsfs}
\usepackage{tikz-cd}
\usepackage{float}
\usepackage{pgfplots}
\definecolor{mytheorembg}{HTML}{F2F2F9}
\definecolor{mytheoremfr}{HTML}{00007B}
\definecolor{doc}{RGB}{0,60,110}
\definecolor{myg}{RGB}{56, 140, 70}
\definecolor{myb}{RGB}{45, 111, 177}
\definecolor{myr}{RGB}{199, 68, 64}

\usetikzlibrary{decorations.pathreplacing,angles,quotes,patterns}
\definecolor{mytheorembg}{HTML}{F2F2F9}
\definecolor{mytheoremfr}{HTML}{00007B}
\definecolor{doc}{RGB}{0,60,110}
\definecolor{myg}{RGB}{56, 140, 70}
\definecolor{myb}{RGB}{45, 111, 177}
\definecolor{myr}{RGB}{199, 68, 64}

\tcbuselibrary{theorems,skins,hooks}
\newtcbtheorem{problem}{Problem}
{%
	enhanced,
	breakable,
	colback = mytheorembg,
	frame hidden,
	boxrule = 0sp,
	borderline west = {2pt}{0pt}{mytheoremfr},
	arc=5pt,
	detach title,
	before upper = \tcbtitle\par\smallskip,
	coltitle = mytheoremfr,
	fonttitle = \bfseries\sffamily,
	description font = \mdseries,
	separator sign none,
	segmentation style={solid, mytheoremfr},
}
{p}

\newtheorem{lemma}{Lemma}
\renewenvironment{proof}{\noindent{\it \textbf{Proof:}}\hspace*{1em}}{\qed\bigskip\\}
% To give references for any problem use like this
% suppose the problem number is p3 then 2 options either 
% \hyperref[p:p3]{<text you want to use to hyperlink> \ref{p:p3}}
%                  or directly 
%                   \ref{p:p3}



\input{../../letterfonts}

\input{../../macros}

\setlength{\parindent}{0pt}

%%%%%%%%%%%%%%%%%%%%%%%%%%%%%%%%%%%%%%%%%%%%%%%%%%%%%%%%%%%%%%%%%%%%%%%%%%%%%%%%%%%%%%%%%%%%%%%%%%%%%%%%%%%%%%%%%%%%%%%%%%%%%%%%%%%%%%%%

\begin{document}
	
	%%%%%%%%%%%%%%%%%%%%%%%%%%%%%%%%%%%%%%%%%%%%%%%%%%%%%%%%%%%%%%%%%%%%%%%%%%%%%%%%%%%%%%%%%%%%%%%%%%%%%%%%%%%%%%%%%%%%%%%%%%%%%%%%%%%%%%%%
	
	\textsf{\noindent \large\textbf{Soham Chatterjee} \hfill \textbf{Assignment - 2}\\
		Email: \href{soham.chatterjee@tifr.res.in}{soham.chatterjee@tifr.res.in} \hfill Dept: STCS\\
		\normalsize Course: Probability Theory \hfill Date: \today}
	
%%%%%%%%%%%%%%%%%%%%%%%%%%%%%%%%%%%%%%%%%%%%%%%%%%%%%%%%%%%%%%%%%%%%%%%%%%%%%%%%%%%%%%%%%%%%%%%%%%%%%%%%%%%%%%%%%%%%%%%%%%%%%%%%%%%%%%%%
% Problem 1
%%%%%%%%%%%%%%%%%%%%%%%%%%%%%%%%%%%%%%%%%%%%%%%%%%%%%%%%%%%%%%%%%%%%%%%%%%%%%%%%%%%%%%%%%%%%%%%%%%%%%%%%%%%%%%%%%%%%%%%%%%%%%%%%%%%%%%%%
	
\begin{problem}{%problem statement
	}{p1% problem reference text
}
We know that independent random variables are uncorrelated. Argue that uncorrelated
jointly Gaussian random variables are independent.

Hint: do this for two random variables first. For $n$ random variables, you might find it
easier to use the characteristic function.
\end{problem}
\solve{
	Let $\ovZ=(Z_1,\dots, Z_n)^T$ be the $n$ uncorrelated jointly Gaussian random variables. Let $K$ be the covarince matrix of $Z$. 
}
%%%%%%%%%%%%%%%%%%%%%%%%%%%%%%%%%%%%%%%%%%%%%%%%%%%%%%%%%%%%%%%%%%%%%%%%%%%%%%%%%%%%%%%%%%%%%%%%%%%%%%%%%%%%%%%%%%%%%%%%%%%%%%%%%%%%%%%%
% Problem 2
%%%%%%%%%%%%%%%%%%%%%%%%%%%%%%%%%%%%%%%%%%%%%%%%%%%%%%%%%%%%%%%%%%%%%%%%%%%%%%%%%%%%%%%%%%%%%%%%%%%%%%%%%%%%%%%%%%%%%%%%%%%%%%%%%%%%%%%%

\begin{problem}{%problem statement
	}{p2% problem reference text
	}
\begin{enumerate}[label=(\roman*)]
	\item * Let $X$ and $Y$ be independent random variables. $X_1\sim N(0,1)$; and $Y=+1$ with probability $p$ and $Y=-1$ with probability $1-p$. We define $X_2=YX_1$. Is $X_2$ Gaussian? Are $X_1,X_2$ jointly Gaussian? Justify your answers.
	
	[See Example 3.3.4 from [G] for a solution]
	\item Repeat (i) if $X_1\sim N(m,1)$ and $m>0$
\end{enumerate}
\end{problem}
\solve{
	
}

%%%%%%%%%%%%%%%%%%%%%%%%%%%%%%%%%%%%%%%%%%%%%%%%%%%%%%%%%%%%%%%%%%%%%%%%%%%%%%%%%%%%%%%%%%%%%%%%%%%%%%%%%%%%%%%%%%%%%%%%%%%%%%%%%%%%%%%%
% Problem 3
%%%%%%%%%%%%%%%%%%%%%%%%%%%%%%%%%%%%%%%%%%%%%%%%%%%%%%%%%%%%%%%%%%%%%%%%%%%%%%%%%%%%%%%%%%%%%%%%%%%%%%%%%%%%%%%%%%%%%%%%%%%%%%%%%%%%%%%%

\begin{problem}{%problem statement
[G] Exercise 3.8
}{p3% problem reference text
	}
\begin{enumerate}[label=(\alph*)]
	\item Let $[K]=\mat{0.75 & 0.25\\ 0.25 & 0.75}$. Show that $1$ and $\frac12$ are eigenvalues of $[K]$ and find the normalized eigenvectors. Express $[K]$ ad $[Q\Lm Q^{-1}]$, where $[\Lm]$ is diagonal and $[Q]$ is orthonormal.
	\item Let $[K']=\alpha[K]$ for real $\alpha\neq 0$. Find the eigenvalues and eigenvectors of $[K']$. Don not use brute force - think!
	\item Find the eigenvalues and eigenvectors of $[K^m]$, where $[K^m]$ is the $mth$ power of $[K]$.
\end{enumerate}
\end{problem}
\solve{
}

%%%%%%%%%%%%%%%%%%%%%%%%%%%%%%%%%%%%%%%%%%%%%%%%%%%%%%%%%%%%%%%%%%%%%%%%%%%%%%%%%%%%%%%%%%%%%%%%%%%%%%%%%%%%%%%%%%%%%%%%%%%%%%%%%%%%%%%%
% Problem 4
%%%%%%%%%%%%%%%%%%%%%%%%%%%%%%%%%%%%%%%%%%%%%%%%%%%%%%%%%%%%%%%%%%%%%%%%%%%%%%%%%%%%%%%%%%%%%%%%%%%%%%%%%%%%%%%%%%%%%%%%%%%%%%%%%%%%%%%%

\begin{problem}{%problem statement 
	}{p4% problem reference text
	}
We derived the p.d.f. of a jointly Gaussian random vector $X=AW$, where $A$ is an $n\times n$ matrix. We used the fact $A$ is invertible. How would you precisely describe the distribution of $X$ if $A$ is not invertible? Describe the underlying geometry of the distribution of $X$. Use the following $A$ as an example: $$A=\matp{1 & 2 &3\\ 1 & 1&1\\ 2&3&4}$$
\end{problem}
\solve{
}
%%%%%%%%%%%%%%%%%%%%%%%%%%%%%%%%%%%%%%%%%%%%%%%%%%%%%%%%%%%%%%%%%%%%%%%%%%%%%%%%%%%%%%%%%%%%%%%%%%%%%%%%%%%%%%%%%%%%%%%%%%%%%%%%%%%%%%%%
% Problem 5
%%%%%%%%%%%%%%%%%%%%%%%%%%%%%%%%%%%%%%%%%%%%%%%%%%%%%%%%%%%%%%%%%%%%%%%%%%%%%%%%%%%%%%%%%%%%%%%%%%%%%%%%%%%%%%%%%%%%%%%%%%%%%%%%%%%%%%%%

\begin{problem}{%problem statement 
		[G] Problem 3.9
	}{p5% problem reference text
	}
Let $X$ and $Y$ be jointly Gaussian with means $m_X$, $m_Y$, variances $\sg_X^2, \sg_Y^2$, and normalized covariance $\rho$. Find the conditional density $f\st_{X\mid Y}(x\mid y)$.
\end{problem}
\solve{
}


%%%%%%%%%%%%%%%%%%%%%%%%%%%%%%%%%%%%%%%%%%%%%%%%%%%%%%%%%%%%%%%%%%%%%%%%%%%%%%%%%%%%%%%%%%%%%%%%%%%%%%%%%%%%%%%%%%%%%%%%%%%%%%%%%%%%%%%%
% Problem 6
%%%%%%%%%%%%%%%%%%%%%%%%%%%%%%%%%%%%%%%%%%%%%%%%%%%%%%%%%%%%%%%%%%%%%%%%%%%%%%%%%%%%%%%%%%%%%%%%%%%%%%%%%%%%%%%%%%%%%%%%%%%%%%%%%%%%%%%%
\parinf\vspace*{1cm}

In the next two problems we will use a common model for communication systems.
The transmitted signal $\vec{X}$ is a Gaussian random vector of size m (vector since there
are several, say $m$, transmit antennas and each component of the vector stands for the
input to a separate antenna). The signal goes over a linear and additive Gaussian noise
channel and is picked up by a receiver which also has $n$ antennas. The received vector
of length $n$ has the form.\begin{equation}
	\vec{Y}=H\vec{X}+\vec{Z},\label{eq}
\end{equation}where $H$ is a constant $n\times m$ vector and $\vec{Z}$ is a Gaussian random vector of size $n$ and independent of $\vec{X}$.

\begin{problem}{%problem statement 
	}{p6% problem reference text
	}
Let us first consider the simpler case of $m=1$ and $n=2$. So $X$ is a scalar random variable. Let $X$ have the standard normal distribution $N(0,1)$. The received signals are $$Y_i=h_iX+Z_i,\qquad i=1,2,$$where $Z_i\sim N(0,\sg^2)$ are i.i.d and independent of $X$. And $h_i'$s are constants which represent the channel ``gains" from the transmit antenna to the receive antennas.\begin{enumerate}[label=(\alph*)]
	\item Find the conditional joint distribution of $Y_1,Y_2$ conditioned on $X=x$.
	\item Find the conditional joint distribution of $X$ conditioned on $Y_1=y_1$, $Y_2=y_2$.
	\item Using (b), what is your estimate of the transmitted signal $X$ if you are told that the receive antennas observed $Y_1=y_1$, $Y_2=y_2$. \textbf{Interpret your results}. Does your answer make intuitive sense? What happens to the estimate when the noise variance $\sg^2$ becomes small? or large?
\end{enumerate}
\end{problem}
\solve{
}

%%%%%%%%%%%%%%%%%%%%%%%%%%%%%%%%%%%%%%%%%%%%%%%%%%%%%%%%%%%%%%%%%%%%%%%%%%%%%%%%%%%%%%%%%%%%%%%%%%%%%%%%%%%%%%%%%%%%%%%%%%%%%%%%%%%%%%%%
% Problem 7
%%%%%%%%%%%%%%%%%%%%%%%%%%%%%%%%%%%%%%%%%%%%%%%%%%%%%%%%%%%%%%%%%%%%%%%%%%%%%%%%%%%%%%%%%%%%%%%%%%%%%%%%%%%%%%%%%%%%%%%%%%%%%%%%%%%%%%%%

\begin{problem}{%problem statement 
	}{p7% problem reference text
	}
Now consider the general model in \eqref{eq} for general $n,m$. Let $\vec{X}\sim N(\vec{0}, K_X)$, $\vec{Z}\sim N(\vec{0},K_Z)$ and $\vec{Z}$ is independent of $\vec{X}$.
\begin{enumerate}[label=(\alph*)]
	\item Show that $\vec{U}=(\vec{X},\vec{Y})$ is jointly Gaussian. You may use any of the equivalent definitions we saw in class
	\item Find a simple condition on $H,K_X,K_Z$ so that $K_U$ is invertible.
	\item What is the conditional distribution of the input $\vec{X}$ given the output $\vec{Y}=\vec{y}$.
\end{enumerate}
\end{problem}
\solve{
}


\end{document}
