\documentclass[a4paper, 11pt]{article}
\usepackage{comment} % enables the use of multi-line comments (\ifx \fi) 
\usepackage{fullpage} % changes the margin
\usepackage[a4paper, total={7in, 10in}]{geometry}
\usepackage{amsmath,mathtools,mathdots}
\usepackage{amssymb,amsthm}  % assumes amsmath package installed
\usepackage{float}
\usepackage{xcolor}
\usepackage{mdframed}
\usepackage[shortlabels]{enumitem}
\usepackage{indentfirst}
\usepackage{hyperref}
\hypersetup{
	colorlinks=true,
	linkcolor=doc!80,
	citecolor=myr,
	filecolor=myr,      
	urlcolor=doc!80,
	pdftitle={Assignment}, %%%%%%%%%%%%%%%%   WRITE ASSIGNMENT PDF NAME  %%%%%%%%%%%%%%%%%%%%
}
\usepackage[most,many,breakable]{tcolorbox}
\usepackage{tikz}
\usepackage{caption}
\usepackage{kpfonts}
\usepackage{libertine}
\usepackage{physics}
\usepackage[ruled,vlined,linesnumbered]{algorithm2e}
\usepackage{mathrsfs}
\usepackage{tikz-cd}
\usepackage{float}

\definecolor{mytheorembg}{HTML}{F2F2F9}
\definecolor{mytheoremfr}{HTML}{00007B}
\definecolor{doc}{RGB}{0,60,110}
\definecolor{myg}{RGB}{56, 140, 70}
\definecolor{myb}{RGB}{45, 111, 177}
\definecolor{myr}{RGB}{199, 68, 64}

\usetikzlibrary{decorations.pathreplacing,angles,quotes,patterns}
\definecolor{mytheorembg}{HTML}{F2F2F9}
\definecolor{mytheoremfr}{HTML}{00007B}
\definecolor{doc}{RGB}{0,60,110}
\definecolor{myg}{RGB}{56, 140, 70}
\definecolor{myb}{RGB}{45, 111, 177}
\definecolor{myr}{RGB}{199, 68, 64}

\tcbuselibrary{theorems,skins,hooks}
\newtcbtheorem{problem}{Problem}
{%
	enhanced,
	breakable,
	colback = mytheorembg,
	frame hidden,
	boxrule = 0sp,
	borderline west = {2pt}{0pt}{mytheoremfr},
	arc=5pt,
	detach title,
	before upper = \tcbtitle\par\smallskip,
	coltitle = mytheoremfr,
	fonttitle = \bfseries\sffamily,
	description font = \mdseries,
	separator sign none,
	segmentation style={solid, mytheoremfr},
}
{p}

\newtheorem{lemma}{Lemma}
\renewenvironment{proof}{\noindent{\it \textbf{Proof:}}\hspace*{1em}}{\qed\bigskip\\}
% To give references for any problem use like this
% suppose the problem number is p3 then 2 options either 
% \hyperref[p:p3]{<text you want to use to hyperlink> \ref{p:p3}}
%                  or directly 
%                   \ref{p:p3}



\input{../../letterfonts}

\input{../../macros}

\setlength{\parindent}{0pt}

%%%%%%%%%%%%%%%%%%%%%%%%%%%%%%%%%%%%%%%%%%%%%%%%%%%%%%%%%%%%%%%%%%%%%%%%%%%%%%%%%%%%%%%%%%%%%%%%%%%%%%%%%%%%%%%%%%%%%%%%%%%%%%%%%%%%%%%%

\begin{document}
	
	%%%%%%%%%%%%%%%%%%%%%%%%%%%%%%%%%%%%%%%%%%%%%%%%%%%%%%%%%%%%%%%%%%%%%%%%%%%%%%%%%%%%%%%%%%%%%%%%%%%%%%%%%%%%%%%%%%%%%%%%%%%%%%%%%%%%%%%%
	
	\textsf{\noindent \large\textbf{Soham Chatterjee} \hfill \textbf{Assignment - 1}\\
%		Email: \href{soham.chatterjee@tifr.res.in}{soham.chatterjee@tifr.res.in} \hfill
		\normalsize Course: Probability Theory \hfill Date: \today}
	
%%%%%%%%%%%%%%%%%%%%%%%%%%%%%%%%%%%%%%%%%%%%%%%%%%%%%%%%%%%%%%%%%%%%%%%%%%%%%%%%%%%%%%%%%%%%%%%%%%%%%%%%%%%%%%%%%%%%%%%%%%%%%%%%%%%%%%%%
% Problem 1
%%%%%%%%%%%%%%%%%%%%%%%%%%%%%%%%%%%%%%%%%%%%%%%%%%%%%%%%%%%%%%%%%%%%%%%%%%%%%%%%%%%%%%%%%%%%%%%%%%%%%%%%%%%%%%%%%%%%%%%%%%%%%%%%%%%%%%%%
	
\begin{problem}{%problem statement
	}{p1% problem reference text
}
\begin{enumerate}[label=(\alph*)]
	\item Prove that if $A_1, A_2, \ldots, A_n$ are events, then
	$$
	\mathbb{P}\left(\bigcup_{i=1}^n A_i\right)=S_1-S_2+S_3-\ldots+(-1)^{n-1} S_n
	$$
	where
	$$
	\begin{aligned}
		S_1 & =\sum_i \mathbb{P}\left(A_i\right) \\
		S_2 & =\sum_{i<j} \mathbb{P}\left(A_i \cap A_j\right) \\
		S_3 & =\sum_{i<j<k} \mathbb{P}\left(A_i \cap A_j \cap A_k\right) \\
		& \ldots \\
		S_n & =\mathbb{P}\left(A_1 \cap A_2 \cap \ldots \cap A_n\right)
	\end{aligned}
	$$
	
	This is also known as the \textit{inclusion-exclusion} principle.
	\item  \textit{Bonferroni inequalities} state that the sum of the first terms in the right-hand side of the identity we proved above is alternately an upper bound and a lower bound for the left-hand side. i.e., for odd $k \leq n$,
	$$
	P\left(\bigcup_{i=1}^n A_i\right) \leq S_1-S_2+\ldots+S_k
	$$
	and for even $k \leq n$
	$$
	P\left(\bigcup_{i=1}^n A_i\right) \geq S_1-S_2+\ldots-S_k
	$$
	
	Note that from what we showed above Bonferroni inequality holds with equality for $k=n$.
	
	Prove Bonferroni inequalities. Observe that the case of $k=1$ is what you know as the \textit{union bound} or Boole's inequality.
\end{enumerate}
\end{problem}
\solve{	
}
%%%%%%%%%%%%%%%%%%%%%%%%%%%%%%%%%%%%%%%%%%%%%%%%%%%%%%%%%%%%%%%%%%%%%%%%%%%%%%%%%%%%%%%%%%%%%%%%%%%%%%%%%%%%%%%%%%%%%%%%%%%%%%%%%%%%%%%%
% Problem 2
%%%%%%%%%%%%%%%%%%%%%%%%%%%%%%%%%%%%%%%%%%%%%%%%%%%%%%%%%%%%%%%%%%%%%%%%%%%%%%%%%%%%%%%%%%%%%%%%%%%%%%%%%%%%%%%%%%%%%%%%%%%%%%%%%%%%%%%%

\begin{problem}{%problem statement
	}{p2% problem reference text
	}
Prove or disprove the following:
\begin{itemize}
	\item The conditional independence of $A$ and $B$ given $C$ implies $A$ and $B$ are independent.
	\item  Independence of $A$ and $B$ implies the conditional independence of $A$ and $B$ given $C$.
\end{itemize}

If you disproved either of the claims above, for which events $C$ is it then the case that the following statement holds: for all events $A$ and $B$, the events $A$ and $B$ are conditionally independent given $C$ if and only if $A$ and $B$ are independent.\end{problem}
\solve{	
}

%%%%%%%%%%%%%%%%%%%%%%%%%%%%%%%%%%%%%%%%%%%%%%%%%%%%%%%%%%%%%%%%%%%%%%%%%%%%%%%%%%%%%%%%%%%%%%%%%%%%%%%%%%%%%%%%%%%%%%%%%%%%%%%%%%%%%%%%
% Problem 3
%%%%%%%%%%%%%%%%%%%%%%%%%%%%%%%%%%%%%%%%%%%%%%%%%%%%%%%%%%%%%%%%%%%%%%%%%%%%%%%%%%%%%%%%%%%%%%%%%%%%%%%%%%%%%%%%%%%%%%%%%%%%%%%%%%%%%%%%

\begin{problem}{%problem statement
	}{p3% problem reference text
	}
Let $A_1, A_2, \ldots$ be a sequence of events. Define
$$
B_n=\bigcup_{m=n}^{\infty} A_m \quad C_n=\bigcap_{m=n}^{\infty} A_m
$$

Clearly $C_n \subseteq A_n \subseteq B_n$. Also, the sequences $\left\{B_n\right\}$ and $\left\{C_n\right\}$ are decreasing respectively. Let
$$
B=\bigcap_{n=1}^{\infty} B_n=\bigcap_{n=1}^{\infty} \bigcup_{m \geq n} A_m \quad C=\bigcup_{n=1}^{\infty} C_n=\bigcup_{n=1}^{\infty} \bigcap_{m \geq \mathrm{n}} A_m
$$

The events $B$ and $C$ are denoted by $\limsup _{n \rightarrow \infty} A_n$ and $\liminf _{n \rightarrow \infty} A_n$ respectively. Show that
\begin{enumerate}[label=(\alph*)]
	\item $B=\left\{\omega \in \Omega: \omega \in A_n\right.$ for infinitely many values of $\left.n\right\}$.
	\item $C=\left\{\omega \in \Omega: \omega \in A_n\right.$ for all but finitely many values of $\left.n\right\}$.
\end{enumerate}	
	We say that a sequence $\left\{A_n\right\}$ converges to a limit $A$ if $B$ and $C$ are the same set $A$. We denote this by $A_n \rightarrow A$. Suppose this is the case, then show that
	
	\begin{enumerate}[resume, label=(\alph*)]
		\item $A$ is an event.
	\item $P\left(A_n\right) \rightarrow P(A)$.
\end{enumerate} 

\end{problem}
\solve{
	\begin{enumerate}[label=(\alph*)]
		\item Let $\om\in B$. Then $\om\in \bigcap\limits_{n=1}^{\infty}\bigcup\limits_{m\geq n}A_m$. Hence $\om \in \bigcup\limits_{m\geq n}A_m$ forall $n\in\bbN$. Hence $\om \in A_k$ for some $k\in \bbN$. Let $k_1$ be the least number such that $\om \in A_{k_1}$. Then we also have $\om \in B_{k_1+1}$. So we have some $k_2\geq k_1+1$ such that  $\om \in A_{k_2}$. Then $\om in B_{k_2+1}$. So there exists $k_3\geq k_2+1$ such that $\om \in A_{k_3}$. Continuing like this at $i^{th}$ step we have some $k_{i+1}\geq k_i+1$ such that $\om \in A_{k_{i+1}}$ and so on. So now we got an strictly increasing infinite sequence of positive integers $\lt\{k_1,k_2,k_3\dots,k_{i},\dots\rt\}$ such that $\om \in A_{k_j}$ for all $j\in\bbN$. Hence $\om \in \left\{\omega \in \Omega: \omega \in A_n\text{ for infinitely many values of }n\right\}$. Hence $$B\subseteq \left\{\omega \in \Omega: \omega \in A_n\text{ for infinitely many values of }n\right\}$$
	\end{enumerate}
}

%%%%%%%%%%%%%%%%%%%%%%%%%%%%%%%%%%%%%%%%%%%%%%%%%%%%%%%%%%%%%%%%%%%%%%%%%%%%%%%%%%%%%%%%%%%%%%%%%%%%%%%%%%%%%%%%%%%%%%%%%%%%%%%%%%%%%%%%
% Problem 4
%%%%%%%%%%%%%%%%%%%%%%%%%%%%%%%%%%%%%%%%%%%%%%%%%%%%%%%%%%%%%%%%%%%%%%%%%%%%%%%%%%%%%%%%%%%%%%%%%%%%%%%%%%%%%%%%%%%%%%%%%%%%%%%%%%%%%%%%

\begin{problem}{%problem statement
	}{p4% problem reference text
	}
 $10 \%$ of the surface of a sphere is coloured white, the rest is black. Show that, irrespective of the manner in which the colours are distributed, it is possible to inscribe a cube in $S$ with all its vertices black.

\textbf{Hint}: For a given distribution of colors, select the cube``uniformly randomly" (you should make this more concrete). First note that it is enough to prove that there is a non-zero probability with which all the vertices of this random cube are colored black (why?). Now try to use the union bound from Problem 1(b) above to show this.
\end{problem}
\solve{
To show that there exists a cube in $S$ with all its vertices black it is enough to show that if a random cube is chosen in $S$ the probability of all vertices black is greater than 0. Now we have $$\underset{\substack{C:\ cube\\ C\text{ is in $S$}}}{\bbP}[\text{All vertices of $C$ is black}]=1-\underset{\substack{C:\ cube\\ C\text{ is in $S$}}}{\bbP}[\text{At least one of the vertices of $C$ is white}]$$So its is enough to show that $\underset{\substack{C:\ cube\\ C\text{ is in $S$}}}{\bbP}[\text{At least one of the vertices of $C$ is white}]<1$. Now we also have \begin{align*}
	\underset{\substack{C:\ cube\\ C\text{ is in $S$}}}{\bbP}[\text{At least one of the vertices of $C$ is white}]
	=\underset{\substack{X_i\in S\\ \forall\ i\in[8]}}{\bbP}\lt[ \exs\ i\in[8]\text{ $X_i$ is colored white}\mid X_1,\dots,X_8 \text{ forms a cube} \rt]\\
\end{align*}
Now by Union Bound we have \begin{multline*}
	\underset{\substack{X_i\in S\\ \forall\ i\in[8]}}{\bbP}\lt[ \exs\ i\in[8]\text{ $X_i$ is colored white}\mid X_1,\dots,X_8 \text{ forms a cube} \rt]\\
	\leq  \sum_{j=1}^8\underset{\substack{X_i\in S\\ \forall\ i\in[8]}}{\bbP}\lt[X_j\text{ is colored white}\mid   X_1,\dots,X_8 \text{ forms a cube}\rt]
\end{multline*}

So now showing $$ \sum_{j=1}^8\underset{\substack{X_i\in S\\ \forall\ i\in[8]}}{\bbP}\lt[X_j\text{ is colored white}\mid   X_1,\dots,X_8 \text{ forms a cube}\rt]<1$$ is enough. Now for any $j\in[8]$, $$\underset{\substack{X_i\in S\\ \forall\ i\in[8]}}{\bbP}\lt[X_j\text{ is colored white}\mid   X_1,\dots,X_8 \text{ forms a cube}\rt]=\underset{\substack{X_i\in S\\ \forall\ i\in[8]}}{\bbP}\lt[X_j\text{ is colored white}\rt]=\frac{1}{10}$$The last equality because $X_j$ is colored white if it is a point picked from the $10\%$ area of the sphere which is colored white and the probability of that is $\frac1{10}$.Therefore we have $$\sum_{j=1}^8\underset{\substack{X_i\in S\\ \forall\ i\in[8]}}{\bbP}\lt[X_j\text{ is colored white}\mid   X_1,\dots,X_8 \text{ forms a cube}\rt]=\sum_{j=1}^8\frac1{10}=\frac{8}{10}<1$$Therefore we have $\underset{\substack{C:\ cube\\ C\text{ is in $S$}}}{\bbP}[\text{At least one of the vertices of $C$ is white}]<1\implies \underset{\substack{C:\ cube\\ C\text{ is in $S$}}}{\bbP}[\text{All vertices of $C$ is black}]>0$. Which means there exists a cube in $S$ with all vertices black
 }
\end{document}
