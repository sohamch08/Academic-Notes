\documentclass{article}
\usepackage{fullpage}
\usepackage{amsmath}
\usepackage{amsfonts}
\usepackage{authblk,caption}
\usepackage{titling}
\usepackage{tikz}
\usepackage{hyperref}
\title{\huge{Classical Mechanics 1, Autumn 2021 CMI \\ Problem set 6\\\hspace{7cm}- Govind S. Krishnaswami}
}
\author{Soham Chatterjee\\Roll: BMC202175}
\date{}

\newcommand{\bma}{\boldsymbol{a}}
\newcommand{\bmb}{\boldsymbol{b}}
\newcommand{\bmc}{\boldsymbol{c}}
\newcommand{\bmr}{\boldsymbol{r}}
\newcommand{\bmv}{\boldsymbol{v}}
\newcommand{\bmp}{\boldsymbol{p}}
\newcommand{\bmF}{\boldsymbol{F}}
\newcommand{\bmf}{\boldsymbol{f}}
\renewcommand\maketitlehooka{\null\mbox{}\vfill}
\renewcommand\maketitlehookd{\vfill\null}

\setlength{\parindent}{1cm}
\begin{document}
	\maketitle\pagebreak
	\begin{enumerate}
		\item \begin{enumerate}
			\item If mass of two particles $A,B$ are $m_a,m_b$ and the distance between them is $d$ then the Gravitational Force $F_G$ between them is given by $$F_G=\frac{G\,m_a\,m_b}{d^2}$$Now given that dimension of $G$ is $[{M^{\alpha}L^{\beta}T^{\gamma}}]$ then \begin{align*}
				& [\text{Force}(F)]=\frac{[G]\, [\text{Mass}(m_a)]\,[\text{Mass}(m_b)]}{[\text{Distance}(d)^2]}\\
				\implies &MLT^{-2}=\frac{{M^{\alpha}L^{\beta}T^{\gamma}}\cdot M\cdot M}{L^2}\\
				\implies & {M^{\alpha}L^{\beta}T^{\gamma}}=MLT^{-2}\cdot L^2\cdot M^{-1}\cdot M^{-1}\\
				\implies & {M^{\alpha}L^{\beta}T^{\gamma}}=M^{-1}L^3T^{-2}
			\end{align*}Therefore $\alpha=-1,\beta=3,\gamma=-2$. Hence dimension of $G$ is $M^{-1}L^3T^2$
		\item $E=h\nu$. Here $[E]=[\text{Energy}] =ML^2T^{-2}$ and $[\nu]=[\text{Frequency}]=T^{-1}$. Given that $[h]=M^{\alpha}L^{\beta}T^{\gamma}$ Then \begin{align*}
			&ML^2T^{-2}=M^{\alpha}L^{\beta}T^{\gamma}\cdot T^{-1}\\
			\implies & M^{\alpha}L^{\beta}T^{\gamma}=ML^2T^{-2}\cdot T\\
			\implies & M^{\alpha}L^{\beta}T^{\gamma}=ML^2T^{-1}\\
		\end{align*}Therefore $\alpha=1,\beta=2,\gamma=-1$. Hence dimension of $h$ is $ML^2T^{-1}$
 \item We found earlier that $[G]=M^{-1}L^3T^{-2}$ and $[c^2]=[\text{Speed of light}^2]=L^2T^{-2}$ and $[M_e]=[\text{Mass}]=M$. Given that $\left[\frac{G\, M_e}{c^2}\right]=M^{\alpha}L^{\beta}T^{\gamma}  $. Then\begin{align*}
 	& M^{\alpha}L^{\beta}T^{\gamma}=\frac{M^{-1}L^3T^{-2}\cdot M}{L^2T^{-2}}\\
 	\implies & M^{\alpha}L^{\beta}T^{\gamma}=M^0LT^0
 \end{align*}Therefore $\alpha=0,\beta=1,\gamma=0$. Hence dimension of $\frac{G\, M_e}{c^2}$ is $M^0LT^0$
\item As given that $\int^{\infty}_{-\infty}\psi^2(x)dx=1$, we can say for every $\psi^2(x)dx$ the dimension of it is $M^0L^0T^0$ Now $[dx]=[\text{Length}]=L$. Given that $[\psi(x)]=M^{\alpha}L^{\beta}T^{\gamma}$ then\begin{align*}
	&M^0L^0T^0=M^{2\alpha}L^{2\beta}T^{2\gamma}\cdot L\\
	\implies&M^{\alpha}L^{\beta}T^{\gamma}=M^0L^{\frac12}T^0
\end{align*}Therefore $\alpha=0,\beta=\frac12,\gamma=0$. Hence dimension of $\psi(x)$ is $M^0L^{\frac12}T^0$
			\end{enumerate}
		
%------------------------------------------------------------------------------
		
		\item \begin{enumerate}
			\item Given the $f$ the potential $V(x)$ can not be uniquely determined because since for both potentials $V(x)$ and $V'(x)=V(x)+c$ where $c$ is a real number the force upon the particle is same. Hence with the force alone we can not determine the potential at that point uniquely.
	\item $f(x)=-kx$ where $k>0$ a fixed constant. Now $f(x)=-V'(x)$. Hence $-V'(x)=-kx\implies kx=V'(x)$. Therefore\begin{align*}
		&\int_{x_0}^x V'(x)dx=\int_{x_0}^x kx\\
		\implies & V(x)-V(x_0)=\frac12kx^2-\frac12kx_0^2\\
		\implies & V(x)=\frac12kx^2+\left( V(x_0)-\frac12kx_0^2\right) 
	\end{align*}Where $x_0$ is the initial position of the spring end. Let $c= V(x_0)-\frac12kx_0^2$ then $V(x)=\frac12kx^2+c$
\begin{center}
	
\begin{tikzpicture}
	\draw[->] (-4.2, 0)node[xshift=5.5cm,yshift=-0.2cm]{$x\longrightarrow$} -- (4.2, 0) node[right] {$x$};
	\draw[->] (0, -1)node[xshift=-0.3cm,yshift=4cm,rotate=90]{$V(x)\longrightarrow$} -- (0, 4.2) node[above] {$y$};
	\draw[scale=0.5, domain=-3:3, smooth, variable=\x, blue] plot ({\x}, {0.75*\x*\x+1});
	\draw (1.5,2.6875)node[xshift=1.2cm]{$V(x)=\frac12kx^2+c$};
	
\end{tikzpicture}

Graph of $V(x)$ as a function of $x$

\end{center}

\hspace{1cm}If a spring is elongated in a certain direction then the restoring force of the spring acts completely opposite of the direction of elongation. Even if the spring is contracted in a certain direction still the restoring force acts opposite to the direction of contraction. Restoring force always acts opposite to the elongation or contraction. That is why to make the force opposite to the direction of $x$ there is a negative sign which indicates that restoring force acts opposite to the direction of $x$.
	\end{enumerate}
\item given that $f_i=-\frac{\partial V }{\partial x_i}$ where $i\in\{1,2,3\}$ and $m\ddot{\bmr}=\bmf$. Now $\bmr=x_1\hat{i}+x_2\hat{j}+x_3\hat{k}$ therefore $\ddot{\bmr}=\ddot{x_1}\hat{i}+\ddot{x_2}\hat{j}+\ddot{x_3}\hat{k}$. Hence we can write\begin{align*}
	&m\left( \ddot{x_1}\hat{i}+\ddot{x_2}\hat{j}+\ddot{x_3}\hat{k}\right) =-\frac{\partial V }{\partial x_1}-\frac{\partial V }{\partial x_2}-\frac{\partial V }{\partial x_3}\\
	\implies & m(\ddot{x_1}\hat{i}+\ddot{x_2}\hat{j}+\ddot{x_3}\hat{k})\cdot (\dot{x_1}\hat{i}+\dot{x_2}\hat{j}+\dot{x_3}\hat{k})+\left(\frac{\partial V }{\partial x_1}+\frac{\partial V }{\partial x_2}+\frac{\partial V }{\partial x_3} \right) \cdot (\dot{x_1}\hat{i}+\dot{x_2}\hat{j}+\dot{x_3}\hat{k})=0\\
	\implies &m (\ddot{x_1}\dot{x_1}\hat{i}+\ddot{x_2}\dot{x_2}\hat{j}+\ddot{x_3}\dot{x_3}\hat{k})+\frac{d V}{d t}=0\\
	\implies &\frac12 m(2\dot{x_1}\ddot{x_1}\hat{i}+2\dot{x_2}\ddot{x_2}\hat{j}+2\dot{x_3}\ddot{x_3}\hat{k})+\frac{d V}{d t}=0\\
	\implies & m \frac{d }{d t}\left(\dot{x_1}^2\hat{i}+\dot{x_2}^2\hat{j}+\dot{x_3}^2\hat{k} \right) +\frac{d V}{d t}=0\\
	\implies &\frac{d }{d t}\left( \frac12m\dot{\bmr}^2+V\right)=0 
\end{align*}Hence the equation of conserved energy ($E$) is $E=\frac12 m\dot{\bmr}^2+V$
\item \begin{enumerate}
	\item The conservative force arising from the potential $V(x)$ is $-V'(x)$. Hence the Newton's Equation of motion of the particle is \begin{equation}
		m\ddot{x}=-V'(x)-\gamma \dot{x}\label{e1}
	\end{equation} where $x$ is actually a function of time $t$. Here in case of $\ddot{x}$ and $\dot{x}$ $x$ is differentiated with respect to time and in case of $V'(x)$, $V(x)$ is differentiated with respect to position $x$. 
	\item Differentiating both sides of $E=\frac12m\dot{x}^2+V(x)$ with respect to time $t$ we get\begin{align*}
		 \frac{d\,E}{dt}\ &=\frac{d}{dt}\left( \frac12m\dot{x}^2+V(x)\right)  \\
		& = \frac12m(2\dot{x})\ddot{x}+V'(x)\dot{x}\\
		& = m\ddot{x}\dot{x}+V'(x)\dot{x}\\
		& = \dot{x}(m\ddot{x}+V(x))\\
		& =\dot{x}(-\gamma \dot{x}) \ [\text{Using \eqref{e1}}]\\
		& = -\gamma\dot{x}^2
	\end{align*}
As given that $\gamma>0$ hence $\frac{d\, E}{dt}\leq 0$. Therefore $E$ is non-increasing function of time.
\item From the part ($b)$ Problem 4 we can see $\gamma$ is non-decreasing over function of time. Hence we can call $\gamma$ as \textbf{Coefficient of Energy Dissipation}.
\end{enumerate}
		\end{enumerate}
\end{document}