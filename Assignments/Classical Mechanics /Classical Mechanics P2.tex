\documentclass{article}
\usepackage{fullpage}
\usepackage{amsmath}
\usepackage{amsfonts}
\usepackage{authblk}
\usepackage{titling}
\usepackage{tikz}

\title{\huge{Classical Mechanics 1, Autumn 2021 CMI \\ Problem set 2\\\hspace{7cm}- Govind S. Krishnaswami}
}
\author{Soham Chatterjee\\Roll: BMC202175}
\date{}


\renewcommand\maketitlehooka{\null\mbox{}\vfill}
\renewcommand\maketitlehookd{\vfill\null}

\setlength{\parindent}{1cm}
\begin{document}
	\maketitle
	\pagebreak
	
\begin{enumerate}
	\item The radius of the circle = $l$ and the angular speed is $\omega$. Hence at any time $t$ if $\boldsymbol{r}(t)$ is the position vector of the particle then$$\boldsymbol{r}(t)=l\cos\omega t \hat{x}+l\sin\omega t\hat{y}$$
	\begin{center}
	\begin{tikzpicture}
		\draw[<->] (0,-1)node[yshift=-0.2cm]{$Y'$} -- (0,4)node[yshift=0.2cm]{$Y$};
		\draw[<->] (-1,0) node[xshift=-0.2cm]{$X'$} -- (4,0)node[xshift=0.2cm]{$X$};
		\draw (3,0) arc [start angle=0,end angle=90, radius=3cm];
		\draw[->] (0,0) -- (1.8,2.4) node[xshift=1.8cm,yshift=0.2cm]{$\boldsymbol{r}(t)=l\cos\omega t \hat{x}+l\sin\omega t\hat{y}$};
		\draw[dashed] (1.8,2.4) -- (1.8,0)node[xshift=0.2cm,yshift=1cm,rotate=-90]{$l\sin\omega t$};
		\draw[dashed] (1.8,2.4) -- (0,2.4) node[xshift=0.7cm,yshift=0.2cm]{$l\cos\omega t$};
		\draw[<->] (0,-0.3) -- (3,-0.3) node[yshift=-0.35cm,xshift=-1cm]{$l$};
	\end{tikzpicture}
	\end{center}
		Hence velocity of the particle $(\boldsymbol{v}(t))$ at any time $t$ is the derivative of $\boldsymbol{r}(t)$. Hence$$\boldsymbol{v}(t)=\dot{\boldsymbol{r}}(t)=-l\omega\sin\omega t\hat{x}+l\omega\cos\omega t\hat{y}$$
		Given that $\boldsymbol{a}(t)$ is the acceleration of the particle. Hence$$\boldsymbol{a}(t)=\dot{\boldsymbol{v}}(t)=-l\omega^2\cos\omega t\hat{x}-l\omega^2\sin\omega t\hat{y}=-\omega^2\boldsymbol{r}(t)$$Therefore$$\dot{\boldsymbol{a}}(t)=-\omega^2\dot{\boldsymbol{r}}(t)=-\omega^2\boldsymbol{v}(t)$$Hence \begin{align*}
			\dot{\boldsymbol{a}}(t)\cdot\boldsymbol{a}(t)\ &=-\omega^2\boldsymbol{v}(t)\cdot (-\omega^2\boldsymbol{r}(t))\\
			& =\omega^4\boldsymbol{v}\cdot\boldsymbol{r}\\
			&=\omega^4(-l\omega\sin\omega t\hat{x}+l\omega\cos\omega t\hat{y})\cdot (l\cos\omega t \hat{x}+l\sin\omega t\hat{y})\\
			&=\omega^4(-l\omega^2\sin\omega t\cos\omega t+l\omega^2\cos\omega t\sin \omega t)=0
		\end{align*}
	Hence $\dot{\boldsymbol{a}}(t)\cdot\boldsymbol{a}(t)=0$. Therefore acceleration and its derivative are perpendicular and henceforth the derivative of the acceleration only changes the direction of acceleration not its value.
	\pagebreak
		\item Let any point whose cartesian coordinate is $(x,y)$ and polar coordinate is $(r,\theta)$. $\hat{x}$, $\hat{y}$ denote the unit vectors in cartesian coordinate system and $\hat{r}$, $\hat{\theta}$ denote the unite vectors in polar coordinate system.
		\begin{center}
			\begin{tikzpicture}
				\draw[<->] (0,-1)node[yshift=-0.2cm]{$Y'$} -- (0,4)node[yshift=0.2cm]{$Y$};
				\draw[<->] (-1,0) node[xshift=-0.2cm]{$X'$} -- (4,0)node[xshift=0.2cm]{$X$};
				\draw (0,0) node[xshift=-0.2cm,yshift=-0.2cm]{$O$};
				\draw (1,0) arc [start angle=0,end angle=37, radius=1cm]node[xshift=0.3cm,yshift=-0.3cm]{$\theta$};
				\draw (0,0) -- (3,2.25);
				\draw[->] (3,2.25)--(4,3)node[xshift=-0.5cm,yshift=-0.2cm]{$\hat{r}$};
				\draw[->] (3,2.25)--(2.25,3.25)node[xshift=0.15cm,yshift=-0.5cm]{$\hat{\theta}$};
				\draw[->] (3,2.25)--(3,3.25)node[xshift=0.15cm,yshift=-0.4cm]{$\hat{y}$};
				\draw[->] (3,2.25)--(4,2.25)node[xshift=-0.25cm,yshift=-0.2cm]{$\hat{x}$};
				\draw (3.5,2.25) arc [start angle=0,end angle=37, radius=0.5cm]node[xshift=0.23cm,yshift=-0.13cm]{$\theta$};
				\draw (3,2.75) arc [start angle=90,end angle=127, radius=0.5cm]node[xshift=0.13cm,yshift=0.2cm]{$\theta$};
			\end{tikzpicture}
		\end{center}
	We can decompose $\hat{r}$ and $\hat{\theta}$ as\begin{equation}
		\hat{r}=\cos\theta\hat{x}+\sin\theta\hat{y}=\frac{x}{r}\hat{x}+\frac{y}{r}\hat{y}\label{e1}
	\end{equation}
\begin{equation}
	\hat{\theta}=-\sin\theta\hat{x}+\cos\theta\hat{y}=-\frac{y}{r}\hat{x}+\frac{x}{r}\hat{y}\label{e2}
\end{equation}Now multiplying \eqref{e1} with $y$ and \eqref{e2} with $x$ and adding them we get\begin{align*}
&y\hat{r}+x\hat{\theta}\ =(\frac{xy}{r}\hat{x}+\frac{y^2}{r}\hat{y})+(-\frac{xy}{r}\hat{x}+\frac{x^2}{r}\hat{y})\\
\implies&y\hat{r}-x\hat{\theta}\ =\frac{x^2+y^2}{r}\hat{y}\\
\implies & \frac{r^2}{r}\hat{y}=y\hat{r}+x\hat{\theta}
\implies & \hat{y}=\frac{y}{r}\hat{r}+\frac{x}{r}\hat{\theta}
\end{align*}
Now multiplying \eqref{e1} with $x$ and \eqref{e2} with $y$ and subtracting \eqref{e2} from \eqref{e1} we get\begin{align*}
	&x\hat{r}-y\hat{\theta}\ =(\frac{x^2}{r}\hat{x}+\frac{xy}{r}\hat{y})+(-\frac{y^2}{r}\hat{x}+\frac{xy}{r}\hat{y})\\
	\implies&x\hat{r}-y\hat{\theta}\ =\frac{x^2+y^2}{r}\hat{y}\\
	\implies & \frac{r^2}{r}\hat{x}=x\hat{r}-y\hat{\theta}\\
	\implies & \hat{x}=\frac{x}{r}\hat{r}-\frac{y}{r}\hat{\theta}
\end{align*}Therefore$$\hat{x}=\frac{x}{r}\hat{r}-\frac{y}{r}\hat{\theta}$$ and $$\hat{y}=\frac{y}{r}\hat{r}+\frac{x}{r}\hat{\theta}$$
\pagebreak
\item Here we are doing the transformation $$(x,y)\mapsto(u,v)=(ax+by,cx+dy)$$Hence for any vector $\boldsymbol{r}=x\hat{x}+y\hat{y}$ will be transformed into$$\boldsymbol{r}'=(ax+by)\hat{u}+(cx+dy)\hat{v}$$For $\boldsymbol{u}$ and $\boldsymbol{v}$ we know$$\boldsymbol{u}=a\hat{x}+b\hat{y}\text{ and }\boldsymbol{v}=c\hat{x}+d\hat{y}$$For $\boldsymbol{u}$ and $\boldsymbol{v}$ to be orthogonal their dot product have to be zero and the modulus of $\boldsymbol{u}$ and $\boldsymbol{v}$ have to be 1.
$$\boldsymbol{u}\cdot\boldsymbol{v}=0\implies(a\hat{x}+b\hat{y})\cdot(c\hat{x}+d\hat{y})=0\implies ac+bd=0$$ $$|\boldsymbol{u}|=1\implies\sqrt{a^2+b^2=1}\implies a^2+b^2=1$$ $$|\boldsymbol{v}|=1\implies\sqrt{c^2+d^2=1}\implies c^2+d^2=1$$Now let $A=\begin{pmatrix}
	a & b\\
	c&d
\end{pmatrix}$Then $A^T=\begin{pmatrix}
a&c\\
b&d
\end{pmatrix}$Then $$AA^T=\begin{pmatrix}
a & b\\
c&d
\end{pmatrix}\begin{pmatrix}
a&c\\
b&d
\end{pmatrix}=\begin{pmatrix}
a^2+b^2&ac+bd\\
ac+bd&c^2+d^2
\end{pmatrix}=\begin{pmatrix}
1&0\\
0&1
\end{pmatrix}=I$$Therefore the necessary condition for the matrix $\begin{pmatrix}
a & b\\
c&d
\end{pmatrix}$ is that it has to be orthogonal.
\item Given that $$\begin{aligned}
	&\hat{r}=\cos \theta \hat{z}+\sin \theta(\cos \phi \hat{x}+\sin \phi \hat{y}) \\
	&\hat{\theta}=-\sin \theta \hat{z}+\cos \theta(\cos \phi \hat{x}+\sin \phi \hat{y}) \\
	&\hat{\phi}=-\sin \phi \hat{x}+\cos \phi \hat{y}
\end{aligned}$$Now \begin{align*}
\hat{r}\times\hat{\theta}\ &=\begin{vmatrix}
	\hat{x}&\hat{y}&\hat{z}\\
	\sin \theta\cos \phi&\sin \theta\sin\phi&\cos \theta\\
	\cos \theta\cos \phi&\cos \theta\sin \phi&-\sin\theta
\end{vmatrix}\\
&=(-\sin^2\theta\sin\phi-\cos^2\theta\sin\phi)\hat{x}+(\cos^2\theta\cos\phi+\sin^2\theta\cos\phi)\hat{y}\\
&\qquad\qquad\qquad\qquad\qquad\qquad+(\sin\theta\cos\theta\sin\phi\cos\phi-\sin\theta\cos\theta\sin\phi\cos\phi)\hat{z}\\
&=-\sin\phi\hat{x}+\cos\phi\hat{y}\\
&=\hat{\phi}\end{align*}
Now to confirm $\hat{r},\hat{\theta},\hat{\phi}$ are orthogonal we have to check if the dot product of any two are 0.\begin{align*}
	\hat{r}\cdot\hat{\theta}\ &=(\cos \theta \hat{z}+\sin \theta(\cos \phi \hat{x}+\sin \phi \hat{y}))\cdot(-\sin \theta \hat{z}+\cos \theta(\cos \phi \hat{x}+\sin \phi \hat{y}))\\
	&=-\sin\theta\cos\theta+\sin\theta\cos\theta\sin^2\phi+\sin\theta\cos\theta\cos^2\phi\\
	&=-\sin\theta\cos\theta+\sin\theta\cos\theta=0\\
	\qquad\\
	\hat{\theta}\cdot\hat{\phi}\ &=(-\sin \theta \hat{z}+\cos \theta(\cos \phi \hat{x}+\sin \phi \hat{y}))\cdot(-\sin \phi \hat{x}+\cos \phi \hat{y})\\
	&=-\cos\theta\cos\phi\sin\phi+\cos\theta\cos\phi\cos\phi=0\\
	\qquad\\
	\hat{\phi}\cdot\hat{r}\ &=(-\sin \phi \hat{x}+\cos \phi \hat{y})\cdot(\cos \theta \hat{z}+\sin \theta(\cos \phi \hat{x}+\sin \phi \hat{y}))\\
	&=-\sin\theta\cos\phi\sin\phi+\sin\theta\cos\phi\cos\phi=0
\end{align*}
Therefore $\hat{r},\hat{\theta},\hat{\phi}$ are perpendicular to each other. Hence $(\hat{r},\hat{\theta},\hat{\phi})$ is. right handed orthogonal.
\item Given that$$\hat{r}=\cos \theta(t) \hat{z}+\sin \theta(t)\cos \phi(t) \hat{x}+\sin \theta(t)\sin \phi(t) \hat{y}$$Hence \begin{align*}
	\dot{\hat{r}}\ &=-\dot{\theta}(t)\sin \theta \hat{z}+(\dot{\theta}(t)\cos\theta(t)\cos\phi(t)-\dot{\phi}(t)\sin\theta(t)\sin\phi(t))\hat{x}\\
	&\qquad\qquad\qquad\qquad\qquad\qquad+(\dot{\theta}(t)\cos\theta(t)\sin\phi(t)+\dot{\phi}(t)\sin\theta(t)\cos\phi(t))\hat{y}\\
	&=\dot{\theta}(t)\big(-\sin \theta \hat{z}+\cos\theta(t)\cos\phi(t)\hat{x}+\cos\theta(t)\sin\phi(t)\hat{y}\big)+\dot{\phi}(t)\big(-\sin\theta(t)\sin\phi(t)\hat{x}+\sin\theta(t)\cos\phi(t)\hat{y}\big)\\
	&=\dot{\theta}(t)\hat{\theta}+\dot{\phi}(t)\sin\theta(t)\big(-\sin\phi(t)\hat{x}+\cos\phi(t)\hat{y}\big)\\
	&=\dot{\theta}(t)\hat{\theta}+\dot{\phi}(t)\sin\theta(t)\hat{\phi}
\end{align*}
Hence$$\dot{\hat{r}}=\dot{\theta}(t)\hat{\theta}+\dot{\phi}(t)\sin\theta(t)\hat{\phi}$$ where coefficient of $\hat{r}$ is 0.
\end{enumerate}
\end{document}