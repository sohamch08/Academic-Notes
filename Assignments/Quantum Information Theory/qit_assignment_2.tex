\documentclass[a4paper, 11pt]{article}
\usepackage{comment} % enables the use of multi-line comments (\ifx \fi) 
\usepackage{fullpage} % changes the margin
\usepackage[a4paper, total={7in, 10in}]{geometry}
\usepackage{amsmath,mathtools,mathdots}
\usepackage{amssymb,amsthm}  % assumes amsmath package installed
\usepackage{float}
\usepackage{xcolor}
\usepackage{mdframed}
\usepackage[shortlabels]{enumitem}
\usepackage{indentfirst}
\usepackage{hyperref}
\hypersetup{
	colorlinks=true,
	linkcolor=doc!80,
	citecolor=myr
	filecolor=magenta,      
	urlcolor=doc!80,
	pdftitle={Assignment}, %%%%%%%%%%%%%%%%   WRITE ASSIGNMENT PDF NAME  %%%%%%%%%%%%%%%%%%%%
}
\usepackage[most,many,breakable]{tcolorbox}
\usepackage{tikz}
\usepackage{caption}
\usepackage{mathpazo}
\usepackage{libertine}
\usepackage{physics}
\usepackage{mathrsfs}


\definecolor{mytheorembg}{HTML}{F2F2F9}
\definecolor{mytheoremfr}{HTML}{00007B}
\definecolor{doc}{RGB}{0,60,110}
\definecolor{myg}{RGB}{56, 140, 70}
\definecolor{myb}{RGB}{45, 111, 177}
\definecolor{myr}{RGB}{199, 68, 64}

\usetikzlibrary{decorations.pathreplacing,angles,quotes,patterns}
\definecolor{mytheorembg}{HTML}{F2F2F9}
\definecolor{mytheoremfr}{HTML}{00007B}
\definecolor{doc}{RGB}{0,60,110}
\definecolor{myg}{RGB}{56, 140, 70}
\definecolor{myb}{RGB}{45, 111, 177}
\definecolor{myr}{RGB}{199, 68, 64}

\tcbuselibrary{theorems,skins,hooks}
\newtcbtheorem{problem}{Problem}
{%
	enhanced,
	breakable,
	colback = mytheorembg,
	frame hidden,
	boxrule = 0sp,
	borderline west = {2pt}{0pt}{mytheoremfr},
	arc=5pt,
	detach title,
	before upper = \tcbtitle\par\smallskip,
	coltitle = mytheoremfr,
	fonttitle = \bfseries\sffamily,
	description font = \mdseries,
	separator sign none,
	segmentation style={solid, mytheoremfr},
}
{p}

% To give references for any problem use like this
% suppose the problem number is p3 then 2 options either 
% \hyperref[p:p3]{<text you want to use to hyperlink> \ref{p:p3}}
%                  or directly 
%                   \ref{p:p3}



\input{../../letterfonts}

\input{../../macros}
\newcommand{\fdcps}{finite dimensional complex inner product space}
\newcommand{\fdcpss}{finite dimensional complex inner product spaces}
\newcommand{\Fdcps}{Finite dimensional complex inner product space}
\newcommand{\Fdcpss}{Finite dimensional complex inner product spaces}
\setlength{\parindent}{0pt}

%%%%%%%%%%%%%%%%%%%%%%%%%%%%%%%%%%%%%%%%%%%%%%%%%%%%%%%%%%%%%%%%%%%%%%%%%%%%%%%%%%%%%%%%%%%%%%%%%%%%%%%%%%%%%%%%%%%%%%%%%%%%%%%%%%%%%%%%

\begin{document}
	
	%%%%%%%%%%%%%%%%%%%%%%%%%%%%%%%%%%%%%%%%%%%%%%%%%%%%%%%%%%%%%%%%%%%%%%%%%%%%%%%%%%%%%%%%%%%%%%%%%%%%%%%%%%%%%%%%%%%%%%%%%%%%%%%%%%%%%%%%
	
	\textsf{\noindent \large\textbf{Soham Chatterjee} \hfill \textbf{Assignment - 2.2 : Quantum Foundations}\\
		Email: \href{sohamc@cmi.ac.in}{sohamc@cmi.ac.in} \hfill Roll: BMC202175\\
		\normalsize Course: Quantum Information Theory \hfill Date: February 29, 2024}
\vspace{1cm}

For all the questions \begin{itemize}
	\item $[k]\coloneqq \{1,2,\dots,k\}$ where $k\in\bbN$.
	\item $\sL(\mcH)\coloneqq $ Linear operators on $\mcH$
	\item $\sR(\mcH)\coloneqq $ Self-adjoint or hermitian operators on $\mcH$
	\item $\sP(\mcH)\coloneqq $ Positive semi-definite operators on $\mcH$
	\item $\sD(\mcH)\coloneqq $ Density operators on $\mcH$
\end{itemize}
%%%%%%%%%%%%%%%%%%%%%%%%%%%%%%%%%%%%%%%%%%%%%%%%%%%%%%%%%%%%%%%%%%%%%%%%%%%%%%%%%%%%%%%%%%%%%%%%%%%%%%%%%%%%%%%%%%%%%%%%%%%%%%%%%%%%%%%%
% Problem 1
%%%%%%%%%%%%%%%%%%%%%%%%%%%%%%%%%%%%%%%%%%%%%%%%%%%%%%%%%%%%%%%%%%%%%%%%%%%%%%%%%%%%%%%%%%%%%%%%%%%%%%%%%%%%%%%%%%%%%%%%%%%%%%%%%%%%%%%%
	
\begin{problem}{%problem statement
	}{p1% problem reference text
}
For $T:\mcH\to \mcH$, prove that  $$\sum_{i=1}^d\la e_i\ket{Te_i}=\sum_{i=1}^d \la f_i\ket{Tf_i}$$if $\{\ket{e_i}\in \mcH\mid 1\leq i\leq d\}$ and $\{\ket{f_i}\in \mcH\mid 1\leq i\leq d\}$ are ONB.
		%Problem		
\end{problem}
	
\solve{
	Let $S:\mcH\to \mcH$ where it maps the basis vectors from $\ket{e_i}\to \ket{f_i}$. Then $S\ket{e_i}=\ket{f_i}$. Hence $S$ is an orthonormal matrix since $$\bra{e_j}S^{\dagger}S\ket{e_i}=\bra{f_j}f_i\ra=\delta_{ji}\quad \text{and} \quad \bra{f_j}SS^{\dagger}\ket{f_i}=\bra{e_j}e_i\ra=\delta_{ji}$$Hence $$\sum_{i=1}^d \la f_i|Tf_i\ra =\sum_{i=1}^d \la e_i|S^{\dagger}TS\ket{e_i}=tr{(S^{\dagger}TS)}=tr(SS^{\dagger}T)=tr(T)=\sum_{i=1}^d\la e_i\ket{Te_i}$$Therefore we have $$\sum_{i=1}^d\la e_i\ket{Te_i}=\sum_{i=1}^d \la f_i\ket{Tf_i}$$
}
%%%%%%%%%%%%%%%%%%%%%%%%%%%%%%%%%%%%%%%%%%%%%%%%%%%%%%%%%%%%%%%%%%%%%%%%%%%%%%%%%%%%%%%%%%%%%%%%%%%%%%%%%%%%%%%%%%%%%%%%%%%%%%%%%%%%%%%%
% Problem 2
%%%%%%%%%%%%%%%%%%%%%%%%%%%%%%%%%%%%%%%%%%%%%%%%%%%%%%%%%%%%%%%%%%%%%%%%%%%%%%%%%%%%%%%%%%%%%%%%%%%%%%%%%%%%%%%%%%%%%%%%%%%%%%%%%%%%%%%%

\begin{problem}{%problem statement
	}{p2% problem reference text
}
If $\{\ket{e_i}\in \mcH_1\mid 1\leq i\leq d\}$ and $\{\ket{f_i}\in \mcH_2\mid 1\leq i\leq d\}$ are ONB, then  $\{  \ket{e_i}\otimes \ket{f_j}\mid 1\leq i,j\leq d\}\subseteq \mcH_1\tensor \mcH_2$ is ONB
	%Problem		
\end{problem}

\solve{
	Let $\ket{\psi}\tensor \ket{\phi}\in \mcH_1\tensor \mcH_2$. Then $\ket{\psi}=\sum\limits_{i=1}^d\alpha_i\ket{e_i}$ where $\alpha_i\in \bbC$ for all $i\in [d]$ since $\{\ket{e_i}\in \mcH_1\mid 1\leq i\leq d\}$ is ONB for $\mcH_1$. Hence $$\ket{\psi}\tensor \ket{\phi}=\sum_{i=1}^d\alpha_{i}\ket{e_i}\tensor \ket{\phi}$$Now $\ket{\phi}=\sum\limits_{i=1}^d\beta_i\ket{f_i}$ where $\beta_i\in \bbC$ for all $i\in [d]$ since $\{\ket{f_i}\in \mcH_2\mid 1\leq i\leq d\}$ is ONB for $\mcH_2$. Hence $$\forall \ i\in [d]\ \ket{e_i}\tensor \ket{phi}=\sum_{j=1}^d\beta_j\ket{e_i}\tensor \ket{f_j}$$Thereofore we get $$\ket{\psi}\tensor \ket{\phi}=\sum_{i=1}^d\alpha_{i}\ket{e_i}\tensor \ket{\phi}=\sum_{i=1}^d\alpha_{i}\sum_{j=1}^d\beta_j\ket{e_i}\tensor \ket{f_j}=\sum_{1\leq i,j\leq d}\alpha_i\beta_j \ket{e_i}\tensor \ket{f_j}$$ Therefore $\{  \ket{e_i}\otimes \ket{f_j}\mid 1\leq i,j\leq d\}$ is a basis of $\mcH_1\tensor \mcH_2$. 
	
	Now for any $i1,i2,j1,j2\in [d]$ $$(\bra{e_{i1}}\tensor \bra{f_{j1}})(\ket{e_{i2}}\tensor\ket{f_{j2}})=\bra{e_{i1}}\ket{e_{i2}} \bra{f_{j1}}\ket{f_{j2}}=\delta_{i1,i2}\, \delta_{j1,j2}$$Therefore $\{  \ket{e_i}\otimes \ket{f_j}\mid 1\leq i,j\leq d\}$ is orthonormal. Therefore $\{  \ket{e_i}\otimes \ket{f_j}\mid 1\leq i,j\leq d\}$ is a ONB for $\mcH_1\tensor \mcH_2$.
	
}

%%%%%%%%%%%%%%%%%%%%%%%%%%%%%%%%%%%%%%%%%%%%%%%%%%%%%%%%%%%%%%%%%%%%%%%%%%%%%%%%%%%%%%%%%%%%%%%%%%%%%%%%%%%%%%%%%%%%%%%%%%%%%%%%%%%%%%%%
% Problem 3
%%%%%%%%%%%%%%%%%%%%%%%%%%%%%%%%%%%%%%%%%%%%%%%%%%%%%%%%%%%%%%%%%%%%%%%%%%%%%%%%%%%%%%%%%%%%%%%%%%%%%%%%%%%%%%%%%%%%%%%%%%%%%%%%%%%%%%%%

\begin{problem}{%problem statement
	}{p3% problem reference text
	}
	Let $\{\ket{g_k}\mid 1\leq i\leq d_2\}\subseteq \mcH_2$ be ONB. For $T\in \sL(\mcH_1\tensor \mcH_2)$, let  $tr_2(T)\in \sL(\mcH_1)$ denote the operator  satisfying $$\bra{u}tr_2(T)\ket{v}= \sum_{k}\bra{u\tensor g_k}T\ket{v\tensor g_k}$$ for any choice $\ket{u},\ket{v}\in \mcH_1$. Prove that $\sum\limits_{k}\bra{u\tensor g_k}T\ket{v\tensor g_k}$ is invariant. 
	%Problem		
\end{problem}

%%%%%%%%%%%%%%%%%%%%%%%%%%%%%%%%%%%%%%%%%%%%%%%%%%%%%%%%%%%%%%%%%%%%%%%%%%%%%%%%%%%%%%%%%%%%%%%%%%%%%%%%%%%%%%%%%%%%%%%%%%%%%%%%%%%%%%%%
% Problem 4
%%%%%%%%%%%%%%%%%%%%%%%%%%%%%%%%%%%%%%%%%%%%%%%%%%%%%%%%%%%%%%%%%%%%%%%%%%%%%%%%%%%%%%%%%%%%%%%%%%%%%%%%%%%%%%%%%%%%%%%%%%%%%%%%%%%%%%%%

\begin{problem}{%problem statement
	}{p4% problem reference text
	}
	Show that the Pauli matrices are all Hermitian, unitary, they square to the identity, and their eigenvalues are $\pm 1$
	%Problem		
\end{problem}

%%%%%%%%%%%%%%%%%%%%%%%%%%%%%%%%%%%%%%%%%%%%%%%%%%%%%%%%%%%%%%%%%%%%%%%%%%%%%%%%%%%%%%%%%%%%%%%%%%%%%%%%%%%%%%%%%%%%%%%%%%%%%%%%%%%%%%%%
% Problem 5
%%%%%%%%%%%%%%%%%%%%%%%%%%%%%%%%%%%%%%%%%%%%%%%%%%%%%%%%%%%%%%%%%%%%%%%%%%%%%%%%%%%%%%%%%%%%%%%%%%%%%%%%%%%%%%%%%%%%%%%%%%%%%%%%%%%%%%%%

\begin{problem}{%problem statement
		Mark Wilde: Exercise 3.3.3
	}{p5% problem reference text
	}
For $S,T\in \sL(\mcH)$, show that $$tr(T)=tr(T^+),\qquad tr(ST)=tr(TS)$$ [Recall $T^+$ denotes adjoint of $T$]. For $\ket{x},\ket{y}\in\mcH$ show $$tr(\ket{x}\bra{y}T)=tr(T\ket{x}\bra{y})=\bra{y}\ket{Tx}$$
%Problem		
\end{problem}

%%%%%%%%%%%%%%%%%%%%%%%%%%%%%%%%%%%%%%%%%%%%%%%%%%%%%%%%%%%%%%%%%%%%%%%%%%%%%%%%%%%%%%%%%%%%%%%%%%%%%%%%%%%%%%%%%%%%%%%%%%%%%%%%%%%%%%%%
% Problem 6
%%%%%%%%%%%%%%%%%%%%%%%%%%%%%%%%%%%%%%%%%%%%%%%%%%%%%%%%%%%%%%%%%%%%%%%%%%%%%%%%%%%%%%%%%%%%%%%%%%%%%%%%%%%%%%%%%%%%%%%%%%%%%%%%%%%%%%%%

\begin{problem}{%problem statement
	}{p6% problem reference text
	}
	Suppose $\mcH$ is \fdcps with $\dim(\mcH)=d$. Show complex dimensionality of $\sL(\mcH)$ is $d^2$, real dimensionality of $\sR(\mcH)$ is $d^2$.\parinn
	
	Suppose $\mcH$ is a real inner product space of $\dim$ $d$, show $\sL(\mcH)$ has dimension $d$ and  the space of all symmetric operators  is a real vector space of dimension $\frac{d(d+1)}{2}$.
	%Problem		
\end{problem}

%%%%%%%%%%%%%%%%%%%%%%%%%%%%%%%%%%%%%%%%%%%%%%%%%%%%%%%%%%%%%%%%%%%%%%%%%%%%%%%%%%%%%%%%%%%%%%%%%%%%%%%%%%%%%%%%%%%%%%%%%%%%%%%%%%%%%%%%
% Problem 7
%%%%%%%%%%%%%%%%%%%%%%%%%%%%%%%%%%%%%%%%%%%%%%%%%%%%%%%%%%%%%%%%%%%%%%%%%%%%%%%%%%%%%%%%%%%%%%%%%%%%%%%%%%%%%%%%%%%%%%%%%%%%%%%%%%%%%%%%

\begin{problem}{%problem statement
	}{p7% problem reference text
	}
Show that $\sD(\mcH)$ is a convex subset of the real vector space of all Hermitian operators on $\mcH$. Show that the extreme points of $\sD(\mcH)$ are pure states, i.e.  rank 1 projection operators.
	%Problem		
\end{problem}

%%%%%%%%%%%%%%%%%%%%%%%%%%%%%%%%%%%%%%%%%%%%%%%%%%%%%%%%%%%%%%%%%%%%%%%%%%%%%%%%%%%%%%%%%%%%%%%%%%%%%%%%%%%%%%%%%%%%%%%%%%%%%%%%%%%%%%%%
% Problem 8
%%%%%%%%%%%%%%%%%%%%%%%%%%%%%%%%%%%%%%%%%%%%%%%%%%%%%%%%%%%%%%%%%%%%%%%%%%%%%%%%%%%%%%%%%%%%%%%%%%%%%%%%%%%%%%%%%%%%%%%%%%%%%%%%%%%%%%%%

\begin{problem}{%problem statement
	}{p8% problem reference text
	}
Show that if $\dim(\mcH)=d$, then $\sD(\mcH)$ can be embedded  into a real vector space  of dimension $n=d^2-1$
	%Problem		
\end{problem}

%%%%%%%%%%%%%%%%%%%%%%%%%%%%%%%%%%%%%%%%%%%%%%%%%%%%%%%%%%%%%%%%%%%%%%%%%%%%%%%%%%%%%%%%%%%%%%%%%%%%%%%%%%%%%%%%%%%%%%%%%%%%%%%%%%%%%%%%
% Problem 9
%%%%%%%%%%%%%%%%%%%%%%%%%%%%%%%%%%%%%%%%%%%%%%%%%%%%%%%%%%%%%%%%%%%%%%%%%%%%%%%%%%%%%%%%%%%%%%%%%%%%%%%%%%%%%%%%%%%%%%%%%%%%%%%%%%%%%%%%

\begin{problem}{%problem statement
	}{p9% problem reference text
	}
Prove the Singular value decomposition theorem stated in class.	%Problem		
\end{problem}

%%%%%%%%%%%%%%%%%%%%%%%%%%%%%%%%%%%%%%%%%%%%%%%%%%%%%%%%%%%%%%%%%%%%%%%%%%%%%%%%%%%%%%%%%%%%%%%%%%%%%%%%%%%%%%%%%%%%%%%%%%%%%%%%%%%%%%%%
% Problem 10
%%%%%%%%%%%%%%%%%%%%%%%%%%%%%%%%%%%%%%%%%%%%%%%%%%%%%%%%%%%%%%%%%%%%%%%%%%%%%%%%%%%%%%%%%%%%%%%%%%%%%%%%%%%%%%%%%%%%%%%%%%%%%%%%%%%%%%%%

\begin{problem}{%problem statement
	}{p10% problem reference text
	}
	Suppose $\ket{\psi}_{AR_1}\in \mcH_{A}\tensor \mcH_{R_1}$, $\ket{\psi}_{AR_2}\in \mcH_A\tensor \mcH_{R_2}$ are purifications of  $\rho_A\in \sD(\mcH_A)$ and $\dim(\mcH_{R_2})\geq \dim(\mcH_{R_1})$, then show that  there exists an isometry $V:\mcH_{R_1}\to \mcH_{R_2}$  such that $$\ket{\psi}_{AR_2}=(V\tensor I)\ket{\psi}_{AR_1}$$
	%Problem		
\end{problem}

%%%%%%%%%%%%%%%%%%%%%%%%%%%%%%%%%%%%%%%%%%%%%%%%%%%%%%%%%%%%%%%%%%%%%%%%%%%%%%%%%%%%%%%%%%%%%%%%%%%%%%%%%%%%%%%%%%%%%%%%%%%%%%%%%%%%%%%%
% Problem 11
%%%%%%%%%%%%%%%%%%%%%%%%%%%%%%%%%%%%%%%%%%%%%%%%%%%%%%%%%%%%%%%%%%%%%%%%%%%%%%%%%%%%%%%%%%%%%%%%%%%%%%%%%%%%%%%%%%%%%%%%%%%%%%%%%%%%%%%%

\begin{problem}{%problem statement
		Mark Wilde: Exercise 3.6.5
	}{p11% problem reference text
	}
	Show that the Bell states form an orthonormal basis: $$\bra{\Phi^{z_1x_1}}\ket{\Phi^{z_2x_2}}=\delta_{z_1,z_2}\,\delta_{x_1,x_2}$$
	%Problem		
\end{problem}

%%%%%%%%%%%%%%%%%%%%%%%%%%%%%%%%%%%%%%%%%%%%%%%%%%%%%%%%%%%%%%%%%%%%%%%%%%%%%%%%%%%%%%%%%%%%%%%%%%%%%%%%%%%%%%%%%%%%%%%%%%%%%%%%%%%%%%%%
% Problem 12
%%%%%%%%%%%%%%%%%%%%%%%%%%%%%%%%%%%%%%%%%%%%%%%%%%%%%%%%%%%%%%%%%%%%%%%%%%%%%%%%%%%%%%%%%%%%%%%%%%%%%%%%%%%%%%%%%%%%%%%%%%%%%%%%%%%%%%%%

\begin{problem}{%problem statement
		Mark Wilde: Exercise 3.7.11
	}{p12% problem reference text
	}
	 Show that the set of states $\{\ket{\Phi^{x,z}}_{AB}\}^{d-1}_{x,z=0}$ forms a complete, orthonormal basis:$$\bra{\Phi^{x_1,z_1}}\ket{\Phi^{x_2,z_2}}=\delta_{x_1,x_2}\,\delta_{z_1,z_2}\qquad \sum_{x,z=0}^d\ket{\Phi^{x,z}}\bra{\Phi^{x,z}}=I_{AB}$$
	%Problem		
\end{problem}

%%%%%%%%%%%%%%%%%%%%%%%%%%%%%%%%%%%%%%%%%%%%%%%%%%%%%%%%%%%%%%%%%%%%%%%%%%%%%%%%%%%%%%%%%%%%%%%%%%%%%%%%%%%%%%%%%%%%%%%%%%%%%%%%%%%%%%%%
% Problem 13
%%%%%%%%%%%%%%%%%%%%%%%%%%%%%%%%%%%%%%%%%%%%%%%%%%%%%%%%%%%%%%%%%%%%%%%%%%%%%%%%%%%%%%%%%%%%%%%%%%%%%%%%%%%%%%%%%%%%%%%%%%%%%%%%%%%%%%%%

\begin{problem}{%problem statement
		Mark Wilde: Exercise 4.1.5
	}{p13% problem reference text
	}
	 Show that the following ensembles have the same density operator: $\lt\{\lt\{\frac12,\ket{0}\rt\},\lt\{\frac12,\ket{1}\rt\}\rt\}$ and $\lt\{\lt\{\frac12,\ket{+}\rt\},\lt\{\frac12,\ket{-}\rt\}\rt\}$
	%Problem		
\end{problem}

%%%%%%%%%%%%%%%%%%%%%%%%%%%%%%%%%%%%%%%%%%%%%%%%%%%%%%%%%%%%%%%%%%%%%%%%%%%%%%%%%%%%%%%%%%%%%%%%%%%%%%%%%%%%%%%%%%%%%%%%%%%%%%%%%%%%%%%%
% Problem 14
%%%%%%%%%%%%%%%%%%%%%%%%%%%%%%%%%%%%%%%%%%%%%%%%%%%%%%%%%%%%%%%%%%%%%%%%%%%%%%%%%%%%%%%%%%%%%%%%%%%%%%%%%%%%%%%%%%%%%%%%%%%%%%%%%%%%%%%%

\begin{problem}{%problem statement
	}{p14% problem reference text
	}
	Show that the set of states $\{\ket{\Phi^{x,z}}_{AB}\}^{d-1}_{x,z=0}$ forms a complete, orthonormal basis:$$\bra{\Phi^{x_1,z_1}}\ket{\Phi^{x_2,z_2}}=\delta_{x_1,x_2}\,\delta_{z_1,z_2}\qquad \sum_{x,z=0}^d\ket{\Phi^{x,z}}\bra{\Phi^{x,z}}=I_{AB}$$
	%Problem		
\end{problem}

%%%%%%%%%%%%%%%%%%%%%%%%%%%%%%%%%%%%%%%%%%%%%%%%%%%%%%%%%%%%%%%%%%%%%%%%%%%%%%%%%%%%%%%%%%%%%%%%%%%%%%%%%%%%%%%%%%%%%%%%%%%%%%%%%%%%%%%%
% Problem 15
%%%%%%%%%%%%%%%%%%%%%%%%%%%%%%%%%%%%%%%%%%%%%%%%%%%%%%%%%%%%%%%%%%%%%%%%%%%%%%%%%%%%%%%%%%%%%%%%%%%%%%%%%%%%%%%%%%%%%%%%%%%%%%%%%%%%%%%%

\begin{problem}{%problem statement
		Mark Wilde: Exercise 4.1.3
	}{p15% problem reference text
	}
	Show that the following ensembles have the same density operator: $\lt\{\lt\{\frac12,\ket{0}\rt\},\lt\{\frac12,\ket{1}\rt\}\rt\}$ and $\lt\{\lt\{\frac12,\ket{+}\rt\},\lt\{\frac12,\ket{-}\rt\}\rt\}$
	%Problem		
\end{problem}

%%%%%%%%%%%%%%%%%%%%%%%%%%%%%%%%%%%%%%%%%%%%%%%%%%%%%%%%%%%%%%%%%%%%%%%%%%%%%%%%%%%%%%%%%%%%%%%%%%%%%%%%%%%%%%%%%%%%%%%%%%%%%%%%%%%%%%%%
% Problem 16
%%%%%%%%%%%%%%%%%%%%%%%%%%%%%%%%%%%%%%%%%%%%%%%%%%%%%%%%%%%%%%%%%%%%%%%%%%%%%%%%%%%%%%%%%%%%%%%%%%%%%%%%%%%%%%%%%%%%%%%%%%%%%%%%%%%%%%%%

\begin{problem}{%problem statement
		Mark Wilde: Exercise 3.7.12
	}{p16% problem reference text
	}
	Show that the following ensembles have the same density operator: $\lt\{\lt\{\frac12,\ket{0}\rt\},\lt\{\frac12,\ket{1}\rt\}\rt\}$ and $\lt\{\lt\{\frac12,\ket{+}\rt\},\lt\{\frac12,\ket{-}\rt\}\rt\}$
	%Problem		
\end{problem}

%%%%%%%%%%%%%%%%%%%%%%%%%%%%%%%%%%%%%%%%%%%%%%%%%%%%%%%%%%%%%%%%%%%%%%%%%%%%%%%%%%%%%%%%%%%%%%%%%%%%%%%%%%%%%%%%%%%%%%%%%%%%%%%%%%%%%%%%
% Problem 17
%%%%%%%%%%%%%%%%%%%%%%%%%%%%%%%%%%%%%%%%%%%%%%%%%%%%%%%%%%%%%%%%%%%%%%%%%%%%%%%%%%%%%%%%%%%%%%%%%%%%%%%%%%%%%%%%%%%%%%%%%%%%%%%%%%%%%%%%

\begin{problem}{%problem statement
	}{p17% problem reference text
	}
	Show that the following ensembles have the same density operator: $\lt\{\lt\{\frac12,\ket{0}\rt\},\lt\{\frac12,\ket{1}\rt\}\rt\}$ and $\lt\{\lt\{\frac12,\ket{+}\rt\},\lt\{\frac12,\ket{-}\rt\}\rt\}$
	%Problem		
\end{problem}

%%%%%%%%%%%%%%%%%%%%%%%%%%%%%%%%%%%%%%%%%%%%%%%%%%%%%%%%%%%%%%%%%%%%%%%%%%%%%%%%%%%%%%%%%%%%%%%%%%%%%%%%%%%%%%%%%%%%%%%%%%%%%%%%%%%%%%%%
% Problem 18
%%%%%%%%%%%%%%%%%%%%%%%%%%%%%%%%%%%%%%%%%%%%%%%%%%%%%%%%%%%%%%%%%%%%%%%%%%%%%%%%%%%%%%%%%%%%%%%%%%%%%%%%%%%%%%%%%%%%%%%%%%%%%%%%%%%%%%%%

\begin{problem}{%problem statement
	}{p18% problem reference text
	}
	Show that the following ensembles have the same density operator: $\lt\{\lt\{\frac12,\ket{0}\rt\},\lt\{\frac12,\ket{1}\rt\}\rt\}$ and $\lt\{\lt\{\frac12,\ket{+}\rt\},\lt\{\frac12,\ket{-}\rt\}\rt\}$
	%Problem		
\end{problem}

%%%%%%%%%%%%%%%%%%%%%%%%%%%%%%%%%%%%%%%%%%%%%%%%%%%%%%%%%%%%%%%%%%%%%%%%%%%%%%%%%%%%%%%%%%%%%%%%%%%%%%%%%%%%%%%%%%%%%%%%%%%%%%%%%%%%%%%%
% Problem 19
%%%%%%%%%%%%%%%%%%%%%%%%%%%%%%%%%%%%%%%%%%%%%%%%%%%%%%%%%%%%%%%%%%%%%%%%%%%%%%%%%%%%%%%%%%%%%%%%%%%%%%%%%%%%%%%%%%%%%%%%%%%%%%%%%%%%%%%%

\begin{problem}{%problem statement
	}{p19% problem reference text
	}
	Show that the following ensembles have the same density operator: $\lt\{\lt\{\frac12,\ket{0}\rt\},\lt\{\frac12,\ket{1}\rt\}\rt\}$ and $\lt\{\lt\{\frac12,\ket{+}\rt\},\lt\{\frac12,\ket{-}\rt\}\rt\}$
	%Problem		
\end{problem}

\end{document}
