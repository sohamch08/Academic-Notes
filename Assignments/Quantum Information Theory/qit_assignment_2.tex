\documentclass[a4paper, 11pt]{article}
\usepackage{comment} % enables the use of multi-line comments (\ifx \fi) 
\usepackage{fullpage} % changes the margin
\usepackage[a4paper, total={7in, 10in}]{geometry}
\usepackage{amsmath,mathtools,mathdots}
\usepackage{amssymb,amsthm}  % assumes amsmath package installed
\usepackage{float}
\usepackage{xcolor}
\usepackage{mdframed}
\usepackage[shortlabels]{enumitem}
\usepackage{indentfirst}
\usepackage{hyperref}
\hypersetup{
	colorlinks=true,
	linkcolor=doc!80,
	citecolor=myr
	filecolor=magenta,      
	urlcolor=doc!80,
	pdftitle={Assignment}, %%%%%%%%%%%%%%%%   WRITE ASSIGNMENT PDF NAME  %%%%%%%%%%%%%%%%%%%%
}
\usepackage[most,many,breakable]{tcolorbox}
\usepackage{tikz}
\usepackage{caption}
\usepackage{mathpazo}
\usepackage{libertine}
\usepackage{physics}
\usepackage{mathrsfs}
\usepackage{tikz-cd}

\definecolor{mytheorembg}{HTML}{F2F2F9}
\definecolor{mytheoremfr}{HTML}{00007B}
\definecolor{doc}{RGB}{0,60,110}
\definecolor{myg}{RGB}{56, 140, 70}
\definecolor{myb}{RGB}{45, 111, 177}
\definecolor{myr}{RGB}{199, 68, 64}

\usetikzlibrary{decorations.pathreplacing,angles,quotes,patterns}
\definecolor{mytheorembg}{HTML}{F2F2F9}
\definecolor{mytheoremfr}{HTML}{00007B}
\definecolor{doc}{RGB}{0,60,110}
\definecolor{myg}{RGB}{56, 140, 70}
\definecolor{myb}{RGB}{45, 111, 177}
\definecolor{myr}{RGB}{199, 68, 64}
\newcounter{problem}
\tcbuselibrary{theorems,skins,hooks}
\newtcbtheorem[use counter=problem]{problem}{Problem}
{%
	enhanced,
	breakable,
	colback = mytheorembg,
	frame hidden,
	boxrule = 0sp,
	borderline west = {2pt}{0pt}{mytheoremfr},
	arc=5pt,
	detach title,
	before upper = \tcbtitle\par\smallskip,
	coltitle = mytheoremfr,
	fonttitle = \bfseries\sffamily,
	description font = \mdseries,
	separator sign none,
	segmentation style={solid, mytheoremfr},
}
{p}

% To give references for any problem use like this
% suppose the problem number is p3 then 2 options either 
% \hyperref[p:p3]{<text you want to use to hyperlink> \ref{p:p3}}
%                  or directly 
%                   \ref{p:p3}



%---------------------------------------
% BlackBoard Math Fonts :-
%---------------------------------------

%Captital Letters
\newcommand{\bbA}{\mathbb{A}}	\newcommand{\bbB}{\mathbb{B}}
\newcommand{\bbC}{\mathbb{C}}	\newcommand{\bbD}{\mathbb{D}}
\newcommand{\bbE}{\mathbb{E}}	\newcommand{\bbF}{\mathbb{F}}
\newcommand{\bbG}{\mathbb{G}}	\newcommand{\bbH}{\mathbb{H}}
\newcommand{\bbI}{\mathbb{I}}	\newcommand{\bbJ}{\mathbb{J}}
\newcommand{\bbK}{\mathbb{K}}	\newcommand{\bbL}{\mathbb{L}}
\newcommand{\bbM}{\mathbb{M}}	\newcommand{\bbN}{\mathbb{N}}
\newcommand{\bbO}{\mathbb{O}}	\newcommand{\bbP}{\mathbb{P}}
\newcommand{\bbQ}{\mathbb{Q}}	\newcommand{\bbR}{\mathbb{R}}
\newcommand{\bbS}{\mathbb{S}}	\newcommand{\bbT}{\mathbb{T}}
\newcommand{\bbU}{\mathbb{U}}	\newcommand{\bbV}{\mathbb{V}}
\newcommand{\bbW}{\mathbb{W}}	\newcommand{\bbX}{\mathbb{X}}
\newcommand{\bbY}{\mathbb{Y}}	\newcommand{\bbZ}{\mathbb{Z}}

%---------------------------------------
% MathCal Fonts :-
%---------------------------------------

%Captital Letters
\newcommand{\mcA}{\mathcal{A}}	\newcommand{\mcB}{\mathcal{B}}
\newcommand{\mcC}{\mathcal{C}}	\newcommand{\mcD}{\mathcal{D}}
\newcommand{\mcE}{\mathcal{E}}	\newcommand{\mcF}{\mathcal{F}}
\newcommand{\mcG}{\mathcal{G}}	\newcommand{\mcH}{\mathcal{H}}
\newcommand{\mcI}{\mathcal{I}}	\newcommand{\mcJ}{\mathcal{J}}
\newcommand{\mcK}{\mathcal{K}}	\newcommand{\mcL}{\mathcal{L}}
\newcommand{\mcM}{\mathcal{M}}	\newcommand{\mcN}{\mathcal{N}}
\newcommand{\mcO}{\mathcal{O}}	\newcommand{\mcP}{\mathcal{P}}
\newcommand{\mcQ}{\mathcal{Q}}	\newcommand{\mcR}{\mathcal{R}}
\newcommand{\mcS}{\mathcal{S}}	\newcommand{\mcT}{\mathcal{T}}
\newcommand{\mcU}{\mathcal{U}}	\newcommand{\mcV}{\mathcal{V}}
\newcommand{\mcW}{\mathcal{W}}	\newcommand{\mcX}{\mathcal{X}}
\newcommand{\mcY}{\mathcal{Y}}	\newcommand{\mcZ}{\mathcal{Z}}



%---------------------------------------
% Bold Math Fonts :-
%---------------------------------------

%Captital Letters
\newcommand{\bmA}{\boldsymbol{A}}	\newcommand{\bmB}{\boldsymbol{B}}
\newcommand{\bmC}{\boldsymbol{C}}	\newcommand{\bmD}{\boldsymbol{D}}
\newcommand{\bmE}{\boldsymbol{E}}	\newcommand{\bmF}{\boldsymbol{F}}
\newcommand{\bmG}{\boldsymbol{G}}	\newcommand{\bmH}{\boldsymbol{H}}
\newcommand{\bmI}{\boldsymbol{I}}	\newcommand{\bmJ}{\boldsymbol{J}}
\newcommand{\bmK}{\boldsymbol{K}}	\newcommand{\bmL}{\boldsymbol{L}}
\newcommand{\bmM}{\boldsymbol{M}}	\newcommand{\bmN}{\boldsymbol{N}}
\newcommand{\bmO}{\boldsymbol{O}}	\newcommand{\bmP}{\boldsymbol{P}}
\newcommand{\bmQ}{\boldsymbol{Q}}	\newcommand{\bmR}{\boldsymbol{R}}
\newcommand{\bmS}{\boldsymbol{S}}	\newcommand{\bmT}{\boldsymbol{T}}
\newcommand{\bmU}{\boldsymbol{U}}	\newcommand{\bmV}{\boldsymbol{V}}
\newcommand{\bmW}{\boldsymbol{W}}	\newcommand{\bmX}{\boldsymbol{X}}
\newcommand{\bmY}{\boldsymbol{Y}}	\newcommand{\bmZ}{\boldsymbol{Z}}
%Small Letters
\newcommand{\bma}{\boldsymbol{a}}	\newcommand{\bmb}{\boldsymbol{b}}
\newcommand{\bmc}{\boldsymbol{c}}	\newcommand{\bmd}{\boldsymbol{d}}
\newcommand{\bme}{\boldsymbol{e}}	\newcommand{\bmf}{\boldsymbol{f}}
\newcommand{\bmg}{\boldsymbol{g}}	\newcommand{\bmh}{\boldsymbol{h}}
\newcommand{\bmi}{\boldsymbol{i}}	\newcommand{\bmj}{\boldsymbol{j}}
\newcommand{\bmk}{\boldsymbol{k}}	\newcommand{\bml}{\boldsymbol{l}}
\newcommand{\bmm}{\boldsymbol{m}}	\newcommand{\bmn}{\boldsymbol{n}}
\newcommand{\bmo}{\boldsymbol{o}}	\newcommand{\bmp}{\boldsymbol{p}}
\newcommand{\bmq}{\boldsymbol{q}}	\newcommand{\bmr}{\boldsymbol{r}}
\newcommand{\bms}{\boldsymbol{s}}	\newcommand{\bmt}{\boldsymbol{t}}
\newcommand{\bmu}{\boldsymbol{u}}	\newcommand{\bmv}{\boldsymbol{v}}
\newcommand{\bmw}{\boldsymbol{w}}	\newcommand{\bmx}{\boldsymbol{x}}
\newcommand{\bmy}{\boldsymbol{y}}	\newcommand{\bmz}{\boldsymbol{z}}


%---------------------------------------
% Scr Math Fonts :-
%---------------------------------------

\newcommand{\sA}{{\mathscr{A}}}   \newcommand{\sB}{{\mathscr{B}}}
\newcommand{\sC}{{\mathscr{C}}}   \newcommand{\sD}{{\mathscr{D}}}
\newcommand{\sE}{{\mathscr{E}}}   \newcommand{\sF}{{\mathscr{F}}}
\newcommand{\sG}{{\mathscr{G}}}   \newcommand{\sH}{{\mathscr{H}}}
\newcommand{\sI}{{\mathscr{I}}}   \newcommand{\sJ}{{\mathscr{J}}}
\newcommand{\sK}{{\mathscr{K}}}   \newcommand{\sL}{{\mathscr{L}}}
\newcommand{\sM}{{\mathscr{M}}}   \newcommand{\sN}{{\mathscr{N}}}
\newcommand{\sO}{{\mathscr{O}}}   \newcommand{\sP}{{\mathscr{P}}}
\newcommand{\sQ}{{\mathscr{Q}}}   \newcommand{\sR}{{\mathscr{R}}}
\newcommand{\sS}{{\mathscr{S}}}   \newcommand{\sT}{{\mathscr{T}}}
\newcommand{\sU}{{\mathscr{U}}}   \newcommand{\sV}{{\mathscr{V}}}
\newcommand{\sW}{{\mathscr{W}}}   \newcommand{\sX}{{\mathscr{X}}}
\newcommand{\sY}{{\mathscr{Y}}}   \newcommand{\sZ}{{\mathscr{Z}}}


%---------------------------------------
% Math Fraktur Font
%---------------------------------------

%Captital Letters
\newcommand{\mfA}{\mathfrak{A}}	\newcommand{\mfB}{\mathfrak{B}}
\newcommand{\mfC}{\mathfrak{C}}	\newcommand{\mfD}{\mathfrak{D}}
\newcommand{\mfE}{\mathfrak{E}}	\newcommand{\mfF}{\mathfrak{F}}
\newcommand{\mfG}{\mathfrak{G}}	\newcommand{\mfH}{\mathfrak{H}}
\newcommand{\mfI}{\mathfrak{I}}	\newcommand{\mfJ}{\mathfrak{J}}
\newcommand{\mfK}{\mathfrak{K}}	\newcommand{\mfL}{\mathfrak{L}}
\newcommand{\mfM}{\mathfrak{M}}	\newcommand{\mfN}{\mathfrak{N}}
\newcommand{\mfO}{\mathfrak{O}}	\newcommand{\mfP}{\mathfrak{P}}
\newcommand{\mfQ}{\mathfrak{Q}}	\newcommand{\mfR}{\mathfrak{R}}
\newcommand{\mfS}{\mathfrak{S}}	\newcommand{\mfT}{\mathfrak{T}}
\newcommand{\mfU}{\mathfrak{U}}	\newcommand{\mfV}{\mathfrak{V}}
\newcommand{\mfW}{\mathfrak{W}}	\newcommand{\mfX}{\mathfrak{X}}
\newcommand{\mfY}{\mathfrak{Y}}	\newcommand{\mfZ}{\mathfrak{Z}}
%Small Letters
\newcommand{\mfa}{\mathfrak{a}}	\newcommand{\mfb}{\mathfrak{b}}
\newcommand{\mfc}{\mathfrak{c}}	\newcommand{\mfd}{\mathfrak{d}}
\newcommand{\mfe}{\mathfrak{e}}	\newcommand{\mff}{\mathfrak{f}}
\newcommand{\mfg}{\mathfrak{g}}	\newcommand{\mfh}{\mathfrak{h}}
\newcommand{\mfi}{\mathfrak{i}}	\newcommand{\mfj}{\mathfrak{j}}
\newcommand{\mfk}{\mathfrak{k}}	\newcommand{\mfl}{\mathfrak{l}}
\newcommand{\mfm}{\mathfrak{m}}	\newcommand{\mfn}{\mathfrak{n}}
\newcommand{\mfo}{\mathfrak{o}}	\newcommand{\mfp}{\mathfrak{p}}
\newcommand{\mfq}{\mathfrak{q}}	\newcommand{\mfr}{\mathfrak{r}}
\newcommand{\mfs}{\mathfrak{s}}	\newcommand{\mft}{\mathfrak{t}}
\newcommand{\mfu}{\mathfrak{u}}	\newcommand{\mfv}{\mathfrak{v}}
\newcommand{\mfw}{\mathfrak{w}}	\newcommand{\mfx}{\mathfrak{x}}
\newcommand{\mfy}{\mathfrak{y}}	\newcommand{\mfz}{\mathfrak{z}}

%---------------------------------------
% Bar
%---------------------------------------

%Captital Letters
\newcommand{\ovA}{\overline{A}}	\newcommand{\ovB}{\overline{B}}
\newcommand{\ovC}{\overline{C}}	\newcommand{\ovD}{\overline{D}}
\newcommand{\ovE}{\overline{E}}	\newcommand{\ovF}{\overline{F}}
\newcommand{\ovG}{\overline{G}}	\newcommand{\ovH}{\overline{H}}
\newcommand{\ovI}{\overline{I}}	\newcommand{\ovJ}{\overline{J}}
\newcommand{\ovK}{\overline{K}}	\newcommand{\ovL}{\overline{L}}
\newcommand{\ovM}{\overline{M}}	\newcommand{\ovN}{\overline{N}}
\newcommand{\ovO}{\overline{O}}	\newcommand{\ovP}{\overline{P}}
\newcommand{\ovQ}{\overline{Q}}	\newcommand{\ovR}{\overline{R}}
\newcommand{\ovS}{\overline{S}}	\newcommand{\ovT}{\overline{T}}
\newcommand{\ovU}{\overline{U}}	\newcommand{\ovV}{\overline{V}}
\newcommand{\ovW}{\overline{W}}	\newcommand{\ovX}{\overline{X}}
\newcommand{\ovY}{\overline{Y}}	\newcommand{\ovZ}{\overline{Z}}
%Small Letters
\newcommand{\ova}{\overline{a}}	\newcommand{\ovb}{\overline{b}}
\newcommand{\ovc}{\overline{c}}	\newcommand{\ovd}{\overline{d}}
\newcommand{\ove}{\overline{e}}	\newcommand{\ovf}{\overline{f}}
\newcommand{\ovg}{\overline{g}}	\newcommand{\ovh}{\overline{h}}
\newcommand{\ovi}{\overline{i}}	\newcommand{\ovj}{\overline{j}}
\newcommand{\ovk}{\overline{k}}	\newcommand{\ovl}{\overline{l}}
\newcommand{\ovm}{\overline{m}}	\newcommand{\ovn}{\overline{n}}
\newcommand{\ovo}{\overline{o}}	\newcommand{\ovp}{\overline{p}}
\newcommand{\ovq}{\overline{q}}	\newcommand{\ovr}{\overline{r}}
\newcommand{\ovs}{\overline{s}}	\newcommand{\ovt}{\overline{t}}
\newcommand{\ovu}{\overline{u}}	\newcommand{\ovv}{\overline{v}}
\newcommand{\ovw}{\overline{w}}	\newcommand{\ovx}{\overline{x}}
\newcommand{\ovy}{\overline{y}}	\newcommand{\ovz}{\overline{z}}

%---------------------------------------
% Tilde
%---------------------------------------

%Captital Letters
\newcommand{\tdA}{\tilde{A}}	\newcommand{\tdB}{\tilde{B}}
\newcommand{\tdC}{\tilde{C}}	\newcommand{\tdD}{\tilde{D}}
\newcommand{\tdE}{\tilde{E}}	\newcommand{\tdF}{\tilde{F}}
\newcommand{\tdG}{\tilde{G}}	\newcommand{\tdH}{\tilde{H}}
\newcommand{\tdI}{\tilde{I}}	\newcommand{\tdJ}{\tilde{J}}
\newcommand{\tdK}{\tilde{K}}	\newcommand{\tdL}{\tilde{L}}
\newcommand{\tdM}{\tilde{M}}	\newcommand{\tdN}{\tilde{N}}
\newcommand{\tdO}{\tilde{O}}	\newcommand{\tdP}{\tilde{P}}
\newcommand{\tdQ}{\tilde{Q}}	\newcommand{\tdR}{\tilde{R}}
\newcommand{\tdS}{\tilde{S}}	\newcommand{\tdT}{\tilde{T}}
\newcommand{\tdU}{\tilde{U}}	\newcommand{\tdV}{\tilde{V}}
\newcommand{\tdW}{\tilde{W}}	\newcommand{\tdX}{\tilde{X}}
\newcommand{\tdY}{\tilde{Y}}	\newcommand{\tdZ}{\tilde{Z}}
%Small Letters
\newcommand{\tda}{\tilde{a}}	\newcommand{\tdb}{\tilde{b}}
\newcommand{\tdc}{\tilde{c}}	\newcommand{\tdd}{\tilde{d}}
\newcommand{\tde}{\tilde{e}}	\newcommand{\tdf}{\tilde{f}}
\newcommand{\tdg}{\tilde{g}}	\newcommand{\tdh}{\tilde{h}}
\newcommand{\tdi}{\tilde{i}}	\newcommand{\tdj}{\tilde{j}}
\newcommand{\tdk}{\tilde{k}}	\newcommand{\tdl}{\tilde{l}}
\newcommand{\tdm}{\tilde{m}}	\newcommand{\tdn}{\tilde{n}}
\newcommand{\tdo}{\tilde{o}}	\newcommand{\tdp}{\tilde{p}}
\newcommand{\tdq}{\tilde{q}}	\newcommand{\tdr}{\tilde{r}}
\newcommand{\tds}{\tilde{s}}	\newcommand{\tdt}{\tilde{t}}
\newcommand{\tdu}{\tilde{u}}	\newcommand{\tdv}{\tilde{v}}
\newcommand{\tdw}{\tilde{w}}	\newcommand{\tdx}{\tilde{x}}
\newcommand{\tdy}{\tilde{y}}	\newcommand{\tdz}{\tilde{z}}

%---------------------------------------
% Vec
%---------------------------------------

%Captital Letters
\newcommand{\vcA}{\vec{A}}	\newcommand{\vcB}{\vec{B}}
\newcommand{\vcC}{\vec{C}}	\newcommand{\vcD}{\vec{D}}
\newcommand{\vcE}{\vec{E}}	\newcommand{\vcF}{\vec{F}}
\newcommand{\vcG}{\vec{G}}	\newcommand{\vcH}{\vec{H}}
\newcommand{\vcI}{\vec{I}}	\newcommand{\vcJ}{\vec{J}}
\newcommand{\vcK}{\vec{K}}	\newcommand{\vcL}{\vec{L}}
\newcommand{\vcM}{\vec{M}}	\newcommand{\vcN}{\vec{N}}
\newcommand{\vcO}{\vec{O}}	\newcommand{\vcP}{\vec{P}}
\newcommand{\vcQ}{\vec{Q}}	\newcommand{\vcR}{\vec{R}}
\newcommand{\vcS}{\vec{S}}	\newcommand{\vcT}{\vec{T}}
\newcommand{\vcU}{\vec{U}}	\newcommand{\vcV}{\vec{V}}
\newcommand{\vcW}{\vec{W}}	\newcommand{\vcX}{\vec{X}}
\newcommand{\vcY}{\vec{Y}}	\newcommand{\vcZ}{\vec{Z}}
%Small Letters
\newcommand{\vca}{\vec{a}}	\newcommand{\vcb}{\vec{b}}
\newcommand{\vcc}{\vec{c}}	\newcommand{\vcd}{\vec{d}}
\newcommand{\vce}{\vec{e}}	\newcommand{\vcf}{\vec{f}}
\newcommand{\vcg}{\vec{g}}	\newcommand{\vch}{\vec{h}}
\newcommand{\vci}{\vec{i}}	\newcommand{\vcj}{\vec{j}}
\newcommand{\vck}{\vec{k}}	\newcommand{\vcl}{\vec{l}}
\newcommand{\vcm}{\vec{m}}	\newcommand{\vcn}{\vec{n}}
\newcommand{\vco}{\vec{o}}	\newcommand{\vcp}{\vec{p}}
\newcommand{\vcq}{\vec{q}}	\newcommand{\vcr}{\vec{r}}
\newcommand{\vcs}{\vec{s}}	\newcommand{\vct}{\vec{t}}
\newcommand{\vcu}{\vec{u}}	\newcommand{\vcv}{\vec{v}}
%\newcommand{\vcw}{\vec{w}}	\newcommand{\vcx}{\vec{x}}
\newcommand{\vcy}{\vec{y}}	\newcommand{\vcz}{\vec{z}}

%---------------------------------------
% Greek Letters:-
%---------------------------------------
\newcommand{\eps}{\epsilon}
\newcommand{\veps}{\varepsilon}
\newcommand{\lm}{\lambda}
\newcommand{\Lm}{\Lambda}
\newcommand{\gm}{\gamma}
\newcommand{\Gm}{\Gamma}
\newcommand{\vph}{\varphi}
\newcommand{\ph}{\phi}
\newcommand{\om}{\omega}
\newcommand{\Om}{\Omega}
\newcommand{\sg}{\sigma}
\newcommand{\Sg}{\Sigma}

\newcommand{\Qed}{\begin{flushright}\qed\end{flushright}}
\newcommand{\parinn}{\setlength{\parindent}{1cm}}
\newcommand{\parinf}{\setlength{\parindent}{0cm}}
\newcommand{\del}[2]{\frac{\partial #1}{\partial #2}}
\newcommand{\Del}[3]{\frac{\partial^{#1} #2}{\partial^{#1} #3}}
\newcommand{\deld}[2]{\dfrac{\partial #1}{\partial #2}}
\newcommand{\Deld}[3]{\dfrac{\partial^{#1} #2}{\partial^{#1} #3}}
\newcommand{\uin}{\mathbin{\rotatebox[origin=c]{90}{$\in$}}}
\newcommand{\usubset}{\mathbin{\rotatebox[origin=c]{90}{$\subset$}}}
\newcommand{\lt}{\left}
\newcommand{\rt}{\right}
\newcommand{\exs}{\exists}
\newcommand{\st}{\strut}
\newcommand{\dps}[1]{\displaystyle{#1}}
\newcommand{\la}{\langle}
\newcommand{\ra}{\rangle}
\newcommand{\cls}[1]{\textsc{#1}}
\newcommand{\prb}[1]{\textsc{#1}}
\newcommand{\comb}[2]{\left(\begin{matrix}
		#1\\ #2
\end{matrix}\right)}
%\newcommand[2]{\quotient}{\faktor{#1}{#2}}
\newcommand\quotient[2]{
	\mathchoice
	{% \displaystyle
		\text{\raise1ex\hbox{$#1$}\Big/\lower1ex\hbox{$#2$}}%
	}
	{% \textstyle
		#1\,/\,#2
	}
	{% \scriptstyle
		#1\,/\,#2
	}
	{% \scriptscriptstyle  
		#1\,/\,#2
	}
}

\newcommand{\tensor}{\otimes}
\newcommand{\xor}{\oplus}

\newcommand{\sol}[1]{\begin{solution}#1\end{solution}}
\newcommand{\solve}[1]{\setlength{\parindent}{0cm}\textbf{\textit{Solution: }}\setlength{\parindent}{1cm}#1 \hfill $\blacksquare$}
\newcommand{\mat}[1]{\left[\begin{matrix}#1\end{matrix}\right]}
\newcommand{\matr}[1]{\begin{matrix}#1\end{matrix}}
\newcommand{\matp}[1]{\lt(\begin{matrix}#1\end{matrix}\rt)}
\newcommand{\detmat}[1]{\lt|\begin{matrix}#1\end{matrix}\rt|}
\newcommand\numberthis{\addtocounter{equation}{1}\tag{\theequation}}
\newcommand{\handout}[3]{
	\noindent
	\begin{center}
		\framebox{
			\vbox{
				\hbox to 6.5in { {\bf Complexity Theory I } \hfill Jan -- May, 2023 }
				\vspace{4mm}
				\hbox to 6.5in { {\Large \hfill #1  \hfill} }
				\vspace{2mm}
				\hbox to 6.5in { {\em #2 \hfill #3} }
			}
		}
	\end{center}
	\vspace*{4mm}
}

\newcommand{\lecture}[3]{\handout{Lecture #1}{Lecturer: #2}{Scribe:	#3}}

\let\marvosymLightning\Lightning
\newcommand{\ctr}{\text{\marvosymLightning}\hspace{0.5ex}} % Requires marvosym package

\newcommand{\ov}[1]{\overline{#1}}
\newcommand{\thmref}[1]{\hyperref[th:#1]{Theorem \ref{th:#1}}}
\newcommand{\propref}[1]{\hyperref[th:#1]{Proposition \ref{th:#1}}}
\newcommand{\lmref}[1]{\hyperref[th:#1]{Lemma \ref{th:#1}}}
\newcommand{\corref}[1]{\hyperref[th:#1]{Corollary \ref{th:#1}}}

\newcommand{\thrmref}[1]{\hyperref[#1]{Theorem \ref{#1}}}
\newcommand{\propnref}[1]{\hyperref[#1]{Proposition \ref{#1}}}
\newcommand{\lemref}[1]{\hyperref[#1]{Lemma \ref{#1}}}
\newcommand{\corrref}[1]{\hyperref[#1]{Corollary \ref{#1}}}

\DeclareMathOperator{\enc}{Enc}
\DeclareMathOperator{\res}{Res}
\DeclareMathOperator{\spec}{Spec}
\DeclareMathOperator{\cov}{Cov}
\DeclareMathOperator{\Var}{Var}
\DeclareMathOperator{\Rank}{rank}
\newcommand{\Tfae}{The following are equivalent:}
\newcommand{\tfae}{the following are equivalent:}
\newcommand{\sparsity}{\textit{sparsity}}

\newcommand{\uddots}{\reflectbox{$\ddots$}} 

\newenvironment{claimwidth}{\begin{center}\begin{adjustwidth}{0.05\textwidth}{0.05\textwidth}}{\end{adjustwidth}\end{center}}
\newcommand{\fdcps}{finite dimensional complex inner product space}
\newcommand{\fdcpss}{finite dimensional complex inner product spaces}
\newcommand{\Fdcps}{Finite dimensional complex inner product space}
\newcommand{\Fdcpss}{Finite dimensional complex inner product spaces}
\setlength{\parindent}{0pt}

%%%%%%%%%%%%%%%%%%%%%%%%%%%%%%%%%%%%%%%%%%%%%%%%%%%%%%%%%%%%%%%%%%%%%%%%%%%%%%%%%%%%%%%%%%%%%%%%%%%%%%%%%%%%%%%%%%%%%%%%%%%%%%%%%%%%%%%%

\begin{document}
	
	%%%%%%%%%%%%%%%%%%%%%%%%%%%%%%%%%%%%%%%%%%%%%%%%%%%%%%%%%%%%%%%%%%%%%%%%%%%%%%%%%%%%%%%%%%%%%%%%%%%%%%%%%%%%%%%%%%%%%%%%%%%%%%%%%%%%%%%%
	
	\textsf{\noindent \large\textbf{Soham Chatterjee} \hfill \textbf{Assignment - 2.2 : Quantum Foundations}\\
		Email: \href{sohamc@cmi.ac.in}{sohamc@cmi.ac.in} \hfill Roll: BMC202175\\
		\normalsize Course: Quantum Information Theory \hfill Date: February 29, 2024}
\vspace{1cm}

For all the questions \begin{itemize}
	\item $[k]\coloneqq \{1,2,\dots,k\}$ where $k\in\bbN$.
	\item $\sL(\mcH)\coloneqq $ Linear operators on $\mcH$
	\item $\sR(\mcH)\coloneqq $ Self-adjoint or hermitian operators on $\mcH$
	\item $\sP(\mcH)\coloneqq $ Positive semi-definite operators on $\mcH$
	\item $\sD(\mcH)\coloneqq $ Density operators on $\mcH$
\end{itemize}
%%%%%%%%%%%%%%%%%%%%%%%%%%%%%%%%%%%%%%%%%%%%%%%%%%%%%%%%%%%%%%%%%%%%%%%%%%%%%%%%%%%%%%%%%%%%%%%%%%%%%%%%%%%%%%%%%%%%%%%%%%%%%%%%%%%%%%%%
% Problem 1
%%%%%%%%%%%%%%%%%%%%%%%%%%%%%%%%%%%%%%%%%%%%%%%%%%%%%%%%%%%%%%%%%%%%%%%%%%%%%%%%%%%%%%%%%%%%%%%%%%%%%%%%%%%%%%%%%%%%%%%%%%%%%%%%%%%%%%%%
	
\begin{problem}{%problem statement
	}{p1% problem reference text
}
For $T:\mcH\to \mcH$, prove that  $$\sum_{i=1}^d\la e_i\ket{Te_i}=\sum_{i=1}^d \la f_i\ket{Tf_i}$$if $\{\ket{e_i}\in \mcH\mid 1\leq i\leq d\}$ and $\{\ket{f_i}\in \mcH\mid 1\leq i\leq d\}$ are ONB.
		%Problem		
\end{problem}
	
\solve{
	Let $S:\mcH\to \mcH$ where it maps the basis vectors from $\ket{e_i}\to \ket{f_i}$. Then $S\ket{e_i}=\ket{f_i}$. Hence $S$ is an unitary matrix since $$\bra{e_j}S^{\dagger}S\ket{e_i}=\bra{f_j}f_i\ra=\delta_{ji}\quad \text{and} \quad \bra{f_j}SS^{\dagger}\ket{f_i}=\bra{e_j}e_i\ra=\delta_{ji}$$Hence $$\sum_{i=1}^d \la f_i|Tf_i\ra =\sum_{i=1}^d \la e_i|S^{\dagger}TS\ket{e_i}=tr{(S^{\dagger}TS)}=tr(SS^{\dagger}T)=tr(T)=\sum_{i=1}^d\la e_i\ket{Te_i}$$Therefore we have $$\sum_{i=1}^d\la e_i\ket{Te_i}=\sum_{i=1}^d \la f_i\ket{Tf_i}$$
}
%%%%%%%%%%%%%%%%%%%%%%%%%%%%%%%%%%%%%%%%%%%%%%%%%%%%%%%%%%%%%%%%%%%%%%%%%%%%%%%%%%%%%%%%%%%%%%%%%%%%%%%%%%%%%%%%%%%%%%%%%%%%%%%%%%%%%%%%
% Problem 2
%%%%%%%%%%%%%%%%%%%%%%%%%%%%%%%%%%%%%%%%%%%%%%%%%%%%%%%%%%%%%%%%%%%%%%%%%%%%%%%%%%%%%%%%%%%%%%%%%%%%%%%%%%%%%%%%%%%%%%%%%%%%%%%%%%%%%%%%

\begin{problem}{%problem statement
	}{p2% problem reference text
}
If $\{\ket{e_i}\in \mcH_1\mid 1\leq i\leq d\}$ and $\{\ket{f_i}\in \mcH_2\mid 1\leq i\leq d\}$ are ONB, then  $\{  \ket{e_i}\otimes \ket{f_j}\mid 1\leq i,j\leq d\}\subseteq \mcH_1\tensor \mcH_2$ is ONB
	%Problem		
\end{problem}

\solve{
	Let $\ket{\psi}\tensor \ket{\phi}\in \mcH_1\tensor \mcH_2$. Then $\ket{\psi}=\sum\limits_{i=1}^d\alpha_i\ket{e_i}$ where $\alpha_i\in \bbC$ for all $i\in [d]$ since $\{\ket{e_i}\in \mcH_1\mid 1\leq i\leq d\}$ is ONB for $\mcH_1$. Hence $$\ket{\psi}\tensor \ket{\phi}=\sum_{i=1}^d\alpha_{i}\ket{e_i}\tensor \ket{\phi}$$Now $\ket{\phi}=\sum\limits_{i=1}^d\beta_i\ket{f_i}$ where $\beta_i\in \bbC$ for all $i\in [d]$ since $\{\ket{f_i}\in \mcH_2\mid 1\leq i\leq d\}$ is ONB for $\mcH_2$. Hence $$\forall \ i\in [d]\ \ket{e_i}\tensor \ket{phi}=\sum_{j=1}^d\beta_j\ket{e_i}\tensor \ket{f_j}$$Thereofore we get $$\ket{\psi}\tensor \ket{\phi}=\sum_{i=1}^d\alpha_{i}\ket{e_i}\tensor \ket{\phi}=\sum_{i=1}^d\alpha_{i}\sum_{j=1}^d\beta_j\ket{e_i}\tensor \ket{f_j}=\sum_{1\leq i,j\leq d}\alpha_i\beta_j \ket{e_i}\tensor \ket{f_j}$$ Therefore $\{  \ket{e_i}\otimes \ket{f_j}\mid 1\leq i,j\leq d\}$ is a basis of $\mcH_1\tensor \mcH_2$. 
	
	Now for any $i1,i2,j1,j2\in [d]$ $$(\bra{e_{i1}}\tensor \bra{f_{j1}})(\ket{e_{i2}}\tensor\ket{f_{j2}})=\bra{e_{i1}}\ket{e_{i2}} \bra{f_{j1}}\ket{f_{j2}}=\delta_{i1,i2}\, \delta_{j1,j2}$$Therefore $\{  \ket{e_i}\otimes \ket{f_j}\mid 1\leq i,j\leq d\}$ is orthonormal. Therefore $\{  \ket{e_i}\otimes \ket{f_j}\mid 1\leq i,j\leq d\}$ is a ONB for $\mcH_1\tensor \mcH_2$.
	
}

%%%%%%%%%%%%%%%%%%%%%%%%%%%%%%%%%%%%%%%%%%%%%%%%%%%%%%%%%%%%%%%%%%%%%%%%%%%%%%%%%%%%%%%%%%%%%%%%%%%%%%%%%%%%%%%%%%%%%%%%%%%%%%%%%%%%%%%%
% Problem 3
%%%%%%%%%%%%%%%%%%%%%%%%%%%%%%%%%%%%%%%%%%%%%%%%%%%%%%%%%%%%%%%%%%%%%%%%%%%%%%%%%%%%%%%%%%%%%%%%%%%%%%%%%%%%%%%%%%%%%%%%%%%%%%%%%%%%%%%%

\begin{problem}{%problem statement
	}{p3% problem reference text
	}
	Let $\{\ket{g_k}\mid 1\leq k\leq d_2\}\subseteq \mcH_2$ be ONB. For $T\in \sL(\mcH_1\tensor \mcH_2)$, let  $tr_2(T)\in \sL(\mcH_1)$ denote the operator  satisfying $$\bra{u}tr_2(T)\ket{v}= \sum_{k}\bra{u\tensor g_k}T\ket{v\tensor g_k}$$ for any choice $\ket{u},\ket{v}\in \mcH_1$. Prove that $\sum\limits_{k}\bra{u\tensor g_k}T\ket{v\tensor g_k}$ is invariant. 
	%Problem		
\end{problem}
\solve{
	Let $\{\ket{f_k}\mid 1\leq k\leq d_2\}\subseteq \mcH_2$ be another ONB.  Suppose $S:\mcH_2\to \mcH$ be a map such that $S\ket{g_k}=\ket{f_k}$. As we previously showed in \hyperref[p:p1]{Problem \ref{p:p1}}, $S$ is unitary. Then for all $k\in [d_2]$ we have $$\ket{f_k}=\sum_{i=1}^{d_2}w_{i,k}\ket{e_i}$$ where $w_{i,k}\in\bbC$. Hence $$\bra{f_i}S^{\dagger}S\ket{f_j}=\sum_{k=1}^{d_2} w_{i,k}^*w_{j,k}=\delta_{i,j}$$Now for any $\ket{u},\ket{v}\in \mcH_1$ we have \begin{align*}
	\bra{u}tr_2(T)\ket{v}_{\{\ket{f_k}\}} & = \bra{u}\lt[ \sum\limits_{k=1}^{d_2}(I\tensor \bra{f_k})T(I\tensor \ket{f_k}) \rt]\ket{v}\\
	& = \bra{u}\lt[ \sum\limits_{k=1}^{d_2}\lt(I\tensor \lt(\sum_{i=1}^{d_2}w^*_{i,k}\bra{g_i}\rt)\rt)T\lt(I\tensor \lt(\sum_{j=1}^{d_2}w_{j,k}\ket{g_j}\rt)\rt) \rt]\ket{v}\\
	& = \sum_{k=1}^{d_2}\sum_{i=1}^{d_2}\sum_{j=1}^{d_2}\bra{u}\Big[ w_{i,k}^*w_{j,k} (I\tensor \bra{g_i})T(I\tensor \ket{g_j})  \Big]\ket{v}\\
	& = \sum_{i=1}^{d_2}\sum_{j=1}^{d_2}\bra{u}\lt[\lt(\sum_{k=1}^{d_2} w_{i,k}^*w_{j,k} \rt)(I\tensor \bra{g_i})T(I\tensor \ket{g_j})  \rt]\ket{v}\\
	& = \sum_{i=1}^{d_2}\sum_{j=1}^{d_2}\bra{u}\lt[\delta_{i,j}(I\tensor \bra{g_i})T(I\tensor \ket{g_j})  \rt]\ket{v}\\
	& = \sum_{i=1}^{d_2}\bra{u}\lt[(I\tensor \bra{g_i})T(I\tensor \ket{g_i})  \rt]\ket{v}\\
	& = \bra{u}tr_2(T)\ket{v}_{\{\ket{g_k}\}}
\end{align*}
Hence $\sum\limits_{k}\bra{u\tensor g_k}T\ket{v\tensor g_k}$ is invariant.
}

%%%%%%%%%%%%%%%%%%%%%%%%%%%%%%%%%%%%%%%%%%%%%%%%%%%%%%%%%%%%%%%%%%%%%%%%%%%%%%%%%%%%%%%%%%%%%%%%%%%%%%%%%%%%%%%%%%%%%%%%%%%%%%%%%%%%%%%%
% Problem 4
%%%%%%%%%%%%%%%%%%%%%%%%%%%%%%%%%%%%%%%%%%%%%%%%%%%%%%%%%%%%%%%%%%%%%%%%%%%%%%%%%%%%%%%%%%%%%%%%%%%%%%%%%%%%%%%%%%%%%%%%%%%%%%%%%%%%%%%%

\begin{problem}{%problem statement
		Mark Wilde: Exercise 3.3.3
	}{p4% problem reference text
	}
	Show that the Pauli matrices are all Hermitian, unitary, they square to the identity, and their eigenvalues are $\pm 1$
	%Problem		
\end{problem}
\solve{
Pauli matrices are 
$$
	X\ket{0}=\ket{1},X\ket{1}=\ket{0}\qquad 
Y\ket{0}=-i\ket{1},Y\ket{1}=i\ket{0}\qquad 
Z\ket{0}=\ket{0},Z\ket{1}=-\ket{1}$$
Therefore we have 
$$
X=\ket{1}\bra{0}+\ket{0}\bra{1}\qquad 
Y=i[\ket{0}\bra{1}-\ket{1}\bra{0}]\qquad 
Z=\ket{0}\bra{0}-\ket{1}\bra{1}$$
Hence 
\begin{align*}
	X^{\dagger} & =(\ket{1}\bra{0})^{\dagger}+(\ket{0}\bra{1})^{\dagger}=\ket{0}\bra{1}+\ket{1}\bra{0}=X       \\
	Y^{\dagger} & =(i\ket{0}\bra{1})^{\dagger}+(-i\ket{1}\bra{0})^{\dagger}=-i\ket{1}\bra{0}+i\ket{0}\bra{1}=Y \\
	Z^{\dagger} & =(\ket{0}\bra{0})^{\dagger}-(\ket{1}\bra{1})^{\dagger}=\ket{0}\bra{0}-\ket{1}\bra{1}=Z
\end{align*}
Therefore they are Hermitian. 

Now \begin{align*}
	X^{\dagger}X=XX^{\dagger}=X^2& =\lt[ \ket{1}\bra{0}+\ket{0}\bra{1} \rt]\lt[ \ket{1}\bra{0}+\ket{0}\bra{1}  \rt]\\
	& = \ket{1}\bra{0}\ket{1}\bra{0}+\ket{1}\bra{0}\ket{0}\bra{1}+\ket{0}\bra{1}\ket{1}\bra{0}+\ket{0}\bra{1}\ket{0}\bra{1} \\
	& = \ket{1}\bra{1}+\ket{0}\bra{0}=I
\end{align*}
\begin{align*}
	Y^{\dagger}Y=Y^{\dagger}=Y^2& =\lt[ i(\ket{0}\bra{1}-\ket{1}\bra{0}) \rt]\lt[ i(\ket{0}\bra{1}-\ket{1}\bra{0})  \rt]\\
	&=-\lt[ \ket{0}\bra{1}\ket{0}\bra{1}-\ket{0}\bra{1}\ket{1}\bra{0}-\ket{1}\bra{0}\ket{0}\bra{1}+\ket{1}\bra{0}\ket{1}\bra{0}\rt] \\
	& = \ket{0}\bra{0}+\ket{1}\bra{1}=I
\end{align*}
\begin{align*}
	Z^{\dagger}Z=Z^{\dagger}=Z^2& =\lt[ \ket{0}\bra{0}-\ket{1}\bra{1} \rt]\lt[ \ket{0}\bra{0}-\ket{1}\bra{1} \rt]\\
	& =\ket{0}\bra{0}\ket{0}\bra{0}-\ket{0}\bra{0}\ket{1}\bra{1}-\ket{1}\bra{1} \ket{0}\bra{0}+\ket{1}\bra{1} \ket{1}\bra{1} \\
	& =\ket{0}\bra{0}+ \ket{1}\bra{1}=I
\end{align*}Therefore $X,Y,Z$ are unitary and they square to the identity.

Since $X\ket{0}=\ket{1}$ and $X\ket{1}=\ket{0}$ we have 
$$X\frac1{\sqrt{2}}\lt(\ket{0}+\ket{1}\rt)=\frac1{\sqrt{2}}\lt(\ket{1}+\ket{0}\rt)\quad X\frac1{\sqrt{2}}\lt(\ket{0} -\ket{1}\rt)=\frac1{\sqrt{2}}\lt(\ket{1}-\ket{0}\rt)=-\frac1{\sqrt{2}}\lt(\ket{0}-\ket{1}\rt)$$So the for the eigenvalue $1$ the corresponding eignevector is $\ket{+}=\frac1{\sqrt{2}}\lt(\ket{0}+\ket{1}\rt)$ and for the eigenvalue $-1$ the corresponding eigenvalue is $\frac1{\sqrt{2}}\lt(\ket{0}-\ket{1}\rt)$.

Since $Y\ket{0}=-i\ket{1}$ and $Y\ket{1}=i\ket{0}$ we have 
$$Y\frac1{\sqrt{2}}\lt(\ket{0}+i\ket{1}\rt)=\frac1{\sqrt{2}}\lt(-i\ket{1}+i^2\ket{0}\rt) = -\frac1{\sqrt{2}}\lt(i\ket{1}+\ket{0}\rt)$$ $$ Y\frac1{\sqrt{2}}\lt(\ket{0} -i\ket{1}\rt)=\frac1{\sqrt{2}}\lt(-i\ket{1}-i^2\ket{0}\rt)=\frac1{\sqrt{2}}\lt(\ket{0}-i\ket{1}\rt)$$So the for the eigenvalue $1$ the corresponding eigenvector is $\ket{0} -i\ket{1}$ and for the eigenvalue $-1$ the corresponding eigenvalue is $\ket{0}+i\ket{1}$.

Since $Z\ket{0}=\ket{0}$ and $Z\ket{1}=-\ket{1}$. So the for the eigenvalue $1$ the corresponding eigenvector is $\ket{0}$ and for the eigenvalue $-1$ the corresponding eigenvalue is $\ket{1}$.
}

%%%%%%%%%%%%%%%%%%%%%%%%%%%%%%%%%%%%%%%%%%%%%%%%%%%%%%%%%%%%%%%%%%%%%%%%%%%%%%%%%%%%%%%%%%%%%%%%%%%%%%%%%%%%%%%%%%%%%%%%%%%%%%%%%%%%%%%%
% Problem 5
%%%%%%%%%%%%%%%%%%%%%%%%%%%%%%%%%%%%%%%%%%%%%%%%%%%%%%%%%%%%%%%%%%%%%%%%%%%%%%%%%%%%%%%%%%%%%%%%%%%%%%%%%%%%%%%%%%%%%%%%%%%%%%%%%%%%%%%%

\begin{problem}{%problem statement
	}{p5% problem reference text
	}
For $S,T\in \sL(\mcH)$, show that $$tr(T)=tr(T^{\dagger})^*,\qquad tr(ST)=tr(TS)$$ [Recall $T^+$ denotes adjoint of $T$]. For $\ket{x},\ket{y}\in\mcH$ show $$tr(\ket{x}\bra{y}T)=tr(T\ket{x}\bra{y})=\bra{y}\ket{Tx}$$
%Problem		
\end{problem}
\solve{
\begin{itemize}
	\item $tr(T)$ is the summation of the diagonal entries of $T$. Now $T^{\dagger}=(T^t)^*$. Now the diagonal elements of $T$ remains in the same same position even after transpose. Hence the diagonal elements of $T^{\dagger}$ are the complex conjugate of the diagonal elements of $T$. Hence sum of the diagonal entries of $T^{\dagger}$ will also be the complex conjugate of the sum of the diagonal entries of $T$. Therefore we get $$tr(T)=tr(T^{\dagger})^*$$
	\item Let $\dim\mcH-d$. Suppose $\{\ket{e_k}\mid k\in [d]\}\subseteq \mcH$ be an ONB of $\mcH$\begin{align*}
		tr(ST) & = \sum_{k=1}^d\bra{e_k}ST\ket{e_k} = \sum_{k=1}^d\bra{e_k}SIT\ket{e_k}\\
		& = \sum_{k=1}^d\bra{e_k}S\lt[ \sum_{i=1}^d \ket{e_i}\bra{e_i} \rt]\ket{e_k}\\
		& = \sum_{k-1}^{d}\sum_{i=1}^d\bra{e_k}S\ket{e_i} \bra{e_i}T\ket{e_k}\\
		& =\sum_{i=1}^d\sum_{k=1}^d\bra{e_i}T\ket{e_k}\bra{e_k}S\ket{e_i}\\
		& = \sum_{i=1}^d \bra{e_i}T\lt[ \sum_{k=1}^d \ket{e_k}\bra{e_k} \rt]S\ket{e_i}\\
		& = \sum_{i=1}^d\bra{e_i}TIS\ket{e_i} = \sum_{i=1}^d\bra{e_i}TS\ket{e_i}=tr({TS})
	\end{align*}
	\item Let $\{\ket{e_i}\mid 1\leq i\leq d\}$ is an ONB of $\mcH$. Then $\ket{x}=\sum\limits_{i=1}^d\alpha_i\ket{e_i}$ and $\ket{y}=\sum\limits_{i=1}^d \beta_i\ket{e_i}$ where $\alpha_i,\beta_i\in \bbC$ for all $i\in [d]$. Now \begin{align*}
		tr(\ket{x}\bra{y}T)&=\sum_{k=1}^d\bra{e_i}\ket{x}\bra{y}T\ket{e_i}\\
		& = \sum_{k=1}^d \bra{y}T\ket{e_i}\bra{e_i}\ket{x}\\
		& = \bra{y}T\lt[ \sum_{k=1}^d\ket{e_i}\bra{e_i} \rt]\ket{x}\\
		& = \bra{y}TI\ket{x}=\bra{y}T\ket{x}
	\end{align*}
	Now 
	$$tr(\ket{x}\bra{y}T)=tr([\ket{x}\bra{y}]T)=tr(T[\ket{x}\bra{y}])$$Therefore we have
	$$tr(\ket{x}\bra{y}T)=tr(T\ket{x}\bra{y})=\bra{y}\ket{Tx}$$
	

\end{itemize}
}

%%%%%%%%%%%%%%%%%%%%%%%%%%%%%%%%%%%%%%%%%%%%%%%%%%%%%%%%%%%%%%%%%%%%%%%%%%%%%%%%%%%%%%%%%%%%%%%%%%%%%%%%%%%%%%%%%%%%%%%%%%%%%%%%%%%%%%%%
% Problem 6
%%%%%%%%%%%%%%%%%%%%%%%%%%%%%%%%%%%%%%%%%%%%%%%%%%%%%%%%%%%%%%%%%%%%%%%%%%%%%%%%%%%%%%%%%%%%%%%%%%%%%%%%%%%%%%%%%%%%%%%%%%%%%%%%%%%%%%%%

\begin{problem}{%problem statement
	}{p6% problem reference text
	}
	Suppose $\mcH$ is \fdcps with $\dim(\mcH)=d$. Show complex dimensionality of $\sL(\mcH)$ is $d^2$, real dimensionality of $\sR(\mcH)$ is $d^2$.\parinn
	
	Suppose $\mcH$ is a real inner product space of $\dim$ $d$, show $\sL(\mcH)$ has dimension $d^2$ and  the space of all symmetric operators  is a real vector space of dimension $\frac{d(d+1)}{2}$.
	%Problem		
\end{problem}
\solve{
	 \begin{itemize}
	 	\item Suppose $\{\ket{e_i}\in\mcH\mid 1\leq i\leq d\}$ is an ONB of $\mcH$. Let $T\in\sL(\mcH)$. Then for all $i\in [d]$ $$T\ket{e_i}=\sum_{j=1}^d\alpha_{i,j}\ket{e_j}$$ where $\alpha_{i,j}\in \bbC$. Hence, the map $T$ is uniquely decided by the numbers $\alpha_{i,j}\in \bbC$ for all $i,j\in [d]$. Hence, there are $d^2$ many numbers which uniquely decides $T$. Therefore $\dim(\sL(\mcH))= d^2$. 
	 
	\item Now let $T\in\sR(\mcH)$. Then $T^{\dagger}=T$. Again suppose $\{\ket{e_i}\in\mcH\mid 1\leq i\leq d\}$ is an ONB of $\mcH$. Let $(i,j)$th element of $T$ is denoted by $t_{i,j}$. Then for all $i\in [d]$, $T_{i,i}\in \bbR$ since $T^{\dagger}=T$. Now for all off diagonal entries $t_{j,i}=t^*_{i,j}$. So there are $\frac{n^2-n}2$ many complex numbers which decides $T$ uniquely apart from the $n$ real entries in the diagonal. Now for each $i,j\in[d]$ let  $t_{i,j}=x_{i,j}+iy_{i,j}$ where $x_{i,j},y_{i,j}\in  \bbR$. Therefore, $$t_{j,i}=t^*_{i,j}=x_{i,j}-iy_{i,j}$$So for each off-diagonal entries there are corresponding 2 real numbers. And there are total $\frac{d^2-d}2$ many off-diagonal entries which participates in uniquely deciding $T$. Hence there are total $$2\times \frac{d^2-d}2+d=d^2$$ real numbers which uniquely decides $T$. Hence $\dim(\sR(\mcH))= d^2$. 
	\item Suppose $\{\ket{e_i}\in\mcH\mid 1\leq i\leq d\}$ is a basis of $\mcH$. Let $T\in\sL(\mcH)$. Then for all $i\in [d]$ $$T\ket{e_i}=\sum_{j=1}^d\alpha_{i,j}\ket{e_j}$$ where $\alpha_{i,j}\in \bbR$. Hence, the map $T$ is uniquely decided by the numbers $\alpha_{i,j}\in \bbC$ for all $i,j\in [d]$. Since there are $d^2$ many numbers which uniquely decides $T$, $\dim(\sL(\mcH))= d^2$. 
	\item Let $T\in\sR(\mcH)$. Then $T^t=T$. Again suppose $\{\ket{e_i}\in\mcH\mid 1\leq i\leq d\}$ is an basis of $\mcH$. Let $(i,j)$th element of $T$ is denoted by $T_{i,j}$. Now for all off diagonal entries $T_{j,i}=T_{i,j}$. So there are $\frac{d^2-d}2$ many real numbers which decides $T$ uniquely apart from the $d$ entries in the diagonal. Therefore,  there are total $\frac{d^2-d}2$ many off-diagonal entries which participates in uniquely deciding $T$. Hence there are total $$ \frac{d^2-d}2+d=\frac{d^2+d}{2}=\frac{d(d+1)}2$$ real numbers which uniquely decides $T$. Hence $\dim(\sR(\mcH))= \frac{d(d+1)}2$. 
	 \end{itemize}
}
\pagebreak
%%%%%%%%%%%%%%%%%%%%%%%%%%%%%%%%%%%%%%%%%%%%%%%%%%%%%%%%%%%%%%%%%%%%%%%%%%%%%%%%%%%%%%%%%%%%%%%%%%%%%%%%%%%%%%%%%%%%%%%%%%%%%%%%%%%%%%%%
% Problem 7
%%%%%%%%%%%%%%%%%%%%%%%%%%%%%%%%%%%%%%%%%%%%%%%%%%%%%%%%%%%%%%%%%%%%%%%%%%%%%%%%%%%%%%%%%%%%%%%%%%%%%%%%%%%%%%%%%%%%%%%%%%%%%%%%%%%%%%%%

\begin{problem}{%problem statement
	}{p7% problem reference text
	}
Show that $\sD(\mcH)$ is a convex subset of the real vector space of all Hermitian operators on $\mcH$. Show that the extreme points of $\sD(\mcH)$ are pure states, i.e.  rank 1 projection operators.
	%Problem		
\end{problem}
\solve{
\begin{itemize}
	\item Let $\rho_1\rho_2\in \sD(\mcH)$. Suppose $\lm\in [0,1]$. Then denote $$\rho=\lm\rho_1+(1-\lm)\rho_2$$Now $\rho\in \sD(\mcH)$ if $\rho\geq 0$ and $tr(\rho)=1$. Now $$tr(\rho)=tr(\lm\rho_1+(1-\lm)\rho_2)=\lm tr(\rho_1)+(1-\lm)tr(\rho_2)=\lm+(1-\lm)=1$$Hence we only need to show that for all $\ket{u}\in \mcH$ we have $\bra{u}\rho\ket{u}\geq 0$. Now $$\bra{u}\rho\ket{u}=\bra{u}(\lm\rho_1+(1-\lm)\rho_2)ket{u}=\lm\bra{u}\rho_1\ket{u}+(1-\lm)\bra{u}\rho_2\ket{u}$$ Now since $\rho_1,\rho_2\in \sD(\mcH)$ we have $\bra{u}\rho_1\ket{u}\geq 0$, $\bra{u}\rho_2\ket{u}\geq 0$. We also have $\lm\geq 0$. Hence, $1-\lm\geq 0$. Therefore, $=\lm\bra{u}\rho_1\ket{u}+(1-\lm)\bra{u}\rho_2\ket{u}\geq 0$. Therefore we got $\bra{u}\rho\ket{u}\geq 0$. Since $\ket{u}$ is any arbitrary vector of $\mcH$, we have $\rho \geq 0$. Hence, $rho\in \sD(\mcH)$. Therefore $\sD(\mcH)$ is convex.

\item $\rho\in \sD(\mcH)$ is an extreme point if a strict convex combination $\rho=\lm\rho_1+(1-\lm)\rho_2$ with $\rho_1,\rho_2\in\sD(\mcH)$ and $\lm\in (0,1)$ is possible only if $\rho=\rho_1=\rho_2$. Suppose $\rho$ is not a pure state. Then $\rho$ represents an ensemble $\{p_X(x),\ket{x}\mid x\in X\}\subseteq \mcH$. By assumption $|X|\geq 2$.  Then $$\rho=\sum_{x\in X}p_X(x)\ket{x}\bra{x}$$Since $|X|\geq 2$ we can partition $X$ into two disjoint sets $U,V$ such that $U\sqcup V=X$. Then take $a=\sum\limits_{x\in U}p_X(x)$. Certainly $a\in (0,1)$. Then take the ensembles $\mcF_1=\lt\{\frac{p_X(x)}{a},\ket{x}\mid x\in U\rt\}$ and  $\mcF_2=\lt\{\frac{p_X(x)}{1-a},\ket{x}\mid x\in V\rt\}$. Certainly $$\sum_{x\in U}\frac{p_X(x)}{a}=\frac{1}{a}\sum_{x\in U}p_X(x)=\frac1a\, a=1$$and$$\sum_{x\in V}\frac{p_X(x)}{1-a}=\frac{1}{1-a}\sum_{x\in V}p_X(x)=\frac1{1-a}\lt[ 1-\sum_{x\in U}p_X(x) \rt] =\frac1{1-a}[1-a]=1$$Hence $\mcF_1$ and $\mcF_2$ are in fact ensembles. Then their corresponding density matrices are $$\rho_1=\frac1a\sum_{x\in U}p_X(x)\ket{x}\bra{x}\qquad \rho_2=\frac1{1-a}\sum_{x\in V}p_X(x)\ket{x}\bra{x}$$Then we have \begin{align*}
	a\rho_1+(1-a)\rho_2&=a\lt[\frac1a\sum_{x\in U}p_X(x)\ket{x}\bra{x}\rt]+(1-a)\lt[\frac1{1-a}\sum_{x\in V}p_X(x)\ket{x}\bra{x}\rt]\\
	& =\sum_{x\in U}p_X(x)\ket{x}\bra{x}+\sum_{x\in V}p_X(x)\ket{x}\bra{x}\\
	& = \sum_{x\in X}p_X(x)\ket{x}\bra{x}=\rho
\end{align*}We can write $\rho$ as a strictly convex sum of two density operators but none of them are equal to $\rho$. Hence contradiction. Hence $\rho$ is a pure state, a rank 1 operator. \parinn

For the other direction let $\rho$ is a pure state. Suppose $$\rho=\lm\rho_1+(1-\lm)\rho_2$$ with $\rho_1,\rho_2\in\sD(\mcH)$ and $\lm\in (0,1)$. Since $\rho=\ket{\phi}\bra{\phi}$ for some $\ket{\phi}\in \mcH$ we have $\rho^2=\rho$. Then  by Cauchy-Schwartz Inequality we have \begin{align*}
	tr(\rho) & =\tr(\rho^2)                                                                       \\
	         & = \lm^2tr(\rho_1^2)+2\lm(1-\lm)tr(\rho_1\rho_2)+(1-\lm)^2tr(\rho_2^2)              \\
	         & \leq\lm^2tr(\rho_1)+(1-\lm)^2tr(\rho_2)+2\lm(1-\lm)\sqrt{tr(\rho_1^2)tr(\rho_2^2)} \\
	         & =\lm^2tr(\rho_1)+(1-\lm)^2tr(\rho_2)+2\lm(1-\lm)\sqrt{tr(\rho_1)tr(\rho_2)}        \\
	         & = \lm^2+(1-\lm)^2+2\lm(1-\lm)=1
\end{align*}
But we know $tr(\rho)=1$. Hence equality is attained. Therefore, in Cauchy-Schwartz Inequality equality is attained. Therefore $tr(\rho_1\rho_2)=\sqrt{tr(\rho_1)tr(\rho_2)}$ i.e. $\rho_1=\rho_2$. Hence $\rho$ is a extreme point. Therefore, pure states are extreme points in $\sD(\mcH)$. 
\end{itemize}
}

%%%%%%%%%%%%%%%%%%%%%%%%%%%%%%%%%%%%%%%%%%%%%%%%%%%%%%%%%%%%%%%%%%%%%%%%%%%%%%%%%%%%%%%%%%%%%%%%%%%%%%%%%%%%%%%%%%%%%%%%%%%%%%%%%%%%%%%%
% Problem 8
%%%%%%%%%%%%%%%%%%%%%%%%%%%%%%%%%%%%%%%%%%%%%%%%%%%%%%%%%%%%%%%%%%%%%%%%%%%%%%%%%%%%%%%%%%%%%%%%%%%%%%%%%%%%%%%%%%%%%%%%%%%%%%%%%%%%%%%%

\begin{problem}{%problem statement
	}{p8% problem reference text
	}
Show that if $\dim(\mcH)=d$, then $\sD(\mcH)$ can be embedded  into a real vector space  of dimension $n=d^2-1$
	%Problem		
\end{problem}
\solve{
In any operator of $\sD(\mcH)$ there are $^2$ entries in the matrix of the operator. But density operator also has one extra condition which is its trace equal's to 1. Hence sum of the diagonal entries is 1. Hence for the diagonal entries it is enough to know about the $d-1$ entries instead of the all $d$ entries because the last entry will be decided by the other $d-1$ entries as their sum is 1. Except the diagonal there are total $d^2-d$ many off diagonal entries. Hence to uniquely characterize a operator in $\sD(\mcH)$ at most $(d^2-d)+(d-1)=d^2-1$ many numbers are needed. Therefore the set of operators, $\sD(\mcH)$ can be embedded into a real vector space of dimension $n=d^2-1$.
}

%%%%%%%%%%%%%%%%%%%%%%%%%%%%%%%%%%%%%%%%%%%%%%%%%%%%%%%%%%%%%%%%%%%%%%%%%%%%%%%%%%%%%%%%%%%%%%%%%%%%%%%%%%%%%%%%%%%%%%%%%%%%%%%%%%%%%%%%
% Problem 9
%%%%%%%%%%%%%%%%%%%%%%%%%%%%%%%%%%%%%%%%%%%%%%%%%%%%%%%%%%%%%%%%%%%%%%%%%%%%%%%%%%%%%%%%%%%%%%%%%%%%%%%%%%%%%%%%%%%%%%%%%%%%%%%%%%%%%%%%

\begin{problem}{%problem statement
	}{p9% problem reference text
	}
Prove the Singular value decomposition theorem stated in class.	%Problem		
\end{problem}
\solve{We will first state the singular value decomposition theorem then we will prove it. \parinf 
	
\textbf{\textit{Singular Value Decomposition Theorem:}} Suppose $T:\mcH\to \mcH$ and $\dim(\mcH)=s$ then $\exs\ U,V\in \sL(H)$ which are unitary and diagonal $D\in \sL(\mcH)$  with non-negative entries so that $$T=UDV$$ 

\textbf{\textit{Proof:}} \parinn Suppose we have an ONB of $\mcH$, $\{\ket{e_i}\mid i\in[d]\}$ of $\mcH$. Let's denote $S=T^{\dagger}T$. Now $S$ is hermitian. Hence by spectral theorem there are unitary matrix $W$ and a diagonal matrix $\Lm$ such that $$S=W\Lm W^{\dagger}$$and also we get an orthonormal eigen basis $\{\ket{v_i}\mid i\in [d]\}\subseteq \mcH$ of $\mcH$ with corresponding eigenvalues $\{\lm_i\mid i\in [d]\}$ of $S$ which are the diagonal entries of $\Lm$. 

Now if $\lm$ is an eigenvalue of $S$ with corresponding eigenvector $\ket{v}$ then $$S\ket{v}=\lm \ket{v}\implies \lm \bra{v}\ket{v}=\bra{v}Sv=(Sv)^{\dagger}\ket{v}=\lm^{\dagger}\bra{v}\ket{v}$$Therefore the eigenvalues are real. Also since $$\lm \bra{v}\ket{v}=\bra{v}S\ket{v}=\bra{v}T^{\dagger}T\ket{v}=(T\ket{v})^{\dagger}(T\ket{v})\geq0$$ we have $\lm\geq 0$ since $\bra{v}\ket{v}\geq0$. Therefore eigenvalues of $S$ are real and non-negative. Therefore entries of $\Lm$ are non-negative.

Let us denote the $i$th eigenvalue of $\Lm$ as $\lm_i$. So now take $\sg_i=\triangleq \sqrt{\lm_i}$. Now create the diagonal matrix $\Sg$ with $i$th eigenvalue of $\Sg$ is $\sg_i$.  Define the vectors $\ket{u_i}=\frac{1}{\sg_i}T\ket{v_i}$ for all $i\in [d]$. Then $$\bra{u_i}\ket{u_j}=\frac1{\sg_i\sg_j}\bra{v_i}T^{\dagger}T\ket{v_j}=\frac1{\sg_i\sg_j}\bra{v_i}S\ket{v_j}=\frac1{\sg_i\sg_j}\bra{v_i}\lm_j\ket{v_j}=\frac{\lm_j}{\sg_i\sg_j}\bra{v_i}\ket{v_j}=\delta_{i,j}$$Hence $\{\ket{u_i}\mid i\in [d]\}$ also forms an orthonormal basis. 

Now we have two maps $V:\mcH\to \mcH$ and $U:\mcH\to \mcH$ which send the orthonormal basis $\{\ket{e_i}\mid i\in [d]\}$ to $\{\ket{v_i}\mid i\in [d]\}$ by $V\ket{e_i}=\ket{v_i}$ and $\{\ket{e_i}\mid i\in [d]\}$ to $\{\ket{u_i}\mid i\in [d]\}$ by $U\ket{e_i}=\ket{u_i}$. Therefore we have the matrix of $U,V$ are orthonormal. By definition of $u_i$ we have  $Tv_i=\sg_iu_i$ for all $i\in[d]$. Hence we have $$TV=U\Sg\implies T=U\Sg V^{\dagger}$$ 
}

%%%%%%%%%%%%%%%%%%%%%%%%%%%%%%%%%%%%%%%%%%%%%%%%%%%%%%%%%%%%%%%%%%%%%%%%%%%%%%%%%%%%%%%%%%%%%%%%%%%%%%%%%%%%%%%%%%%%%%%%%%%%%%%%%%%%%%%%
% Problem 10
%%%%%%%%%%%%%%%%%%%%%%%%%%%%%%%%%%%%%%%%%%%%%%%%%%%%%%%%%%%%%%%%%%%%%%%%%%%%%%%%%%%%%%%%%%%%%%%%%%%%%%%%%%%%%%%%%%%%%%%%%%%%%%%%%%%%%%%%

\begin{problem}{%problem statement
	}{p10% problem reference text
	}
	Suppose $\ket{\psi}_{AR_1}\in \mcH_{A}\tensor \mcH_{R_1}$, $\ket{\psi}_{AR_2}\in \mcH_A\tensor \mcH_{R_2}$ are purifications of  $\rho_A\in \sD(\mcH_A)$ and $\dim(\mcH_{R_2})\geq \dim(\mcH_{R_1})$, then show that  there exists an isometry $V:\mcH_{R_1}\to \mcH_{R_2}$  such that $$\ket{\psi}_{AR_2}=(V\tensor I)\ket{\psi}_{AR_1}$$
	%Problem		
\end{problem}

%%%%%%%%%%%%%%%%%%%%%%%%%%%%%%%%%%%%%%%%%%%%%%%%%%%%%%%%%%%%%%%%%%%%%%%%%%%%%%%%%%%%%%%%%%%%%%%%%%%%%%%%%%%%%%%%%%%%%%%%%%%%%%%%%%%%%%%%
% Problem 11
%%%%%%%%%%%%%%%%%%%%%%%%%%%%%%%%%%%%%%%%%%%%%%%%%%%%%%%%%%%%%%%%%%%%%%%%%%%%%%%%%%%%%%%%%%%%%%%%%%%%%%%%%%%%%%%%%%%%%%%%%%%%%%%%%%%%%%%%

\begin{problem}{%problem statement
		Mark Wilde: Exercise 3.6.5
	}{p11% problem reference text
	}
	Show that the Bell states form an orthonormal basis: $$\bra{\Phi^{z_1x_1}}\ket{\Phi^{z_2x_2}}=\delta_{z_1,z_2}\,\delta_{x_1,x_2}$$
	%Problem		
\end{problem}
\solve{By definition we have
$$\ket{\Phi^{z,x}}=(Z^z\tensor I)(X^x\tensor I)\ket{\Phi^+}=(Z^zX^x\tensor I)\ket{\Phi^+}$$We have $\ket{\Phi^+}=\frac1{\sqrt{2}}(\ket{00}+\ket{11})$. Therefore, \begin{align*}
	\bra{\Phi^{z_1x_1}}\ket{\Phi^{z_2x_2}} & = \bra{\Phi^{+}}(X^{x_1}Z^{z_1}\tensor I)(Z^{z_2}X^{x_2}\tensor I)\ket{\Phi^{+}}\\
	& = \bra{\Phi^{+}}(X^{x_1}Z^{z_1}Z^{z_2}X^{x_2})\tensor I\ket{\Phi^{+}}\\
	& = \bra{\Phi^{+}}(X^{x_1}Z^{z_1\oplus z_2}X^{x_2})\tensor I\ket{\Phi^{+}}\\
	& = \frac12 \Big[\bra{0}X^{x_1}Z^{z_1\oplus z_2}X^{x_2}\ket{0}\bra{0}\ket{0}+\bra{1}X^{x_1}Z^{z_1\oplus z_2}X^{x_2}\ket{1}\bra{1}\ket{1}\Big]\\
	& = \frac12 \Big[\bra{0}X^{x_1}Z^{z_1\oplus z_2}X^{x_2}\ket{0}+\bra{1}X^{x_1}Z^{z_1\oplus z_2}X^{x_2}\ket{1}\Big]
\end{align*}
Now we will do case wise analysis. 
\subsection*{Case 1: $\boldsymbol{z_1\oplus z_2=0}$}
Then $Z^{z_1\oplus z_2}=I$. Therefore $$X^{x_1}Z^{z_1\oplus z_2}X^{x_2}=X^{x_1}X^{x_2}=X^{x_1\oplus x_2}$$ Hence \begin{align*}
	 & \frac12\Big[   \bra{0}X^{x_1\oplus x_2}\ket{0}+  \bra{1}X^{x_1\oplus x_2}\ket{1} \Big]=\delta_{0,x_1\oplus x_2}=\delta_{x_1,x_2}   =\delta_{x_1,x_2}\times 1=\delta_{0,z_1\oplus z_2}\delta_{x_1,x_2}=\delta_{z_1,z_2}\delta_{x_1,x_2}
\end{align*}
\subsection*{Case 2: $\boldsymbol{z_1\oplus z_2=1}$}
So now $Z^{z_1\xor z_2}=Z$. Now we will  analyze  all possible cases
\subsubsection*{Case 2.I: $\boldsymbol{x_1=0,x_2=0}$}
Then $X^{x_1}Z^{z_1\oplus z_2}X^{x_2}=Z$. Hence \begin{align*}
	& \frac12 \Big[ \bra{0}Z\ket{0}+\bra{1}Z\ket{1}  \Big]\\
	& = \frac12 \Big[ \bra{0}\ket{0}-\bra{1}\ket{1}   \Big]\\
	& = 0 = \delta_{z_1,z_2}=\delta_{z_1,z_2}\times 1=\delta_{z_1,z_2}\delta_{x_1,x_2}
\end{align*}
\subsubsection*{Case 2.II: $\boldsymbol{x_1=0,x_2=1}$}
Now $X^{x_1}Z^{z_1\oplus z_2}X^{x_2}=ZX$. 
So \begin{align*}
	&\frac12 \Big[\bra{0}ZX\ket{0}+\bra{1}ZX\ket{1}\Big] =\frac12 \Big[\bra{0}Z\ket{1}+\bra{1}Z\ket{0}\Big]=\frac12 \Big[-\bra{0}\ket{1}+\bra{1}\ket{0}\Big]	= 0=\delta_{z_1,z_2}\delta_{x_1,x_2}
\end{align*}
\subsubsection*{Case 2.III: $\boldsymbol{x_1=1,x_2=0}$}
Now $X^{x_1}Z^{z_1\oplus z_2}X^{x_2}=XZ$. 
So \begin{align*}
	&\frac12 \Big[\bra{0}XZ\ket{0}+\bra{1}ZX\ket{1}\Big] =\frac12 \Big[\bra{0}X\ket{0}-\bra{1}X\ket{1}\Big]=\frac12 \Big[\bra{0}\ket{1}-\bra{1}\ket{0}\Big]	= 0 = 0\times 0=\delta_{z_1,z_2}\delta_{x_1,x_2}
\end{align*}
\subsubsection*{Case 2.IV: $\boldsymbol{x_1=1,x_2=1}$}
Now $X^{x_1}Z^{z_1\oplus z_2}X^{x_2}=XZX$. 
So \begin{align*}
	&\frac12 \Big[\bra{0}XZX\ket{0}+\bra{1}ZX\ket{1}\Big]=\frac12 \Big[\bra{0}XZ\ket{1}+\bra{1}XZ\ket{0}\Big]= \frac12 \Big[-\bra{0}X\ket{1}+\bra{1}X\ket{0}\Big]\\
	=& \frac12 \Big[-\bra{0}\ket{0}+\bra{1}\ket{1}\Big]	=  \frac12[-1+1]= 0 = 0\times 1 =\delta_{z_1,z_2}\delta_{x_1,x_2}
\end{align*}Therefore we can say $$\bra{\Phi^{z_1x_1}}\ket{\Phi^{z_2x_2}}=\delta_{z_1,z_2}\,\delta_{x_1,x_2}$$Hence Bell States form an orthonormal basis.

}

%%%%%%%%%%%%%%%%%%%%%%%%%%%%%%%%%%%%%%%%%%%%%%%%%%%%%%%%%%%%%%%%%%%%%%%%%%%%%%%%%%%%%%%%%%%%%%%%%%%%%%%%%%%%%%%%%%%%%%%%%%%%%%%%%%%%%%%%
% Problem 12
%%%%%%%%%%%%%%%%%%%%%%%%%%%%%%%%%%%%%%%%%%%%%%%%%%%%%%%%%%%%%%%%%%%%%%%%%%%%%%%%%%%%%%%%%%%%%%%%%%%%%%%%%%%%%%%%%%%%%%%%%%%%%%%%%%%%%%%%

\begin{problem}{%problem statement
		Mark Wilde: Exercise 3.7.11
	}{p12% problem reference text
	}
	 Show that the set of states $\{\ket{\Phi^{x,z}}_{AB}\}^{d-1}_{x,z=0}$ forms a complete, orthonormal basis:$$\bra{\Phi^{x_1,z_1}}\ket{\Phi^{x_2,z_2}}=\delta_{x_1,x_2}\,\delta_{z_1,z_2}\qquad \sum_{x,z=0}^d\ket{\Phi^{x,z}}\bra{\Phi^{x,z}}=I_{AB}$$
	%Problem		
\end{problem}
\solve{
We have $$X(x)\ket{j}=\ket{j\xor x}\qquad Z(z)\ket{j}=\exp{\frac{2\pi izj}{d}}$$Therefore $$X(x)Z(z)\ket{j}=\exp{\frac{2\pi izj}{d}}\ket{j\xor x}$$Now 
\begin{align*}
	\ket{\Phi^{x,z}}_{AB}&=(X_A(x)Z_A(z)\tensor I_B)\ket{\Phi}_{AB}\qquad \ket{\Phi}_{AB}=\frac1{\sqrt{d}}\sum_{j=0}^{d-1}\ket{j}_A\ket{j}_B
\end{align*}Therefore we have 
\begin{align*}
	\ket{\Phi^{x,z}}_{AB} & =X_A(x)Z_A(z)\tensor I_B\ket{\Phi}_{AB}                                                    \\
	                      & =\frac1{\sqrt{d}}\sum_{j=0}^{d-1}X_A(x)Z_A(z)\ket{j}_A\tensor \ket{j}_B                    \\
	                      & =\frac1{\sqrt{d}}\sum_{j=0}^{d-1}\exp{\frac{2\pi ijzj}{d}}\ket{j\xor x}_A\tensor \ket{j}_B
\end{align*}
Hence \begin{align*}
	\bra{\Phi^{x_1,z_1}}\ket{\Phi^{x_2,z_2}} & = \frac1d\lt[\sum_{j=0}^{d-1}\exp{-\frac{2\pi iz_1j}{d}}\bra{j\xor x_1}_A\tensor \bra{j}_B\rt]\lt[ \sum_{k=0}^{d-1}\exp{\frac{2\pi iz_2k}{d}}\ket{k\xor x_2}_A\tensor \ket{k}_B \rt]\\
	& = \frac1d\lt[   \sum_{j=0}^{d-1}\sum_{k=1}^{d-1} \exp{\frac{2\pi i}{d}(kz_2-jz_1)}\bra{j\xor x_1}_A\ket{k\xor x_2}_A\bra{j}_B\ket{k}_B   \rt]\\
	& = \frac1d\lt[   \sum_{j=0}^{d-1}  \exp{\frac{2\pi ij}{d}(z_2-z_1)}\bra{j\xor x_1}_A\ket{j\xor x_2}_A   \rt]\\
	& = \frac1d\lt[\delta_{x_1,x_2} \sum_{i=0}^{d-1} \exp{\frac{2\pi ij}{d}(z_2-z_1)}\rt]\\[3mm]
	& = \begin{cases}
		\dfrac{1}{d}\, d\,\delta_{x_1,x_2}=\delta_{x_1,x_2} & \text{when $z_1=z_2$}\\
		\dfrac{1}{d}\, \delta_{x_1,x_2}\,\dfrac{1-\exp{\frac{2\pi i d}{d}}}{1-\exp{\frac{2\pi i}{d}}}= \dfrac{1}{d}\, \delta_{x_1,x_2}\,\dfrac{1-\exp{2\pi i d}}{1-\exp{\frac{2\pi i}{d}}}=0 & \text{when $z_1\neq z_2$}
	\end{cases}\\[3mm]
	& = \delta_{x_1,x_2}\delta_{z_1,z_2}
\end{align*}Also 
\begin{align*}
\sum_{x,z=0}^d\ket{\Phi^{x,z}} 	\bra{\Phi^{x,z}} & = \frac1d \sum_{x,z=0}^d\lt[ \sum_{j=0}^{d-1}\exp{\frac{2\pi izj}{d}}\ket{j\xor x}_A\tensor \ket{j}_B \rt]\lt[\sum_{k=0}^{d-1}\exp{-\frac{2\pi izk}{d}}\bra{k\xor x}_A\tensor \bra{k}_B\rt]\\
& = \frac1d\sum_{x,z=0}^d\lt[   \sum_{j=0}^{d-1}\sum_{k=1}^{d-1} \exp{\frac{2\pi iz}{d}(k-j)}\ket{j\xor x}_A\bra{k\xor x}_A\tensor\ket{j}_B\bra{k}_B   \rt]\\
& = \frac1d\sum_{x=0}^d\sum_{j=0}^{d-1}\sum_{k=1}^{d-1} \lt[  \sum_{z=0}^d \exp{\frac{2\pi iz}{d}(k-j)} \rt]\ket{j\xor x}_A\bra{k\xor x}_A\tensor\ket{j}_B\bra{k}_B  \\
& = \frac1d\sum_{x=0}^d\sum_{j=0}^{d-1} d\ket{j\xor x}_A\bra{k\xor x}_A\tensor\ket{j}_B\bra{k}_B \\
& =  \sum_{x=0}^d\sum_{j=0}^{d-1} \ket{j\xor x}_A\bra{j\xor x}_A\tensor\ket{j}_B\bra{j}_B\\
& = \sum_{j=0}^{d-1} \lt[ \sum_{x=0}^d]\ket{j\xor x}_A\bra{j\xor x}_A\rt]\tensor\ket{j}_B\bra{j}_B\\
& = \sum_{j=0}^{d-1} I_A\tensor\ket{j}_B\bra{j}_B= I_A\tensor I_B=I_{A,B}
\end{align*}
}

%%%%%%%%%%%%%%%%%%%%%%%%%%%%%%%%%%%%%%%%%%%%%%%%%%%%%%%%%%%%%%%%%%%%%%%%%%%%%%%%%%%%%%%%%%%%%%%%%%%%%%%%%%%%%%%%%%%%%%%%%%%%%%%%%%%%%%%%
% Problem 13
%%%%%%%%%%%%%%%%%%%%%%%%%%%%%%%%%%%%%%%%%%%%%%%%%%%%%%%%%%%%%%%%%%%%%%%%%%%%%%%%%%%%%%%%%%%%%%%%%%%%%%%%%%%%%%%%%%%%%%%%%%%%%%%%%%%%%%%%

\begin{problem}{%problem statement
		Mark Wilde: Exercise 4.1.5
	}{p13% problem reference text
	}
	 Show that the following ensembles have the same density operator: $\lt\{\lt\{\frac12,\ket{0}\rt\},\lt\{\frac12,\ket{1}\rt\}\rt\}$ and $\lt\{\lt\{\frac12,\ket{+}\rt\},\lt\{\frac12,\ket{-}\rt\}\rt\}$
	%Problem		
\end{problem}
\solve{
The density operator of the ensemble $\lt\{\lt\{\frac12,\ket{0}\rt\},\lt\{\frac12,\ket{1}\rt\}\rt\}$ is $$\rho_1=\frac12\ket{0}\bra{0}+\frac12\ket{1}\bra{1}=\frac12\Big[ \ket{0}\bra{0}+\ket{1}\bra{1} \Big]$$

Now$$\ket{+}\bra{+}=\frac12\Big[ \ket{0}+\ket{1} \Big]\Big[ \bra{0}+\bra{1} \Big]=\frac12\Big[\ket{0}\bra{0}+\ket{0}\bra{1}+\ket{1}\bra{0}+\ket{1}\bra{1}   \Big]$$and similarly $$\ket{-}\bra{-}=\frac12\Big[ \ket{0}-\ket{1} \Big]\Big[ \bra{0}-\bra{1} \Big]=\frac12\Big[\ket{0}\bra{0}-\ket{0}\bra{1}-\ket{1}\bra{0}-\ket{1}\bra{1}   \Big]$$
The density operator of the ensemble $\lt\{\lt\{\frac12,\ket{+}\rt\},\lt\{\frac12,\ket{-}\rt\}\rt\}$ is 
\begin{align*}
	\rho_2 &= \frac12\ket{+}\bra{+}+\frac12\ket{-}\bra{-}\\
	& = \frac14\Big[\ket{0}\bra{0}+\ket{0}\bra{1}+\ket{1}\bra{0}+\ket{1}\bra{1}   \Big]+\frac14\Big[\ket{0}\bra{0}-\ket{0}\bra{1}-\ket{1}\bra{0}-\ket{1}\bra{1}   \Big]\\
	& = \frac14\Big[2\ket{0}\bra{0}+2\ket{1}\bra{1}\Big]=\frac12\Big[\ket{0}\bra{0}+\ket{1}\bra{1}\Big]=\rho_1
\end{align*} 
Hence both the ensembles have the same density operator. 
}

%%%%%%%%%%%%%%%%%%%%%%%%%%%%%%%%%%%%%%%%%%%%%%%%%%%%%%%%%%%%%%%%%%%%%%%%%%%%%%%%%%%%%%%%%%%%%%%%%%%%%%%%%%%%%%%%%%%%%%%%%%%%%%%%%%%%%%%%
% Problem 14
%%%%%%%%%%%%%%%%%%%%%%%%%%%%%%%%%%%%%%%%%%%%%%%%%%%%%%%%%%%%%%%%%%%%%%%%%%%%%%%%%%%%%%%%%%%%%%%%%%%%%%%%%%%%%%%%%%%%%%%%%%%%%%%%%%%%%%%%

\begin{problem}{%problem statement
	}{p14% problem reference text
	}
	For $T\in \sL(\mcH_A\tensor \mcH_B)$ specified through $T=\sum\limits_{i,j}\gm_{i,j}A_i\tensor B_j$, where $A_i\in \sL(\mcH_A)$, $B_i\in \sL(\mcH_B)$, $\gm_{i,j}\in \bbC$ we define $$f(T)=\sum_{i,j}\gm_{i,j}tr(B_j)A_i$$Let $g:\sL(\mcH_A\tensor \mcH_B) \to \sL(\mcH_A)$ be defined as a map satisfying $$\bra{u}g(Z)\ket{v}=\sum_k\bra{u\tensor e_k}Z\ket{v\tensor e_k}$$ for any ONB $\{\ket{e_k}\mid 1\leq k\leq K\}\subseteq \mcH_B$. \begin{enumerate}
		\item Prove that $f(T)=g(T)$ and both are well defined. \parinn
		\item Prove if $\{M_r\mid 1\leq r\leq R\}\subseteq \sL(\mcH_A)$ is a measurement then $$tr(M_rg(Z)M^{\dagger}_r)=tr[(M_r\tensor I_B)Z(M^{\dagger}_r\tensor I_B)]$$
	\end{enumerate}
	%Problem		
\end{problem}

%%%%%%%%%%%%%%%%%%%%%%%%%%%%%%%%%%%%%%%%%%%%%%%%%%%%%%%%%%%%%%%%%%%%%%%%%%%%%%%%%%%%%%%%%%%%%%%%%%%%%%%%%%%%%%%%%%%%%%%%%%%%%%%%%%%%%%%%
% Problem 15
%%%%%%%%%%%%%%%%%%%%%%%%%%%%%%%%%%%%%%%%%%%%%%%%%%%%%%%%%%%%%%%%%%%%%%%%%%%%%%%%%%%%%%%%%%%%%%%%%%%%%%%%%%%%%%%%%%%%%%%%%%%%%%%%%%%%%%%%

\begin{problem}{%problem statement
		Mark Wilde: Exercise 4.1.3
	}{p15% problem reference text
	}
	 Prove the following equality:$$tr(A)=\bra{\Gm}_{RS}I_R\tensor A_S\ket{\Gm}_S$$where $A$ is a square operator acting on a Hilbert space $\mcH_S$, $I_R$  is the operator acting on a Hilbert space $\mcH_R$ isormorphic to $\mcH_S$ and $\ket{\Gm}_{RS}$ is the unnormalized  maximally entangled  vector $$\ket{\Gm}_{RS}=\sum_{i=0}^{d-1}\ket{i}_R\ket{i}_S$$
	%Problem		
\end{problem}
\solve{
\begin{align*}
	\bra{\Gm}_{RS}I_R\tensor A_S\ket{\Gm}_S & = \lt[\sum_{i=0}^{d-1}\bra{i}_R\bra{i}_S   \rt]I_R\tensor A_S\lt[ \sum_{j=0}^{d-1}\ket{j}_R\ket{j}_S \rt]\\ 
	& = \sum_{i=0}^{d-1}\sum_{j=0}^{d-1}\bra{i}_R\ket{j}_R \bra{i}_SA_S\ket{j}_S\\
	& = \sum_{i=0}^{d-1}\bra{i}_SA_S\ket{i}_S= tr(A)
\end{align*}
}

%%%%%%%%%%%%%%%%%%%%%%%%%%%%%%%%%%%%%%%%%%%%%%%%%%%%%%%%%%%%%%%%%%%%%%%%%%%%%%%%%%%%%%%%%%%%%%%%%%%%%%%%%%%%%%%%%%%%%%%%%%%%%%%%%%%%%%%%
% Problem 16
%%%%%%%%%%%%%%%%%%%%%%%%%%%%%%%%%%%%%%%%%%%%%%%%%%%%%%%%%%%%%%%%%%%%%%%%%%%%%%%%%%%%%%%%%%%%%%%%%%%%%%%%%%%%%%%%%%%%%%%%%%%%%%%%%%%%%%%%

\begin{problem}{%problem statement
		Mark Wilde: Exercise 3.7.12
	}{p16% problem reference text
	}
	Show that the following ``transpose trick” or ``ricochet” property holds for a maximally entangled state $\ket{\Phi}_{AB}$ defined as $$\ket{\Phi}_{AB}=\frac1{\sqrt{d}}\sum_{i=0}^{d-1}\ket{i}_A\ket{i}_B$$ and any $d \times d$ matrix $M$ $$(M_A\tensor I_B)\ket{\Phi}_{AB}=(I_A\tensor M_B^t)\ket{\Phi}_{AB}$$where $M^t$ is the transpose of the operator $M$ with respect to the basis $\{\ket{i}_B\}$ from  the definition of $\ket{\Phi}_{AB}$. The implication is that some local action of Alice on $\ket{\Phi}_{AB}$ is equivalent to Bob performing the transpose of this action on his share of the state.
	Of course, the same equality is true for the unnormalized maximally entangled
	vector  $\ket{\Gm}_{AB}$ of the from \hyperref[p:p15]{Problem \ref{p:p15}}$$(M_A\tensor I_B)\ket{\Gm}_{AB}=(I_A\tensor M_B^t)\ket{\Gm}_{AB}$$
	%Problem		
\end{problem}
\solve{$M_{i,j}$ denotes the $(i,j)$th entry of $M$ and $M^t_{i,j}$ denotes the $(i,j)$th entry of $M^t$ which is equal to $M_{j,i}$. Then we can say $$M\ket{i}=\sum_{j=0}^{d-1}M_{i,j}\ket{j}$$\begin{align*}
		(M_A\tensor I_B)\ket{\Phi}_{AB}&=\frac{1}{\sqrt{d}}\sum_{i=0}^{d-1}M_A\ket{i}_A \tensor \ket{i}_B\\
		& = \frac{1}{\sqrt{d}}\sum_{i=0}^{d-1}\sum_{j=0}^{d-1}M_{{i,j}}\ket{j}_A\tensor \ket{i}_B\\
		& = \frac{1}{\sqrt{d}}\sum_{i=0}^{d-1}\sum_{j=0}^{d-1}M_{{j,i}}^t\ket{j}_A\tensor \ket{i}_B\\
		& = \frac{1}{\sqrt{d}}\sum_{i=0}^{d-1}\sum_{j=0}^{d-1}\ket{j}_A\tensor M_{{j,i}}^t\ket{i}_B\\
		& = \frac{1}{\sqrt{d}}\sum_{j=0}^{d-1}\ket{j}_A\tensor \sum_{i=0}^{d-1}M_{{j,i}}^t\ket{i}_B\\		
		& = \frac1{\sqrt{d}}\sum_{j=0}^{d-1}\ket{j}_A\tensor M^t_B\ket{j}_B\\
		& = (I_A\tensor M^t_B)\frac1{\sqrt{d}}\sum_{j=0}^{d-1}\ket{j}_A\tensor \ket{j}_B\\
		& = (I_A\tensor M^t_B)\ket{\Phi}_{AB}
	\end{align*}

}

%%%%%%%%%%%%%%%%%%%%%%%%%%%%%%%%%%%%%%%%%%%%%%%%%%%%%%%%%%%%%%%%%%%%%%%%%%%%%%%%%%%%%%%%%%%%%%%%%%%%%%%%%%%%%%%%%%%%%%%%%%%%%%%%%%%%%%%%
% Problem 17
%%%%%%%%%%%%%%%%%%%%%%%%%%%%%%%%%%%%%%%%%%%%%%%%%%%%%%%%%%%%%%%%%%%%%%%%%%%%%%%%%%%%%%%%%%%%%%%%%%%%%%%%%%%%%%%%%%%%%%%%%%%%%%%%%%%%%%%%

\begin{problem}{%problem statement
	}{p17% problem reference text
	}
	For each of the maps \begin{enumerate}[label=(\roman*)]
		\item State whether the map is a quantum evolution i.e. a CPTP linear map
		\item If the answer to (i) is `Yes' then identify the Choi-Krous Operators specifying the map
		\item Identify the components - reference Hilbert Space, isometry - that specify the map through Shine spring citation i.e. if the map $\sE:\sL(\mcH)\to \sL(\mcH_B)$ is a CPTP map. Identify a Hilbert space $\mcH_R$ and an isometry $V:\mcH_A\to \mcH_R\tensor \mcH_A\tensor \mcH_B$ so that $\sE(S)=tr_{RA}(VSV^{\dagger})$
	\end{enumerate}
		\begin{enumerate}[label=(\arabic*)]
			\item $\sE:\bbC\to \sD(\mcH)$, where $1\mapsto \rho$
			\item $\sE:\sL(\mcH_A)\to \sL(\mcH_A\tensor \mcH_B)$, where $S\mapsto S\tensor \rho_B$ where $\rho_B\in \sD(\mcH_B)$ is a fixed density operator.
			\item $\sE:\sL(\mcH_A)\to \sL(\mcH_A\tensor \mcH_B)$ where $S\mapsto tr_B(S)$
			\item $\sE:\sL(\mcH_A)\to \sL(\mcH_A\tensor \mcH_B)$ where $S\mapsto VSV^{\dagger}$ where $V:\mcH_A\to\mcH_A\tensor \mcH_B$ is an isometry
			\item Suppose $\mcH_x=span\{\ket{x}\mid x\in \mcX\}$ where $\mcX$ is a finite set such that $$\bra{\tilde{x}}ket{x}=\delta_{\tilde{x},x}$$and similarly $\mcH_y = span\{\ket{y}\mid y\in \mcY\}$ where $\mcY$ is a finite set and $$\bra{\tilde{y}}\ket{y}=\delta_{\tilde{y},y}$$Suppose $p_{Y|X}(y|x)$, $(x,y)\in \mcX\times \mcY$ is a stochastic matrix i.e. $\sum\limits_{y\in \mcY}p_{Y|X}(y|x)=1$ for all $x\in \mcX$ and $p_{Y|X}(y|x)\geq 0$.
				
			Then consider the maps \begin{center}
				\begin{tikzcd}
					\sum\limits_{x\in \mcX}p_X(x)\ket{x}\bra{x}\arrow[r] & \sum\limits_{x,y}p_X(x)p_{Y|X}(y|x)\ket{xy}\bra{xy}\\
					\sum\limits_{x\in \mcX}p_X(x)\ket{x}\bra{x}\arrow[r] & \sum\limits_{y\in \mcY}p_Y(y)\ket{y}\bra{y}
				\end{tikzcd}
			\end{center}where $p_Y(y)=\sum\limits_{x\in \mcX}p_X(x)p_{Y|X}(y|x)$.
			\item Suppose $\{M_k\mid 1\leq k\leq K\}$ form a measurement, then $$S\mapsto \sum_{k=1}^KM_kSM_k^{\dagger}\tensor \ket{k}\bra{k}$$
		\end{enumerate}

	%Problem		
\end{problem}

%%%%%%%%%%%%%%%%%%%%%%%%%%%%%%%%%%%%%%%%%%%%%%%%%%%%%%%%%%%%%%%%%%%%%%%%%%%%%%%%%%%%%%%%%%%%%%%%%%%%%%%%%%%%%%%%%%%%%%%%%%%%%%%%%%%%%%%%
% Problem 18
%%%%%%%%%%%%%%%%%%%%%%%%%%%%%%%%%%%%%%%%%%%%%%%%%%%%%%%%%%%%%%%%%%%%%%%%%%%%%%%%%%%%%%%%%%%%%%%%%%%%%%%%%%%%%%%%%%%%%%%%%%%%%%%%%%%%%%%%

\begin{problem}{%problem statement
	}{p18% problem reference text
	}
	If $A\in \sL(\mcH_1)$, $B\in \sL(\mcH_2)$ and $A\geq 0$, $B\geq 0$ prove or disprove (with counter example) $A\tensor B\geq 0$.
	
	[Recall that for any operator $S\in \sL(\mcH)$ we say $S\geq 0$ if $\bra{u}S\ket{S}\geq 0$ $\forall\ \ket{u}\in \mcH$]
	%Problem		
\end{problem}

%%%%%%%%%%%%%%%%%%%%%%%%%%%%%%%%%%%%%%%%%%%%%%%%%%%%%%%%%%%%%%%%%%%%%%%%%%%%%%%%%%%%%%%%%%%%%%%%%%%%%%%%%%%%%%%%%%%%%%%%%%%%%%%%%%%%%%%%
% Problem 19
%%%%%%%%%%%%%%%%%%%%%%%%%%%%%%%%%%%%%%%%%%%%%%%%%%%%%%%%%%%%%%%%%%%%%%%%%%%%%%%%%%%%%%%%%%%%%%%%%%%%%%%%%%%%%%%%%%%%%%%%%%%%%%%%%%%%%%%%

\begin{problem}{%problem statement
	}{p19% problem reference text
	}
	Suppose $\mcH_A$, $\mcH_B$ are Hilbert spaces. State the definition of when a map $\sE:\sL(\mcH_A)\to \sL(\mcH_B)$ is a completely positive map. Let us say a map $\sF:\sL(\mcH_A)\to \sL(\mcH_B)$ is positive if whenever $X\in \sP(\mcH_A)$ is positive semidefinite, we have $\sF(X)\in \sP(\mcH_B)$ is positive definite. Provide an example of $\mcH_A$, $\mcH_B$ and a map $\sF:\sL(\mcH_A)\to \sL(\mcH_B)$ is positive by not completely positive.\parinn
	
	Identify $\rho\in \sD(\mcH_A)$, a density operator, so that $(i_R\tensor \sF)(\ket{\phi_{\rho}}\bra{\phi_{\rho}})$ is not positive semi-definite, wherein $\ket{\phi_{\rho}}\in \mcH_R\tensor \mcH_A$ is a purification of $\rho$.
	%Problem		
\end{problem}

\end{document}
