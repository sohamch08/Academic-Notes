\documentclass{article}
\usepackage{fullpage}
\usepackage{amsmath}
\usepackage{amsfonts}
\usepackage{authblk}
\usepackage{titling}

\title{\huge{Analysis Assignment 2\\ \hspace{5cm}- Rajeeva L. Karandikar}}
\author{Soham Chatterjee\\Roll: BMC202175}
\date{}


\newcommand{\mN}{\mathbb{N}}
\newcommand{\mR}{\mathbb{R}}
\newcommand{\mQ}{\mathbb{Q}}
\renewcommand\maketitlehooka{\null\mbox{}\vfill}
\renewcommand\maketitlehookd{\vfill\null}

\setlength{\parindent}{1cm}

\begin{document}
	\maketitle\pagebreak
	\begin{enumerate}
		\item Let the set $F$ is unbounded. Then for any  $n\in\mN$ $\exists\  x_n\in [a,b]$ such that $|f(x_n)|>n$. Now as $\forall\ n\in\mN$ $x_n\in[a,b]$, hence the sequence $\{x_n\}$ is bounded.
		
		\hspace{1cm}Now in Analysis Assignment 1, Question No. 1.(iii) we proved that  for a bounded sequence $\{a_n\}$ $\exists\ \{n_j\mid j\geq 1\}$ where $n_j<n_{j+1}$ and $n_j\in\mN$ such that $$\lim\limits_{n\to\infty}a_{n_j}=\lim\limits_{n\to\infty}\sup a_n$$Hence there exists a sequence $\{x_{n_k}\}$ where $n_k,k\in\mN$ and $n_k<n_{k+1}$ such that $\lim\limits_{k\to \infty}x_{n_k}=\lim\limits_{n\to\infty}\sup x_n$. Let $\alpha=\lim\limits_{n\to\infty}\sup x_n$.
		
		\hspace{1cm}Since $f(x_n)>n$ hence $f(x_{n_k})>n_k\geq k$ and the sequence $\{n\}$ diverges. Therefore the sequence $\{f(x_{n_k})\}$ also diverges. But as the sequence $\{x_{n_k}\}$ converges to $\alpha$ and $f$ is continuous in $[a,b]$, $\{f(x_{n_k})\}$ should converge to $f(\alpha)$. Contradiction. Therefore the set $F$ is bounded. [Proved]
		\item Since we already proved that $f$ is closed and bounded, suppose $M$ is the least upper bound of the set $F$ where $F=\{f(x)\mid x\in[a,b]\}$. Now  we construct a sequence $\{x_n\}$, where $x_n\in[a,b]\ \forall\ n\in\mN$ in such that $$|M-f(x_n)|<\frac1n$$Now such $x_n$ will always exist because if no such $x_n$ exists then $\forall x\in[a,b]$ $f(x)<M-\frac1n$ and then $M-\frac1n $ would be the upper bound less than least upper bound which is not possible. Hence $\{f(x_n)\}$ converges to $M$. Therefore $x_n$ is also convergent. Let's say $\{x_n\}$ converges to $\alpha$. As $f$ is continuous $f$ should converge to $f(\alpha)$. Now $\{x_n\} $ converges to $\alpha$ $\{f(x_n)\}$ converges to $M$ and $f(\alpha)$. As the limit should be unique hence $f(\alpha)=M$.
		
		\hspace{1cm}Similarly suppose $m$ is the greatest lower bound and we construct a sequence $\{y_n\}$ such that $$|m-f(y_n)|<\frac1n$$. Hence $f(y_n)$ converges to $m$. Suppose $\{y_n\}$ converges to $\beta$. As $f$ is continuous $f(y_n)$ should converge to $f(\beta)$.  Therefore $f(\beta)=m$. Hence $f$ attains its extremes.
		
		\hspace{1cm}Now $f$ is a continuous function from closed bounded interval $[a,b]$ to a closed bounded interval $[m,M]$. Hence $f$ is uniformly continuous (Source: Lecture Notes of 19.10.2021). Now whenever $g$ is composed upon $f$ its domain becomes the range of $f$ i.e. $[m,M]$. Therefore $g:[m,M]\to\mR$ is a continuous function on a closed bounded interval. Therefore $g$ is uniformly continuous.
		
		\hspace{1cm}As $f$ is uniformly continuous $\forall\ \epsilon_f>0$ $\exists \ \delta_f>0$ such that $$|f(x)-f(y)|<\epsilon_f\text{ whenever }|x-y|<\delta_f$$where $x,y\in[a,b]$. As $g$ is uniformly continuous $\forall\ \epsilon_g>0$ $\exists \ \delta_g>0$ such that $$|g(x)-g(y)|<\epsilon_g\text{ whenever }|x-y|<\delta_g$$where $x,y\in[m,M]$. Now if we take $\epsilon_f\leq \delta_g$ then $\forall\ \epsilon_g>0$ $\exists \ \delta_f>0$ such that $$|(g\circ f)(x)-(g\circ f)(y)|<\epsilon_g\text{ whenever }|x-y|<\delta_f$$where $x,y\in[a,b]$. Hence $(g\circ f)$ is also uniformly continuous. [Proved]
		\item\begin{enumerate}
			\item As $f$ is continuous in $(a,b)$ and $g(x)=f(x)$ in $(a,b)$, $g$ is also continuous in $(a,b)$. Hence only when $G$ can be discontinuous is at $x=a$, $x=b$.
			
			\hspace{1cm}Given that $$g(a)=\alpha=\lim\limits_{x\to\ a}f(x)=\lim\limits_{x\to\ a}g(x)$$Hence $g$ is continuous at $x=a$. Again  $$g(b)=\beta=\lim\limits_{x\to\ b}f(x)=\lim\limits_{x\to\ b}g(x)$$Therefore $g$ is also continuous at $x=b$. Hence $g$ is continuous on $[a,b]$. [Proved]
			\item Now $g$ is continuous in the closed interval $[a,b]$. Using the result in Problem 1 we can say $g$ is uniformly continuous in $[a,b]$. Therefore $g$ is uniformly continuous in $(a,b)$. Now as $g(x)=f(x)$ in $(a,b)$, $f$ is also uniformly continuous in $(a,b)$. [Proved]
		\end{enumerate}
	\end{enumerate}
\end{document}