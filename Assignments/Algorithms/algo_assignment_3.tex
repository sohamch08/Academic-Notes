\documentclass[a4paper, 11pt]{article}
\usepackage{comment} % enables the use of multi-line comments (\ifx \fi) 
\usepackage{fullpage} % changes the margin
\usepackage[a4paper, total={7in, 10in}]{geometry}
\usepackage{amsmath,mathtools,mathdots}
\usepackage{amssymb,amsthm}  % assumes amsmath package installed
\usepackage{xcolor}
\usepackage{mdframed}
\usepackage[shortlabels]{enumitem}
\usepackage{indentfirst}
\usepackage{hyperref}
\hypersetup{
	colorlinks=true,
	linkcolor=doc!80,
	citecolor=myr,
	filecolor=myr,      
	urlcolor=doc!80,
	pdftitle={Assignment}, %%%%%%%%%%%%%%%%   WRITE ASSIGNMENT PDF NAME  %%%%%%%%%%%%%%%%%%%%
}
\usepackage[most,many,breakable]{tcolorbox}
\usepackage{tikz}
\usepackage{caption}
\usepackage{kpfonts}
\usepackage{libertine}
\usepackage{physics}
\usepackage[ruled,vlined,linesnumbered]{algorithm2e}
\usepackage{mathrsfs}
\usepackage{tikz-cd}
\usepackage{float}
\usepackage[normalem]{ulem}
\definecolor{mytheorembg}{HTML}{F2F2F9}
\definecolor{mytheoremfr}{HTML}{00007B}
\definecolor{doc}{RGB}{0,60,110}
\definecolor{myg}{RGB}{56, 140, 70}
\definecolor{myb}{RGB}{45, 111, 177}
\definecolor{myr}{RGB}{199, 68, 64}

\usetikzlibrary{decorations.pathreplacing,angles,quotes,patterns}
\definecolor{mytheorembg}{HTML}{F2F2F9}
\definecolor{mytheoremfr}{HTML}{00007B}
\definecolor{doc}{RGB}{0,60,110}
\definecolor{myg}{RGB}{56, 140, 70}
\definecolor{myb}{RGB}{45, 111, 177}
\definecolor{myr}{RGB}{199, 68, 64}
\newcounter{problem}
\tcbuselibrary{theorems,skins,hooks}
\newtcbtheorem[use counter=problem]{problem}{Problem}
{%
	enhanced,
	breakable,
	colback = mytheorembg,
	frame hidden,
	boxrule = 0sp,
	borderline west = {2pt}{0pt}{mytheoremfr},
	arc=5pt,
	detach title,
	before upper = \tcbtitle\par\smallskip,
	coltitle = mytheoremfr,
	fonttitle = \bfseries\sffamily,
	description font = \mdseries,
	separator sign none,
	segmentation style={solid, mytheoremfr},
}
{p}

\newtheorem{lemma}{Lemma}
\renewenvironment{proof}{\noindent{\it \textbf{Proof:}}\hspace*{1em}}{\qed\bigskip\\}
% To give references for any problem use like this
% suppose the problem number is p3 then 2 options either 
% \hyperref[p:p3]{<text you want to use to hyperlink> \ref{p:p3}}
%                  or directly 
%                   \ref{p:p3}


\newcommand*\circled[1]{\tikz[baseline=(char.base)]{
		\node[shape=circle,draw,inner sep=1pt] (char) {#1};}}
\newcommand\getcurrentref[1]{%
	\ifnumequal{\value{#1}}{0}
	{??}
	{\the\value{#1}}%
}
%---------------------------------------
% BlackBoard Math Fonts :-
%---------------------------------------

%Captital Letters
\newcommand{\bbA}{\mathbb{A}}	\newcommand{\bbB}{\mathbb{B}}
\newcommand{\bbC}{\mathbb{C}}	\newcommand{\bbD}{\mathbb{D}}
\newcommand{\bbE}{\mathbb{E}}	\newcommand{\bbF}{\mathbb{F}}
\newcommand{\bbG}{\mathbb{G}}	\newcommand{\bbH}{\mathbb{H}}
\newcommand{\bbI}{\mathbb{I}}	\newcommand{\bbJ}{\mathbb{J}}
\newcommand{\bbK}{\mathbb{K}}	\newcommand{\bbL}{\mathbb{L}}
\newcommand{\bbM}{\mathbb{M}}	\newcommand{\bbN}{\mathbb{N}}
\newcommand{\bbO}{\mathbb{O}}	\newcommand{\bbP}{\mathbb{P}}
\newcommand{\bbQ}{\mathbb{Q}}	\newcommand{\bbR}{\mathbb{R}}
\newcommand{\bbS}{\mathbb{S}}	\newcommand{\bbT}{\mathbb{T}}
\newcommand{\bbU}{\mathbb{U}}	\newcommand{\bbV}{\mathbb{V}}
\newcommand{\bbW}{\mathbb{W}}	\newcommand{\bbX}{\mathbb{X}}
\newcommand{\bbY}{\mathbb{Y}}	\newcommand{\bbZ}{\mathbb{Z}}

%---------------------------------------
% MathCal Fonts :-
%---------------------------------------

%Captital Letters
\newcommand{\mcA}{\mathcal{A}}	\newcommand{\mcB}{\mathcal{B}}
\newcommand{\mcC}{\mathcal{C}}	\newcommand{\mcD}{\mathcal{D}}
\newcommand{\mcE}{\mathcal{E}}	\newcommand{\mcF}{\mathcal{F}}
\newcommand{\mcG}{\mathcal{G}}	\newcommand{\mcH}{\mathcal{H}}
\newcommand{\mcI}{\mathcal{I}}	\newcommand{\mcJ}{\mathcal{J}}
\newcommand{\mcK}{\mathcal{K}}	\newcommand{\mcL}{\mathcal{L}}
\newcommand{\mcM}{\mathcal{M}}	\newcommand{\mcN}{\mathcal{N}}
\newcommand{\mcO}{\mathcal{O}}	\newcommand{\mcP}{\mathcal{P}}
\newcommand{\mcQ}{\mathcal{Q}}	\newcommand{\mcR}{\mathcal{R}}
\newcommand{\mcS}{\mathcal{S}}	\newcommand{\mcT}{\mathcal{T}}
\newcommand{\mcU}{\mathcal{U}}	\newcommand{\mcV}{\mathcal{V}}
\newcommand{\mcW}{\mathcal{W}}	\newcommand{\mcX}{\mathcal{X}}
\newcommand{\mcY}{\mathcal{Y}}	\newcommand{\mcZ}{\mathcal{Z}}



%---------------------------------------
% Bold Math Fonts :-
%---------------------------------------

%Captital Letters
\newcommand{\bmA}{\boldsymbol{A}}	\newcommand{\bmB}{\boldsymbol{B}}
\newcommand{\bmC}{\boldsymbol{C}}	\newcommand{\bmD}{\boldsymbol{D}}
\newcommand{\bmE}{\boldsymbol{E}}	\newcommand{\bmF}{\boldsymbol{F}}
\newcommand{\bmG}{\boldsymbol{G}}	\newcommand{\bmH}{\boldsymbol{H}}
\newcommand{\bmI}{\boldsymbol{I}}	\newcommand{\bmJ}{\boldsymbol{J}}
\newcommand{\bmK}{\boldsymbol{K}}	\newcommand{\bmL}{\boldsymbol{L}}
\newcommand{\bmM}{\boldsymbol{M}}	\newcommand{\bmN}{\boldsymbol{N}}
\newcommand{\bmO}{\boldsymbol{O}}	\newcommand{\bmP}{\boldsymbol{P}}
\newcommand{\bmQ}{\boldsymbol{Q}}	\newcommand{\bmR}{\boldsymbol{R}}
\newcommand{\bmS}{\boldsymbol{S}}	\newcommand{\bmT}{\boldsymbol{T}}
\newcommand{\bmU}{\boldsymbol{U}}	\newcommand{\bmV}{\boldsymbol{V}}
\newcommand{\bmW}{\boldsymbol{W}}	\newcommand{\bmX}{\boldsymbol{X}}
\newcommand{\bmY}{\boldsymbol{Y}}	\newcommand{\bmZ}{\boldsymbol{Z}}
%Small Letters
\newcommand{\bma}{\boldsymbol{a}}	\newcommand{\bmb}{\boldsymbol{b}}
\newcommand{\bmc}{\boldsymbol{c}}	\newcommand{\bmd}{\boldsymbol{d}}
\newcommand{\bme}{\boldsymbol{e}}	\newcommand{\bmf}{\boldsymbol{f}}
\newcommand{\bmg}{\boldsymbol{g}}	\newcommand{\bmh}{\boldsymbol{h}}
\newcommand{\bmi}{\boldsymbol{i}}	\newcommand{\bmj}{\boldsymbol{j}}
\newcommand{\bmk}{\boldsymbol{k}}	\newcommand{\bml}{\boldsymbol{l}}
\newcommand{\bmm}{\boldsymbol{m}}	\newcommand{\bmn}{\boldsymbol{n}}
\newcommand{\bmo}{\boldsymbol{o}}	\newcommand{\bmp}{\boldsymbol{p}}
\newcommand{\bmq}{\boldsymbol{q}}	\newcommand{\bmr}{\boldsymbol{r}}
\newcommand{\bms}{\boldsymbol{s}}	\newcommand{\bmt}{\boldsymbol{t}}
\newcommand{\bmu}{\boldsymbol{u}}	\newcommand{\bmv}{\boldsymbol{v}}
\newcommand{\bmw}{\boldsymbol{w}}	\newcommand{\bmx}{\boldsymbol{x}}
\newcommand{\bmy}{\boldsymbol{y}}	\newcommand{\bmz}{\boldsymbol{z}}


%---------------------------------------
% Scr Math Fonts :-
%---------------------------------------

\newcommand{\sA}{{\mathscr{A}}}   \newcommand{\sB}{{\mathscr{B}}}
\newcommand{\sC}{{\mathscr{C}}}   \newcommand{\sD}{{\mathscr{D}}}
\newcommand{\sE}{{\mathscr{E}}}   \newcommand{\sF}{{\mathscr{F}}}
\newcommand{\sG}{{\mathscr{G}}}   \newcommand{\sH}{{\mathscr{H}}}
\newcommand{\sI}{{\mathscr{I}}}   \newcommand{\sJ}{{\mathscr{J}}}
\newcommand{\sK}{{\mathscr{K}}}   \newcommand{\sL}{{\mathscr{L}}}
\newcommand{\sM}{{\mathscr{M}}}   \newcommand{\sN}{{\mathscr{N}}}
\newcommand{\sO}{{\mathscr{O}}}   \newcommand{\sP}{{\mathscr{P}}}
\newcommand{\sQ}{{\mathscr{Q}}}   \newcommand{\sR}{{\mathscr{R}}}
\newcommand{\sS}{{\mathscr{S}}}   \newcommand{\sT}{{\mathscr{T}}}
\newcommand{\sU}{{\mathscr{U}}}   \newcommand{\sV}{{\mathscr{V}}}
\newcommand{\sW}{{\mathscr{W}}}   \newcommand{\sX}{{\mathscr{X}}}
\newcommand{\sY}{{\mathscr{Y}}}   \newcommand{\sZ}{{\mathscr{Z}}}


%---------------------------------------
% Math Fraktur Font
%---------------------------------------

%Captital Letters
\newcommand{\mfA}{\mathfrak{A}}	\newcommand{\mfB}{\mathfrak{B}}
\newcommand{\mfC}{\mathfrak{C}}	\newcommand{\mfD}{\mathfrak{D}}
\newcommand{\mfE}{\mathfrak{E}}	\newcommand{\mfF}{\mathfrak{F}}
\newcommand{\mfG}{\mathfrak{G}}	\newcommand{\mfH}{\mathfrak{H}}
\newcommand{\mfI}{\mathfrak{I}}	\newcommand{\mfJ}{\mathfrak{J}}
\newcommand{\mfK}{\mathfrak{K}}	\newcommand{\mfL}{\mathfrak{L}}
\newcommand{\mfM}{\mathfrak{M}}	\newcommand{\mfN}{\mathfrak{N}}
\newcommand{\mfO}{\mathfrak{O}}	\newcommand{\mfP}{\mathfrak{P}}
\newcommand{\mfQ}{\mathfrak{Q}}	\newcommand{\mfR}{\mathfrak{R}}
\newcommand{\mfS}{\mathfrak{S}}	\newcommand{\mfT}{\mathfrak{T}}
\newcommand{\mfU}{\mathfrak{U}}	\newcommand{\mfV}{\mathfrak{V}}
\newcommand{\mfW}{\mathfrak{W}}	\newcommand{\mfX}{\mathfrak{X}}
\newcommand{\mfY}{\mathfrak{Y}}	\newcommand{\mfZ}{\mathfrak{Z}}
%Small Letters
\newcommand{\mfa}{\mathfrak{a}}	\newcommand{\mfb}{\mathfrak{b}}
\newcommand{\mfc}{\mathfrak{c}}	\newcommand{\mfd}{\mathfrak{d}}
\newcommand{\mfe}{\mathfrak{e}}	\newcommand{\mff}{\mathfrak{f}}
\newcommand{\mfg}{\mathfrak{g}}	\newcommand{\mfh}{\mathfrak{h}}
\newcommand{\mfi}{\mathfrak{i}}	\newcommand{\mfj}{\mathfrak{j}}
\newcommand{\mfk}{\mathfrak{k}}	\newcommand{\mfl}{\mathfrak{l}}
\newcommand{\mfm}{\mathfrak{m}}	\newcommand{\mfn}{\mathfrak{n}}
\newcommand{\mfo}{\mathfrak{o}}	\newcommand{\mfp}{\mathfrak{p}}
\newcommand{\mfq}{\mathfrak{q}}	\newcommand{\mfr}{\mathfrak{r}}
\newcommand{\mfs}{\mathfrak{s}}	\newcommand{\mft}{\mathfrak{t}}
\newcommand{\mfu}{\mathfrak{u}}	\newcommand{\mfv}{\mathfrak{v}}
\newcommand{\mfw}{\mathfrak{w}}	\newcommand{\mfx}{\mathfrak{x}}
\newcommand{\mfy}{\mathfrak{y}}	\newcommand{\mfz}{\mathfrak{z}}

%---------------------------------------
% Bar
%---------------------------------------

%Captital Letters
\newcommand{\ovA}{\overline{A}}	\newcommand{\ovB}{\overline{B}}
\newcommand{\ovC}{\overline{C}}	\newcommand{\ovD}{\overline{D}}
\newcommand{\ovE}{\overline{E}}	\newcommand{\ovF}{\overline{F}}
\newcommand{\ovG}{\overline{G}}	\newcommand{\ovH}{\overline{H}}
\newcommand{\ovI}{\overline{I}}	\newcommand{\ovJ}{\overline{J}}
\newcommand{\ovK}{\overline{K}}	\newcommand{\ovL}{\overline{L}}
\newcommand{\ovM}{\overline{M}}	\newcommand{\ovN}{\overline{N}}
\newcommand{\ovO}{\overline{O}}	\newcommand{\ovP}{\overline{P}}
\newcommand{\ovQ}{\overline{Q}}	\newcommand{\ovR}{\overline{R}}
\newcommand{\ovS}{\overline{S}}	\newcommand{\ovT}{\overline{T}}
\newcommand{\ovU}{\overline{U}}	\newcommand{\ovV}{\overline{V}}
\newcommand{\ovW}{\overline{W}}	\newcommand{\ovX}{\overline{X}}
\newcommand{\ovY}{\overline{Y}}	\newcommand{\ovZ}{\overline{Z}}
%Small Letters
\newcommand{\ova}{\overline{a}}	\newcommand{\ovb}{\overline{b}}
\newcommand{\ovc}{\overline{c}}	\newcommand{\ovd}{\overline{d}}
\newcommand{\ove}{\overline{e}}	\newcommand{\ovf}{\overline{f}}
\newcommand{\ovg}{\overline{g}}	\newcommand{\ovh}{\overline{h}}
\newcommand{\ovi}{\overline{i}}	\newcommand{\ovj}{\overline{j}}
\newcommand{\ovk}{\overline{k}}	\newcommand{\ovl}{\overline{l}}
\newcommand{\ovm}{\overline{m}}	\newcommand{\ovn}{\overline{n}}
\newcommand{\ovo}{\overline{o}}	\newcommand{\ovp}{\overline{p}}
\newcommand{\ovq}{\overline{q}}	\newcommand{\ovr}{\overline{r}}
\newcommand{\ovs}{\overline{s}}	\newcommand{\ovt}{\overline{t}}
\newcommand{\ovu}{\overline{u}}	\newcommand{\ovv}{\overline{v}}
\newcommand{\ovw}{\overline{w}}	\newcommand{\ovx}{\overline{x}}
\newcommand{\ovy}{\overline{y}}	\newcommand{\ovz}{\overline{z}}

%---------------------------------------
% Tilde
%---------------------------------------

%Captital Letters
\newcommand{\tdA}{\tilde{A}}	\newcommand{\tdB}{\tilde{B}}
\newcommand{\tdC}{\tilde{C}}	\newcommand{\tdD}{\tilde{D}}
\newcommand{\tdE}{\tilde{E}}	\newcommand{\tdF}{\tilde{F}}
\newcommand{\tdG}{\tilde{G}}	\newcommand{\tdH}{\tilde{H}}
\newcommand{\tdI}{\tilde{I}}	\newcommand{\tdJ}{\tilde{J}}
\newcommand{\tdK}{\tilde{K}}	\newcommand{\tdL}{\tilde{L}}
\newcommand{\tdM}{\tilde{M}}	\newcommand{\tdN}{\tilde{N}}
\newcommand{\tdO}{\tilde{O}}	\newcommand{\tdP}{\tilde{P}}
\newcommand{\tdQ}{\tilde{Q}}	\newcommand{\tdR}{\tilde{R}}
\newcommand{\tdS}{\tilde{S}}	\newcommand{\tdT}{\tilde{T}}
\newcommand{\tdU}{\tilde{U}}	\newcommand{\tdV}{\tilde{V}}
\newcommand{\tdW}{\tilde{W}}	\newcommand{\tdX}{\tilde{X}}
\newcommand{\tdY}{\tilde{Y}}	\newcommand{\tdZ}{\tilde{Z}}
%Small Letters
\newcommand{\tda}{\tilde{a}}	\newcommand{\tdb}{\tilde{b}}
\newcommand{\tdc}{\tilde{c}}	\newcommand{\tdd}{\tilde{d}}
\newcommand{\tde}{\tilde{e}}	\newcommand{\tdf}{\tilde{f}}
\newcommand{\tdg}{\tilde{g}}	\newcommand{\tdh}{\tilde{h}}
\newcommand{\tdi}{\tilde{i}}	\newcommand{\tdj}{\tilde{j}}
\newcommand{\tdk}{\tilde{k}}	\newcommand{\tdl}{\tilde{l}}
\newcommand{\tdm}{\tilde{m}}	\newcommand{\tdn}{\tilde{n}}
\newcommand{\tdo}{\tilde{o}}	\newcommand{\tdp}{\tilde{p}}
\newcommand{\tdq}{\tilde{q}}	\newcommand{\tdr}{\tilde{r}}
\newcommand{\tds}{\tilde{s}}	\newcommand{\tdt}{\tilde{t}}
\newcommand{\tdu}{\tilde{u}}	\newcommand{\tdv}{\tilde{v}}
\newcommand{\tdw}{\tilde{w}}	\newcommand{\tdx}{\tilde{x}}
\newcommand{\tdy}{\tilde{y}}	\newcommand{\tdz}{\tilde{z}}

%---------------------------------------
% Vec
%---------------------------------------

%Captital Letters
\newcommand{\vcA}{\vec{A}}	\newcommand{\vcB}{\vec{B}}
\newcommand{\vcC}{\vec{C}}	\newcommand{\vcD}{\vec{D}}
\newcommand{\vcE}{\vec{E}}	\newcommand{\vcF}{\vec{F}}
\newcommand{\vcG}{\vec{G}}	\newcommand{\vcH}{\vec{H}}
\newcommand{\vcI}{\vec{I}}	\newcommand{\vcJ}{\vec{J}}
\newcommand{\vcK}{\vec{K}}	\newcommand{\vcL}{\vec{L}}
\newcommand{\vcM}{\vec{M}}	\newcommand{\vcN}{\vec{N}}
\newcommand{\vcO}{\vec{O}}	\newcommand{\vcP}{\vec{P}}
\newcommand{\vcQ}{\vec{Q}}	\newcommand{\vcR}{\vec{R}}
\newcommand{\vcS}{\vec{S}}	\newcommand{\vcT}{\vec{T}}
\newcommand{\vcU}{\vec{U}}	\newcommand{\vcV}{\vec{V}}
\newcommand{\vcW}{\vec{W}}	\newcommand{\vcX}{\vec{X}}
\newcommand{\vcY}{\vec{Y}}	\newcommand{\vcZ}{\vec{Z}}
%Small Letters
\newcommand{\vca}{\vec{a}}	\newcommand{\vcb}{\vec{b}}
\newcommand{\vcc}{\vec{c}}	\newcommand{\vcd}{\vec{d}}
\newcommand{\vce}{\vec{e}}	\newcommand{\vcf}{\vec{f}}
\newcommand{\vcg}{\vec{g}}	\newcommand{\vch}{\vec{h}}
\newcommand{\vci}{\vec{i}}	\newcommand{\vcj}{\vec{j}}
\newcommand{\vck}{\vec{k}}	\newcommand{\vcl}{\vec{l}}
\newcommand{\vcm}{\vec{m}}	\newcommand{\vcn}{\vec{n}}
\newcommand{\vco}{\vec{o}}	\newcommand{\vcp}{\vec{p}}
\newcommand{\vcq}{\vec{q}}	\newcommand{\vcr}{\vec{r}}
\newcommand{\vcs}{\vec{s}}	\newcommand{\vct}{\vec{t}}
\newcommand{\vcu}{\vec{u}}	\newcommand{\vcv}{\vec{v}}
%\newcommand{\vcw}{\vec{w}}	\newcommand{\vcx}{\vec{x}}
\newcommand{\vcy}{\vec{y}}	\newcommand{\vcz}{\vec{z}}

%---------------------------------------
% Greek Letters:-
%---------------------------------------
\newcommand{\eps}{\epsilon}
\newcommand{\veps}{\varepsilon}
\newcommand{\lm}{\lambda}
\newcommand{\Lm}{\Lambda}
\newcommand{\gm}{\gamma}
\newcommand{\Gm}{\Gamma}
\newcommand{\vph}{\varphi}
\newcommand{\ph}{\phi}
\newcommand{\om}{\omega}
\newcommand{\Om}{\Omega}
\newcommand{\sg}{\sigma}
\newcommand{\Sg}{\Sigma}

\newcommand{\Qed}{\begin{flushright}\qed\end{flushright}}
\newcommand{\parinn}{\setlength{\parindent}{1cm}}
\newcommand{\parinf}{\setlength{\parindent}{0cm}}
\newcommand{\del}[2]{\frac{\partial #1}{\partial #2}}
\newcommand{\Del}[3]{\frac{\partial^{#1} #2}{\partial^{#1} #3}}
\newcommand{\deld}[2]{\dfrac{\partial #1}{\partial #2}}
\newcommand{\Deld}[3]{\dfrac{\partial^{#1} #2}{\partial^{#1} #3}}
\newcommand{\uin}{\mathbin{\rotatebox[origin=c]{90}{$\in$}}}
\newcommand{\usubset}{\mathbin{\rotatebox[origin=c]{90}{$\subset$}}}
\newcommand{\lt}{\left}
\newcommand{\rt}{\right}
\newcommand{\exs}{\exists}
\newcommand{\st}{\strut}
\newcommand{\dps}[1]{\displaystyle{#1}}
\newcommand{\la}{\langle}
\newcommand{\ra}{\rangle}
\newcommand{\cls}[1]{\textsc{#1}}
\newcommand{\prb}[1]{\textsc{#1}}
\newcommand{\comb}[2]{\left(\begin{matrix}
		#1\\ #2
\end{matrix}\right)}
%\newcommand[2]{\quotient}{\faktor{#1}{#2}}
\newcommand\quotient[2]{
	\mathchoice
	{% \displaystyle
		\text{\raise1ex\hbox{$#1$}\Big/\lower1ex\hbox{$#2$}}%
	}
	{% \textstyle
		#1\,/\,#2
	}
	{% \scriptstyle
		#1\,/\,#2
	}
	{% \scriptscriptstyle  
		#1\,/\,#2
	}
}

\newcommand{\tensor}{\otimes}
\newcommand{\xor}{\oplus}

\newcommand{\sol}[1]{\begin{solution}#1\end{solution}}
\newcommand{\solve}[1]{\setlength{\parindent}{0cm}\textbf{\textit{Solution: }}\setlength{\parindent}{1cm}#1 \hfill $\blacksquare$}
\newcommand{\mat}[1]{\left[\begin{matrix}#1\end{matrix}\right]}
\newcommand{\matr}[1]{\begin{matrix}#1\end{matrix}}
\newcommand{\matp}[1]{\lt(\begin{matrix}#1\end{matrix}\rt)}
\newcommand{\detmat}[1]{\lt|\begin{matrix}#1\end{matrix}\rt|}
\newcommand\numberthis{\addtocounter{equation}{1}\tag{\theequation}}
\newcommand{\handout}[3]{
	\noindent
	\begin{center}
		\framebox{
			\vbox{
				\hbox to 6.5in { {\bf Complexity Theory I } \hfill Jan -- May, 2023 }
				\vspace{4mm}
				\hbox to 6.5in { {\Large \hfill #1  \hfill} }
				\vspace{2mm}
				\hbox to 6.5in { {\em #2 \hfill #3} }
			}
		}
	\end{center}
	\vspace*{4mm}
}

\newcommand{\lecture}[3]{\handout{Lecture #1}{Lecturer: #2}{Scribe:	#3}}

\let\marvosymLightning\Lightning
\newcommand{\ctr}{\text{\marvosymLightning}\hspace{0.5ex}} % Requires marvosym package

\newcommand{\ov}[1]{\overline{#1}}
\newcommand{\thmref}[1]{\hyperref[th:#1]{Theorem \ref{th:#1}}}
\newcommand{\propref}[1]{\hyperref[th:#1]{Proposition \ref{th:#1}}}
\newcommand{\lmref}[1]{\hyperref[th:#1]{Lemma \ref{th:#1}}}
\newcommand{\corref}[1]{\hyperref[th:#1]{Corollary \ref{th:#1}}}

\newcommand{\thrmref}[1]{\hyperref[#1]{Theorem \ref{#1}}}
\newcommand{\propnref}[1]{\hyperref[#1]{Proposition \ref{#1}}}
\newcommand{\lemref}[1]{\hyperref[#1]{Lemma \ref{#1}}}
\newcommand{\corrref}[1]{\hyperref[#1]{Corollary \ref{#1}}}

\DeclareMathOperator{\enc}{Enc}
\DeclareMathOperator{\res}{Res}
\DeclareMathOperator{\spec}{Spec}
\DeclareMathOperator{\cov}{Cov}
\DeclareMathOperator{\Var}{Var}
\DeclareMathOperator{\Rank}{rank}
\newcommand{\Tfae}{The following are equivalent:}
\newcommand{\tfae}{the following are equivalent:}
\newcommand{\sparsity}{\textit{sparsity}}

\newcommand{\uddots}{\reflectbox{$\ddots$}} 

\newenvironment{claimwidth}{\begin{center}\begin{adjustwidth}{0.05\textwidth}{0.05\textwidth}}{\end{adjustwidth}\end{center}}

\setlength{\parindent}{0pt}
\makeatletter
\newenvironment{listalgorithm}
{\par\noindent\hspace*{-\@totalleftmargin}%
	\minipage{\textwidth}\algorithm[H]}
{\endalgorithm\endminipage}
\makeatother
%%%%%%%%%%%%%%%%%%%%%%%%%%%%%%%%%%%%%%%%%%%%%%%%%%%%%%%%%%%%%%%%%%%%%%%%%%%%%%%%%%%%%%%%%%%%%%%%%%%%%%%%%%%%%%%%%%%%%%%%%%%%%%%%%%%%%%%%

\begin{document}
	
	%%%%%%%%%%%%%%%%%%%%%%%%%%%%%%%%%%%%%%%%%%%%%%%%%%%%%%%%%%%%%%%%%%%%%%%%%%%%%%%%%%%%%%%%%%%%%%%%%%%%%%%%%%%%%%%%%%%%%%%%%%%%%%%%%%%%%%%%
	
	\textsf{\noindent \large\textbf{Soham Chatterjee} \hfill \textbf{Assignment - 3}\\
		Email: \href{soham.chatterjee@tifr.res.in}{soham.chatterjee@tifr.res.in} \hfill Dept: STCS\\
		\normalsize Course: Algorithms \hfill Date: \today}
	
%%%%%%%%%%%%%%%%%%%%%%%%%%%%%%%%%%%%%%%%%%%%%%%%%%%%%%%%%%%%%%%%%%%%%%%%%%%%%%%%%%%%%%%%%%%%%%%%%%%%%%%%%%%%%%%%%%%%%%%%%%%%%%%%%%%%%%%%
% Problem 1
%%%%%%%%%%%%%%%%%%%%%%%%%%%%%%%%%%%%%%%%%%%%%%%%%%%%%%%%%%%%%%%%%%%%%%%%%%%%%%%%%%%%%%%%%%%%%%%%%%%%%%%%%%%%%%%%%%%%%%%%%%%%%%%%%%%%%%%%
\addtocounter{problem}{1}
\begin{problem}{%problem statement
		\hfill  (15 marks)
	}{p1% problem reference text
}
Given a directed graph $G=(V,E)$ with special vertices $s,t$, we define the following sets. Let $X$ be the set of vertices that \textit{always} lie on the side of $s$ in any minimum cut (e.g., $s\in X$). Let $Y$ be the set of  vertices that \textit{always} lie on the side of $t$ in any minimum cut (e.g., $t\in Y$). Let $Z=V\setminus (X\cup Y)$. Give an $O$(time for max-flow computation)-time algorithm to partition $V$ into $X,Y$ and $Z$
\end{problem}
\solve{	Consider the $s-t$ cut $(S,V\setminus S)$ with smallest number of vertices where $s\in S$. We will show that the set of vertices that always lie on the side of $s$ in any minimum cut $(X,V\setminus X)$ is actually $S$. First we define the notation $cap_T(T')=\sum\limits_{e=(u,v):u\in T,v\in T'}c_e$ for any subset $T,T'\subseteq V$ and $T\cap T'=\emptyset$
	\begin{lemma}\label{minmincut}
		$S$ is contained in every $s-t$ mincut $(X,V\setminus X)$ where $s\in X$.
	\end{lemma}
\begin{proof}
	Suppose $S$ is not in every $s-t$ mincut. Suppose there exists a mincut $(X,V\setminus X)$ such that $S\not\subseteq X$. Then consider the sets $X\setminus S$, $S\setminus X$. Since $S\not\subseteq X$ and $S$ has smallest size we have $X\setminus S\neq \emptyset$. Now for all $v\in X\setminus S$ we have $\textit{cap}_S(v)\geq 0$ since otherwise $f(S\cup \{v\})=f(S)+\textit{cap}_S(v)< f(S)$ which is not possible as $S$ is a min cut. Now consider the set $X\cap S$. $X\cap S\neq \emptyset$ as $s\in X\cap S$ and $|X\cap S|<|S|$. Hence $X\cap S$ is not a mincut. Therefore $f(X\cap S)>f(S)$. Now notice that \begin{align*}
		f(S) & = \textit{cap}_{V\setminus(X\cup S)}(X\cap S)+\textit{cap}_{V\setminus(X\cup S)}(S\setminus X)+\textit{cap}_{X\setminus S}(X\cap S)+\textit{cap}_{X\setminus S}(S\setminus X)\\
		f(X) & = \textit{cap}_{V\setminus(X\cup S)}(X\cap S)+\textit{cap}_{V\setminus(X\cup S)}(X\setminus S)+\textit{cap}_{S\setminus X}(X\cap S)+\textit{cap}_{S\setminus X}(X\setminus S)\\
		f(X\cap S) & = \textit{cap}_{V\setminus(X\cup S)}(X\cap S)+\textit{cap}_{X\setminus S}(X\cap S)+\textit{cap}_{S\setminus X}(X\cap S)\\
		f(X\cup S) & = \textit{cap}_{V\setminus(X\cup S)}(X\cap S)+\textit{cap}_{V\setminus(X\cup S)}(X\setminus S)+\textit{cap}_{V\setminus(X\cup S)}(S\setminus X)
	\end{align*}Therefore we have$$
	f(S)>f(S)+f(X)-f(X\cap S) =f(X\cap S)+\textit{cap}_{X\setminus S}(S\setminus X)+\textit{cap}_{S\setminus X}(X\setminus S)\geq f(X\cup S)$$Hence we obtain a $s-t$ cut $X\cup S$ which has capacity lesser than the $S$ which is not possible. Hence contradiction. Therefore $S$ is contained in every $s-t$ mincut $(X,V\setminus X)$ where $s\in X$.
\end{proof}
Hence by \lemref{minmincut} we have that the set of vertices which lies on  every $s-t$ mincut is $S$ since $S$ is the smallest $s-t$ mincut where $s\in S$. This lemma also suggests that there in an unique smallest size $s-t$ cut. So all we have to do now is find the mincut which has the smallest number of vertices. Now we will show that for any max-flow the set of vertices reachable in the corresponding residue graph is indeed the set $S$ which also forms a mincut.
\begin{lemma}
	For any max flow $f$ in the residue graph $G_f$ the set of vertices reachable from $s$ is the set $S$.
\end{lemma}
\begin{proof}
	We know the set of vertices reachable from $s$ in the residue graph $G_f$ is a $s-t$ mincut. Let the set of vertices reachable from $s$ in $G_f$ is the set $X$. By \lemref{minmincut} $S\subseteq X$. Suppose $S\neq X$. Then $S\subsetneq X\implies X\setminus S\neq \emptyset$. So $\exs\ x\in X\setminus S$.  Now since $f$ is max flow we have $$\sum_{e=(u,v):u\in S,v\notin S}c_e=|f|=\sum_{e=(u,v)u\in S,v\notin S}f(u,v)$$Since $c_e\geq f(e)$ for all $e\in E$ we have that $c_e=f_e$ for all $r=(u,v)\in E$ where $u\in S$ and $v\notin S$. Therefore all the forward edges between $S$ and $V\setminus S$ in $G_f$ are saturated and therefore all the edges between $S$ and $V\setminus S$ in $G_f$ are backward edges. Hence no vertex of $V\setminus S$ are reachable from $s$ in $G_f$. Hence $x$ is not reachable. But $x\in X$ and $X$ is the set of vertices in $G_f$ reachable from $s$. Hence contradiction. Therefore $S$ is the set of vertices reachable from $s$ in $G_f$. Hence we have the lemma.
\end{proof}
So now we have proved that if we run the max flow algorithm on the graph with the capacities we get the maxflow corresponding to the minimum cut which has the smallest size. Now from a maxflow we can get the corresponding mincut by running a Depth First Search Algorithm on the residue graph of the max flow and the cut set contains all the vertices reachable from $s$. Now to find the the set of vertices which always lie on the side of $t$ in any minimum cut we reverse the direction of edges in the graph then run the same maxflow algorithm on the new graph with  the same capacities. This will return the minimum cut $(V\setminus T, T)$ where $t\in T$ and $T$ has the smallest size. So the algorithm steps follows like this:
\begin{algorithm}
	\DontPrintSemicolon
	\KwIn{$G=(V,E)$, $c_e\in\bbZ_0$ for all $e\in E$}
	\KwOut{$(X,Y,Z)$ such that $V=X\sqcup Y\sqcup Z$}
	\Begin{\For{$(u,v)\in E$}{$\tdc_{v,u}\longleftarrow c_{u,v}$}
	$f\longleftarrow$ \prb{Edmond-Karp}($G=(V,E),\{c_e\colon e\in E\}$)\;
	\For{$e=(u,v)\in E$}{\If{$f(u,v)>0$}{$(v,u)\in E_f$}
	\If{$f(u,v)<c_{u,v}$}{$(u,v)\in E_f$}}
$X\longleftarrow$ Set of visited vertices on running \prb{DFS} on the graph $G_f=(V,E_f)$\;
\For{$(u,v)\in E$}{$(v,u)\in E'$}
$g\longleftarrow$ \prb{Edmond-Karp}($G=(V,E'),\{\tdc_e\colon e\in E'\}$)\;
\For{$e=(u,v)\in E'$}{\If{$g(u,v)>0$}{$(v,u)\in E'_g$}
	\If{$g(u,v)<c_{u,v}$}{$(u,v)\in E'_g$}}
$Y\longleftarrow$ Set of visited vertices on running \prb{DFS} on the graph $G_g=(V,E'_g)$\;
\Return{$(X,Y,V\setminus (X\cup Y))$}
}
\caption{Partitioning $V$}
\end{algorithm}

For the for loops at line 2,5,11,14 it takes $O(|E|)$ time. And the DFS at lines 10,19 takes time $O(|V|+|E|)$ time. But we know \prb{Edmond-Karp} algorithm takes time $O(|E|^2\times |V|)\textit{poly}(c_e)$. Hence the algorithm takes $O$(Time for Maxflow algorithm). Hence the total runtime of the algorithm $O$(Time for max-flow computation).
}\parinf

[Me, Soumyadeep and Soumyajit discussed the problem together. I also discussed with Vivek and Shubham]\parinn
%%%%%%%%%%%%%%%%%%%%%%%%%%%%%%%%%%%%%%%%%%%%%%%%%%%%%%%%%%%%%%%%%%%%%%%%%%%%%%%%%%%%%%%%%%%%%%%%%%%%%%%%%%%%%%%%%%%%%%%%%%%%%%%%%%%%%%%%
% Problem 2
%%%%%%%%%%%%%%%%%%%%%%%%%%%%%%%%%%%%%%%%%%%%%%%%%%%%%%%%%%%%%%%%%%%%%%%%%%%%%%%%%%%%%%%%%%%%%%%%%%%%%%%%%%%%%%%%%%%%%%%%%%%%%%%%%%%%%%%%

\begin{problem}{%problem statement
		\hfill  (5 marks)
	}{p2% problem reference text
	}
Given a set $S$ of $n$ items, a function $f:2^S\to\bbR$ is said to be \textit{submodular} if, for all sets $A\subseteq B$ and elements $x\notin B$, $$f(A\cup\{x\})-f(A)\geq f(B\cup \{x\}))-f(B)$$That is, the marginal value of an element ot a smaller set, is at least it's marginal value to a larger set.

Prove that a function $f$ is submodular if and only if it satisfies, for any sets $X,Y\subseteq S$, $$f(X)+f(Y)\geq f(X\cup Y)+f(X\cap Y)$$
\end{problem}
\solve{	Suppose $f$ is submodular. Let $B\setminus A=\{v_1,\dots, v_k\}$. Now denote $T_i=\{v_1,\dots, v_i\}$ for all $i\in\{0,\dots, k\}$ where $T_0=\emptyset$. Now for $T_i$ where $i<k$ we have $(A\cap B)\cup T_i\subseteq A\cup T_i$. Hence  \begin{align*}
	&f\big(((A\cap B)\cup T_i)\cup \{v_{i+1}\}\big)-f\big((A\cap B)\cup T_i\big)\geq f\big((A\cup T_i)\cup \{v_{i+1}\}\big)-f(A\cup T_i)\\
	\implies & f\big((A\cap B)\cup T_{i+1}\big)-f\big((A\cap B)\cup T_i\big)\geq f(A\cup T_{i+1})-f(A\cup T_i)
\end{align*}Now we sum these expressions for all $i=0$ to $i=k-1$ and we get \begin{align*}
f\big((A\cap B)\cup T_{k}\big)-f\big((A\cap B)\cup T_0\big)\geq f(A\cup T_{k})-f(A\cup T_0)\implies f(B)-f(A\cap B)\geq f(A\cup B)-f(A)
\end{align*}Hence we have $f(A)+f(B)\geq f(A\cup B)+f(A\cap B)$. 

Now we will show that if $f$ satisfies the condition that $f(X)+f(Y)\geq f(X\cup Y)+f(X\cap Y)$ for any two subsets $X,Y\subseteq S$ then $f$ is submodular. Let $A,B\subseteq S$ where $A\subseteq B$. Now let $x\notin B$ and $x\in S$. Then we have $$
	f(A\cup \{i\})+f(B)\geq f(A\cup \{i\}\cup B)+f\big((A\cup \{i\})\cap B\big)	\implies f(A\cup \{i\})+f(B)\geq f(B\cup \{i\})+f(A)$$Hence we have $f(A\cup \{i\})-f(A)\geq f(B\cup \{i\})-f(B)$. Therefore $f$ is submodular. 
}


%%%%%%%%%%%%%%%%%%%%%%%%%%%%%%%%%%%%%%%%%%%%%%%%%%%%%%%%%%%%%%%%%%%%%%%%%%%%%%%%%%%%%%%%%%%%%%%%%%%%%%%%%%%%%%%%%%%%%%%%%%%%%%%%%%%%%%%%
% Problem 3
%%%%%%%%%%%%%%%%%%%%%%%%%%%%%%%%%%%%%%%%%%%%%%%%%%%%%%%%%%%%%%%%%%%%%%%%%%%%%%%%%%%%%%%%%%%%%%%%%%%%%%%%%%%%%%%%%%%%%%%%%%%%%%%%%%%%%%%%

\begin{problem}{%problem statement
		\hfill  (5 marks)
	}{p3% problem reference text
	}
Let $G=(V,E)$ be a directed graph with nonnegative integral capacity $c_e$ on each edge. Define the \textit{cut function} $f:2^V\to \bbZ_+$ as $$f(S)=\sum_{e=(u,v):u\in S,v\notin S}c_e$$Show that cut function $f$ is submodular.
\end{problem}
\solve{
	Using \hyperref[p:p2]{Problem \ref{p:p2}} it suffices to show that for any $S,T\subseteq V$ we have $f(S)+f(T)\geq f(S\cup T)+f(S\cap T)$. Like in \hyperref[p:p1]{Problem \ref{p:p1}} we define the notation $cap_T(T')=\sum\limits_{e=(u,v):u\in T,v\in T'}c_e$ for any subset $T,T'\subseteq V$ and $T\cap T'=\emptyset$. Then we have \begin{align*}
		f(S)&=\textit{cap}_{V\setminus (S\cup T)}(S\cap T)+\textit{cap}_{V\setminus (S\cup T)}(S\setminus T)+\textit{cap}_{T\setminus S}(S\cap T)+\textit{cap}_{T\setminus S}(S\setminus T)\\
		f(T)&=\textit{cap}_{V\setminus (S\cup T)}(S\cap T)+\textit{cap}_{V\setminus (S\cup T)}(T\setminus S)+\textit{cap}_{S\setminus T}(S\cap T)+\textit{cap}_{S\setminus T}(T\setminus S)\\
		f(S\cup T)& =\textit{cap}_{V\setminus (S\cup T)}(S\cap T)+\textit{cap}_{V\setminus (S\cup T)}(S\setminus T)+\textit{cap}_{V\setminus (S\cup T)}(T\setminus S)\\
		f(S\cap T)& = \textit{cap}_{V\setminus (S\cup T)}(S\cap T)+\textit{cap}_{S\setminus T}(S\cap T)+\textit{cap}_{T\setminus S}(S\cap T)
	\end{align*}Hence we have $f(S)+f(T)-(f(S\cup T)f(S\cap T))=2\textit{cap}_{T\setminus S}(S\setminus T)\geq 0$. Hence by \hyperref[p:p2]{Problem \ref{p:p2}} $f$ is submodular. 
}


%%%%%%%%%%%%%%%%%%%%%%%%%%%%%%%%%%%%%%%%%%%%%%%%%%%%%%%%%%%%%%%%%%%%%%%%%%%%%%%%%%%%%%%%%%%%%%%%%%%%%%%%%%%%%%%%%%%%%%%%%%%%%%%%%%%%%%%%
% Problem 4
%%%%%%%%%%%%%%%%%%%%%%%%%%%%%%%%%%%%%%%%%%%%%%%%%%%%%%%%%%%%%%%%%%%%%%%%%%%%%%%%%%%%%%%%%%%%%%%%%%%%%%%%%%%%%%%%%%%%%%%%%%%%%%%%%%%%%%%%

\begin{problem}{%problem statement
		\hfill  (10 marks)
	}{p4% problem reference text
	}
Let $G=(V,E)$ be a directed graph with nonnegative integral capacities on the edges, and let $s,t,$ be two special vertices in the graph. Let $(S,V\setminus S)$ be a minimum $s-t$ cut with vertices $u,v$ in $S$, \textbf{so that there exists a minimum $u-v$ cut $(U,V\setminus U)$ with $t\notin U$.} Then show that there exists a minimum cut $u-v$ cut $(U',V\setminus U')$ so that $U'\subseteq S$ \sout{or $V\setminus U\subseteq S$}.

Problems 3, 4 may be useful in solving this.
\end{problem}
\solve{
	Since $t\notin U$ and $u,v\in S$, $S\cup U$ forms a $s-t$ cut and $S\cap U$ forms a $u-v$ cut. Now since the \textit{cut function} is submodular by \hyperref[p:p3]{Problem \ref{p:p3}}. Hence we have using \hyperref[p:p2]{Problem \ref{p:p2}} $$f(S)+f(U)\geq f(S\cup U)+f(S\cap U)$$Now since $S\cap U$ is a $s-t$ cut and $S$ is $s-t$ min cut we have $f(S\cup U)\geq f(S)$. And similarly since $U$ is $u-v$ mincut and $S\cap U$ is a $u-v$ cut we have $f(S\cap U)\geq f(U)$. Therefore $$f(S\cup U)+f(S\cap U)=f(S)+f(U)\implies f(S\cup U)+f(S\cap U)=f(S)+f(U)$$ Now if $f(S\cup U)>f(S)$ then we have $f(S\cup U)+f(S\cap U)>f(S)+f(U)$ but this is not true because of the submodularity of the cut function. Hence $f(S\cup U)=f(S)$. Therefore we get $f(S\cap U)=f(U)$. Hence $S\cap U$ is a $u-v$ mincut which is a subset of $S$. 
}\parinf

[Me and Soumyadeep discussed with the instructor]\parinn


%%%%%%%%%%%%%%%%%%%%%%%%%%%%%%%%%%%%%%%%%%%%%%%%%%%%%%%%%%%%%%%%%%%%%%%%%%%%%%%%%%%%%%%%%%%%%%%%%%%%%%%%%%%%%%%%%%%%%%%%%%%%%%%%%%%%%%%%
% Problem 5
%%%%%%%%%%%%%%%%%%%%%%%%%%%%%%%%%%%%%%%%%%%%%%%%%%%%%%%%%%%%%%%%%%%%%%%%%%%%%%%%%%%%%%%%%%%%%%%%%%%%%%%%%%%%%%%%%%%%%%%%%%%%%%%%%%%%%%%%

\begin{problem}{%problem statement
	Exercise 21-2 (Depth determination) from CLRS	\hfill  (25 marks)
}{p5% problem reference text
}
In the \textbf{\textit{depth-determination problem}}, we maintain a forest $\mcF=\{T_i\}$ of rooted
trees under three operations:
\begin{itemize}[align=left, leftmargin=1cm,itemsep=-0.2cm]
	\item[\prb{Make-Tree}$(v)$] creates a tree whose only node is $v$.
	\item[\prb{Find-Depth}$(v)$] returns the depth of node $v$ within its tree.
	\item[\prb{Graft}$(r,v)$] makes node $r$, which is assumed to be the root of a tree, become the
	child of node $v$, which is assumed to be in a different tree than $r$ but may or may
	not itself be a root.
\end{itemize}
\begin{enumerate}[label=\bfseries\itshape\alph*.]
	\item Suppose that we use a tree representation similar to a disjoint-set forest: $v.p$
	is the parent of node $v$, except that $v.p=v$ if $v$ is a root. Suppose further
	that we implement \prb{Graft}$(v)$ by setting $r.p=v$ and \prb{Find-Depth}$(v)$ by
	following the find path up to the root, returning a count of all nodes other than $v$
	encountered. Show that the worst-case running time of a sequence of $m$ \prb{Make-Tree}, \prb{Find-Depth} , and \prb{Graft} operations is $\Theta(m^2)$.
\end{enumerate}
	By using the union-by-rank and path-compression heuristics, we can reduce the
worst-case running time. We use the disjoint-set forest $\mcD=\{S_i\}$, where each
set $S_i$ (which is itself a tree) corresponds to a tree $T_i$ in the forest $\mcF $. The tree
structure within a set $S_i$, however, does not necessarily correspond to that of $T_i$. In
fact, the implementation of $S_i$ does not record the exact parent-child relationships
but nevertheless allows us to determine any node’s depth in $T_i$.
The key idea is to maintain in each node $v$ a ``pseudodistance” $v.d$, which is
defined so that the sum of the pseudodistances along the simple path from $v$ to the rot of its set $S_i$ equals the depth of $v$ in $T_i$. That is, if the simple path from $v$ to its root in $S_i$ is $v_0,v_1,\dots, v_k$ where $v_0=v$ and $v_k$ is $S_i$'s root, then the depth of $v$ in $T_i$ is $\sum\limits_{j=0}^kv_j.d$.
\begin{enumerate}[resume, label=\bfseries\itshape\alph*.]
	\item Give an implementation of \prb{Make-Tree}
	\item Show how to modify \prb{Find-Set} to implement \prb{Find-Depth}. Your implementation should perform path compression, and its running time should be linear in the length of the find path. Make sure that your implementation updates psedodistances correctly.
	\item Show how to implement \prb{Graft}$(r,v)$ which combines the sets containing $r$ and $v$, by modifying \prb{Union} and \prb{Link} procedures. Make sure that your implementation updates pseudodistances correctly. Note that the root of a set $S_i$ is not necessarily the root of the corresponding tree $T_i$.
	\item Find a tight bound on the worst-case running time of a sequence of $m$ \prb{Make-Tree}, \prb{Find-Depth} , and \prb{Graft} operations, $n$ of which are \prb{Make-Tree} operations.
\end{enumerate}
\end{problem}
\solve{ 
\begin{enumerate}[label=\bfseries\itshape\alph*.]
\item \prb{Make-Tree} and \prb{Graft} both takes constant time. \prb{Find-Depth} takes time linear in the depth of the node. Now the depth can be at most the number of nodes present in the tree.  So if $k$ of the $m$ commands are \prb{Make-Tree} and \prb{Graft} operations and rest are \prb{Find-Tree} operations then each \prb{Find-Tree} takes at most $O(k)$ time. Hence the total time is $\theta(k)+(m-k)\theta(k)=\theta(k(m-k))$. Now $k(m-k)$ is largest when $k=m-k\implies k=\frac{m}2$. Therefore it takes $\theta(m^2)$ time in worst case for a sequence of $m$ \prb{Make-Tree}, \prb{Find-Depth} and \prb{Graft} operations.
\item Sine \prb{Make-Tree} creates a new tree with only one node, we create a new set with only a single node.
\begin{algorithm}
	$v.p\longleftarrow v$\;
	$v.\textit{rank}\longleftarrow 0$\;
	$v.d\longleftarrow 0$
	\caption{\prb{Make-Tree}$(v)$}
\end{algorithm}
\item Like suggested in the question we will do a path compression and while doing a path compression we update the add $v.d$ to the value obtained by doing a path compression from $v.p$ and this will give the depth.
\begin{algorithm}
	\If{$v\neq v.p$}{
	$(v.p.d)\longleftarrow \prb{Find-Set}(v.p)$\;
	$v.d\longleftarrow v.d+d$\;
}
\Return{$(v.p,v.d)$}
	\caption{\prb{Find-Set}$(v)$}
\end{algorithm}
So  $\prb{Find-Depth}(v)=\prb{Find-Set}(v)[1]$
\item Now to implement \prb{Graft} we will need to find the roots of the tree $r,v$ is in and then find the depths of $r,v$. We will add the trees according to their ranks and then update the pseudodistances.
\begin{algorithm}
	$x\longleftarrow \prb{Find-Set}(v)[0]$\;
	$(y,\textit{depth})\longleftarrow \prb{Find-Set}(v)$\;
	\If{$x.rank>y.rank$}{
	$y.p\longleftarrow x$, 
	$y.d\longleftarrow y.d-x.d$
}
\Else{
	$x.p\longleftarrow y$, 
	$x.d\longleftarrow x.d+\textit{depth}+1-y.d$\;
	\If{$x.rank==y.rank$}{$y.rank\longleftarrow x.rank+1$}
}
	\caption{\prb{Graft}$(r,v)$}
\end{algorithm}
\item \prb{Make-Tree} takes constant time. For \prb{Find-Depth} takes $O(\prb{Find-Set})$ time and $\prb{Graft}$ takes $O(\prb{Union})$ time. Hence after a sequence of $m$ opertions, $n$ of which are \prb{Make-Time} operations, takes $O(n)$ time. Now other $m-n=O(m)$ operations are \prb{Find-Depth} and \prb{Graft}. So it takes $O(m\log^*n)$ time in worst case. 
\end{enumerate}
}


%%%%%%%%%%%%%%%%%%%%%%%%%%%%%%%%%%%%%%%%%%%%%%%%%%%%%%%%%%%%%%%%%%%%%%%%%%%%%%%%%%%%%%%%%%%%%%%%%%%%%%%%%%%%%%%%%%%%%%%%%%%%%%%%%%%%%%%%
% Problem 6
%%%%%%%%%%%%%%%%%%%%%%%%%%%%%%%%%%%%%%%%%%%%%%%%%%%%%%%%%%%%%%%%%%%%%%%%%%%%%%%%%%%%%%%%%%%%%%%%%%%%%%%%%%%%%%%%%%%%%%%%%%%%%%%%%%%%%%%%

\begin{problem}{%problem statement
	Exercise 16-4(a) from CLRS (Scheduling variations) \hfill  (10 marks)
}{p6% problem reference text
}
Consider the following algorithm for the problem from section 16.5 of scheduling unit-time tasks with deadlines and penalties. Let all $n$ time slots be initially empty, where time slot $i$ is the unit-length slot of time that finishes at time $i$. We consider the tasks in order of monotonically decreasing penalty. When considering task $a_j$, if there exists a time slot at or before $a_j$'s deadline $d_j$ that is still empty, assign $a_j$ to the latest such slot, filling it. If there is no such slot, assign task $a_j$ to the latest of the as yet unfilled slots. Argue that this algorithm always gives an optimal answer.
\end{problem}
\solve{
Suppose $E=\{a_j\colon j\in[n]\}$ are the jobs with deadline $d_1,\dots, d_n$.  Let $S\subseteq J$ is independent i.e. $S\in \mcI$ if $S$ has a feasible schedule. Now we know $(E,\mcI)$ forms a matroid. Define $N_t(S)=|\{j\in S\colon d_j\leq t\}|$. We know $S\in \mcI\iff N_t(S)\leq t$ for all $t$.   For any $j\in[n]$, if there exists a time slot  at  $d_j$ or before the algorithm assigns $a_j$ to the latest such slot and otherwise the algorithm $a_j$ assigns $a_j$ to the latest of the as yet unfilled slots. Therefore at any iteration the set of jobs that are filled in the time slots at or before their deadlines is an independent set. 

Suppose till some iteration the set of operations that are filled in the time slots at or before their deadlines is $S$. Hence $S\in \mcI$. Therefore in the next iteration if the algorithm assigns $a_j$ within it's deadline then $S\cup \{a_j\}\in \mcI$. Now let  $S\cup \{a_j\}\in\mcI$. Suppose the algorithm is not able to place $a_j$ in the time slot at or before $d_j$. That means all the time slots till $d_j$ are filled before assigning $a_j$ in a time slot. That means $N_{d_j}(S)=d_j$ and $N_{d_j}(S\cup \{a_j\})=d_j+1$. But since $S\cup \{a_j\}\in \mcI$ we have $N_t(S\cup \{a_j\})\leq t$ for all $t$. Hence contradiction. Therefore the algorithm assigns $a_j$ in the time slot at or before $d_j$ if and only if $S\cup \{a_j\}\in \mcI$.

Now in order to show that the algorithm outputs a optimal answer we have to show that the penalty is minimum. Since $(E,\mcI)$ is a matroid. Hence the greedy algorithm works to find the minimum weight base. And in the given algorithm each $a_j$ is added in time slot at or before it's deadline if and only if adding to the set of jobs completed within deadline in previous iteration is an independent set. Hence the algorithm forms the greedy approach to minimize the penalty. Therefore the algorithm outputs an optimal solution. 
}\parinf

[I discussed with Vivek]


%%%%%%%%%%%%%%%%%%%%%%%%%%%%%%%%%%%%%%%%%%%%%%%%%%%%%%%%%%%%%%%%%%%%%%%%%%%%%%%%%%%%%%%%%%%%%%%%%%%%%%%%%%%%%%%%%%%%%%%%%%%%%%%%%%%%%%%%
% Problem 7
%%%%%%%%%%%%%%%%%%%%%%%%%%%%%%%%%%%%%%%%%%%%%%%%%%%%%%%%%%%%%%%%%%%%%%%%%%%%%%%%%%%%%%%%%%%%%%%%%%%%%%%%%%%%%%%%%%%%%%%%%%%%%%%%%%%%%%%%


\begin{problem}{%problem statement
		\hfill  (30 marks)
	}{p7% problem reference text
	}
We are given two red-black trees $T_1$ and $T_2$ and an element $x$, with the guarantee that, for
any $x_1 \in T_1$ and $x_2 \in T_2$, $x_1.key < x.key < x_2.key$. Our problem is to implement the
procedure \textsc{RB-Join} that forms a single red-black tree from the elements in $T_1$, $T_2$, and $x$.
Let $n$ be the total number of nodes in $T_1$ and $T_2$.
\begin{enumerate}[label=(\roman*)]
	\item (5 marks) Given a red-black tree with $n'$ nodes, show that the black-height of the tree can be obtained in time $O(\log n')$. Let $T.bh$ store this information for each red-black tree $T$.
	\item (5 marks) Assume that $T_1.bh\geq T_2.bh$. Give an $O(\log n)$ times algorithm that finds a black node $y$ in $T_1$ with the largest key from among all nodes in $T_1$ with black-height $T_2.bh$.
	\item (5 marks) Let $T_y$ be the subtree rooted at $y$. Describe how $T_y\cup \{x\}\cup T_2$ can replace $T_y$ in $O(1)$ time without destroying the binary search tree property. 
	\item (10 marks) What color should $x$ be so that the red-black properties 1,3,4 (from section 13.1 of CLRS) are maintained? Describe how to enforce properties 2 and 4 in $O(\log n)$ time.
	\item (5 marks) Complete the description of \prb{RB-Join}, and show the running time.
\end{enumerate}
\end{problem}
\solve{
\begin{enumerate}[label=(\roman*)]
	\item Since in a Red-Black tree in every path from root to any leaf the number of black nodes visited is same. Therefore for any vertex if we start to go up till we reach the root and go down till we reach a leaf the total number of visited black nodes is the black-height of the red-black tree. 
		\begin{algorithm}
		\SetKwComment{Comment}{// }{}
		\DontPrintSemicolon
		\KwIn{Red-Black tree with $n'$ nodes}
		\KwOut{Find the black-height of the tree}
		\Begin{
	$v\longleftarrow root$ of $T$, 
$bh\longleftarrow 0$\;
\While{True}{
	\If{$v.\textit{color}==\textit{black}$}{$bh\longleftarrow bh+1$}
	\If{$v.\textit{left}\neq \prb{NIL}$}{$v\longleftarrow v.\textit{left}$}
	\ElseIf{$v.\textit{right}\neq \prb{NIL}$}{$v\longleftarrow v.\textit{right}$}
	\Else{\Return{bh}}
}	
	}
		\caption{\prb{Find-Black-Height(T)}}
	\end{algorithm}

Since the height of a red-black tree with $n'$ nodes is $O(\log n')$ nodes both the for loops run for $O(\log n')$ many iterations. Now this returns the black-height correctly since from the root on every encounter of a black nodes it increments $bh$ by 1 and at the end when it reaches a leaf it increments $bh$ and then returns the value. So the black-height of the tree can be obtained in time $O(\log n')$.
\item So to find a black node which has the largest key we will follow the right most path from the root and we will return the node on encountering the $(T_1.bh-T_2.bh+1)^{th}$ black node since that black node has height $T_2.bh$. So the algorithm for finding the black node with largest key and height $T_2.bh$:
\end{enumerate}\newpage

\begin{algorithm}[H]
	\SetKwComment{Comment}{// }{}
	\DontPrintSemicolon
	\KwIn{Red-black trees $T_1,T_2$}
	\KwOut{Find the black node with largest key with height $T_2.bh$}
	\Begin{
		\If{$T_1.bh<T_2.bh$}{\Return{None}}
		$v\longleftarrow root$ of $T_1$\;
		$b\longleftarrow 0$\;
		\While{$b<T_1.bh-T_2.bh+1$}{
			\If{$v.\textit{color}==\textit{black}$}{$b\longleftarrow b+1$}
			\If{$v.\textit{right}\neq \prb{NIL}$}{$v\longleftarrow v.\textit{right}$}
			\ElseIf{$v.\textit{left}\neq \prb{NIL}$}{$v\longleftarrow v.\textit{left}$}
			\Else{\Return{None}}
		}	
	\Return{$v$}
	}
	\caption{\prb{Largest-Key-Height}$(T_1,T_2.bh)$}
\end{algorithm}
\begin{enumerate}[label=(\roman*),resume]
\item Since  for all $x_1\in T_1$ we have $x_1.key<x$ and for all $x_2\in T_2$ we have $x_2.key>y.key$ we will put $x$ in place of $y$ and then make $y$ the left child of $x$ and $T_2$ the right child of $x$. So we do the following operation:
\begin{algorithm}
	\DontPrintSemicolon
	$v\longleftarrow$ root of $T_2$\;
	$x.left\longleftarrow y$\;
	$x.right\longleftarrow v$\;
	$v.p\longleftarrow x$\;
	$y.p\longleftarrow x$\;
	$x.p\longleftarrow y.p$\;
	\caption{\prb{Replace-Subtree}$(y,x,T_2)$}
\end{algorithm}
\item Since we know the color of $y$ is black and the root of $T_2$ is also black  we set $x.\textit{color}=\textit{red}$. If $x.p.\textit{color}=\textit{black}$ then we are okay. Otherwise we have $x.p.\textit{color}=\textit{red}$. Now a red node can not have a red child. In this case we will do recolor and rotations as necessary like we did for \prb{Insert} operation in  a red-black tree in order to balance the coloring. First we define the \prb{Left-Rotate} and \prb{Right-Rotate} and  \prb{Recolor}:
\end{enumerate}
\begin{minipage}{0.48\textwidth}
	\begin{algorithm}[H]
		\DontPrintSemicolon
		$v\longleftarrow u.\textit{left}$\;
		$v.p\longleftarrow u.p$\;
		$u.p\longleftarrow v$\;
		$u.\textit{left}\longleftarrow v.\textit{right}$\;
		$v.\textit{right}\longleftarrow u$\;
		\caption{\prb{Right-Rotate}$(u)$}
	\end{algorithm}
\end{minipage}
\hfill
\begin{minipage}{0.48\textwidth}
	\begin{algorithm}[H]
		\DontPrintSemicolon
		$v\longleftarrow u.\textit{right}$\;
		$v.p\longleftarrow u.p$\;
		$u.p\longleftarrow v$\;
		$u.\textit{right}\longleftarrow v.\textit{left}$\;
		$v.\textit{left}\longleftarrow u$\;
		\caption{\prb{Left-Rotate}$(u)$}
	\end{algorithm}
\end{minipage}
\begin{minipage}{0.48\textwidth}
\begin{algorithm}[H]
	\DontPrintSemicolon
	$u.\textit{color}\longleftarrow \textit{black}$\;
	$u.\textit{left}\longleftarrow \textit{red}$\;
	$u.\textit{right}\longleftarrow \textit{red}$\;
	\caption{\prb{Recolor-Black}($u$)}
\end{algorithm}
\end{minipage}
\hfill
\begin{minipage}{0.48\textwidth}
\begin{algorithm}[H]
	\DontPrintSemicolon
	$u.\textit{color}\longleftarrow \textit{red}$\;
	$u.\textit{left}\longleftarrow \textit{black}$\;
	$u.\textit{right}\longleftarrow \textit{black}$\;
	\caption{\prb{Recolor-Red}($u$)}
\end{algorithm}
\end{minipage}

\begin{enumerate}[label=(\roman*),resume]
\item[] Now we use the pointer $v.\textit{uncle}$ to point to the uncle of $v$ i.e. sibling of $v.p$. If $v.\textit{uncle}.\textit{color}=red$ then we do  on \prb{Recolor-Red}($v.p.p$) and then we move to $v.p.p$ and check if the colors of $v.p.p$ and $v.p.p.p$ are same or not other wise if $v.u.\textit{color}=black$ then we do a \prb{Left-Rotate}$(v.p)$ and then a \prb{Right-Rotate}$(v.p.p)$ and then  $\prb{Recolor-Black}(v)$ and then we have a proper red-black tree. S we have the following algorithm:
\begin{algorithm}
	\DontPrintSemicolon
\Begin{
	$x.\textit{color}\longleftarrow red$\;
	$v\longleftarrow x$\;
	\While{True}{
		\If{$v.p.\textit{color}==\textit{black}$}{
			\Return{}
		}	
		\If{$v.\textit{uncle}.\textit{color}==\textit{red}$}{
			\prb{Recolor-Red}($v.p.p$)\;
			$v\longleftarrow v.p.p$\;
		}	
		\ElseIf{$v.\textit{uncle}.\textit{color}==\textit{black}$}{
			\prb{Left-Rotate}($v.p$)\;
			\prb{Right-Rotate}($v.p.p$)\;
			\prb{Recolor-Black}($v$)\;
			\Return{}
		}
	}}
\caption{\prb{Color-New-Node}($x$)}
\end{algorithm}
In the algorithm in each iteration the we move upwards till we reach the root and there we stop since the root has the color black. And we know height of the red-black tree is $O(\log n)$. Hence the number of iterations of the while loop is $O(\log n)$. Each iteration takes constant time. Therefore this algorithm takes $O(\log n)$ time and it ensures all the properties of the red-black tree are maintained.
\item So the final complete description for the \prb{RB-Join}  is following:
\begin{algorithm}
\Begin{
$bh_1\longleftarrow \prb{Find-Black-Height}(T_1)$\;
 $bh_2\longleftarrow \prb{Find-Black-Tree}(T_2)$\;
		\If{$bh_1<bh_2$}{\Return{None}}
$y\longleftarrow\prb{Largest-Key-Height}(T_1,bh_2)$\;
$\prb{Replace-Subtree}(y,x,T_2)$\;
$\prb{Color-New-Node}(x)$\;
\Return{$T_1$}
}
\caption{\prb{RB-Join}($T_1,x,T_2$)}
\end{algorithm}
\end{enumerate}
}
%}\parinf
%
%[Me and Soumyadeep discussed the problem together]\parinn

\end{document}
