\documentclass[a4paper, 11pt]{article}
\usepackage{comment} % enables the use of multi-line comments (\ifx \fi) 
\usepackage{fullpage} % changes the margin
\usepackage[a4paper, total={7in, 10in}]{geometry}
\usepackage{amsmath,mathtools,mathdots}
\usepackage{amssymb,amsthm}  % assumes amsmath package installed
\usepackage{xcolor}
\usepackage{mdframed}
\usepackage[shortlabels]{enumitem}
\usepackage{indentfirst}
\usepackage{hyperref}
\hypersetup{
	colorlinks=true,
	linkcolor=doc!80,
	citecolor=myr,
	filecolor=myr,      
	urlcolor=doc!80,
	pdftitle={Assignment}, %%%%%%%%%%%%%%%%   WRITE ASSIGNMENT PDF NAME  %%%%%%%%%%%%%%%%%%%%
}
\usepackage[most,many,breakable]{tcolorbox}
\usepackage{tikz}
\usepackage{caption}
\usepackage{kpfonts}
\usepackage{libertine}
\usepackage{physics}
\usepackage[ruled,vlined,linesnumbered]{algorithm2e}
\usepackage{mathrsfs}
\usepackage{tikz-cd}
\usepackage{float}

\definecolor{mytheorembg}{HTML}{F2F2F9}
\definecolor{mytheoremfr}{HTML}{00007B}
\definecolor{doc}{RGB}{0,60,110}
\definecolor{myg}{RGB}{56, 140, 70}
\definecolor{myb}{RGB}{45, 111, 177}
\definecolor{myr}{RGB}{199, 68, 64}

\usetikzlibrary{decorations.pathreplacing,angles,quotes,patterns}
\definecolor{mytheorembg}{HTML}{F2F2F9}
\definecolor{mytheoremfr}{HTML}{00007B}
\definecolor{doc}{RGB}{0,60,110}
\definecolor{myg}{RGB}{56, 140, 70}
\definecolor{myb}{RGB}{45, 111, 177}
\definecolor{myr}{RGB}{199, 68, 64}
\newcounter{problem}
\tcbuselibrary{theorems,skins,hooks}
\newtcbtheorem[use counter=problem]{problem}{Problem}
{%
	enhanced,
	breakable,
	colback = mytheorembg,
	frame hidden,
	boxrule = 0sp,
	borderline west = {2pt}{0pt}{mytheoremfr},
	arc=5pt,
	detach title,
	before upper = \tcbtitle\par\smallskip,
	coltitle = mytheoremfr,
	fonttitle = \bfseries\sffamily,
	description font = \mdseries,
	separator sign none,
	segmentation style={solid, mytheoremfr},
}
{p}

\newtheorem{lemma}{Lemma}
\renewenvironment{proof}{\noindent{\it \textbf{Proof:}}\hspace*{1em}}{\qed\bigskip\\}
% To give references for any problem use like this
% suppose the problem number is p3 then 2 options either 
% \hyperref[p:p3]{<text you want to use to hyperlink> \ref{p:p3}}
%                  or directly 
%                   \ref{p:p3}


\newcommand*\circled[1]{\tikz[baseline=(char.base)]{
		\node[shape=circle,draw,inner sep=1pt] (char) {#1};}}
\newcommand\getcurrentref[1]{%
	\ifnumequal{\value{#1}}{0}
	{??}
	{\the\value{#1}}%
}
%---------------------------------------
% BlackBoard Math Fonts :-
%---------------------------------------

%Captital Letters
\newcommand{\bbA}{\mathbb{A}}	\newcommand{\bbB}{\mathbb{B}}
\newcommand{\bbC}{\mathbb{C}}	\newcommand{\bbD}{\mathbb{D}}
\newcommand{\bbE}{\mathbb{E}}	\newcommand{\bbF}{\mathbb{F}}
\newcommand{\bbG}{\mathbb{G}}	\newcommand{\bbH}{\mathbb{H}}
\newcommand{\bbI}{\mathbb{I}}	\newcommand{\bbJ}{\mathbb{J}}
\newcommand{\bbK}{\mathbb{K}}	\newcommand{\bbL}{\mathbb{L}}
\newcommand{\bbM}{\mathbb{M}}	\newcommand{\bbN}{\mathbb{N}}
\newcommand{\bbO}{\mathbb{O}}	\newcommand{\bbP}{\mathbb{P}}
\newcommand{\bbQ}{\mathbb{Q}}	\newcommand{\bbR}{\mathbb{R}}
\newcommand{\bbS}{\mathbb{S}}	\newcommand{\bbT}{\mathbb{T}}
\newcommand{\bbU}{\mathbb{U}}	\newcommand{\bbV}{\mathbb{V}}
\newcommand{\bbW}{\mathbb{W}}	\newcommand{\bbX}{\mathbb{X}}
\newcommand{\bbY}{\mathbb{Y}}	\newcommand{\bbZ}{\mathbb{Z}}

%---------------------------------------
% MathCal Fonts :-
%---------------------------------------

%Captital Letters
\newcommand{\mcA}{\mathcal{A}}	\newcommand{\mcB}{\mathcal{B}}
\newcommand{\mcC}{\mathcal{C}}	\newcommand{\mcD}{\mathcal{D}}
\newcommand{\mcE}{\mathcal{E}}	\newcommand{\mcF}{\mathcal{F}}
\newcommand{\mcG}{\mathcal{G}}	\newcommand{\mcH}{\mathcal{H}}
\newcommand{\mcI}{\mathcal{I}}	\newcommand{\mcJ}{\mathcal{J}}
\newcommand{\mcK}{\mathcal{K}}	\newcommand{\mcL}{\mathcal{L}}
\newcommand{\mcM}{\mathcal{M}}	\newcommand{\mcN}{\mathcal{N}}
\newcommand{\mcO}{\mathcal{O}}	\newcommand{\mcP}{\mathcal{P}}
\newcommand{\mcQ}{\mathcal{Q}}	\newcommand{\mcR}{\mathcal{R}}
\newcommand{\mcS}{\mathcal{S}}	\newcommand{\mcT}{\mathcal{T}}
\newcommand{\mcU}{\mathcal{U}}	\newcommand{\mcV}{\mathcal{V}}
\newcommand{\mcW}{\mathcal{W}}	\newcommand{\mcX}{\mathcal{X}}
\newcommand{\mcY}{\mathcal{Y}}	\newcommand{\mcZ}{\mathcal{Z}}



%---------------------------------------
% Bold Math Fonts :-
%---------------------------------------

%Captital Letters
\newcommand{\bmA}{\boldsymbol{A}}	\newcommand{\bmB}{\boldsymbol{B}}
\newcommand{\bmC}{\boldsymbol{C}}	\newcommand{\bmD}{\boldsymbol{D}}
\newcommand{\bmE}{\boldsymbol{E}}	\newcommand{\bmF}{\boldsymbol{F}}
\newcommand{\bmG}{\boldsymbol{G}}	\newcommand{\bmH}{\boldsymbol{H}}
\newcommand{\bmI}{\boldsymbol{I}}	\newcommand{\bmJ}{\boldsymbol{J}}
\newcommand{\bmK}{\boldsymbol{K}}	\newcommand{\bmL}{\boldsymbol{L}}
\newcommand{\bmM}{\boldsymbol{M}}	\newcommand{\bmN}{\boldsymbol{N}}
\newcommand{\bmO}{\boldsymbol{O}}	\newcommand{\bmP}{\boldsymbol{P}}
\newcommand{\bmQ}{\boldsymbol{Q}}	\newcommand{\bmR}{\boldsymbol{R}}
\newcommand{\bmS}{\boldsymbol{S}}	\newcommand{\bmT}{\boldsymbol{T}}
\newcommand{\bmU}{\boldsymbol{U}}	\newcommand{\bmV}{\boldsymbol{V}}
\newcommand{\bmW}{\boldsymbol{W}}	\newcommand{\bmX}{\boldsymbol{X}}
\newcommand{\bmY}{\boldsymbol{Y}}	\newcommand{\bmZ}{\boldsymbol{Z}}
%Small Letters
\newcommand{\bma}{\boldsymbol{a}}	\newcommand{\bmb}{\boldsymbol{b}}
\newcommand{\bmc}{\boldsymbol{c}}	\newcommand{\bmd}{\boldsymbol{d}}
\newcommand{\bme}{\boldsymbol{e}}	\newcommand{\bmf}{\boldsymbol{f}}
\newcommand{\bmg}{\boldsymbol{g}}	\newcommand{\bmh}{\boldsymbol{h}}
\newcommand{\bmi}{\boldsymbol{i}}	\newcommand{\bmj}{\boldsymbol{j}}
\newcommand{\bmk}{\boldsymbol{k}}	\newcommand{\bml}{\boldsymbol{l}}
\newcommand{\bmm}{\boldsymbol{m}}	\newcommand{\bmn}{\boldsymbol{n}}
\newcommand{\bmo}{\boldsymbol{o}}	\newcommand{\bmp}{\boldsymbol{p}}
\newcommand{\bmq}{\boldsymbol{q}}	\newcommand{\bmr}{\boldsymbol{r}}
\newcommand{\bms}{\boldsymbol{s}}	\newcommand{\bmt}{\boldsymbol{t}}
\newcommand{\bmu}{\boldsymbol{u}}	\newcommand{\bmv}{\boldsymbol{v}}
\newcommand{\bmw}{\boldsymbol{w}}	\newcommand{\bmx}{\boldsymbol{x}}
\newcommand{\bmy}{\boldsymbol{y}}	\newcommand{\bmz}{\boldsymbol{z}}


%---------------------------------------
% Scr Math Fonts :-
%---------------------------------------

\newcommand{\sA}{{\mathscr{A}}}   \newcommand{\sB}{{\mathscr{B}}}
\newcommand{\sC}{{\mathscr{C}}}   \newcommand{\sD}{{\mathscr{D}}}
\newcommand{\sE}{{\mathscr{E}}}   \newcommand{\sF}{{\mathscr{F}}}
\newcommand{\sG}{{\mathscr{G}}}   \newcommand{\sH}{{\mathscr{H}}}
\newcommand{\sI}{{\mathscr{I}}}   \newcommand{\sJ}{{\mathscr{J}}}
\newcommand{\sK}{{\mathscr{K}}}   \newcommand{\sL}{{\mathscr{L}}}
\newcommand{\sM}{{\mathscr{M}}}   \newcommand{\sN}{{\mathscr{N}}}
\newcommand{\sO}{{\mathscr{O}}}   \newcommand{\sP}{{\mathscr{P}}}
\newcommand{\sQ}{{\mathscr{Q}}}   \newcommand{\sR}{{\mathscr{R}}}
\newcommand{\sS}{{\mathscr{S}}}   \newcommand{\sT}{{\mathscr{T}}}
\newcommand{\sU}{{\mathscr{U}}}   \newcommand{\sV}{{\mathscr{V}}}
\newcommand{\sW}{{\mathscr{W}}}   \newcommand{\sX}{{\mathscr{X}}}
\newcommand{\sY}{{\mathscr{Y}}}   \newcommand{\sZ}{{\mathscr{Z}}}


%---------------------------------------
% Math Fraktur Font
%---------------------------------------

%Captital Letters
\newcommand{\mfA}{\mathfrak{A}}	\newcommand{\mfB}{\mathfrak{B}}
\newcommand{\mfC}{\mathfrak{C}}	\newcommand{\mfD}{\mathfrak{D}}
\newcommand{\mfE}{\mathfrak{E}}	\newcommand{\mfF}{\mathfrak{F}}
\newcommand{\mfG}{\mathfrak{G}}	\newcommand{\mfH}{\mathfrak{H}}
\newcommand{\mfI}{\mathfrak{I}}	\newcommand{\mfJ}{\mathfrak{J}}
\newcommand{\mfK}{\mathfrak{K}}	\newcommand{\mfL}{\mathfrak{L}}
\newcommand{\mfM}{\mathfrak{M}}	\newcommand{\mfN}{\mathfrak{N}}
\newcommand{\mfO}{\mathfrak{O}}	\newcommand{\mfP}{\mathfrak{P}}
\newcommand{\mfQ}{\mathfrak{Q}}	\newcommand{\mfR}{\mathfrak{R}}
\newcommand{\mfS}{\mathfrak{S}}	\newcommand{\mfT}{\mathfrak{T}}
\newcommand{\mfU}{\mathfrak{U}}	\newcommand{\mfV}{\mathfrak{V}}
\newcommand{\mfW}{\mathfrak{W}}	\newcommand{\mfX}{\mathfrak{X}}
\newcommand{\mfY}{\mathfrak{Y}}	\newcommand{\mfZ}{\mathfrak{Z}}
%Small Letters
\newcommand{\mfa}{\mathfrak{a}}	\newcommand{\mfb}{\mathfrak{b}}
\newcommand{\mfc}{\mathfrak{c}}	\newcommand{\mfd}{\mathfrak{d}}
\newcommand{\mfe}{\mathfrak{e}}	\newcommand{\mff}{\mathfrak{f}}
\newcommand{\mfg}{\mathfrak{g}}	\newcommand{\mfh}{\mathfrak{h}}
\newcommand{\mfi}{\mathfrak{i}}	\newcommand{\mfj}{\mathfrak{j}}
\newcommand{\mfk}{\mathfrak{k}}	\newcommand{\mfl}{\mathfrak{l}}
\newcommand{\mfm}{\mathfrak{m}}	\newcommand{\mfn}{\mathfrak{n}}
\newcommand{\mfo}{\mathfrak{o}}	\newcommand{\mfp}{\mathfrak{p}}
\newcommand{\mfq}{\mathfrak{q}}	\newcommand{\mfr}{\mathfrak{r}}
\newcommand{\mfs}{\mathfrak{s}}	\newcommand{\mft}{\mathfrak{t}}
\newcommand{\mfu}{\mathfrak{u}}	\newcommand{\mfv}{\mathfrak{v}}
\newcommand{\mfw}{\mathfrak{w}}	\newcommand{\mfx}{\mathfrak{x}}
\newcommand{\mfy}{\mathfrak{y}}	\newcommand{\mfz}{\mathfrak{z}}

%---------------------------------------
% Bar
%---------------------------------------

%Captital Letters
\newcommand{\ovA}{\overline{A}}	\newcommand{\ovB}{\overline{B}}
\newcommand{\ovC}{\overline{C}}	\newcommand{\ovD}{\overline{D}}
\newcommand{\ovE}{\overline{E}}	\newcommand{\ovF}{\overline{F}}
\newcommand{\ovG}{\overline{G}}	\newcommand{\ovH}{\overline{H}}
\newcommand{\ovI}{\overline{I}}	\newcommand{\ovJ}{\overline{J}}
\newcommand{\ovK}{\overline{K}}	\newcommand{\ovL}{\overline{L}}
\newcommand{\ovM}{\overline{M}}	\newcommand{\ovN}{\overline{N}}
\newcommand{\ovO}{\overline{O}}	\newcommand{\ovP}{\overline{P}}
\newcommand{\ovQ}{\overline{Q}}	\newcommand{\ovR}{\overline{R}}
\newcommand{\ovS}{\overline{S}}	\newcommand{\ovT}{\overline{T}}
\newcommand{\ovU}{\overline{U}}	\newcommand{\ovV}{\overline{V}}
\newcommand{\ovW}{\overline{W}}	\newcommand{\ovX}{\overline{X}}
\newcommand{\ovY}{\overline{Y}}	\newcommand{\ovZ}{\overline{Z}}
%Small Letters
\newcommand{\ova}{\overline{a}}	\newcommand{\ovb}{\overline{b}}
\newcommand{\ovc}{\overline{c}}	\newcommand{\ovd}{\overline{d}}
\newcommand{\ove}{\overline{e}}	\newcommand{\ovf}{\overline{f}}
\newcommand{\ovg}{\overline{g}}	\newcommand{\ovh}{\overline{h}}
\newcommand{\ovi}{\overline{i}}	\newcommand{\ovj}{\overline{j}}
\newcommand{\ovk}{\overline{k}}	\newcommand{\ovl}{\overline{l}}
\newcommand{\ovm}{\overline{m}}	\newcommand{\ovn}{\overline{n}}
\newcommand{\ovo}{\overline{o}}	\newcommand{\ovp}{\overline{p}}
\newcommand{\ovq}{\overline{q}}	\newcommand{\ovr}{\overline{r}}
\newcommand{\ovs}{\overline{s}}	\newcommand{\ovt}{\overline{t}}
\newcommand{\ovu}{\overline{u}}	\newcommand{\ovv}{\overline{v}}
\newcommand{\ovw}{\overline{w}}	\newcommand{\ovx}{\overline{x}}
\newcommand{\ovy}{\overline{y}}	\newcommand{\ovz}{\overline{z}}

%---------------------------------------
% Tilde
%---------------------------------------

%Captital Letters
\newcommand{\tdA}{\tilde{A}}	\newcommand{\tdB}{\tilde{B}}
\newcommand{\tdC}{\tilde{C}}	\newcommand{\tdD}{\tilde{D}}
\newcommand{\tdE}{\tilde{E}}	\newcommand{\tdF}{\tilde{F}}
\newcommand{\tdG}{\tilde{G}}	\newcommand{\tdH}{\tilde{H}}
\newcommand{\tdI}{\tilde{I}}	\newcommand{\tdJ}{\tilde{J}}
\newcommand{\tdK}{\tilde{K}}	\newcommand{\tdL}{\tilde{L}}
\newcommand{\tdM}{\tilde{M}}	\newcommand{\tdN}{\tilde{N}}
\newcommand{\tdO}{\tilde{O}}	\newcommand{\tdP}{\tilde{P}}
\newcommand{\tdQ}{\tilde{Q}}	\newcommand{\tdR}{\tilde{R}}
\newcommand{\tdS}{\tilde{S}}	\newcommand{\tdT}{\tilde{T}}
\newcommand{\tdU}{\tilde{U}}	\newcommand{\tdV}{\tilde{V}}
\newcommand{\tdW}{\tilde{W}}	\newcommand{\tdX}{\tilde{X}}
\newcommand{\tdY}{\tilde{Y}}	\newcommand{\tdZ}{\tilde{Z}}
%Small Letters
\newcommand{\tda}{\tilde{a}}	\newcommand{\tdb}{\tilde{b}}
\newcommand{\tdc}{\tilde{c}}	\newcommand{\tdd}{\tilde{d}}
\newcommand{\tde}{\tilde{e}}	\newcommand{\tdf}{\tilde{f}}
\newcommand{\tdg}{\tilde{g}}	\newcommand{\tdh}{\tilde{h}}
\newcommand{\tdi}{\tilde{i}}	\newcommand{\tdj}{\tilde{j}}
\newcommand{\tdk}{\tilde{k}}	\newcommand{\tdl}{\tilde{l}}
\newcommand{\tdm}{\tilde{m}}	\newcommand{\tdn}{\tilde{n}}
\newcommand{\tdo}{\tilde{o}}	\newcommand{\tdp}{\tilde{p}}
\newcommand{\tdq}{\tilde{q}}	\newcommand{\tdr}{\tilde{r}}
\newcommand{\tds}{\tilde{s}}	\newcommand{\tdt}{\tilde{t}}
\newcommand{\tdu}{\tilde{u}}	\newcommand{\tdv}{\tilde{v}}
\newcommand{\tdw}{\tilde{w}}	\newcommand{\tdx}{\tilde{x}}
\newcommand{\tdy}{\tilde{y}}	\newcommand{\tdz}{\tilde{z}}

%---------------------------------------
% Vec
%---------------------------------------

%Captital Letters
\newcommand{\vcA}{\vec{A}}	\newcommand{\vcB}{\vec{B}}
\newcommand{\vcC}{\vec{C}}	\newcommand{\vcD}{\vec{D}}
\newcommand{\vcE}{\vec{E}}	\newcommand{\vcF}{\vec{F}}
\newcommand{\vcG}{\vec{G}}	\newcommand{\vcH}{\vec{H}}
\newcommand{\vcI}{\vec{I}}	\newcommand{\vcJ}{\vec{J}}
\newcommand{\vcK}{\vec{K}}	\newcommand{\vcL}{\vec{L}}
\newcommand{\vcM}{\vec{M}}	\newcommand{\vcN}{\vec{N}}
\newcommand{\vcO}{\vec{O}}	\newcommand{\vcP}{\vec{P}}
\newcommand{\vcQ}{\vec{Q}}	\newcommand{\vcR}{\vec{R}}
\newcommand{\vcS}{\vec{S}}	\newcommand{\vcT}{\vec{T}}
\newcommand{\vcU}{\vec{U}}	\newcommand{\vcV}{\vec{V}}
\newcommand{\vcW}{\vec{W}}	\newcommand{\vcX}{\vec{X}}
\newcommand{\vcY}{\vec{Y}}	\newcommand{\vcZ}{\vec{Z}}
%Small Letters
\newcommand{\vca}{\vec{a}}	\newcommand{\vcb}{\vec{b}}
\newcommand{\vcc}{\vec{c}}	\newcommand{\vcd}{\vec{d}}
\newcommand{\vce}{\vec{e}}	\newcommand{\vcf}{\vec{f}}
\newcommand{\vcg}{\vec{g}}	\newcommand{\vch}{\vec{h}}
\newcommand{\vci}{\vec{i}}	\newcommand{\vcj}{\vec{j}}
\newcommand{\vck}{\vec{k}}	\newcommand{\vcl}{\vec{l}}
\newcommand{\vcm}{\vec{m}}	\newcommand{\vcn}{\vec{n}}
\newcommand{\vco}{\vec{o}}	\newcommand{\vcp}{\vec{p}}
\newcommand{\vcq}{\vec{q}}	\newcommand{\vcr}{\vec{r}}
\newcommand{\vcs}{\vec{s}}	\newcommand{\vct}{\vec{t}}
\newcommand{\vcu}{\vec{u}}	\newcommand{\vcv}{\vec{v}}
%\newcommand{\vcw}{\vec{w}}	\newcommand{\vcx}{\vec{x}}
\newcommand{\vcy}{\vec{y}}	\newcommand{\vcz}{\vec{z}}

%---------------------------------------
% Greek Letters:-
%---------------------------------------
\newcommand{\eps}{\epsilon}
\newcommand{\veps}{\varepsilon}
\newcommand{\lm}{\lambda}
\newcommand{\Lm}{\Lambda}
\newcommand{\gm}{\gamma}
\newcommand{\Gm}{\Gamma}
\newcommand{\vph}{\varphi}
\newcommand{\ph}{\phi}
\newcommand{\om}{\omega}
\newcommand{\Om}{\Omega}
\newcommand{\sg}{\sigma}
\newcommand{\Sg}{\Sigma}

\newcommand{\Qed}{\begin{flushright}\qed\end{flushright}}
\newcommand{\parinn}{\setlength{\parindent}{1cm}}
\newcommand{\parinf}{\setlength{\parindent}{0cm}}
\newcommand{\del}[2]{\frac{\partial #1}{\partial #2}}
\newcommand{\Del}[3]{\frac{\partial^{#1} #2}{\partial^{#1} #3}}
\newcommand{\deld}[2]{\dfrac{\partial #1}{\partial #2}}
\newcommand{\Deld}[3]{\dfrac{\partial^{#1} #2}{\partial^{#1} #3}}
\newcommand{\uin}{\mathbin{\rotatebox[origin=c]{90}{$\in$}}}
\newcommand{\usubset}{\mathbin{\rotatebox[origin=c]{90}{$\subset$}}}
\newcommand{\lt}{\left}
\newcommand{\rt}{\right}
\newcommand{\exs}{\exists}
\newcommand{\st}{\strut}
\newcommand{\dps}[1]{\displaystyle{#1}}
\newcommand{\la}{\langle}
\newcommand{\ra}{\rangle}
\newcommand{\cls}[1]{\textsc{#1}}
\newcommand{\prb}[1]{\textsc{#1}}
\newcommand{\comb}[2]{\left(\begin{matrix}
		#1\\ #2
\end{matrix}\right)}
%\newcommand[2]{\quotient}{\faktor{#1}{#2}}
\newcommand\quotient[2]{
	\mathchoice
	{% \displaystyle
		\text{\raise1ex\hbox{$#1$}\Big/\lower1ex\hbox{$#2$}}%
	}
	{% \textstyle
		#1\,/\,#2
	}
	{% \scriptstyle
		#1\,/\,#2
	}
	{% \scriptscriptstyle  
		#1\,/\,#2
	}
}

\newcommand{\tensor}{\otimes}
\newcommand{\xor}{\oplus}

\newcommand{\sol}[1]{\begin{solution}#1\end{solution}}
\newcommand{\solve}[1]{\setlength{\parindent}{0cm}\textbf{\textit{Solution: }}\setlength{\parindent}{1cm}#1 \hfill $\blacksquare$}
\newcommand{\mat}[1]{\left[\begin{matrix}#1\end{matrix}\right]}
\newcommand{\matr}[1]{\begin{matrix}#1\end{matrix}}
\newcommand{\matp}[1]{\lt(\begin{matrix}#1\end{matrix}\rt)}
\newcommand{\detmat}[1]{\lt|\begin{matrix}#1\end{matrix}\rt|}
\newcommand\numberthis{\addtocounter{equation}{1}\tag{\theequation}}
\newcommand{\handout}[3]{
	\noindent
	\begin{center}
		\framebox{
			\vbox{
				\hbox to 6.5in { {\bf Complexity Theory I } \hfill Jan -- May, 2023 }
				\vspace{4mm}
				\hbox to 6.5in { {\Large \hfill #1  \hfill} }
				\vspace{2mm}
				\hbox to 6.5in { {\em #2 \hfill #3} }
			}
		}
	\end{center}
	\vspace*{4mm}
}

\newcommand{\lecture}[3]{\handout{Lecture #1}{Lecturer: #2}{Scribe:	#3}}

\let\marvosymLightning\Lightning
\newcommand{\ctr}{\text{\marvosymLightning}\hspace{0.5ex}} % Requires marvosym package

\newcommand{\ov}[1]{\overline{#1}}
\newcommand{\thmref}[1]{\hyperref[th:#1]{Theorem \ref{th:#1}}}
\newcommand{\propref}[1]{\hyperref[th:#1]{Proposition \ref{th:#1}}}
\newcommand{\lmref}[1]{\hyperref[th:#1]{Lemma \ref{th:#1}}}
\newcommand{\corref}[1]{\hyperref[th:#1]{Corollary \ref{th:#1}}}

\newcommand{\thrmref}[1]{\hyperref[#1]{Theorem \ref{#1}}}
\newcommand{\propnref}[1]{\hyperref[#1]{Proposition \ref{#1}}}
\newcommand{\lemref}[1]{\hyperref[#1]{Lemma \ref{#1}}}
\newcommand{\corrref}[1]{\hyperref[#1]{Corollary \ref{#1}}}

\DeclareMathOperator{\enc}{Enc}
\DeclareMathOperator{\res}{Res}
\DeclareMathOperator{\spec}{Spec}
\DeclareMathOperator{\cov}{Cov}
\DeclareMathOperator{\Var}{Var}
\DeclareMathOperator{\Rank}{rank}
\newcommand{\Tfae}{The following are equivalent:}
\newcommand{\tfae}{the following are equivalent:}
\newcommand{\sparsity}{\textit{sparsity}}

\newcommand{\uddots}{\reflectbox{$\ddots$}} 

\newenvironment{claimwidth}{\begin{center}\begin{adjustwidth}{0.05\textwidth}{0.05\textwidth}}{\end{adjustwidth}\end{center}}

\setlength{\parindent}{0pt}
\makeatletter
\newenvironment{listalgorithm}
{\par\noindent\hspace*{-\@totalleftmargin}%
	\minipage{\textwidth}\algorithm[H]}
{\endalgorithm\endminipage}
\makeatother
%%%%%%%%%%%%%%%%%%%%%%%%%%%%%%%%%%%%%%%%%%%%%%%%%%%%%%%%%%%%%%%%%%%%%%%%%%%%%%%%%%%%%%%%%%%%%%%%%%%%%%%%%%%%%%%%%%%%%%%%%%%%%%%%%%%%%%%%

\begin{document}
	
	%%%%%%%%%%%%%%%%%%%%%%%%%%%%%%%%%%%%%%%%%%%%%%%%%%%%%%%%%%%%%%%%%%%%%%%%%%%%%%%%%%%%%%%%%%%%%%%%%%%%%%%%%%%%%%%%%%%%%%%%%%%%%%%%%%%%%%%%
	
	\textsf{\noindent \large\textbf{Soham Chatterjee} \hfill \textbf{Assignment - 2}\\
		Email: \href{soham.chatterjee@tifr.res.in}{soham.chatterjee@tifr.res.in} \hfill Dept: STCS\\
		\normalsize Course: Algorithms \hfill Date: \today}
	
%%%%%%%%%%%%%%%%%%%%%%%%%%%%%%%%%%%%%%%%%%%%%%%%%%%%%%%%%%%%%%%%%%%%%%%%%%%%%%%%%%%%%%%%%%%%%%%%%%%%%%%%%%%%%%%%%%%%%%%%%%%%%%%%%%%%%%%%
% Problem 1
%%%%%%%%%%%%%%%%%%%%%%%%%%%%%%%%%%%%%%%%%%%%%%%%%%%%%%%%%%%%%%%%%%%%%%%%%%%%%%%%%%%%%%%%%%%%%%%%%%%%%%%%%%%%%%%%%%%%%%%%%%%%%%%%%%%%%%%%
	
\begin{problem}{%problem statement
		P2\hfill  (10 marks)
	}{p1% problem reference text
}
Show that $n-1$ comparisons are necessary and sufficient to find the minimum element in an unsorted array of $n$ elements.

\end{problem}
\solve{	
Suppose there is an algorithm which can find minimum element with less than $n-1$ comparisons. Let $A=(a_1,\dots, a_n)$ be the unsorted array of $n$ elements. Create a graph of vertices $[n]$ and put an edge between $i$ and $j$ if the elements $a_i$ and $a_j$ are compared. Since there are $,n-1$ comparisons in the graph we constructed there are at most $n-2$ edges. Hence the graph has at least $2$ connected components. Let $C_1$ and $C_2$ be two connected components. No two elements one from each component is compared. Let the concluded minimal element from $C_1$ is $x$ and the concluded minimal element from $C_2$ is $y$. Hence we can make any of them bigger than the other depending on the algorithm output to make the algorithm wrong. Hence at least $n-1$ comparisons are necessary to find the minimum element.

Now we can also find the minimum element of an unsorted array of $n$ elements with $n-1$ comparisons by the following algorithm:

\begin{algorithm}
	\DontPrintSemicolon
	\KwIn{Array $A=(a_1,\dots, a_n)$}
	\KwOut{Minimum element of $A$}
	\Begin{
	$MinVal\longleftarrow A[1]$\;
	\For{$i=2,\dots, n$}{
	\If{$ MinVal>A[i]$}{
		$MinVal\longleftarrow A[i]$\;
	}
	}
	\Return{$MinVal$}
}
\caption{\prb{Find-Minimum}}
\end{algorithm}

This algorithm uses only $n-1$ comparisons to find the minimum element of the array. Therefore $n-1$ comparisons are sufficient to find the minimum element in an unsorted array of $n$ elements.
}
%%%%%%%%%%%%%%%%%%%%%%%%%%%%%%%%%%%%%%%%%%%%%%%%%%%%%%%%%%%%%%%%%%%%%%%%%%%%%%%%%%%%%%%%%%%%%%%%%%%%%%%%%%%%%%%%%%%%%%%%%%%%%%%%%%%%%%%%
% Problem 2
%%%%%%%%%%%%%%%%%%%%%%%%%%%%%%%%%%%%%%%%%%%%%%%%%%%%%%%%%%%%%%%%%%%%%%%%%%%%%%%%%%%%%%%%%%%%%%%%%%%%%%%%%%%%%%%%%%%%%%%%%%%%%%%%%%%%%%%%

\begin{problem}{%problem statement
		P3\hfill  (15 marks)
	}{p2% problem reference text
	}
Show a comparison based algorithm for finding the minimum and maximum in an unsorted array of $n$ elements using $\lt\lfloor \frac{3n}2\rt\rfloor-2$ comparisons. Also show that $\lt\lfloor \frac{3n}2\rt\rfloor-2$ comparisons are necessary to find the minimum and maximum.
\end{problem}
\solve{	
First assume that all the elements are distinct. We will show the algorithm with which you only need to do $\lt\lceil\frac{3n}2\rt\rceil-2$ comparisons. 

So first we partition the elements into pairs. If $n$ is odd then the last element will not in any groups. Hence there are total $\lt\lfloor\frac{n}2\rt\rfloor$ pairs. Now for each pair we compare then. Hence we have done now $\lt\lfloor\frac{n}2\rt\rfloor$ many comparisons. Now in each pair we have an element which was lesser than the other. So from each pair we take the lesser element and make a new set $S$. So $S$ has $\lt\lfloor\frac{n}2\rt\rfloor$ elements. And from each pair we take the greater element and make another new set $T$. If $n$ is odd then the last element which didn't take part in the pairs is  added to both the sets $S$ and  $T$. So $S$ and $T$  both have now $\lt\lceil\frac{n}2\rt\rceil$ elements. 

Now since $S$ has the lesser elements the minimum element of the array is in $S$. And since $T$ has the greater elements, the maximum element of the array is in $T$. So by the \hyperref[p:p1]{Problem \ref{p:p1}: P2} we have an algorithm to find out the minimum element of $S$ using $\lt\lfloor\frac{n}2\rt\rfloor-1$ comparisons. And since finding maximum element of array is very similar as finding minimum element using the same algorithm but using less than instead of greater than we get the maximum element of $T$ using $\lt\lceil\frac{n}2\rt\rceil-1$ comparisons. Hence now we have the minimum and the maximum element of the array.

\begin{algorithm}[H]
	\DontPrintSemicolon
	\KwIn{Array $A=(a_1,\dots, a_n)$}
	\KwOut{Minimum and Maximum element of $A$}
	\Begin{
	 $n\longleftarrow A.length$\;
	 \If{$n==1$}{\Return{$(A[1],A[1])$}}
	 $k\longleftarrow \lt\lfloor \frac{n}{2}\rt\rfloor$,  $S\longleftarrow \emptyset$, $T\longleftarrow \emptyset$\;
	 \For{$i=1,\dots, k$}{
		\If{$A[2i-1]<A[2i]$}{$S\longleftarrow S\cup \{A[2i-1]\}$,		$T\longleftarrow T\cup \{A[2i]\}$} 
		\Else{$S\longleftarrow S\cup \{A[2i]\}$,			$T\longleftarrow T\cup \{A[2i-1]\}$}
 	}
 \If{$n \bmod 2==1$}{$S\longleftarrow S\cup \{A[n]\}$, $T\longleftarrow T\cup \{A[n]\}$}
 $Minval\longleftarrow \infty$, $Maxval\longleftarrow -\infty$\;
 \For{$i\in S$}{
 	\If{$i=1$}{$Minval\longleftarrow S[1]$}
 	\ElseIf{$Minval>S[i]$}{$Minval\longleftarrow S[i]$}
 }
 \For{$j\in T$}{
	\If{$j=1$}{$Maxval\longleftarrow T[1]$}
	\ElseIf{$Maxval<T[i]$}{$Maxval\longleftarrow T[i]$}
}
\Return{($Minval, Maxval$)}
	}
\caption{\prb{Find-Minimum-Maximum}$(A)$}
\end{algorithm}


Hence total comparisons needed to find the minimum and maximum element is $$\lt\lfloor\frac{n}2\rt\rfloor+ \lt(\lt\lceil\frac{n}2\rt\rceil-1\rt)+\lt(\lt\lceil\frac{n}2\rt\rceil-1\rt)=\lt\lfloor\frac{n}2\rt\rfloor+\lt\lceil\frac{n}2\rt\rceil+\lt\lceil\frac{n}2\rt\rceil-2= \lt\lceil\frac{3n}2\rt\rceil-2 $$Hence this algorithm can find out the minimum and maximum using $\lt\lceil\frac{3n}2\rt\rceil-2$ comparisons. Now we need to prove that at least this many comparisons are needed. 


Now we work as an adversary to the algorithm and will force the algorithm to do at least $\lt\lceil \frac{3n}{2}\rt\rceil-2$ many comparisons. So for each element in the array we keep the tags $M$ and $m$. The $M$ tag represents that the element can be the maximum element of the array and the $m$ tag represents the element can be the minimum element of the array. Now on each comparison of two elements the algorithm does we remove tags for example if two elements both having a $M$ tag and a $m$ tag is compared then the smaller one can not be  maximum so the $M$ tag from the smaller one is removed and the larger one can not be the minimum so the $m$ tag from the larger one is removed.

Initially before the algorithm starts all elements have one $M$ tag and one $m$ tag. After the algorithm successfully find the maximum element and the minimum element there is only one $M$ tag on some element and only one $m$ tag on some element of the array. So initially there are $2n$ tags and at the end there are $2$ tags. Hence $2n-2$ tags have to be removed in order to find the maximum and minimum of the array. 

Now at any point during the execution of the algorithm if there is an element with both the $M$ and $m$ tags then this element has not been compared to other element because otherwise if its compared then either it becomes the larger and then the $m$ tag is removed or it becomes the smaller and then the $M$ tag is removed. Now consider the following cases:
\begin{itemize}
	\item Two $(M,m)$ tagged elements are compared: Then the adversary makes one of the larger and other one smaller randomly.
	\item If $(M,m)$ tagged element, $a$ is compared with $M$ tagged element, $b$: $a$ has not been compared to any one. So the adversary simply makes $b$ larger. So only the $M$ tag from $a$ is removed.
	\item If $(M,m)$ tagged element, $a$ is compared with $m$ tagged element, $b$: $a$ has not been compared to any one. So the adversary simply makes $b$ smaller. So only the $m$ tag from $a$ is removed.
	\item If $M$ tagged element, $a$ is compared with $M$ tagged element, $b$: Then the whatever the adversary makes larger only one $M$ tag will be removed. 
	\item If $m$ tagged element, $a$ is compared with $m$ tagged element, $b$: Then similarly like the above case whatever the adversary makes smaller only one $m$ tag will be removed. 
	\item If $M$ tagged element, $a$ is compared with an element $b$ with no tag: Then if $a$ becomes smaller then $M$ tag of $a$ is removed and in the other case no tag is removed.
	\item If $m$ tagged element, $a$ is compared with an element $b$ with no tag: Then if $a$ becomes larger then $m$ tag of $a$ is removed and in the other case no tag is removed.
\end{itemize}
Apart from the first case in all the above case at most $1$ tag is getting removed. For first case $2$ tags are removed. But at most $\lt\lfloor\frac{n}{2}\rt\rfloor$ many such comparisons possible. So the only case remaining is when a $M$ tagged element, $a$ is compared with a $m$ tagged element $b$. We want to claim if the adversary makes the $a$ larger than $b$ then no contradictions will arise.

We want to argue for this using the graphs. We have $n$ vertices which are basically the elements of the array. So for each comparison if $a_i$ and $a_j$ are compared we put a directed edge $a_i\to a_j \iff a_i>a_j$. This way we construct the graph $G=(V,E)$. Let ${G^{(t)}=(V,E^{(t)})}$ be the graph constructed by the comparisons which happened before the $t^{th}$ comparison in the algorithm. So in $G^{(0)}$, $E^{(0)}=\emptyset$. Now we know for any $t$, $G^{(t)}$ is a directed acyclic graph. So suppose at $t^{th}$ comparison an element $a$ with tag $M$ is compared with an element $b$ with tag $m$. Consider the weakly connected components of $G^{(t)}$. A weakly connected component is a subset of vertices such that for any two vertices $u,v$ in that set either there is a path from $u\rightsquigarrow v$ or there is a path from $v\rightsquigarrow u$. 

If $a$ and $b$ are in same weakly connected component then there is a path from $a\rightsquigarrow b$ or a path from $b\rightsquigarrow a$. If there is a path from $a\rightsquigarrow b$. Then naturally we conclude that $a>b$. So assume there is a path from $b\rightsquigarrow a$. In that case let the last edge of the path is $b'\to a$. Since the edge $b'\to a\in E^{(t)}$ they were compared before the $t^{th}$ comparison and in that comparison $a$ was smaller than $b'$. Then the $M$ tag of $a$ should have been removed but $a$ still has the tag. Contradiction. If $a$ and $b$ are in same weakly connected component then there is a path from $a\rightsquigarrow b$ and therefore $a>b$.

Now assume $a$ and $b$ are in different weakly connected component. Then there is no path from $a\rightsquigarrow b$ or $b\rightsquigarrow a$. Hence the adversary can make $a$ larger than $b$ and there will not be any contradictions. Cause making $a>b$ only affects the vertices which are reachable from $a$ and $b$. But $a$ is not reachable from the vertices reachable from $b$ and similarly $b$ is not reachable from the vertices reachable from $a$. Therefore the adversary can make $a$ larger than $b$ and then no tags will be removed.

So now we shown that in at most $\lt\lfloor \frac{n}2\rt\rfloor$ many comparisons $22$ tags will be removed. So the for the remaining at least $2n-2\lt\lfloor \frac{n}2\rt\rfloor$ tags for rest of the comparisons in each comparison at most $1$ tag is removed. And at the end $2$ tags are remaining. So the number of comparisons needed is:$$\lt[(2n-2)-2\lt\lfloor \frac{n}2\rt\rfloor\rt]+\lt\lfloor \frac{n}2\rt\rfloor=2n-\lt\lfloor \frac{n}2\rt\rfloor-2$$We will show that $2n-\lt\lfloor \frac{n}2\rt\rfloor=\lt\lceil\frac{3n}2\rt\rceil$. If $n=2k$ for some $k\in\bbN$ then $2n-\lt\lfloor \frac{n}2\rt\rfloor=4k-k=3k=\lt\lceil \frac{3n}2\rt\rceil$. If $n=2k+1$ for some $k\in\bbN$ then $2n-\lt\lfloor \frac{n}2\rt\rfloor=4k+2-k=3k+2$ and on the other hand $ \lt\lceil \frac{3n}2\rt\rceil=\lt\lceil \frac{6k+3}{2}\rt\rceil=\lt\lceil3k+\frac32\rt\rceil=3k+2$. So we have $2n-\lt\lfloor \frac{n}2\rt\rfloor=\lt\lceil\frac{3n}2\rt\rceil$. Hence at least $\lt\lceil\frac{3n}2\rt\rceil-2$ many comparisons are needed for any algorithm to find out the minimum and maximum element of an unsorted array of $n$ elements.
}\parinf

[I discussed the solution with Soumyadeep]

%%%%%%%%%%%%%%%%%%%%%%%%%%%%%%%%%%%%%%%%%%%%%%%%%%%%%%%%%%%%%%%%%%%%%%%%%%%%%%%%%%%%%%%%%%%%%%%%%%%%%%%%%%%%%%%%%%%%%%%%%%%%%%%%%%%%%%%%
% Problem 3
%%%%%%%%%%%%%%%%%%%%%%%%%%%%%%%%%%%%%%%%%%%%%%%%%%%%%%%%%%%%%%%%%%%%%%%%%%%%%%%%%%%%%%%%%%%%%%%%%%%%%%%%%%%%%%%%%%%%%%%%%%%%%%%%%%%%%%%%

\begin{problem}{%problem statement
		P4\hfill  (10 marks)
	}{p3% problem reference text
	}
Let $G=(V,E)$ be a directed acyclic graph $G=(V,E)$. Additionally, you are given a nonnegative, integral weight $w_e$ on each edge $e\in E$, and two special vertices $s,t\in V$. Give an algorithm to find a max-weight path from $s$ to $t$.
\end{problem}
\solve{
We can solve this problem using dynamic programing. Let $dist(v)$ denotes the weight of maximum weight path from $s\rightsquigarrow v$ for any $v\in V$. $dist(s)=0$. Now for any $v\in V$ we have the following relation: $$dist(v)=\max\limits_{u:(u,v)\in E}\{dist(u)+w(u,v)\}$$Hence we have the following algorithm:

\begin{algorithm}[H]
	\DontPrintSemicolon
	\KwIn{A directec acyclic graph $G=(V,E)$ with $2$ vertices $s,t$ and weights on edges $W$.}
	\KwOut{Maximum weight path from $s\rightsquigarrow t$}
	\Begin{
		$dist(s)\longleftarrow 0$, $s.parent\longleftarrow \prb{Null}$\;
		\For{$v\in V-\{s\}$}{$dist(v)\longleftarrow -\infty$\;
		$v.parent\longleftarrow \prb{Null}$}
		$U\longleftarrow \{s\}$\;
		\While{$U\neq \emptyset$}{
		$u\longleftarrow $ Extract first element of $U$\;
		\For{$(u,v)\in E$}{
			\If{$\prb{MaxPath}[v]\leq \prb{MaxPath}[u]+w(u,v)$}{$\prb{MaxPath}[v]\longleftarrow \prb{MaxPath}[u]+w(u,v)$\;
				$v.parent\longleftarrow u$}
			$U\longleftarrow U\cup \{v\}$\;
	}	
	}
$P\longleftarrow \emptyset$, $p\longleftarrow t$\;
\While{\prb{True}}{
	$P\longleftarrow P\cup \{p\}$\;
	\If{$p.parent ==\prb{Null}$}{\prb{Break}}
	$p\longleftarrow p.parent$\;
	
}
$P\longleftarrow \prb{Reverse}(P)$\;
\Return{$P$}
		
		
	}
\caption{\prb{Find-Max-Path}$(G,s,t,W)$}
\end{algorithm}\parinf

\textbf{Time Complexity:} The lines 2-3 takes $O(n)$ time and line 5 takes constant time. The while loop at line 7 picks a vertex and then goes through all its neighbors  and adds them to $U$. So the while loop visits at most every vertex and and every edges. In each iteration it takes constant time. Therefore the while loop takes $O(|V|+|E|)=O(n^2)$ time. Now for the while loop at line 14 it can be at most $n$ many iterations and in each iteration it takes constant time. The \prb{Reverse} function also take linear time. Therefore the algorithm runs in $O(n^2)$ time or $O(|V|+|E|)$ time.

}

\newpage

%%%%%%%%%%%%%%%%%%%%%%%%%%%%%%%%%%%%%%%%%%%%%%%%%%%%%%%%%%%%%%%%%%%%%%%%%%%%%%%%%%%%%%%%%%%%%%%%%%%%%%%%%%%%%%%%%%%%%%%%%%%%%%%%%%%%%%%%
% Problem 4
%%%%%%%%%%%%%%%%%%%%%%%%%%%%%%%%%%%%%%%%%%%%%%%%%%%%%%%%%%%%%%%%%%%%%%%%%%%%%%%%%%%%%%%%%%%%%%%%%%%%%%%%%%%%%%%%%%%%%%%%%%%%%%%%%%%%%%%%

\begin{problem}{%problem statement
		P5\hfill  (15 marks)
	}{p4% problem reference text
	}
Given a matroid $(S,\mcI)$, show that  $(S,\mcI')$ is also a matroid, where $A\in \mcI'$ if $S\setminus A$ contains a maximal independent in $\mcI$.
\end{problem}
\solve{
\begin{enumerate}[label=\bfseries\tiny\protect\circled{\small\arabic*}]
	\item Downward Closure: Let $A\in \mcI'$ and $B\subseteq A\implies S\setminus A\subseteq S\setminus B$. Since $A\in \mcI'$, $\exs\ X\in \mcI$ such that $X$ is a maximal independent set such that $X\subseteq S\setminus A$. Therefore we have $X\subseteq S\setminus A\subseteq S\setminus B$. Therefore $B\in \mcI'$. Hence $\mcI'$ follows the downward closure property.
	\item Exchange Property: Let $A,B\in\mcI'$ and $|B|<|A|$.  Let $X,Y\in\mcI$ be maximal independent sets such that $X\subseteq S\setminus A$ and $Y\subseteq S\setminus B$. We have $|X|=|Y|$. Now consider the set $X\setminus B\in \mcI$ as $X\in\mcI$. We can extend $X\setminus B$ to a maximal independent set $Z$ by adding elements from $Y$ not in $X\setminus Y$. Then let $Z=(X\setminus B)\sqcup T$ where $T\subseteq Y\setminus (X\setminus B)=Y\setminus X$. \parinn
	
	First of all we have $(X\setminus B)\cap (A-B)=\emptyset$ since $X\cap A=\emptyset$. Also we have $$X\cap B\subseteq (S\setminus A)\cap B=B\setminus A$$Now we want to claim $A\setminus B\not\subseteq T$ because if it is true then $\exs\ e\in A\setminus B$ such that $e\notin T$. Then $Z\subseteq S\setminus (B\cup \{e\})$. So suppose $A\setminus B\subseteq T$. Then $|T|\geq |A\setminus B|$. Now $$|B\setminus A|=|B|-|A\cap B|<|A|-|A\cap B|=|A\setminus B|$$Hence $$|T|\geq |A\setminus B|> |B-A|$$Now we have $$|Z|=|X\setminus B|+|T|>|X\setminus B|+|B\setminus A| \geq |X\setminus B|+|X\cap B|=|X| $$Therefore we have $|Z|>|X|$ but both are maximal independent sets of $(S,\mcI)$ and their size should be equal. Hence contradiction. Therefore $A\setminus B\not\subseteq T$. 
	
	Therefore $\exs\ e\in A\setminus B$ such that $e\notin T$. Then $e\notin Z\implies Z\subseteq S\setminus B\cup \{x\}$. Therefore $B\cup \{x\}\in \mcI'$. Hence $(S,\mcI')$ follows the exchange property.
\end{enumerate}
Therefore $(S,\mcI')$ is a matroid.
}


%%%%%%%%%%%%%%%%%%%%%%%%%%%%%%%%%%%%%%%%%%%%%%%%%%%%%%%%%%%%%%%%%%%%%%%%%%%%%%%%%%%%%%%%%%%%%%%%%%%%%%%%%%%%%%%%%%%%%%%%%%%%%%%%%%%%%%%%
% Problem 5
%%%%%%%%%%%%%%%%%%%%%%%%%%%%%%%%%%%%%%%%%%%%%%%%%%%%%%%%%%%%%%%%%%%%%%%%%%%%%%%%%%%%%%%%%%%%%%%%%%%%%%%%%%%%%%%%%%%%%%%%%%%%%%%%%%%%%%%%

\begin{problem}{%problem statement
	P6\hfill  (15 marks)
}{p5% problem reference text
}
In class, we showed that if $(S,\mcI)$ is a matroid, then for any nonnegative weights $w$ no the  elements of $S$, the greedy algorithm obtains a maximum weight independent set. Show that this is only true if $(S,\mcI)$ a matroid. That is, for a fixed downward-closed set system $(S,\mcI)$, if the greedy algorithm obtains a maximum weight element of $\mcI$ for every assignment of nonnegative weights to elements of $S$, then $(S,\mcI)$ is a matroid.
\end{problem}
\solve{ 
	Here we only have to prove that $(S,\mcI)$ follows the exchange property. So let $A,B\in \mcI$ and $|B|<|A|$. So we create a weight function $W$. Let $|S|=n$. Then $\forall\ b\in B$ we have  $W(b)=\frac1{n}$. And $\forall\ a\in A-B$, we have $W(a)=\frac1{n+1}$. Now $$W(B)\leq \frac{|B|}{n}$$Now suppose $|A|=k$, $|B|=l$ and $|A-B|=m$ then we have $n\geq k>l$ and $m>0$ and $|A\cap B|=k-m\leq l$. Hence $$ W(A)=\frac{k-m}{n}+\frac{m}{n+1}=\frac{(k-m)(n+1)+nm}{n(n+1)}=\frac{(n+1)k-m}{n(n+1)}$$ Now the algorithm sorts the elements by their weight and after sorting firstly the algorithm will encounter the elements of $B$ and then the elements of $A-B$. Since $B\in\mcI$ the algorithm will pick all the elements of $B$ and construct the set $B$ when it starts to encounter the elements of $A-B$. Now the algorithm picks any element from $A-B$ if there is already a set $S\in\mcI$ such that $W(S)>W(B)$. Now we claim $W(A)>W(B)$. 
	
	$$W(A)= \frac{(n+1)k-m}{n(n+1)}>\frac{l}{n}=W(B)\iff \frac{(n+1)k-m}{n+1}>l\iff {(n+1)k-m}>l(n+1)\iff (n+1)(k-l)>m$$The last inequality is true since $k>l$ and $m\leq n$. Hence we have a set which has weight greater than the weight of $B$. So the algorithm will pick an element $x$ from $A-B$ such that $B\cup \{x\}\in \mcI$.
	
	Hence $(S,\mcI)$ follows the base exchange property. Therefore $(S,\mcI)$ is a matroid. 
}


%%%%%%%%%%%%%%%%%%%%%%%%%%%%%%%%%%%%%%%%%%%%%%%%%%%%%%%%%%%%%%%%%%%%%%%%%%%%%%%%%%%%%%%%%%%%%%%%%%%%%%%%%%%%%%%%%%%%%%%%%%%%%%%%%%%%%%%%
% Problem 6
%%%%%%%%%%%%%%%%%%%%%%%%%%%%%%%%%%%%%%%%%%%%%%%%%%%%%%%%%%%%%%%%%%%%%%%%%%%%%%%%%%%%%%%%%%%%%%%%%%%%%%%%%%%%%%%%%%%%%%%%%%%%%%%%%%%%%%%%

\begin{problem}{%problem statement
	P7: Exercise 10.4-6 (on tree representations with pointers) from CLRS.\hfill  (10 marks)
}{p6% problem reference text
}

The \textit{left-child}, \textit{right-sibling} representation of an arbitrary rooted tree uses three
pointers in each node: \textit{left-child}, \textit{right-sibling}, and \textit{parent}. From any node, its
parent can be reached and identified in constant time and all its children can be
reached and identified in time linear in the number of children. Show how to use
only two pointers and one boolean value in each node so that the parent of a node
or all of its children can be reached and identified in time linear in the number of
children.
\end{problem}
\solve{
	Fir each node in the tree the we will keep a boolean value called \textit{isRightSibling}. If value of \textit{isRightSibling} is 1 then there is a right sibling of the current node and there is a \textit{right-sibling} pointer of the node and the node has not \textit{parent} pointer pointing towards its parent. If value of \textit{isRightSibling} is 0 then there is not \textit{right-sibling} pointer of the current node. Instead the node has a \textit{parent} pointer pointing towards its parent. 
	
	So the following are two algorithms for going to a parent of a given node and going to a  given specific child of a given node.\parinf
	
%\begin{minipage}{0.45\textwidth}
	\begin{algorithm}[H]
		\DontPrintSemicolon
		\Begin{
			\While{\prb{True}}{
				\If{$node.isRightSibling==1$}{$node\longleftarrow node.right-sibling$}
				\Else{\Return{$node.parent$}}
			}		
		}
		\caption{\prb{Parent}$(node)$}
	\end{algorithm}
%\end{minipage}\hfil
%\begin{minipage}{0.55\textwidth}
\begin{algorithm}[H]
	\DontPrintSemicolon
	\Begin{
		$\prb{Children}\longleftarrow\emptyset$\;
		$\prb{Children}\longleftarrow \prb{Children}\cup \{node.left-child\}$\;
		\While{$node.isRightSibling==1$}{
			$\prb{Children}\longleftarrow\prb{Children}\cup \{node.right-sibling\}$\;
			$node\longleftarrow node.right-sibling$\;
			
		}	
	\Return{\prb{Children}}	
	}
	\caption{\prb{Child}$(node)$}
\end{algorithm}

The algorithm for finding the parent goes through all the children of the parent of the given node to go to the right most sibling of the given node then returns the parent to the node. So it takes linear time in the number of children. The algorithm for finding all the children it goes to the left child of the given node and then goes through all the right siblings of the children one by one and then returns the array of all the children. Hence this algorithm also reaches all the children of the given node in linear time in the number of children.
%\end{minipage}

}


%%%%%%%%%%%%%%%%%%%%%%%%%%%%%%%%%%%%%%%%%%%%%%%%%%%%%%%%%%%%%%%%%%%%%%%%%%%%%%%%%%%%%%%%%%%%%%%%%%%%%%%%%%%%%%%%%%%%%%%%%%%%%%%%%%%%%%%%
% Problem 7
%%%%%%%%%%%%%%%%%%%%%%%%%%%%%%%%%%%%%%%%%%%%%%%%%%%%%%%%%%%%%%%%%%%%%%%%%%%%%%%%%%%%%%%%%%%%%%%%%%%%%%%%%%%%%%%%%%%%%%%%%%%%%%%%%%%%%%%%

\begin{problem}{%problem statement
		P8\hfill  (10 marks)
	}{p7% problem reference text
	}
Given a directed graph $G=(V,E)$ with weights on the edges, and which has a negative-weight directed cycle that is reachable from the source $s$, Give an efficient algorithm to list the vertices of such a cycle.
\end{problem}
\solve{
	If the graph if there is a negative-weight cycle reachable from $s$ then there is a vertex $j$ reachable from $s$ such that the shortest path from $j\rightsquigarrow j$ has negative weight. So we will use the Floyd Warshal Algorithm. But we will also keep track of the paths. We are assuming that $G$ is given as an adjacency matrix. 
	
	Here we introduce a $[n+1]\times [n]\times [n]$ $D=\lt(d_{i,j}^{(k)}\rt)$ where $k\in[n+1]$. So $d_{i,j}^{(k)}$ is the weight of the shortest path from $i$ to $j$ using the vertices $[k]$. Naturally we have the relation: $$d_{i,{j}}^{(k+1)}=\min \lt\{d_{i,{k+1}}^{(k)}+d_{{k+1},j}^{(k)},\; d_{i,{j}}^{(k)}  \rt\}$$Here we also need to keep the path. So when the minimum path uses the $k+1$ vertex then we just concatenate the paths from $i\rightsquigarrow k+1\rightsquigarrow j$ and otherwise the path is same as it was not using the vertex $k+1$. So here is the algorithm:
	
\begin{algorithm}[H]
\DontPrintSemicolon
\KwIn{A directec graph $G=(V=[n],E)$ with source vertex $s$ and weights on edges $W$ with promise that $G$ has a negative-weighted cycle reachable from $s$}
\KwOut{Find a negative-weighted cycle reachable from $s$}
\Begin{
Create a $[n+1]\times [n]\times [n]$ array $D=\lt(d_{i,j}^{(k)}\rt)$ for all $k\in[n+1]$, $i,j\in [n]$ with all entries $(\infty,\prb{Null})$\;
\For{$i,j\in[n]$}{$d^{(0)}_{i,j}=\big(A[i,j],[i,j]\big)$}

\For{$k=1,\dots, n$}{
	\For{$i=1,\dots, n$}{
	\For{$j=1,\dots, n$}{
		\If{$d_{i,j}^{(k-1)}[1]  >  d_{i,k}^{(k-1)}[1]+d_{k,j}^{(k-1)}[1]$ }{$ d_{i,j}^{(k)}[2]\longleftarrow d_{i,k}^{(k-1)}[2]\ ++\ d_{k,j}^{(k-1)}[2]$}
		\Else{{$ d_{i,j}^{(k)}[2]\longleftarrow d_{i,j}^{(k-1)}[2]$}}
}
}
}
$Visited\longleftarrow \{s\}$, $U\longleftarrow \{s\}$\;
Create a $n$ length array $VIS$ with all entries $0$\;
$VIS[s]=1$\;
\While{$U\neq \emptyset$ }{
	$u\longleftarrow $ Extract first element of $U$\;
	\For{$(u,v)\in E$}{
		\If{$VIS[v]==0$}{
			$U\longleftarrow U\cup\{v\}$\; 
			$Visited\longleftarrow Visited\cup \{v\}$\;
			$VIS[v]\longleftarrow 1$}
	}
	
}
\For{$v\in Visited$}{\If{$d_{v,v}^{(n)}[1]<0$}{\Return $d_{i,i}^{(n)}[2]$}}
}
\caption{\prb{Find-Negative-Cycle}$(A,s,W)$}
\end{algorithm}\parinf

\textbf{Time Complexity:} The line 2 takes $O(n^3)$ time to create the array. In line 3 the for loop has $n^2$ iterations where each iteration takes constant time. In line 5-7 it takes total $n^3$ iterations and each iteration takes constant time. So the for loops in line 5-7 in total takes $O(n^3)$ time. In line 12-14 it takes linear time. In line 15 the while loop and the for loop at line 17 goes through each vertex and then also goes through each neighbor of it. Hence there are total $O(|V|+|E|)=O(n^2)$ iterations each of which takes constant time. In line 22 the for loop has $O(n)$ many iterations at most and each iteration it takes constant time. Hence the algorithm runs in $O(n^3)$ time. 
}


%%%%%%%%%%%%%%%%%%%%%%%%%%%%%%%%%%%%%%%%%%%%%%%%%%%%%%%%%%%%%%%%%%%%%%%%%%%%%%%%%%%%%%%%%%%%%%%%%%%%%%%%%%%%%%%%%%%%%%%%%%%%%%%%%%%%%%%%
% Problem 8
%%%%%%%%%%%%%%%%%%%%%%%%%%%%%%%%%%%%%%%%%%%%%%%%%%%%%%%%%%%%%%%%%%%%%%%%%%%%%%%%%%%%%%%%%%%%%%%%%%%%%%%%%%%%%%%%%%%%%%%%%%%%%%%%%%%%%%%%

\begin{problem}{%problem statement
	P9\hfill  (15 marks)
}{p8% problem reference text
}
Let us modify the ``cut rule" (in the implementation of decrease-key operation for a Fibonacci heap) to cut a node $x$ from its parent as soon as it loses its 3rd child. Recall that the rule that we studied in class was when a node loses its 2nd child. Can we still upper bound the maximum degree of a node of an $n-$node Fibonacci heap with $O(\log n)$.
\end{problem}
\solve{
}


%%%%%%%%%%%%%%%%%%%%%%%%%%%%%%%%%%%%%%%%%%%%%%%%%%%%%%%%%%%%%%%%%%%%%%%%%%%%%%%%%%%%%%%%%%%%%%%%%%%%%%%%%%%%%%%%%%%%%%%%%%%%%%%%%%%%%%%%
% Problem 9
%%%%%%%%%%%%%%%%%%%%%%%%%%%%%%%%%%%%%%%%%%%%%%%%%%%%%%%%%%%%%%%%%%%%%%%%%%%%%%%%%%%%%%%%%%%%%%%%%%%%%%%%%%%%%%%%%%%%%%%%%%%%%%%%%%%%%%%%

\begin{problem}{%problem statement
	P10\hfill  (15 marks)
}{p9% problem reference text
}
The following are Fibonacci-heap operations: \textit{extract-min}$(\cdot)$, \textit{decrease-key}$(\cdot,\cdot)$, and also \textit{create-node}$(x,k)$ which creates a node $x$ in th root list with key value $k$. Show a sequence of these operations that results in a Fibonacci heap consisting of just one tree that is a linear chain of $n$ nodes.
\end{problem}
\solve{
}
\end{document}
