\documentclass[a4paper, 11pt]{article}
\usepackage{comment} % enables the use of multi-line comments (\ifx \fi) 
\usepackage{fullpage} % changes the margin
\usepackage[a4paper, total={7in, 10in}]{geometry}
\usepackage{amsmath,mathtools,mathdots}
\usepackage{amssymb,amsthm}  % assumes amsmath package installed
\usepackage{float}
\usepackage{xcolor}
\usepackage{mdframed}
\usepackage[shortlabels]{enumitem}
\usepackage{indentfirst}
\usepackage{hyperref}
\hypersetup{
	colorlinks=true,
	linkcolor=doc!80,
	citecolor=myr,
	filecolor=myr,      
	urlcolor=doc!80,
	pdftitle={Assignment}, %%%%%%%%%%%%%%%%   WRITE ASSIGNMENT PDF NAME  %%%%%%%%%%%%%%%%%%%%
}
\usepackage[most,many,breakable]{tcolorbox}
\usepackage{tikz}
\usepackage{caption}
\usepackage{kpfonts}
\usepackage{libertine}
\usepackage{physics}
\usepackage[ruled,vlined,linesnumbered]{algorithm2e}
\usepackage{mathrsfs}
\usepackage{tikz-cd}
\usepackage{float}

\definecolor{mytheorembg}{HTML}{F2F2F9}
\definecolor{mytheoremfr}{HTML}{00007B}
\definecolor{doc}{RGB}{0,60,110}
\definecolor{myg}{RGB}{56, 140, 70}
\definecolor{myb}{RGB}{45, 111, 177}
\definecolor{myr}{RGB}{199, 68, 64}

\usetikzlibrary{decorations.pathreplacing,angles,quotes,patterns}
\definecolor{mytheorembg}{HTML}{F2F2F9}
\definecolor{mytheoremfr}{HTML}{00007B}
\definecolor{doc}{RGB}{0,60,110}
\definecolor{myg}{RGB}{56, 140, 70}
\definecolor{myb}{RGB}{45, 111, 177}
\definecolor{myr}{RGB}{199, 68, 64}

\tcbuselibrary{theorems,skins,hooks}
\newtcbtheorem{problem}{Problem}
{%
	enhanced,
	breakable,
	colback = mytheorembg,
	frame hidden,
	boxrule = 0sp,
	borderline west = {2pt}{0pt}{mytheoremfr},
	arc=5pt,
	detach title,
	before upper = \tcbtitle\par\smallskip,
	coltitle = mytheoremfr,
	fonttitle = \bfseries\sffamily,
	description font = \mdseries,
	separator sign none,
	segmentation style={solid, mytheoremfr},
}
{p}

\newtheorem{lemma}{Lemma}
\renewenvironment{proof}{\noindent{\it \textbf{Proof:}}\hspace*{1em}}{\qed\bigskip\\}
% To give references for any problem use like this
% suppose the problem number is p3 then 2 options either 
% \hyperref[p:p3]{<text you want to use to hyperlink> \ref{p:p3}}
%                  or directly 
%                   \ref{p:p3}



\input{../../letterfonts}

\input{../../macros}

\setlength{\parindent}{0pt}

%%%%%%%%%%%%%%%%%%%%%%%%%%%%%%%%%%%%%%%%%%%%%%%%%%%%%%%%%%%%%%%%%%%%%%%%%%%%%%%%%%%%%%%%%%%%%%%%%%%%%%%%%%%%%%%%%%%%%%%%%%%%%%%%%%%%%%%%

\begin{document}
	
	%%%%%%%%%%%%%%%%%%%%%%%%%%%%%%%%%%%%%%%%%%%%%%%%%%%%%%%%%%%%%%%%%%%%%%%%%%%%%%%%%%%%%%%%%%%%%%%%%%%%%%%%%%%%%%%%%%%%%%%%%%%%%%%%%%%%%%%%
	
	\textsf{\noindent \large\textbf{Soham Chatterjee} \hfill \textbf{Assignment - 1}\\
		Email: \href{soham.chatterjee@tifr.res.in}{soham.chatterjee@tifr.res.in} \hfill Dept: STCS\\
		\normalsize Course: Algorithms \hfill Date: \today}
	
%%%%%%%%%%%%%%%%%%%%%%%%%%%%%%%%%%%%%%%%%%%%%%%%%%%%%%%%%%%%%%%%%%%%%%%%%%%%%%%%%%%%%%%%%%%%%%%%%%%%%%%%%%%%%%%%%%%%%%%%%%%%%%%%%%%%%%%%
% Problem 1
%%%%%%%%%%%%%%%%%%%%%%%%%%%%%%%%%%%%%%%%%%%%%%%%%%%%%%%%%%%%%%%%%%%%%%%%%%%%%%%%%%%%%%%%%%%%%%%%%%%%%%%%%%%%%%%%%%%%%%%%%%%%%%%%%%%%%%%%
	
\begin{problem}{%problem statement
		P3\hfill  (15 marks)
	}{p1% problem reference text
}
Solve the recurrences: \begin{enumerate}[label=(\roman*)]
	\item $T(n)=2 T(n / 2)+n \log n$, 
	\item $T(n)=7 T(n / 3)+n^2$, 
	\item $T(n)=\sqrt{n} T(\sqrt{n})+n$.
\end{enumerate}

\end{problem}
\solve{	
}
%%%%%%%%%%%%%%%%%%%%%%%%%%%%%%%%%%%%%%%%%%%%%%%%%%%%%%%%%%%%%%%%%%%%%%%%%%%%%%%%%%%%%%%%%%%%%%%%%%%%%%%%%%%%%%%%%%%%%%%%%%%%%%%%%%%%%%%%
% Problem 2
%%%%%%%%%%%%%%%%%%%%%%%%%%%%%%%%%%%%%%%%%%%%%%%%%%%%%%%%%%%%%%%%%%%%%%%%%%%%%%%%%%%%%%%%%%%%%%%%%%%%%%%%%%%%%%%%%%%%%%%%%%%%%%%%%%%%%%%%

\begin{problem}{%problem statement
		P4\hfill  (5 marks)
	}{p2% problem reference text
	}
 Give the best upper bounds you can on the $n$th Fibonacci number $F_n$, where $F_n=F_{n-1}+F_{n-2}$ and $F_1=F_2=1$ are conditionally independent given $C$ if and only if $A$ and $B$ are independent.
\end{problem}
\solve{	
}

%%%%%%%%%%%%%%%%%%%%%%%%%%%%%%%%%%%%%%%%%%%%%%%%%%%%%%%%%%%%%%%%%%%%%%%%%%%%%%%%%%%%%%%%%%%%%%%%%%%%%%%%%%%%%%%%%%%%%%%%%%%%%%%%%%%%%%%%
% Problem 3
%%%%%%%%%%%%%%%%%%%%%%%%%%%%%%%%%%%%%%%%%%%%%%%%%%%%%%%%%%%%%%%%%%%%%%%%%%%%%%%%%%%%%%%%%%%%%%%%%%%%%%%%%%%%%%%%%%%%%%%%%%%%%%%%%%%%%%%%

\begin{problem}{%problem statement
		P5\hfill  (10 marks)
	}{p3% problem reference text
	}
Consider two sets $A$ and $B$, each having $n$ integers in the range from 0 to $10 n$. We wish to compute the Cartesian sum of $A$ and $B$, defined by
$$
C=\{x+y: x \in A, y \in B\}
$$

Note that the integers in $C$ are in the range $0$ to $20n$ . We want to find the elements in $C$ and the number of times each element of $C$ is realized as a sum of elements in $A$ and $B$. Give an algorithm that solves the problem in $O(n \log n)$ time, and prove correctness.

\end{problem}
\solve{
}



%%%%%%%%%%%%%%%%%%%%%%%%%%%%%%%%%%%%%%%%%%%%%%%%%%%%%%%%%%%%%%%%%%%%%%%%%%%%%%%%%%%%%%%%%%%%%%%%%%%%%%%%%%%%%%%%%%%%%%%%%%%%%%%%%%%%%%%%
% Problem 4
%%%%%%%%%%%%%%%%%%%%%%%%%%%%%%%%%%%%%%%%%%%%%%%%%%%%%%%%%%%%%%%%%%%%%%%%%%%%%%%%%%%%%%%%%%%%%%%%%%%%%%%%%%%%%%%%%%%%%%%%%%%%%%%%%%%%%%%%

\begin{problem}{%problem statement
		P6\hfill  (20 marks)
	}{p4% problem reference text
	}
	Define $[n]:=\{1,2, \ldots, n\}$. You are given $n$, and oracle access to a function $f:[n] \times[n] \rightarrow[n] \times[n]$ that takes as input two positive integers of value at most $n$, and returns two positive integers of value at most $n$. Let $f_1\left(x_1, x_2\right)$ and $f_2\left(x_1, x_2\right)$ be the first and second coordinates of $f\left(x_1, x_2\right)$, respectively. You are also told that $f_i$ is monotone nondecreasing in coordinate $i$ when coordinate $3-i$ is kept fixed, and monotone nonincreasing in coordinate $3-i$ when coordinate $i$ is kept fixed. That is, given $x_1 \leq x_1^{\prime} \in[n]$ and $x_2 \leq x_2^{\prime} \in[n]$, $f_1\left(x_1, x_2\right) \leq f_1\left(x_1^{\prime}, x_2\right)$, and $f_1\left(x_1, x_2\right) \geq f_1\left(x_1, x_2^{\prime}\right)$. Similarly, $f_2\left(x_1, x_2\right) \geq f_2\left(x_1^{\prime}, x_2\right)$, and $f_2\left(x_1, x_2\right) \leq f_2\left(x_1, x_2^{\prime}\right)$.\parinn
	
	The problem is to find a fixed point of the function, i.e., values $x_1, x_2 \in[n]$ so that $f\left(x_1, x_2\right)=$ $\left(x_1, x_2\right)$. Give an algorithm that given $n$ and oracle access to such a function $f$, finds a fixed point of $f$ in time $O(\operatorname{poly}(\log n))$. You must also give a proof of correctness, and running time analysis.
\end{problem}
\solve{
	
}
%%%%%%%%%%%%%%%%%%%%%%%%%%%%%%%%%%%%%%%%%%%%%%%%%%%%%%%%%%%%%%%%%%%%%%%%%%%%%%%%%%%%%%%%%%%%%%%%%%%%%%%%%%%%%%%%%%%%%%%%%%%%%%%%%%%%%%%%
% Problem 5
%%%%%%%%%%%%%%%%%%%%%%%%%%%%%%%%%%%%%%%%%%%%%%%%%%%%%%%%%%%%%%%%%%%%%%%%%%%%%%%%%%%%%%%%%%%%%%%%%%%%%%%%%%%%%%%%%%%%%%%%%%%%%%%%%%%%%%%%

\begin{problem}{%problem statement
	P7\hfill  (15 marks)
}{p5% problem reference text
}
A palindrome is a nonempty string over some alphabet that reads the same forward and backward. Examples of palindromes are all strings of length 1, civic, racecar, and aibohphobia. Give an efficient algorithm, with proof of correctness and run-time analysis, to find the longest palindrome that is a subsequence of a given input string. For example, given the input string character, your algorithm should return carac.
\end{problem}
\solve{

}
%%%%%%%%%%%%%%%%%%%%%%%%%%%%%%%%%%%%%%%%%%%%%%%%%%%%%%%%%%%%%%%%%%%%%%%%%%%%%%%%%%%%%%%%%%%%%%%%%%%%%%%%%%%%%%%%%%%%%%%%%%%%%%%%%%%%%%%%
% Problem 6
%%%%%%%%%%%%%%%%%%%%%%%%%%%%%%%%%%%%%%%%%%%%%%%%%%%%%%%%%%%%%%%%%%%%%%%%%%%%%%%%%%%%%%%%%%%%%%%%%%%%%%%%%%%%%%%%%%%%%%%%%%%%%%%%%%%%%%%%

\begin{problem}{%problem statement
	P8\hfill  (25 marks)
}{p6% problem reference text
}
The purpose of this question is to extend the closest-points algorithm seen in the first lecture, to give an $O\left(n \log ^2 n\right)$ algorithm for finding the closest pair of points in 3 dimensions. All points in this question are in $\mathbb{R}^3$.
\begin{enumerate}[label=(\alph*)]
	\item (5 marks) Prove that, if all points are at least distance $\delta$ apart, a cube with each dimension of size $2 \delta$ contains at most a constant (say $k$ ) number of points.
	\item (10 marks) You are now given 2 sets of points $S_1$ and $S_2$, each containing $n$ points. The distance between any pair of points in $S_1$ is at least $\delta$, and further, each point in $S_1$ has $z$-coordinate in $[0, \delta]$. Similarly, the distance between any pair of points in $S_2$ is at least $\delta$, and each point in $S_2$ has $z$-coordinate in $[-\delta, 0]$.
	
Extend the algorithm discussed in class to give an $O(n \log n)$-time algorithm for finding the closest pair of points in $S_1 \cup S_2$. Note that, by the first part of the question, any cube with each dimension at most $2 \delta$, contains at most $2 k$ points from $S_1 \cup S_2$.
\item  (10 marks) Given a set $S$ of $n$ points in $\mathbb{R}^3$, now give an $O\left(n \log ^2 n\right)$-time algorithm to find the closest pair of points.

\end{enumerate}

\end{problem}
\solve{

}
%%%%%%%%%%%%%%%%%%%%%%%%%%%%%%%%%%%%%%%%%%%%%%%%%%%%%%%%%%%%%%%%%%%%%%%%%%%%%%%%%%%%%%%%%%%%%%%%%%%%%%%%%%%%%%%%%%%%%%%%%%%%%%%%%%%%%%%%
% Problem 7
%%%%%%%%%%%%%%%%%%%%%%%%%%%%%%%%%%%%%%%%%%%%%%%%%%%%%%%%%%%%%%%%%%%%%%%%%%%%%%%%%%%%%%%%%%%%%%%%%%%%%%%%%%%%%%%%%%%%%%%%%%%%%%%%%%%%%%%%

\begin{problem}{%problem statement
		P9\hfill  (10 marks)
	}{p7% problem reference text
	}
This problem relates to one of the questions asked in class. For any $p, q \geq 1$, and any points $x, y$, and $z \in \mathbb{R}^2$, prove or disprove the following:
		$$
		\|x-y\|_p \leq\|x-z\|_p \Leftrightarrow\|x-y\|_q \leq\|x-z\|_q
		$$

	
	That is, prove or disprove that $y$ is closer to $x$ than $z$ in the $L_p$ distance metric if and only if it is closer to $x$ in the $L_q$ distance metric
	As usual, $\|x-y\|_p=\left(\left(x_1-y_1\right)^p+\left(x_2-y_2\right)^p\right)^{1 / p}$.
\end{problem}
\solve{
	
}
\end{document}
