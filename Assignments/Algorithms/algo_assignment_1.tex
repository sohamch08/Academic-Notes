\documentclass[a4paper, 11pt]{article}
\usepackage{comment} % enables the use of multi-line comments (\ifx \fi) 
\usepackage{fullpage} % changes the margin
\usepackage[a4paper, total={7in, 10in}]{geometry}
\usepackage{amsmath,mathtools,mathdots}
\usepackage{amssymb,amsthm}  % assumes amsmath package installed
\usepackage{float}
\usepackage{xcolor}
\usepackage{mdframed}
\usepackage[shortlabels]{enumitem}
\usepackage{indentfirst}
\usepackage{hyperref}
\hypersetup{
	colorlinks=true,
	linkcolor=doc!80,
	citecolor=myr,
	filecolor=myr,      
	urlcolor=doc!80,
	pdftitle={Assignment}, %%%%%%%%%%%%%%%%   WRITE ASSIGNMENT PDF NAME  %%%%%%%%%%%%%%%%%%%%
}
\usepackage[most,many,breakable]{tcolorbox}
\usepackage{tikz}
\usepackage{caption}
\usepackage{kpfonts}
\usepackage{libertine}
\usepackage{physics}
\usepackage[ruled,vlined,linesnumbered]{algorithm2e}
\usepackage{mathrsfs}
\usepackage{tikz-cd}
\usepackage{float}

\definecolor{mytheorembg}{HTML}{F2F2F9}
\definecolor{mytheoremfr}{HTML}{00007B}
\definecolor{doc}{RGB}{0,60,110}
\definecolor{myg}{RGB}{56, 140, 70}
\definecolor{myb}{RGB}{45, 111, 177}
\definecolor{myr}{RGB}{199, 68, 64}

\usetikzlibrary{decorations.pathreplacing,angles,quotes,patterns}
\definecolor{mytheorembg}{HTML}{F2F2F9}
\definecolor{mytheoremfr}{HTML}{00007B}
\definecolor{doc}{RGB}{0,60,110}
\definecolor{myg}{RGB}{56, 140, 70}
\definecolor{myb}{RGB}{45, 111, 177}
\definecolor{myr}{RGB}{199, 68, 64}
\newcounter{problem}
\tcbuselibrary{theorems,skins,hooks}
\newtcbtheorem[use counter=problem]{problem}{Problem}
{%
	enhanced,
	breakable,
	colback = mytheorembg,
	frame hidden,
	boxrule = 0sp,
	borderline west = {2pt}{0pt}{mytheoremfr},
	arc=5pt,
	detach title,
	before upper = \tcbtitle\par\smallskip,
	coltitle = mytheoremfr,
	fonttitle = \bfseries\sffamily,
	description font = \mdseries,
	separator sign none,
	segmentation style={solid, mytheoremfr},
}
{p}

\newtheorem{lemma}{Lemma}
\renewenvironment{proof}{\noindent{\it \textbf{Proof:}}\hspace*{1em}}{\qed\bigskip\\}
% To give references for any problem use like this
% suppose the problem number is p3 then 2 options either 
% \hyperref[p:p3]{<text you want to use to hyperlink> \ref{p:p3}}
%                  or directly 
%                   \ref{p:p3}



%---------------------------------------
% BlackBoard Math Fonts :-
%---------------------------------------

%Captital Letters
\newcommand{\bbA}{\mathbb{A}}	\newcommand{\bbB}{\mathbb{B}}
\newcommand{\bbC}{\mathbb{C}}	\newcommand{\bbD}{\mathbb{D}}
\newcommand{\bbE}{\mathbb{E}}	\newcommand{\bbF}{\mathbb{F}}
\newcommand{\bbG}{\mathbb{G}}	\newcommand{\bbH}{\mathbb{H}}
\newcommand{\bbI}{\mathbb{I}}	\newcommand{\bbJ}{\mathbb{J}}
\newcommand{\bbK}{\mathbb{K}}	\newcommand{\bbL}{\mathbb{L}}
\newcommand{\bbM}{\mathbb{M}}	\newcommand{\bbN}{\mathbb{N}}
\newcommand{\bbO}{\mathbb{O}}	\newcommand{\bbP}{\mathbb{P}}
\newcommand{\bbQ}{\mathbb{Q}}	\newcommand{\bbR}{\mathbb{R}}
\newcommand{\bbS}{\mathbb{S}}	\newcommand{\bbT}{\mathbb{T}}
\newcommand{\bbU}{\mathbb{U}}	\newcommand{\bbV}{\mathbb{V}}
\newcommand{\bbW}{\mathbb{W}}	\newcommand{\bbX}{\mathbb{X}}
\newcommand{\bbY}{\mathbb{Y}}	\newcommand{\bbZ}{\mathbb{Z}}

%---------------------------------------
% MathCal Fonts :-
%---------------------------------------

%Captital Letters
\newcommand{\mcA}{\mathcal{A}}	\newcommand{\mcB}{\mathcal{B}}
\newcommand{\mcC}{\mathcal{C}}	\newcommand{\mcD}{\mathcal{D}}
\newcommand{\mcE}{\mathcal{E}}	\newcommand{\mcF}{\mathcal{F}}
\newcommand{\mcG}{\mathcal{G}}	\newcommand{\mcH}{\mathcal{H}}
\newcommand{\mcI}{\mathcal{I}}	\newcommand{\mcJ}{\mathcal{J}}
\newcommand{\mcK}{\mathcal{K}}	\newcommand{\mcL}{\mathcal{L}}
\newcommand{\mcM}{\mathcal{M}}	\newcommand{\mcN}{\mathcal{N}}
\newcommand{\mcO}{\mathcal{O}}	\newcommand{\mcP}{\mathcal{P}}
\newcommand{\mcQ}{\mathcal{Q}}	\newcommand{\mcR}{\mathcal{R}}
\newcommand{\mcS}{\mathcal{S}}	\newcommand{\mcT}{\mathcal{T}}
\newcommand{\mcU}{\mathcal{U}}	\newcommand{\mcV}{\mathcal{V}}
\newcommand{\mcW}{\mathcal{W}}	\newcommand{\mcX}{\mathcal{X}}
\newcommand{\mcY}{\mathcal{Y}}	\newcommand{\mcZ}{\mathcal{Z}}



%---------------------------------------
% Bold Math Fonts :-
%---------------------------------------

%Captital Letters
\newcommand{\bmA}{\boldsymbol{A}}	\newcommand{\bmB}{\boldsymbol{B}}
\newcommand{\bmC}{\boldsymbol{C}}	\newcommand{\bmD}{\boldsymbol{D}}
\newcommand{\bmE}{\boldsymbol{E}}	\newcommand{\bmF}{\boldsymbol{F}}
\newcommand{\bmG}{\boldsymbol{G}}	\newcommand{\bmH}{\boldsymbol{H}}
\newcommand{\bmI}{\boldsymbol{I}}	\newcommand{\bmJ}{\boldsymbol{J}}
\newcommand{\bmK}{\boldsymbol{K}}	\newcommand{\bmL}{\boldsymbol{L}}
\newcommand{\bmM}{\boldsymbol{M}}	\newcommand{\bmN}{\boldsymbol{N}}
\newcommand{\bmO}{\boldsymbol{O}}	\newcommand{\bmP}{\boldsymbol{P}}
\newcommand{\bmQ}{\boldsymbol{Q}}	\newcommand{\bmR}{\boldsymbol{R}}
\newcommand{\bmS}{\boldsymbol{S}}	\newcommand{\bmT}{\boldsymbol{T}}
\newcommand{\bmU}{\boldsymbol{U}}	\newcommand{\bmV}{\boldsymbol{V}}
\newcommand{\bmW}{\boldsymbol{W}}	\newcommand{\bmX}{\boldsymbol{X}}
\newcommand{\bmY}{\boldsymbol{Y}}	\newcommand{\bmZ}{\boldsymbol{Z}}
%Small Letters
\newcommand{\bma}{\boldsymbol{a}}	\newcommand{\bmb}{\boldsymbol{b}}
\newcommand{\bmc}{\boldsymbol{c}}	\newcommand{\bmd}{\boldsymbol{d}}
\newcommand{\bme}{\boldsymbol{e}}	\newcommand{\bmf}{\boldsymbol{f}}
\newcommand{\bmg}{\boldsymbol{g}}	\newcommand{\bmh}{\boldsymbol{h}}
\newcommand{\bmi}{\boldsymbol{i}}	\newcommand{\bmj}{\boldsymbol{j}}
\newcommand{\bmk}{\boldsymbol{k}}	\newcommand{\bml}{\boldsymbol{l}}
\newcommand{\bmm}{\boldsymbol{m}}	\newcommand{\bmn}{\boldsymbol{n}}
\newcommand{\bmo}{\boldsymbol{o}}	\newcommand{\bmp}{\boldsymbol{p}}
\newcommand{\bmq}{\boldsymbol{q}}	\newcommand{\bmr}{\boldsymbol{r}}
\newcommand{\bms}{\boldsymbol{s}}	\newcommand{\bmt}{\boldsymbol{t}}
\newcommand{\bmu}{\boldsymbol{u}}	\newcommand{\bmv}{\boldsymbol{v}}
\newcommand{\bmw}{\boldsymbol{w}}	\newcommand{\bmx}{\boldsymbol{x}}
\newcommand{\bmy}{\boldsymbol{y}}	\newcommand{\bmz}{\boldsymbol{z}}


%---------------------------------------
% Scr Math Fonts :-
%---------------------------------------

\newcommand{\sA}{{\mathscr{A}}}   \newcommand{\sB}{{\mathscr{B}}}
\newcommand{\sC}{{\mathscr{C}}}   \newcommand{\sD}{{\mathscr{D}}}
\newcommand{\sE}{{\mathscr{E}}}   \newcommand{\sF}{{\mathscr{F}}}
\newcommand{\sG}{{\mathscr{G}}}   \newcommand{\sH}{{\mathscr{H}}}
\newcommand{\sI}{{\mathscr{I}}}   \newcommand{\sJ}{{\mathscr{J}}}
\newcommand{\sK}{{\mathscr{K}}}   \newcommand{\sL}{{\mathscr{L}}}
\newcommand{\sM}{{\mathscr{M}}}   \newcommand{\sN}{{\mathscr{N}}}
\newcommand{\sO}{{\mathscr{O}}}   \newcommand{\sP}{{\mathscr{P}}}
\newcommand{\sQ}{{\mathscr{Q}}}   \newcommand{\sR}{{\mathscr{R}}}
\newcommand{\sS}{{\mathscr{S}}}   \newcommand{\sT}{{\mathscr{T}}}
\newcommand{\sU}{{\mathscr{U}}}   \newcommand{\sV}{{\mathscr{V}}}
\newcommand{\sW}{{\mathscr{W}}}   \newcommand{\sX}{{\mathscr{X}}}
\newcommand{\sY}{{\mathscr{Y}}}   \newcommand{\sZ}{{\mathscr{Z}}}


%---------------------------------------
% Math Fraktur Font
%---------------------------------------

%Captital Letters
\newcommand{\mfA}{\mathfrak{A}}	\newcommand{\mfB}{\mathfrak{B}}
\newcommand{\mfC}{\mathfrak{C}}	\newcommand{\mfD}{\mathfrak{D}}
\newcommand{\mfE}{\mathfrak{E}}	\newcommand{\mfF}{\mathfrak{F}}
\newcommand{\mfG}{\mathfrak{G}}	\newcommand{\mfH}{\mathfrak{H}}
\newcommand{\mfI}{\mathfrak{I}}	\newcommand{\mfJ}{\mathfrak{J}}
\newcommand{\mfK}{\mathfrak{K}}	\newcommand{\mfL}{\mathfrak{L}}
\newcommand{\mfM}{\mathfrak{M}}	\newcommand{\mfN}{\mathfrak{N}}
\newcommand{\mfO}{\mathfrak{O}}	\newcommand{\mfP}{\mathfrak{P}}
\newcommand{\mfQ}{\mathfrak{Q}}	\newcommand{\mfR}{\mathfrak{R}}
\newcommand{\mfS}{\mathfrak{S}}	\newcommand{\mfT}{\mathfrak{T}}
\newcommand{\mfU}{\mathfrak{U}}	\newcommand{\mfV}{\mathfrak{V}}
\newcommand{\mfW}{\mathfrak{W}}	\newcommand{\mfX}{\mathfrak{X}}
\newcommand{\mfY}{\mathfrak{Y}}	\newcommand{\mfZ}{\mathfrak{Z}}
%Small Letters
\newcommand{\mfa}{\mathfrak{a}}	\newcommand{\mfb}{\mathfrak{b}}
\newcommand{\mfc}{\mathfrak{c}}	\newcommand{\mfd}{\mathfrak{d}}
\newcommand{\mfe}{\mathfrak{e}}	\newcommand{\mff}{\mathfrak{f}}
\newcommand{\mfg}{\mathfrak{g}}	\newcommand{\mfh}{\mathfrak{h}}
\newcommand{\mfi}{\mathfrak{i}}	\newcommand{\mfj}{\mathfrak{j}}
\newcommand{\mfk}{\mathfrak{k}}	\newcommand{\mfl}{\mathfrak{l}}
\newcommand{\mfm}{\mathfrak{m}}	\newcommand{\mfn}{\mathfrak{n}}
\newcommand{\mfo}{\mathfrak{o}}	\newcommand{\mfp}{\mathfrak{p}}
\newcommand{\mfq}{\mathfrak{q}}	\newcommand{\mfr}{\mathfrak{r}}
\newcommand{\mfs}{\mathfrak{s}}	\newcommand{\mft}{\mathfrak{t}}
\newcommand{\mfu}{\mathfrak{u}}	\newcommand{\mfv}{\mathfrak{v}}
\newcommand{\mfw}{\mathfrak{w}}	\newcommand{\mfx}{\mathfrak{x}}
\newcommand{\mfy}{\mathfrak{y}}	\newcommand{\mfz}{\mathfrak{z}}

%---------------------------------------
% Bar
%---------------------------------------

%Captital Letters
\newcommand{\ovA}{\overline{A}}	\newcommand{\ovB}{\overline{B}}
\newcommand{\ovC}{\overline{C}}	\newcommand{\ovD}{\overline{D}}
\newcommand{\ovE}{\overline{E}}	\newcommand{\ovF}{\overline{F}}
\newcommand{\ovG}{\overline{G}}	\newcommand{\ovH}{\overline{H}}
\newcommand{\ovI}{\overline{I}}	\newcommand{\ovJ}{\overline{J}}
\newcommand{\ovK}{\overline{K}}	\newcommand{\ovL}{\overline{L}}
\newcommand{\ovM}{\overline{M}}	\newcommand{\ovN}{\overline{N}}
\newcommand{\ovO}{\overline{O}}	\newcommand{\ovP}{\overline{P}}
\newcommand{\ovQ}{\overline{Q}}	\newcommand{\ovR}{\overline{R}}
\newcommand{\ovS}{\overline{S}}	\newcommand{\ovT}{\overline{T}}
\newcommand{\ovU}{\overline{U}}	\newcommand{\ovV}{\overline{V}}
\newcommand{\ovW}{\overline{W}}	\newcommand{\ovX}{\overline{X}}
\newcommand{\ovY}{\overline{Y}}	\newcommand{\ovZ}{\overline{Z}}
%Small Letters
\newcommand{\ova}{\overline{a}}	\newcommand{\ovb}{\overline{b}}
\newcommand{\ovc}{\overline{c}}	\newcommand{\ovd}{\overline{d}}
\newcommand{\ove}{\overline{e}}	\newcommand{\ovf}{\overline{f}}
\newcommand{\ovg}{\overline{g}}	\newcommand{\ovh}{\overline{h}}
\newcommand{\ovi}{\overline{i}}	\newcommand{\ovj}{\overline{j}}
\newcommand{\ovk}{\overline{k}}	\newcommand{\ovl}{\overline{l}}
\newcommand{\ovm}{\overline{m}}	\newcommand{\ovn}{\overline{n}}
\newcommand{\ovo}{\overline{o}}	\newcommand{\ovp}{\overline{p}}
\newcommand{\ovq}{\overline{q}}	\newcommand{\ovr}{\overline{r}}
\newcommand{\ovs}{\overline{s}}	\newcommand{\ovt}{\overline{t}}
\newcommand{\ovu}{\overline{u}}	\newcommand{\ovv}{\overline{v}}
\newcommand{\ovw}{\overline{w}}	\newcommand{\ovx}{\overline{x}}
\newcommand{\ovy}{\overline{y}}	\newcommand{\ovz}{\overline{z}}

%---------------------------------------
% Tilde
%---------------------------------------

%Captital Letters
\newcommand{\tdA}{\tilde{A}}	\newcommand{\tdB}{\tilde{B}}
\newcommand{\tdC}{\tilde{C}}	\newcommand{\tdD}{\tilde{D}}
\newcommand{\tdE}{\tilde{E}}	\newcommand{\tdF}{\tilde{F}}
\newcommand{\tdG}{\tilde{G}}	\newcommand{\tdH}{\tilde{H}}
\newcommand{\tdI}{\tilde{I}}	\newcommand{\tdJ}{\tilde{J}}
\newcommand{\tdK}{\tilde{K}}	\newcommand{\tdL}{\tilde{L}}
\newcommand{\tdM}{\tilde{M}}	\newcommand{\tdN}{\tilde{N}}
\newcommand{\tdO}{\tilde{O}}	\newcommand{\tdP}{\tilde{P}}
\newcommand{\tdQ}{\tilde{Q}}	\newcommand{\tdR}{\tilde{R}}
\newcommand{\tdS}{\tilde{S}}	\newcommand{\tdT}{\tilde{T}}
\newcommand{\tdU}{\tilde{U}}	\newcommand{\tdV}{\tilde{V}}
\newcommand{\tdW}{\tilde{W}}	\newcommand{\tdX}{\tilde{X}}
\newcommand{\tdY}{\tilde{Y}}	\newcommand{\tdZ}{\tilde{Z}}
%Small Letters
\newcommand{\tda}{\tilde{a}}	\newcommand{\tdb}{\tilde{b}}
\newcommand{\tdc}{\tilde{c}}	\newcommand{\tdd}{\tilde{d}}
\newcommand{\tde}{\tilde{e}}	\newcommand{\tdf}{\tilde{f}}
\newcommand{\tdg}{\tilde{g}}	\newcommand{\tdh}{\tilde{h}}
\newcommand{\tdi}{\tilde{i}}	\newcommand{\tdj}{\tilde{j}}
\newcommand{\tdk}{\tilde{k}}	\newcommand{\tdl}{\tilde{l}}
\newcommand{\tdm}{\tilde{m}}	\newcommand{\tdn}{\tilde{n}}
\newcommand{\tdo}{\tilde{o}}	\newcommand{\tdp}{\tilde{p}}
\newcommand{\tdq}{\tilde{q}}	\newcommand{\tdr}{\tilde{r}}
\newcommand{\tds}{\tilde{s}}	\newcommand{\tdt}{\tilde{t}}
\newcommand{\tdu}{\tilde{u}}	\newcommand{\tdv}{\tilde{v}}
\newcommand{\tdw}{\tilde{w}}	\newcommand{\tdx}{\tilde{x}}
\newcommand{\tdy}{\tilde{y}}	\newcommand{\tdz}{\tilde{z}}

%---------------------------------------
% Vec
%---------------------------------------

%Captital Letters
\newcommand{\vcA}{\vec{A}}	\newcommand{\vcB}{\vec{B}}
\newcommand{\vcC}{\vec{C}}	\newcommand{\vcD}{\vec{D}}
\newcommand{\vcE}{\vec{E}}	\newcommand{\vcF}{\vec{F}}
\newcommand{\vcG}{\vec{G}}	\newcommand{\vcH}{\vec{H}}
\newcommand{\vcI}{\vec{I}}	\newcommand{\vcJ}{\vec{J}}
\newcommand{\vcK}{\vec{K}}	\newcommand{\vcL}{\vec{L}}
\newcommand{\vcM}{\vec{M}}	\newcommand{\vcN}{\vec{N}}
\newcommand{\vcO}{\vec{O}}	\newcommand{\vcP}{\vec{P}}
\newcommand{\vcQ}{\vec{Q}}	\newcommand{\vcR}{\vec{R}}
\newcommand{\vcS}{\vec{S}}	\newcommand{\vcT}{\vec{T}}
\newcommand{\vcU}{\vec{U}}	\newcommand{\vcV}{\vec{V}}
\newcommand{\vcW}{\vec{W}}	\newcommand{\vcX}{\vec{X}}
\newcommand{\vcY}{\vec{Y}}	\newcommand{\vcZ}{\vec{Z}}
%Small Letters
\newcommand{\vca}{\vec{a}}	\newcommand{\vcb}{\vec{b}}
\newcommand{\vcc}{\vec{c}}	\newcommand{\vcd}{\vec{d}}
\newcommand{\vce}{\vec{e}}	\newcommand{\vcf}{\vec{f}}
\newcommand{\vcg}{\vec{g}}	\newcommand{\vch}{\vec{h}}
\newcommand{\vci}{\vec{i}}	\newcommand{\vcj}{\vec{j}}
\newcommand{\vck}{\vec{k}}	\newcommand{\vcl}{\vec{l}}
\newcommand{\vcm}{\vec{m}}	\newcommand{\vcn}{\vec{n}}
\newcommand{\vco}{\vec{o}}	\newcommand{\vcp}{\vec{p}}
\newcommand{\vcq}{\vec{q}}	\newcommand{\vcr}{\vec{r}}
\newcommand{\vcs}{\vec{s}}	\newcommand{\vct}{\vec{t}}
\newcommand{\vcu}{\vec{u}}	\newcommand{\vcv}{\vec{v}}
%\newcommand{\vcw}{\vec{w}}	\newcommand{\vcx}{\vec{x}}
\newcommand{\vcy}{\vec{y}}	\newcommand{\vcz}{\vec{z}}

%---------------------------------------
% Greek Letters:-
%---------------------------------------
\newcommand{\eps}{\epsilon}
\newcommand{\veps}{\varepsilon}
\newcommand{\lm}{\lambda}
\newcommand{\Lm}{\Lambda}
\newcommand{\gm}{\gamma}
\newcommand{\Gm}{\Gamma}
\newcommand{\vph}{\varphi}
\newcommand{\ph}{\phi}
\newcommand{\om}{\omega}
\newcommand{\Om}{\Omega}
\newcommand{\sg}{\sigma}
\newcommand{\Sg}{\Sigma}

\newcommand{\Qed}{\begin{flushright}\qed\end{flushright}}
\newcommand{\parinn}{\setlength{\parindent}{1cm}}
\newcommand{\parinf}{\setlength{\parindent}{0cm}}
\newcommand{\del}[2]{\frac{\partial #1}{\partial #2}}
\newcommand{\Del}[3]{\frac{\partial^{#1} #2}{\partial^{#1} #3}}
\newcommand{\deld}[2]{\dfrac{\partial #1}{\partial #2}}
\newcommand{\Deld}[3]{\dfrac{\partial^{#1} #2}{\partial^{#1} #3}}
\newcommand{\uin}{\mathbin{\rotatebox[origin=c]{90}{$\in$}}}
\newcommand{\usubset}{\mathbin{\rotatebox[origin=c]{90}{$\subset$}}}
\newcommand{\lt}{\left}
\newcommand{\rt}{\right}
\newcommand{\exs}{\exists}
\newcommand{\st}{\strut}
\newcommand{\dps}[1]{\displaystyle{#1}}
\newcommand{\la}{\langle}
\newcommand{\ra}{\rangle}
\newcommand{\cls}[1]{\textsc{#1}}
\newcommand{\prb}[1]{\textsc{#1}}
\newcommand{\comb}[2]{\left(\begin{matrix}
		#1\\ #2
\end{matrix}\right)}
%\newcommand[2]{\quotient}{\faktor{#1}{#2}}
\newcommand\quotient[2]{
	\mathchoice
	{% \displaystyle
		\text{\raise1ex\hbox{$#1$}\Big/\lower1ex\hbox{$#2$}}%
	}
	{% \textstyle
		#1\,/\,#2
	}
	{% \scriptstyle
		#1\,/\,#2
	}
	{% \scriptscriptstyle  
		#1\,/\,#2
	}
}

\newcommand{\tensor}{\otimes}
\newcommand{\xor}{\oplus}

\newcommand{\sol}[1]{\begin{solution}#1\end{solution}}
\newcommand{\solve}[1]{\setlength{\parindent}{0cm}\textbf{\textit{Solution: }}\setlength{\parindent}{1cm}#1 \hfill $\blacksquare$}
\newcommand{\mat}[1]{\left[\begin{matrix}#1\end{matrix}\right]}
\newcommand{\matr}[1]{\begin{matrix}#1\end{matrix}}
\newcommand{\matp}[1]{\lt(\begin{matrix}#1\end{matrix}\rt)}
\newcommand{\detmat}[1]{\lt|\begin{matrix}#1\end{matrix}\rt|}
\newcommand\numberthis{\addtocounter{equation}{1}\tag{\theequation}}
\newcommand{\handout}[3]{
	\noindent
	\begin{center}
		\framebox{
			\vbox{
				\hbox to 6.5in { {\bf Complexity Theory I } \hfill Jan -- May, 2023 }
				\vspace{4mm}
				\hbox to 6.5in { {\Large \hfill #1  \hfill} }
				\vspace{2mm}
				\hbox to 6.5in { {\em #2 \hfill #3} }
			}
		}
	\end{center}
	\vspace*{4mm}
}

\newcommand{\lecture}[3]{\handout{Lecture #1}{Lecturer: #2}{Scribe:	#3}}

\let\marvosymLightning\Lightning
\newcommand{\ctr}{\text{\marvosymLightning}\hspace{0.5ex}} % Requires marvosym package

\newcommand{\ov}[1]{\overline{#1}}
\newcommand{\thmref}[1]{\hyperref[th:#1]{Theorem \ref{th:#1}}}
\newcommand{\propref}[1]{\hyperref[th:#1]{Proposition \ref{th:#1}}}
\newcommand{\lmref}[1]{\hyperref[th:#1]{Lemma \ref{th:#1}}}
\newcommand{\corref}[1]{\hyperref[th:#1]{Corollary \ref{th:#1}}}

\newcommand{\thrmref}[1]{\hyperref[#1]{Theorem \ref{#1}}}
\newcommand{\propnref}[1]{\hyperref[#1]{Proposition \ref{#1}}}
\newcommand{\lemref}[1]{\hyperref[#1]{Lemma \ref{#1}}}
\newcommand{\corrref}[1]{\hyperref[#1]{Corollary \ref{#1}}}

\DeclareMathOperator{\enc}{Enc}
\DeclareMathOperator{\res}{Res}
\DeclareMathOperator{\spec}{Spec}
\DeclareMathOperator{\cov}{Cov}
\DeclareMathOperator{\Var}{Var}
\DeclareMathOperator{\Rank}{rank}
\newcommand{\Tfae}{The following are equivalent:}
\newcommand{\tfae}{the following are equivalent:}
\newcommand{\sparsity}{\textit{sparsity}}

\newcommand{\uddots}{\reflectbox{$\ddots$}} 

\newenvironment{claimwidth}{\begin{center}\begin{adjustwidth}{0.05\textwidth}{0.05\textwidth}}{\end{adjustwidth}\end{center}}

\setlength{\parindent}{0pt}

%%%%%%%%%%%%%%%%%%%%%%%%%%%%%%%%%%%%%%%%%%%%%%%%%%%%%%%%%%%%%%%%%%%%%%%%%%%%%%%%%%%%%%%%%%%%%%%%%%%%%%%%%%%%%%%%%%%%%%%%%%%%%%%%%%%%%%%%

\begin{document}
	
	%%%%%%%%%%%%%%%%%%%%%%%%%%%%%%%%%%%%%%%%%%%%%%%%%%%%%%%%%%%%%%%%%%%%%%%%%%%%%%%%%%%%%%%%%%%%%%%%%%%%%%%%%%%%%%%%%%%%%%%%%%%%%%%%%%%%%%%%
	
	\textsf{\noindent \large\textbf{Soham Chatterjee} \hfill \textbf{Assignment - 1}\\
		Email: \href{sohamc@cmi.ac.in}{sohamc@cmi.ac.in} \hfill Roll: BMC202175\\
		\normalsize Course: Algebra and Computation \hfill Date: \today}
	
%%%%%%%%%%%%%%%%%%%%%%%%%%%%%%%%%%%%%%%%%%%%%%%%%%%%%%%%%%%%%%%%%%%%%%%%%%%%%%%%%%%%%%%%%%%%%%%%%%%%%%%%%%%%%%%%%%%%%%%%%%%%%%%%%%%%%%%%
% Problem 1
%%%%%%%%%%%%%%%%%%%%%%%%%%%%%%%%%%%%%%%%%%%%%%%%%%%%%%%%%%%%%%%%%%%%%%%%%%%%%%%%%%%%%%%%%%%%%%%%%%%%%%%%%%%%%%%%%%%%%%%%%%%%%%%%%%%%%%%%
	
\begin{problem}{%problem statement
		Problem Set 1: P5
	}{p1% problem reference text
}
For a prime $p$ and a positive integer $e$, prove that $\bbZ_{p^e}^*$ is cyclic.
\end{problem}
\solve{
	We will prove this in 3 stages: $e=1$, $e=2$, $e> 2$.
	
	\section*{Case 1: $e=1$}
	\begin{lemma}\label{totprop}
		$\sum\limits_{d\mid n} \varphi(d)=n$
	\end{lemma}
	\begin{proof}
		Consider the list of numbers $S=\lt\{\frac1n,\frac2n,\dots,\frac{n}{n}\rt\}$. If we express every number in $S$ as simplified form i.e. $\frac{p}{q}$ form where $gcd(p,q)=1$. Then the denominators are all the divisors of $n$. 
		
		
		Then for any $k\in[n]$ we have $$\frac{k}{n}=\frac{\frac{k}{gcd(k,n)}}{\frac{n}{gcd(k,n)}}$$Denote $d_k\coloneqq\frac{n}{gcd(k,n)}$ then $d_k$ is a factor of $n$. And since $gcd\lt( \frac{k}{gcd(k,n)}, \frac{n}{gcd(k,n)} \rt)=1$ we have $\frac{k}{gcd(k,n)}\in \bbZ_{d_k}^*$. Let $k\in\bbZ_{d}^*$ then suppose $l$ is such that $d\times l=n$ then the fraction $\frac{k}{d}=\frac{k\times l}{n}\in S$ and its simplified form is infact $\frac{k}{d}$.
		
		Hence for any $d\mid n$, the number of fractions with denominator $d$ is $\varphi(d)$, since for all such fractions the numerators are the elements of $\bbZ_{d}^*$. Therefore we have $\sum\limits_{d\mid n} \varphi(d)=n$.
	\end{proof}
	
	Now define for $d$ such that $d\mid p-1$, $S_d=\{a\in \bbZ_p^*\mid ord(a)=d\}$. Then we have the following lemma:
	
	\begin{lemma}
		$|S_d|=\varphi(d)$
	\end{lemma}
	\begin{proof}
		First we will show that $|S_d|\in\{0,\varphi(d)\}$ then we will show that $|S_d|=\varphi(d)$. Now if $|S_d|\neq 0$ then $\exs\ a\in S_d$ such that $ord(a)=d$. Then consider the polynomial $x^d-1$ over $\bbF_p$. $1,a,a^2,\dots,a^{p-1}$ are its distinct roots. Since the degree is $d$ these are the only roots of the polynomial. Now $a^k$ has order $\frac{d}{gcd(d,k)}$. Then the elements which has order $d$ are $a^k$ where $gcd(k,d)=1$. Hence there are $\varphi(d)$ many powers of $a$ which has order $d$. Therefore $|S_d|\in\{0,\varphi(d)\}$.
		
		Now we have by \lmref{totprop} $$\sum_{d\mid p-1}\varphi(d)=p-1$$Now $\{S_d\colon d\mid p-1\}$ is a partition of $\bbZ_P^*$. Therefore $\sum\limits_{d\mid p-1}|S_d|=p-1$. Hence $$p-1=\sum\limits_{d\mid p-1}|S_d|\leq \sum\limits_{d\mid p-1}\varphi(d)=p-1\iff |S_d|=\varphi(d)\ \forall\ d\text{ such that } d\mid p-1$$
	\end{proof}
	
	Hence the number of elements in $\bbZ_p^*$ which has order $d$ such that $d\mid p-1$
	
	Now we will introduce another definition. Let $H$ be a group. Then Exponent of $H$ is the smallest number $n$ such that $\forall a\in H$, $a^n=1$. Now we will show that every finite abelian group has an element which has the order to be exponent of the group. Then we will show that $\bbZ_p^*$ has exponent $p-1$. With that we can say $\bbZ_p^*$ has an element which has order $p-1$. Therefore $\bbZ_p^*$ is cyclic since $|\bbZ_p^*|=p-1$ because $\bbZ_p^*$ is a finite abelian group.
	\begin{lemma}
		If $G$ is a finite abelian group with exponent $n$ then $\exs\ g\in G$ such that $ord(g)=n$.
	\end{lemma}
	\begin{proof}
		By structure theorem we have $$G\cong \bbZ_{q_1}\times \cdots \times \bbZ_{q_m}$$ where $q_1,\dots, q_m$ are primes powers. Now $\forall\ g\in G$, $ord(g)\mid lcm(q_1,\dots,q_m)$. The element in $\bbZ_{q_1}\times \cdots \times \bbZ_{q_m}$, $(1,1,\dots, 1)$ has order $lcm(q_1,\dots,q_m)$. So the exponent of $G$ is $lcm(q_1,\dots,q_m)$ and the corresponding element of $(1,\dots,1)$ has order $lcm(q_1,\dots,q_m)$.
	\end{proof}
	\begin{lemma}
		$\bbZ_p^*$ has exponent $p-1$.
	\end{lemma}
	\begin{proof}
		Over $\bbF_p$ the equation $x^{p-1}-1$ has $p-1$ roots which are all the elements of $\bbZ_p^*$. There does not exists any polynomial of lower degree which satisfies this property. Hence the exponent of $\bbZ_p^*$ is $p-1$.
	\end{proof}
	
	Therefore there exists an element of $\bbZ_p^*$ which has order $p-1$. Therefore the group $\bbZ_p^*$ is cyclic.
	\section*{Case 2: $e=2$}
	\begin{lemma}
		Let $g$ be generator of the group $\bbZ_p^*$. Then either $g$ or $g+p$ is generator for $\bbZ_{p^2}^*$.
	\end{lemma}
	
	\begin{proof}
		We have $|\bbZ_{p^2}^*|\varphi(p^2)=p(p-1)$. Let $g$ has order $m$ in $\bbZ_{p^2}^*$. Then $g^p\equiv 1\bmod p$. Hence $p-1\mid m$. Therefore $m=p(p-1)$ or $m=p-1$ since $m\mid p(p-1)$. If its the first case then we are done. For the later take the element $g+p$. Again let its order is $m'$. Then $(g+p)^{m'}\equiv 1\bmod p$. So $p-1\mid m'$. Hence $m'$ can be either $p-1$ or $p(p-1)$. If it is also $p-1$ then we have \begin{align*}
			1\equiv (g+p)^{p-1} & \equiv g^{p-1}+(p-1)g^{p-2}p+p^2(\cdots)\bmod {p^2}\\
			& \equiv g^{p-1}+p(p-1)g^{p-2}\bmod {p^2}\\
			& \equiv 1+p(p-1)g^{p-2}\bmod {p^2}
		\end{align*}
		Therefore $$p(p-1)g^{p-2}\equiv 0\bmod{p^2}\iff p\mid (p-1)g^{p-2}$$ which is not possible since $gcd(p,p-1)=1$ and $gcd(p,g)=1$. Contradiction. Hence at least one of $g$ and $g+p$ has order $p(p-1)$.  
	\end{proof}
	
	With this lemma we have an element of $\bbZ_{p^2}^*$ which has order $p(p-1)=|\bbZ_{p^2}^*|$. So $\bbZ_{p^2}^*$ is cyclic.
	
	
	
	
	\section*{Case 3: $e> 2$}
	\begin{lemma}
		$(1+p)^{p^k}\equiv 1+p^{k+1}\bmod {p^{k+2}}$
	\end{lemma}
	\begin{proof}
		\begin{align*}
			(1+p)^{p^k}& \equiv \lt((1+p)^p   \rt)^{p^{k-1}}\\
			& \equiv \lt( 1+p^2+\binom{p}{2}p^2 \rt)^{p^{k-1}}\bmod {p^{k+2}}\\
			& \equiv 1+p^2\times p^{k-1}\bmod p^{k+2}\\
			&\equiv 1+p^{k+1}\bmod{p^{k+2}}
		\end{align*}
	\end{proof}
	Therefore $$(1+p)^{p^{k+1}}\equiv(1+p^{k+1})^{p}\equiv 1+p\times p^{k+1}\equiv 1+p^{k+2}\equiv 1\bmod{p^{k+2}}$$Hence $(1+p)$ has order $p^{k+1}$ in $\bbZ_{p^{k+2}}^*$. Or we can say $1+p$ has order $p^{e-1}$ is $\bbZ_{p^e}^*$. 
	
	Let $g$ be the generator of $\bbZ_p^*$. Then let the order of $g$ in $\bbZ_{p^e}^*$ is $m$. Then $p-1\mid m$. So $g$ has order $p^k(p-1)$. Therefore the number $g(1+p)\bmod {p^e}$ has order $p^{e-1}(1-p)=\varphi(p^e)$. Therefore $\bbZ_{p^e}^*$ is a cyclic group.
	
}
%%%%%%%%%%%%%%%%%%%%%%%%%%%%%%%%%%%%%%%%%%%%%%%%%%%%%%%%%%%%%%%%%%%%%%%%%%%%%%%%%%%%%%%%%%%%%%%%%%%%%%%%%%%%%%%%%%%%%%%%%%%%%%%%%%%%%%%%
% Problem 2
%%%%%%%%%%%%%%%%%%%%%%%%%%%%%%%%%%%%%%%%%%%%%%%%%%%%%%%%%%%%%%%%%%%%%%%%%%%%%%%%%%%%%%%%%%%%%%%%%%%%%%%%%%%%%%%%%%%%%%%%%%%%%%%%%%%%%%%%

\begin{problem}{%problem statement
		Problem Set 1: P6
	}{p2% problem reference text
	}
Let $N=p_1p_2\cdots p_k$ be a Carmichael number and $p_i$'s are primes. In class we have seen that given $N$ as input, a single pass of Miller-Robin primality test outputs a nontrivial factor of $N$ with probability $\geq \frac12$. We can do a finer calculation and get better success probability. Show that a single pass of Miller-Robin primality test outputs a nontrivial factor of $N$ with probability $1-\frac1{2^{k-1}}$.
\end{problem}
\solve{
	Let $\phi$ be the isomorphism of $$\bbZ_N^*\cong \bbZ_{p_1}^*\times \cdots \times \bbZ_{p_k}^*$$Now suppose $N-1=2^vm$ where $m$ is odd. Let $a\in \{2,\dots, N-2\}$ Let $l_a$ be the minimum such that  $a^{2^{l_a+1}m} \bmod N\equiv 1$. Surely for all $a$, $l_a>0$ and $l_a\leq N-1$ Now take $l=\max\{l_a\mid a\in \{2,\dots, N-2\}\}$. Therefore $l>0$ and $l\leq N-1$. For all $k<l$ there exists $a\in\{2,\dots, N-2\}$ such that  $a^{2^{k+1}m}\not\equiv 1\bmod N$. 
	
	Now consider the group $$G_N=\{a\in \bbZ_N^*\mid a^{2^lm}\equiv \pm 1\bmod N\}$$Now there exists at least one $\tda$ such that $\tda^{2^lm}\equiv -1\bmod N$ since otherwise for all $a\in \{2,\dots N-2\}$, $l_a\leq l-1$. Then $\max\{l_a\mid a\in\{2,\dots ,N-2\}\}\leq l-1$ which contradicts that the value we got is $l$. Hence there exist a $\tda\in \bbZ_N^*$ such that $\tda^{2^lm}\equiv -1\bmod N$.
	
	Now $\phi(\tda^{2^lm})=(-1,\dots,-1)$. Suppose $\phi(\tda)=(\tda_1,\dots, \tda_k)$. Then we have $$\forall\ i\in[k],\ \tda_i^{2^lm}\equiv -1\bmod p_i$$Now for any $i\in[k]$ the corresponding element in $\bbZ_N^*$ of $(1,\dots,1,\tda_i,1,\dots,1)$ denote by $g$. Then $g^{2^lm}\not\equiv -1\bmod N$. There are $k$ many slots here and in each slot we have 2 options $1$ or $\tda_i$. Hence with above like construction we can have at most $2^k$ many elements. Among these the elements $(1,\dots, 1)$ and $(\tda_1,\dots,\tda_k)$ are in $G_N$. All the other elements remain in distict cosets of $G_N$ in $\bbZ_N^*/G_N$. Hence $$Pr_{a\in_R \bbZ_N^*}[a\in \bbZ_N^*-G_N]\geq \frac{2^k-2}{2^k}=1-\frac1{2^{k-1}}$$Hence $$Pr[\text{Primality Test outputs a nontrivial factor of $N$}]\geq 1-\frac1{2^{k-1}}$$
	
}

%%%%%%%%%%%%%%%%%%%%%%%%%%%%%%%%%%%%%%%%%%%%%%%%%%%%%%%%%%%%%%%%%%%%%%%%%%%%%%%%%%%%%%%%%%%%%%%%%%%%%%%%%%%%%%%%%%%%%%%%%%%%%%%%%%%%%%%%
% Problem 3
%%%%%%%%%%%%%%%%%%%%%%%%%%%%%%%%%%%%%%%%%%%%%%%%%%%%%%%%%%%%%%%%%%%%%%%%%%%%%%%%%%%%%%%%%%%%%%%%%%%%%%%%%%%%%%%%%%%%%%%%%%%%%%%%%%%%%%%%

\begin{problem}{%problem statement
		Problem Set 1: P7
	}{p3% problem reference text
	}
	Design a randomized polynomial time algorithm such that given $N$ and $\varphi(N)$ as inputs, it outputs a nontrivial factor of $N$ with probability at least $\frac12$, where $\varphi(\cdot)$ is the Euler’s totient function
\end{problem}
\solve{
	We first run Miler-Robin Test. If it outputs prime then we output that. Otherwise if it outputs a factor we also output that. If it outputs `Composite' then we do the following:
	
	Let $\varphi(N)=2^st(p-1)$ where $p\mid N$ and $t$ is odd. If $a$ is a non quadratic residue then $$a^{\frac{\varphi(N)}{2^{s+1}}}\bmod N\equiv \lt[a^{\frac{p-1}{2}}\rt]^{t}\bmod p\equiv -1\bmod p$$ Let for $p_i$ the corresponding $s,t$ are denoted by $s_i,t_i$. WLOG assume $s_1\geq s_2\geq \cdots \geq s_k$. Then if $a$ is a Non-Quadratic Residue wrt $p_1$ and Quadratic Residue wrt $p_2$ then $$a^{\frac{\phi(N)}{2^{s_1+1}}}\bmod N\equiv \lt[a^{\frac{p_1-1}{2}}\rt]^{t_1}\bmod p_1\equiv -1\bmod N\text{ but }\lt[a^{\frac{\phi(N)}{2^{s_2+1}}}\rt]^{2^{s_2-s_1}}\bmod p_1\equiv \lt[a^{\frac{p_2-1}{2}}\rt]^{2^{s_2-s_1}\cdot t_1}\bmod p_2\equiv 1\bmod p_2$$Hence $$a^{\frac{\varphi(N)}{2^{s_1+1}}}\not\equiv \pm 1\bmod N$$Now probability that any number is Non-Quadratic Residue modulo $p_1$ but Quadratic residue modulo $p_2$ is $\frac14$. Therefore $a^{\frac{\varphi{N}}{2^{t_1+1}}}$ has a common factor $p_1$ with $N$ but $N$ does not divides it. 
	
	Therefore in the algorithm if the Miller Robin test returns `Composite' then we will take a random $a\in\{2,\dots,N-2\}$ then we will compute lowest number $l$ such that 
	$\lt[a^{\frac{\phi(N)}{2^{l+1}}}\rt]^m\equiv 1\bmod N$ 
	where $m$ is odd and $\phi(N)=2^k\cdot m$ then we will take 
	$gcd\lt(\lt[a^{\frac{\phi(N)}{2^{l+1}}}\rt]^m,N\rt)$. 
	This will return a nontrivial factor of $N$. We will do this procedure $3$ times if the $gcd$ returned is 1. And after 3 times gcd returned 1 we will output prime. 
	
	Hence this procedure fails to give a  nontrivial factor is $1-\lt(1-\frac14\rt)^3=1-\frac{27}{64}>\frac12$. 
}

%%%%%%%%%%%%%%%%%%%%%%%%%%%%%%%%%%%%%%%%%%%%%%%%%%%%%%%%%%%%%%%%%%%%%%%%%%%%%%%%%%%%%%%%%%%%%%%%%%%%%%%%%%%%%%%%%%%%%%%%%%%%%%%%%%%%%%%%
% Problem 4
%%%%%%%%%%%%%%%%%%%%%%%%%%%%%%%%%%%%%%%%%%%%%%%%%%%%%%%%%%%%%%%%%%%%%%%%%%%%%%%%%%%%%%%%%%%%%%%%%%%%%%%%%%%%%%%%%%%%%%%%%%%%%%%%%%%%%%%%

\begin{problem}{%problem statement
		Problem Set 1: P13
	}{p4% problem reference text
	}
	Design a deterministic polynomial time algorithm to compute the gcd of two univariate polynomials
	using resultants and linear system solving.
\end{problem}
\solve{
	Let $p,q\in \bbF[x]$ where $\deg p=m$ and $\deg q=n$. The Sylvester matrix generated by $p,q$ is $S_{p,q}$. Let for any $k\in\bbN$, $\bbF_k\coloneqq \{f\in \bbF[x]\mid \deg f<k\}$. Then for $(u,v)\in\bbF_n\times \bbF_m$, $S_{p,q}(u,v)=up+vq$.
	
	Let $gcd(p,q)=h$ and $\deg h=d$.
	\begin{lemma}
		$\dim \ker S_{p,q}=\deg gcd(p,q)$
	\end{lemma}
	\begin{proof}
		Let $(x,y)\in \ker S_{p,q}$. Then $px+qy=0$. Now denote $p=hp_0$ and $q=hq_0$. Hence $gcd(p_0,q_0)=1$. Therefore $$px+qy=0\iff p_0x+q_0y=0\iff p_0x=-q_0y$$Therefore $q_0\mid x$ and $p_0\mid y$. So let $x=q_0g_x$ and $y=p_0g_y$. Then $$p_0x+q_0y=0\iff p_0q_0g_x+q_0p_0g_y=0\iff p_0q_0(g_x+g_y)=0\iff g_x=-g_y$$
		So denote $g=g_x=-g_y$. So $x=q_0g$, $y=-p_0g$. Now  $$\deg x<\deg q\iff \deg q_0+\deg g<\deg q_0+\deg h\iff \deg g<\deg h $$  Hence for each $(x,y)\in \ket S_{p,q}$ there exists unique $g\in \bbF_d$ such that $x=q_0g$ and $y=-p_0g$ and also for each $g\in \bbF_d$ we have $x=q_0g$ and $y=-p_0g$ such that $px+qy=0$. Hence there exists a bijection $\bbF_d \cong \ker S_{p,q}$ by $g\mapsto (q_0g,-p_0g)$
	\end{proof}
	
	Therefore by Rank-Nullity Theorem $$\rank(S_{p,q})+\dim \ker S_{p,q}=m+n$$ Therefore $\rank(S_{p,q})=m+n-d$. Hence the last $d$ rows of the row echelon form of the $S_{p,q}^T$ are zeros. Let $(S_{p,q}^T)^*$ denote the row echelon form of $S^T_{p,q}$. Let $e_i$ denote the $i$th row of $(S_{p,q}^T)^*$. Hence the last nonzero row of $(S^T_{p,q})^*$ is $e_{m+n-d}$. We have $\deg e_{m+n-d}\leq d$. Now for $i\in[n]$ the $i$th row of $S^T_{p,q}$ is just $x^{n-i}p$ and for $n+1\leq j\leq n+m$ the $j$th row is $x^{m+n-j}q$. Hence $$e_{m+n-d}=\sum_{i=1}^{n}\alpha_i x^{n-i}p+\sum_{i=n+1}^{m+n}\alpha_ix^{m+n-i}q$$ The LHS has degree $\leq d$ and the RHS is divisible by $h$ since $h\mid p$ and $h\mid q$. Hence $h=e_{m+n-d}$ up to some unit multiplication. Therefore we can say $e_{m+n-d}$ is the gcd of $p,q$. Therefore the algorithm will be 
	\parinf
	
	\textbf{Algorithm:}\begin{enumerate}[label=Step \arabic*,leftmargin=1.5cm]
		\item Construct $S_{p,q}$
		\item Find Row Echelon Form of $S_{p,q}^T$ by Gaussian Elimination
		\item Output the last nonzero row 
	\end{enumerate}
	\parinn
}

%%%%%%%%%%%%%%%%%%%%%%%%%%%%%%%%%%%%%%%%%%%%%%%%%%%%%%%%%%%%%%%%%%%%%%%%%%%%%%%%%%%%%%%%%%%%%%%%%%%%%%%%%%%%%%%%%%%%%%%%%%%%%%%%%%%%%%%%
% Problem 5
%%%%%%%%%%%%%%%%%%%%%%%%%%%%%%%%%%%%%%%%%%%%%%%%%%%%%%%%%%%%%%%%%%%%%%%%%%%%%%%%%%%%%%%%%%%%%%%%%%%%%%%%%%%%%%%%%%%%%%%%%%%%%%%%%%%%%%%%

\begin{problem}{%problem statement
		Problem Set 1: P14
	}{p5% problem reference text
	}
	Give a polynomial time algorithm to compute the gcd of two bivariate polynomials, without using
	bivariate factorization.
\end{problem}
\solve{
		\begin{lemma}\label{gcdlc}
		Let $R$ be an Euclidean Domain. Let $p\in R$ be a prime and $f,g\in R[x]$ be nonzero. Let $h=gcd(f,g)\in R[x]$. Denote $\ov{f}=f\bmod p$ and $\ov{g}=g\bmod p$ and $d=\deg h$ and $\alpha=lc(h)$. Assume $p\nmid b=gcd(lc(f),lc(g))\in R$ and $\ovd=\deg gcd(\ovf,\ovg)$. Then \begin{enumerate}
			\item $\alpha\mid b$
			\item $\ovd\geq d$
			\item $d=\ovd\iff \ov{\alpha}\cdot gcd(\ovf,\ovg)=\ovh\iff p\nmid \res\lt(\frac{f}{h},\frac{g}{h}\rt) $
		\end{enumerate}
	\end{lemma}
	
	
	\begin{proof}
		\begin{enumerate}
			\item 	Now $h$ divides both $f,g$. Therefore $lc(h)$ divides both $lc(f)$ and $lc(g)$ in $R$. Hence $\alpha\mid b$
			\item Let $u=\frac{f}{h}$ and $v=\frac{g}{h}$. Since $p\nmid b\implies p\nmid lc(h)$. Hence $\deg h=\deg \ovh=d$. Now $$\ovu\ovh=\ovf\text{ and }\ovv\ovh=\ovg$$ Hence $\ovh\mid \ovf$ and $\ovh\mid \ovg\implies \ovh\mid gcd(\ovf,\ovg)$. Therefore $\deg gcd(\ovf,\ovg)\geq \deg \ovh\implies \ovd\geq d$.
			\item $d=\ovd\iff \deg \ovh=\deg gcd(\ovf,\ovg)$. Now $p\nmid b$ and $\alpha\mid b$ so $p\nmid \alpha$. Hence $\alpha$ is a unit in $R/\la p\ra$ as $R/\la p\ra$ is a field. In a field gcd is always taken to be monic. Now $\ov{\alpha}=lc(\ovh)$. Since $\deg \ovh=\deg gcd(\ovf,\ovg)$ we can say $\ovh=u\cdot gcd(\ovf,\ovg)$ for some unit $u\in R/\la p\ra$. Now since $gcd(\ovf,\ovg)$ is monic we have $u=\ov{\alpha}$ Therefore $d=\ovd\implies \ov{\alpha}\cdot gcd(\ovf,\ovg)=\ovh$. Other direction obviously becomes true as $\ov{\alpha}$ is a unit in $R/\la p\ra$. 
			
			Now $p\nmid b\implies p$ divides at most one of $lc(u)$ or $lc(v)$. WLOG assume $p\nmid lc(u)$. We know $$p\mid \res (u,v)\iff gcd(\ovu,\ovv)\neq 1\text{ in }R/\la p\ra$$So \begin{align*}
				gcd(\ovf,\ovg)=gcd(\ovu,\ovv)\cdot \frac{\ovh}{\ov{\alpha}} & \iff \ov{\alpha} gcd(\ovf,\ovg)=gcd(\ovu,\ovv)\ovh \\
				                                                            & \iff \ovh=gcd(\ovu,\ovv)\ovh                       \\
				                                                            & \iff gcd(\ovu,\ovv)=1                              \\
				                                                            & \iff p\nmid \res(\ovu,\ovv)                        \\
				                                                            & \iff p\nmid \res(u,v)
			\end{align*}
			
		\end{enumerate}
	\end{proof}
	

	
	\begin{algorithm}[H]
		\KwIn{\begin{enumerate}
				\item Primitive Polynomials $f,g\in\bbF[x,y]=R[x]$
				\item $\deg_x f=n\geq \deg_x g\geq 1$
				\item $\deg_y f,\deg _yg\leq d$
		\end{enumerate}}
	\KwOut{$h=gcd(f,g)\in \bbF[x,y]$}
	\DontPrintSemicolon
		\Begin{
		$b\longleftarrow gcd(lc(f),lc(g))$, $FAIL\longleftarrow 1$\;
		\While{$FAIL$}{
			Choose a random monic irreducible polynomial $p\in \bbF[y]$ with $\deg p=d+1+\deg b$\;
			$\ov{f}\longleftarrow f\bmod p$, $\ov{g}\longleftarrow g\bmod p$\;
			Use Extended Euclidean Algorithm over $\bbF[y]/\la p\ra$ on $\ov{f}$ and $\ov{g}$ to compute the monic $v\in R/\la p\ra$\;
			Compute $w,f',g'\in R[x]$ with $\deg_yw,\deg_yf',\deg_yg'<\deg p$ such that:$$w\equiv bv\bmod p\qquad f'w\equiv bf\bmod p\qquad g'w\equiv bg \bmod p$$\;
			\If{$\deg_y(f'w)=\deg_y(bf)$ and $\deg_y(g'w)=\deg(bg)$}{$FAIL\longleftarrow 0$}
			\Return{ premitive part of $w$ w.r.t $x$}
		}
		}
		\caption{Modular Bivariate GCD Algorithm}
	\end{algorithm}
	
		Now in $\bbF[x,y]$ let $gcd(f,g)=h$ and $r=\res_x\lt(\frac{f}{h},\frac{g}{h}\rt)\in \bbF[y]$. Now $\deg_yb<\deg_yp=\deg p$ and hence $p\nmid b$. Assume $p\nmid r$ then by \lmref{gcdlc} we have $\alpha \cdot v \equiv h\bmod p$ and $\alpha\mid b$. Therefore $$w\equiv bv\equiv \lt(\frac{b}{\alpha}\rt)h\bmod p$$ Now primitive part of $w$=premitive part of $\lt(\frac{b}{\alpha}\rt)h$=$h$. Hence correct result is returned.
		
				Now if $p\mid r$ then by \lmref{gcdlc} we have $\deg_xgcd(\ovf,\ovg)>\deg_x h$. If the degree conditions in step 8 are satisfied then the congruences in step 7 would be equalities and the primitive part of $w$ will be a common divisor of $f$ and $g$ of higher degree than $\deg_x h$. Contradiction. So the degree condtions will not be satisfied.
 }
\end{document}
