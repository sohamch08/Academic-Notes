\documentclass[a4paper, 11pt]{article}
\usepackage{comment} % enables the use of multi-line comments (\ifx \fi) 
\usepackage{fullpage} % changes the margin
\usepackage[a4paper, total={7in, 10in}]{geometry}
\usepackage{amsmath,mathtools,mathdots}
\usepackage{amssymb,amsthm}  % assumes amsmath package installed
\usepackage{float}
\usepackage{xcolor}
\usepackage{mdframed}
\usepackage[shortlabels]{enumitem}
\usepackage{indentfirst}
\usepackage{hyperref}
\hypersetup{
	colorlinks=true,
	linkcolor=blue,
	filecolor=magenta,      
	urlcolor=blue!70!red,
	pdftitle={Assignment}, %%%%%%%%%%%%%%%%   WRITE ASSIGNMENT PDF NAME  %%%%%%%%%%%%%%%%%%%%
}
\usepackage[most,many,breakable]{tcolorbox}
\usepackage{tikz}
\usetikzlibrary{decorations.pathreplacing,angles,quotes,patterns}
\usepackage{caption}
\usepackage{bbm}
\usepackage{mathpazo}
\usepackage{libertine}
\definecolor{mytheorembg}{HTML}{F2F2F9}
\definecolor{mytheoremfr}{HTML}{00007B}


\tcbuselibrary{theorems,skins,hooks}
\newtcbtheorem{problem}{Problem}
{%
	enhanced,
	breakable,
	colback = mytheorembg,
	frame hidden,
	boxrule = 0sp,
	borderline west = {2pt}{0pt}{mytheoremfr},
	sharp corners,
	detach title,
	before upper = \tcbtitle\par\smallskip,
	coltitle = mytheoremfr,
	fonttitle = \bfseries\sffamily,
	description font = \mdseries,
	separator sign none,
	segmentation style={solid, mytheoremfr},
}
{p}

% To give references for any problem use like this
% suppose the problem number is p3 then 2 options either 
% \hyperref[p:p3]{<text you want to use to hyperlink> \ref{p:p3}}
%                  or directly 
%                   \ref{p:p3}



\input{../../letterfonts}

\input{../../macros}

\setlength{\parindent}{0pt}

%%%%%%%%%%%%%%%%%%%%%%%%%%%%%%%%%%%%%%%%%%%%%%%%%%%%%%%%%%%%%%%%%%%%%%%%%%%%%%%%%%%%%%%%%%%%%%%%%%%%%%%%%%%%%%%%%%%%%%%%%%%%%%%%%%%%%%%%

\begin{document}
	
	%%%%%%%%%%%%%%%%%%%%%%%%%%%%%%%%%%%%%%%%%%%%%%%%%%%%%%%%%%%%%%%%%%%%%%%%%%%%%%%%%%%%%%%%%%%%%%%%%%%%%%%%%%%%%%%%%%%%%%%%%%%%%%%%%%%%%%%%
	
	\textsf{\noindent \large\textbf{Soham Chatterjee} \hfill \textbf{Assignment - 1}\\
		Email: \href{sohamc@cmi.ac.in}{sohamc@cmi.ac.in} \hfill Roll: BMC202175\\
		\normalsize Course: Expander Graphs and Applications \hfill Date: March 5, 2024}
	
%%%%%%%%%%%%%%%%%%%%%%%%%%%%%%%%%%%%%%%%%%%%%%%%%%%%%%%%%%%%%%%%%%%%%%%%%%%%%%%%%%%%%%%%%%%%%%%%%%%%%%%%%%%%%%%%%%%%%%%%%%%%%%%%%%%%%%%%
% Problem 1
%%%%%%%%%%%%%%%%%%%%%%%%%%%%%%%%%%%%%%%%%%%%%%%%%%%%%%%%%%%%%%%%%%%%%%%%%%%%%%%%%%%%%%%%%%%%%%%%%%%%%%%%%%%%%%%%%%%%%%%%%%%%%%%%%%%%%%%%
	
\begin{problem}{%problem statement
		Problem 4.9 (The Replacement Product): Pseudorandomness By Salil Vadhan
	}{p1% problem reference text
}
Given a $D_1$-regular graph $G_1$ on $N_1$ vertices and a $D_2$-regular graph $G_2$ on $D_1$ vertices consider the following graph $G_1\textcircled{r}G_2$ on vertex set $[N_1]\times [D_1]$: vertex $(u,i)$ is connected to $(v,j)$ iff \begin{enumerate}[label=(\alph*)]
	\item $u=v$ and $(i,j)$ is an edge in $G_2$ or,
	\item $v$ is the $i$'th neighbour of $u$ in $G_1$ and $u$ is the $j$th neighbor of $v$. 
\end{enumerate}
That is, we ``replace" each vertex $v$ in $G_1$ with a copy of $G_2$, associating edge incident to $v$ with one vertex of $G_2$.
\begin{enumerate}
	\item Prove that there is a function $g$ such that if $G_1$ has spectral expansion $\gm_1>0$ and $G_2$ has spectral expansion $\gm_2>0$ (and both graphs are undirected) then $G_1\textcircled{r}G_2$ has spectral expansion $g(\gm_1,\gm_2,D_2)>0$.
	
	[Hint: Note that $(G_1\textcircled{r}G_2)^3$ has $G_1\textcircled{z}G_2$ as a subgraph]
	\item Show how to convert an explicit construction of constant degree (spectral) expanders into an explicit construction of degree 3 (spectral) expanders.
	\item Without using Theorem 4.14, prove an analogue of Part 1 for edge expansion. That is, there is a function $h$ such that if $G_1$ is an $\lt(\frac{N_1}2,\eps_1\rt)$ edge expander and $G_2$ is a $\lt(\frac{D_1}2,\eps_2\rt)$ edge expander then $G_1\textcircled{r}G_2$ is a $\lt(\frac{N_1D_1}{2},h(\eps_1,\eps_2,D_2)\rt)$ edge expander where $h(\eps_1,\eps_2,D_2)>0$ if $\eps_1,\eps_2>0$. 
	
	[Hint: Given any set $S$ of vertices of $G_1\textcircled{r}G_2$, partition $S$ into the clouds that are more than ``half-full" and those that are not]
	\item Prove that the functions $g(\gm_1,\gm_2,D_2)$ and $h(\eps_1,\eps_2,D_2)$ must depend on $D_2$ by showing that $G_1\textcircled{r}G_2$ cannot be a $\lt(\frac{N_1D_1}2,\eps\rt)$ edge expander if $\eps>\frac1{D_1+1}$ and $N_1\geq 2$
\end{enumerate}
\end{problem}
\solve{
\begin{enumerate}
	\item Let $A_1$ and $A_2$ denote the normalized adjacency matrices of $G_1$ and $G_2$ respectively. The degree of the new graph $G_1\textcircled{r}G_2$ is $D_2+1$. Now denote $B\triangleq I_{N_1}\otimes A_2$ and $A$ be a $N_1\cdot D_1\times N_1\cdot D_1$ matrix where $$A[(u,i),(v,j)]=\begin{cases}
		1 & \text{when $i$th neighbor of $u$ is $v$ and $j$th neighbor of $v$ is $u$ in $G_1$}\\
		0 & \text{otherwise}
	\end{cases}$$Therefore the adjacency matrix of the graph $G_1\textcircled{r}G_1$ is $A+D_2B$. Therefore the normalized adjacency matrix, $M$ $$M\triangleq \frac{A+D_2B}{D_2+1}$$
	
	Now notice the graph $(G_1\textcircled{r}G_2)^3$ contains the graph $G_1\textcircled{z}G_2$ as a subgraph. Hence $$M^3=\lt[\frac{A+D_2B}{D_2+1}\rt]^3=\frac{D_2^2}{(D_2+1)^3}BAB+\lt[1-\frac{D_2^2}{(D_2+1)^3}\rt]C$$ for some matrix $C$. Lets denote $p\coloneqq \frac{D_2^2}{(D_2+1)^3}$. Then $M^3=pBAB+(1-p)C$. Hence for any $v\perp u$ where $u$ is the uniform vector we have $$\|M^3v\|\leq p\|BABv\|+(1-p)\|Cv\|$$
	
	Now we can think as $C$ is a normalized adjacency matrix of an undirected graph. Hence for all $v\perp u$ we have $\|Cv\|\leq \|v\|$. Now we know for all $v\perp u$ $$\|BABv\|\leq (\lm_1+\lm_2+\lm_2^2)\|v\|$$ where $\lm_1=1-\gm_1$ and $\lm_2=1-\gm_2$. Hence $$\|M^3v\|\leq p(\lm_1+\lm_2+\lm_2^2)\|v\|+(1-p)\|v\|=[p(\lm_1+\lm_2+\lm_2^2)+(1-p)]\|v\|$$
	
	Now\begin{align*}
		1-[p(\lm_1+\lm_2+\lm_2^2)+(1-p)] & = 1-(1-p)-p(\lm_1+\lm_2+\lm_2^2)\\
		& =p-p(\lm_1+\lm_2+\lm_2^2)\\
		& =p[1-(\lm_1+\lm_2+\lm_2^2)]
	\end{align*}Now we know $$\lm_1+\lm_2+\lm_2^2<1\iff 0<1-(\lm_1+\lm_2+\lm_2^2)<1\qquad \text{and}\qquad 0<p<1$$Then $0<p[1-(\lm_1+\lm_2+\lm_2^2)]<1$. Hence $$0<p(\lm_1+\lm_2+\lm_2^2)+(1-p)<1$$ Therefore for all $v\perp u$, $$\|Mv\|\leq \lt[  p(\lm_1+\lm_2+\lm_2^2)+(1-p)\rt]^{\frac13}\|v\|$$ Now 
	\begin{align*}
		\lt[p(\lm_1+\lm_2+\lm_2^2)+(1-p)\rt]^{\frac13} & = \lt[1- p[1-(\lm_1+\lm_2+\lm_2^2)] \rt]^{\frac13}\\
		& \leq 1-\frac13 p[1-(\lm_1+\lm_2+\lm_2^2)]<1
	\end{align*}So $$g(\gm_1,\gm_2,D_2)=1-	\lt[p(\lm_1+\lm_2+\lm_2^2)+(1-p)\rt]^{\frac13} >0$$
	\item 
\end{enumerate}
}

%%%%%%%%%%%%%%%%%%%%%%%%%%%%%%%%%%%%%%%%%%%%%%%%%%%%%%%%%%%%%%%%%%%%%%%%%%%%%%%%%%%%%%%%%%%%%%%%%%%%%%%%%%%%%%%%%%%%%%%%%%%%%%%%%%%%%%%%
% Problem 2
%%%%%%%%%%%%%%%%%%%%%%%%%%%%%%%%%%%%%%%%%%%%%%%%%%%%%%%%%%%%%%%%%%%%%%%%%%%%%%%%%%%%%%%%%%%%%%%%%%%%%%%%%%%%%%%%%%%%%%%%%%%%%%%%%%%%%%%%

\begin{problem}{%problem statement
		Problem 4.10 (Unbalanced Vertex Expanders and Data Structures): Pseudorandomness By Salil Vadhan
	}{p2% problem reference text
	}
	Consider a $(K,(1-\eps)D)$ bipartite vertex expander $G$ with $N$ left vertices, $M$ right vertices and left degree $D$.
	\begin{enumerate}
		\item For a set $S$ of left vertices, a $y\in N(S)$ is called a \textit{unique} neighbor of $S$ if $y$ is incident to exactly one edge from $S$. Prove that every left-set $S$ of size at most $K$ has at least $(1-2\eps)D|S|$ unique neighbors.
		\item For a set $S$ of size at most $\frac{K}2$, prove that at most $\frac{|S|}2$ vertices outside $S$ have at least $\delta D$ neighbors in $N(S)$ for $\delta=O(\eps)$.
	\end{enumerate}
\end{problem}
\solve{
\begin{enumerate}
	\item Let $U$ be the set of unique neighbors in $N(S)$. Denote $T=\Gm(S)-U$. Then we have $|U\cup T|\geq (1-\eps)D|S|$. Now we will count the number of edges between $S$ and $\Gm(S)$. From each vertex in $S$ there are $D$ edges going out. Hence total $D|S|$ many edges are going out from $S$. Now in $\Gm(S)$ for each vertex in $U$ there is exactly one edge coming from $S$ and for each edge in $T$ there are at least 2 edges coming from $S$. Hence there are at least $|U|+2|T|$ many edges are coming towards $\Gm(S)$. Hence we have:
	
	\begin{multline*}
		|U|+2|T|\leq D|S|\iff |U|+2(|\Gm(S)|-|U|)\leq D|S|\\
		\iff |U|\geq 2|\Gm(S)|-D|S|\geq (1-\eps)D|S|-D|S|=(1-2\eps)D|S|
	\end{multline*}Hence there are at least  $(1-2\eps)D|S|$ unique neighbors.
	\item 
\end{enumerate}
}

%%%%%%%%%%%%%%%%%%%%%%%%%%%%%%%%%%%%%%%%%%%%%%%%%%%%%%%%%%%%%%%%%%%%%%%%%%%%%%%%%%%%%%%%%%%%%%%%%%%%%%%%%%%%%%%%%%%%%%%%%%%%%%%%%%%%%%%%
% Problem 3
%%%%%%%%%%%%%%%%%%%%%%%%%%%%%%%%%%%%%%%%%%%%%%%%%%%%%%%%%%%%%%%%%%%%%%%%%%%%%%%%%%%%%%%%%%%%%%%%%%%%%%%%%%%%%%%%%%%%%%%%%%%%%%%%%%%%%%%%

\begin{problem}{%problem statement
		Problem 5.5 (LDPC Codes): Pseudorandomness By Salil Vadhan
	}{p3% problem reference text
	}
	Given a $D_1$-regular graph $G_1$ on $N_1$ vertices and a $D_2$-regular graph $G_2$ on $D_1$ vertices consider the following graph $G_1\textcircled{r}G_2$ on vertex set $[N_1]\times [D_1]$: vertex $(u,i)$ is connected to $(v,j)$ iff \begin{enumerate}[label=(\alph*)]
		\item $u=v$ and $(i,j)$ is an edge in $G_2$ or,
		\item $v$ is the $i$'th neighbor of $u$ in $G_1$ and $u$ is the $j$th neighbor of $v$. 
	\end{enumerate}
	That is, we ``replace" each vertex $v$ in $G_1$ with a copy of $G_2$, associating edge incident to $v$ with one vertex of $G_2$.
	\begin{enumerate}
		\item Prove that there is a function $g$ such that if $G_1$ has spectral expansion $\gm_1>0$ and $G_2$ has spectral expansion $\gm_2>0$ (and both graphs are undirected) then $G_1\textcircled{r}G_2$ has spectral expansion $g(\gm_1,\gm_2,D_2)>0$.
		
		[Hint: Note that $(G_1\textcircled{r}G_2)^3$ has $G_1\textcircled{z}G_2$ as a subgraph]
		\item Show how to convert an explicit construction of constant degree (spectral) expanders into an explicit construction of degree 3 (spectral) expanders.
		\item Without using Theorem 4.14, prove an analogue of Part 1 for edge expansion. That is, there is a function $h$ such that if $G_1$ is an $\lt(\frac{N_1}2,\eps_1\rt)$ edge expander and $G_2$ is a $\lt(\frac{D_1}2,\eps_2\rt)$ edge expander then $G_1\textcircled{r}G_2$ is a $\lt(\frac{N_1D_1}{2},h(\eps_1,\eps_2,D_2)\rt)$ edge expander where $h(\eps_1,\eps_2,D_2)>0$ if $\eps_1,\eps_2>0$. 
		
		[Hint: Given any set $S$ of vertices of $G_1\textcircled{r}G_2$, partition $S$ into the clouds that are more than ``half-full" and those that are not]
		\item Prove that the functions $g(\gm_1,\gm_2,D_2)$ and $h(\eps_1,\eps_2,D_2)$ must depend on $D_2$ by showing that $G_1\textcircled{r}G_2$ cannot be a $\lt(\frac{N_1D_1}2,\eps\rt)$ edge expander if $\eps>\frac1{D_1+1}$ and $N_1\geq 2$
	\end{enumerate}
\end{problem}

%%%%%%%%%%%%%%%%%%%%%%%%%%%%%%%%%%%%%%%%%%%%%%%%%%%%%%%%%%%%%%%%%%%%%%%%%%%%%%%%%%%%%%%%%%%%%%%%%%%%%%%%%%%%%%%%%%%%%%%%%%%%%%%%%%%%%%%%
% Problem 1
%%%%%%%%%%%%%%%%%%%%%%%%%%%%%%%%%%%%%%%%%%%%%%%%%%%%%%%%%%%%%%%%%%%%%%%%%%%%%%%%%%%%%%%%%%%%%%%%%%%%%%%%%%%%%%%%%%%%%%%%%%%%%%%%%%%%%%%%

\begin{problem}{%problem statement
	}{p1% problem reference text
	}
	Given a $D_1$-regular graph $G_1$ on $N_1$ vertices and a $D_2$-regular graph $G_2$ on $D_1$ vertices consider the following graph $G_1\textcircled{r}G_2$ on vertex set $[N_1]\times [D_1]$: vertex $(u,i)$ is connected to $(v,j)$ iff \begin{enumerate}[label=(\alph*)]
		\item $u=v$ and $(i,j)$ is an edge in $G_2$ or,
		\item $v$ is the $i$'th neighbor of $u$ in $G_1$ and $u$ is the $j$th neighbour of $v$. 
	\end{enumerate}
	That is, we ``replace" each vertex $v$ in $G_1$ with a copy of $G_2$, associating edge incident to $v$ with one vertex of $G_2$.
	\begin{enumerate}
		\item Prove that there is a function $g$ such that if $G_1$ has spectral expansion $\gm_1>0$ and $G_2$ has spectral expansion $\gm_2>0$ (and both graphs are undirected) then $G_1\textcircled{r}G_2$ has spectral expansion $g(\gm_1,\gm_2,D_2)>0$.
		
		[Hint: Note that $(G_1\textcircled{r}G_2)^3$ has $G_1\textcircled{z}G_2$ as a subgraph]
		\item Show how to convert an explicit construction of constant degree (spectral) expanders into an explicit construction of degree 3 (spectral) expanders.
		\item Without using Theorem 4.14, prove an analogue of Part 1 for edge expansion. That is, there is a function $h$ such that if $G_1$ is an $\lt(\frac{N_1}2,\eps_1\rt)$ edge expander and $G_2$ is a $\lt(\frac{D_1}2,\eps_2\rt)$ edge expander then $G_1\textcircled{r}G_2$ is a $\lt(\frac{N_1D_1}{2},h(\eps_1,\eps_2,D_2)\rt)$ edge expander where $h(\eps_1,\eps_2,D_2)>0$ if $\eps_1,\eps_2>0$. 
		
		[Hint: Given any set $S$ of vertices of $G_1\textcircled{r}G_2$, partition $S$ into the clouds that are more than ``half-full" and those that are not]
		\item Prove that the functions $g(\gm_1,\gm_2,D_2)$ and $h(\eps_1,\eps_2,D_2)$ must depend on $D_2$ by showing that $G_1\textcircled{r}G_2$ cannot be a $\lt(\frac{N_1D_1}2,\eps\rt)$ edge expander if $\eps>\frac1{D_1+1}$ and $N_1\geq 2$
	\end{enumerate}
\end{problem}
\end{document}
