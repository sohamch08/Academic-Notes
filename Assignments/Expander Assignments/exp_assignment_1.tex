\documentclass[a4paper, 11pt]{article}
\usepackage{comment} % enables the use of multi-line comments (\ifx \fi) 
\usepackage{fullpage} % changes the margin
\usepackage[a4paper, total={7in, 10in}]{geometry}
\usepackage{amsmath,mathtools,mathdots}
\usepackage{amssymb,amsthm}  % assumes amsmath package installed
\usepackage{float}
\usepackage{xcolor}
\usepackage{mdframed}
\usepackage[shortlabels]{enumitem}
\usepackage{indentfirst}
\usepackage{hyperref}
\hypersetup{
	colorlinks=true,
	linkcolor=blue,
	filecolor=magenta,      
	urlcolor=blue!70!red,
	pdftitle={Assignment}, %%%%%%%%%%%%%%%%   WRITE ASSIGNMENT PDF NAME  %%%%%%%%%%%%%%%%%%%%
}
\usepackage[most,many,breakable]{tcolorbox}
\usepackage{tikz}
\usetikzlibrary{decorations.pathreplacing,angles,quotes,patterns}
\usepackage{caption}
\usepackage{bbm}
\usepackage{mathpazo}
\usepackage{libertine}
\definecolor{mytheorembg}{HTML}{F2F2F9}
\definecolor{mytheoremfr}{HTML}{00007B}


\tcbuselibrary{theorems,skins,hooks}
\newtcbtheorem{problem}{Problem}
{%
	enhanced,
	breakable,
	colback = mytheorembg,
	frame hidden,
	boxrule = 0sp,
	borderline west = {2pt}{0pt}{mytheoremfr},
	sharp corners,
	detach title,
	before upper = \tcbtitle\par\smallskip,
	coltitle = mytheoremfr,
	fonttitle = \bfseries\sffamily,
	description font = \mdseries,
	separator sign none,
	segmentation style={solid, mytheoremfr},
}
{p}

% To give references for any problem use like this
% suppose the problem number is p3 then 2 options either 
% \hyperref[p:p3]{<text you want to use to hyperlink> \ref{p:p3}}
%                  or directly 
%                   \ref{p:p3}



%---------------------------------------
% BlackBoard Math Fonts :-
%---------------------------------------

%Captital Letters
\newcommand{\bbA}{\mathbb{A}}	\newcommand{\bbB}{\mathbb{B}}
\newcommand{\bbC}{\mathbb{C}}	\newcommand{\bbD}{\mathbb{D}}
\newcommand{\bbE}{\mathbb{E}}	\newcommand{\bbF}{\mathbb{F}}
\newcommand{\bbG}{\mathbb{G}}	\newcommand{\bbH}{\mathbb{H}}
\newcommand{\bbI}{\mathbb{I}}	\newcommand{\bbJ}{\mathbb{J}}
\newcommand{\bbK}{\mathbb{K}}	\newcommand{\bbL}{\mathbb{L}}
\newcommand{\bbM}{\mathbb{M}}	\newcommand{\bbN}{\mathbb{N}}
\newcommand{\bbO}{\mathbb{O}}	\newcommand{\bbP}{\mathbb{P}}
\newcommand{\bbQ}{\mathbb{Q}}	\newcommand{\bbR}{\mathbb{R}}
\newcommand{\bbS}{\mathbb{S}}	\newcommand{\bbT}{\mathbb{T}}
\newcommand{\bbU}{\mathbb{U}}	\newcommand{\bbV}{\mathbb{V}}
\newcommand{\bbW}{\mathbb{W}}	\newcommand{\bbX}{\mathbb{X}}
\newcommand{\bbY}{\mathbb{Y}}	\newcommand{\bbZ}{\mathbb{Z}}

%---------------------------------------
% MathCal Fonts :-
%---------------------------------------

%Captital Letters
\newcommand{\mcA}{\mathcal{A}}	\newcommand{\mcB}{\mathcal{B}}
\newcommand{\mcC}{\mathcal{C}}	\newcommand{\mcD}{\mathcal{D}}
\newcommand{\mcE}{\mathcal{E}}	\newcommand{\mcF}{\mathcal{F}}
\newcommand{\mcG}{\mathcal{G}}	\newcommand{\mcH}{\mathcal{H}}
\newcommand{\mcI}{\mathcal{I}}	\newcommand{\mcJ}{\mathcal{J}}
\newcommand{\mcK}{\mathcal{K}}	\newcommand{\mcL}{\mathcal{L}}
\newcommand{\mcM}{\mathcal{M}}	\newcommand{\mcN}{\mathcal{N}}
\newcommand{\mcO}{\mathcal{O}}	\newcommand{\mcP}{\mathcal{P}}
\newcommand{\mcQ}{\mathcal{Q}}	\newcommand{\mcR}{\mathcal{R}}
\newcommand{\mcS}{\mathcal{S}}	\newcommand{\mcT}{\mathcal{T}}
\newcommand{\mcU}{\mathcal{U}}	\newcommand{\mcV}{\mathcal{V}}
\newcommand{\mcW}{\mathcal{W}}	\newcommand{\mcX}{\mathcal{X}}
\newcommand{\mcY}{\mathcal{Y}}	\newcommand{\mcZ}{\mathcal{Z}}



%---------------------------------------
% Bold Math Fonts :-
%---------------------------------------

%Captital Letters
\newcommand{\bmA}{\boldsymbol{A}}	\newcommand{\bmB}{\boldsymbol{B}}
\newcommand{\bmC}{\boldsymbol{C}}	\newcommand{\bmD}{\boldsymbol{D}}
\newcommand{\bmE}{\boldsymbol{E}}	\newcommand{\bmF}{\boldsymbol{F}}
\newcommand{\bmG}{\boldsymbol{G}}	\newcommand{\bmH}{\boldsymbol{H}}
\newcommand{\bmI}{\boldsymbol{I}}	\newcommand{\bmJ}{\boldsymbol{J}}
\newcommand{\bmK}{\boldsymbol{K}}	\newcommand{\bmL}{\boldsymbol{L}}
\newcommand{\bmM}{\boldsymbol{M}}	\newcommand{\bmN}{\boldsymbol{N}}
\newcommand{\bmO}{\boldsymbol{O}}	\newcommand{\bmP}{\boldsymbol{P}}
\newcommand{\bmQ}{\boldsymbol{Q}}	\newcommand{\bmR}{\boldsymbol{R}}
\newcommand{\bmS}{\boldsymbol{S}}	\newcommand{\bmT}{\boldsymbol{T}}
\newcommand{\bmU}{\boldsymbol{U}}	\newcommand{\bmV}{\boldsymbol{V}}
\newcommand{\bmW}{\boldsymbol{W}}	\newcommand{\bmX}{\boldsymbol{X}}
\newcommand{\bmY}{\boldsymbol{Y}}	\newcommand{\bmZ}{\boldsymbol{Z}}
%Small Letters
\newcommand{\bma}{\boldsymbol{a}}	\newcommand{\bmb}{\boldsymbol{b}}
\newcommand{\bmc}{\boldsymbol{c}}	\newcommand{\bmd}{\boldsymbol{d}}
\newcommand{\bme}{\boldsymbol{e}}	\newcommand{\bmf}{\boldsymbol{f}}
\newcommand{\bmg}{\boldsymbol{g}}	\newcommand{\bmh}{\boldsymbol{h}}
\newcommand{\bmi}{\boldsymbol{i}}	\newcommand{\bmj}{\boldsymbol{j}}
\newcommand{\bmk}{\boldsymbol{k}}	\newcommand{\bml}{\boldsymbol{l}}
\newcommand{\bmm}{\boldsymbol{m}}	\newcommand{\bmn}{\boldsymbol{n}}
\newcommand{\bmo}{\boldsymbol{o}}	\newcommand{\bmp}{\boldsymbol{p}}
\newcommand{\bmq}{\boldsymbol{q}}	\newcommand{\bmr}{\boldsymbol{r}}
\newcommand{\bms}{\boldsymbol{s}}	\newcommand{\bmt}{\boldsymbol{t}}
\newcommand{\bmu}{\boldsymbol{u}}	\newcommand{\bmv}{\boldsymbol{v}}
\newcommand{\bmw}{\boldsymbol{w}}	\newcommand{\bmx}{\boldsymbol{x}}
\newcommand{\bmy}{\boldsymbol{y}}	\newcommand{\bmz}{\boldsymbol{z}}


%---------------------------------------
% Scr Math Fonts :-
%---------------------------------------

\newcommand{\sA}{{\mathscr{A}}}   \newcommand{\sB}{{\mathscr{B}}}
\newcommand{\sC}{{\mathscr{C}}}   \newcommand{\sD}{{\mathscr{D}}}
\newcommand{\sE}{{\mathscr{E}}}   \newcommand{\sF}{{\mathscr{F}}}
\newcommand{\sG}{{\mathscr{G}}}   \newcommand{\sH}{{\mathscr{H}}}
\newcommand{\sI}{{\mathscr{I}}}   \newcommand{\sJ}{{\mathscr{J}}}
\newcommand{\sK}{{\mathscr{K}}}   \newcommand{\sL}{{\mathscr{L}}}
\newcommand{\sM}{{\mathscr{M}}}   \newcommand{\sN}{{\mathscr{N}}}
\newcommand{\sO}{{\mathscr{O}}}   \newcommand{\sP}{{\mathscr{P}}}
\newcommand{\sQ}{{\mathscr{Q}}}   \newcommand{\sR}{{\mathscr{R}}}
\newcommand{\sS}{{\mathscr{S}}}   \newcommand{\sT}{{\mathscr{T}}}
\newcommand{\sU}{{\mathscr{U}}}   \newcommand{\sV}{{\mathscr{V}}}
\newcommand{\sW}{{\mathscr{W}}}   \newcommand{\sX}{{\mathscr{X}}}
\newcommand{\sY}{{\mathscr{Y}}}   \newcommand{\sZ}{{\mathscr{Z}}}


%---------------------------------------
% Math Fraktur Font
%---------------------------------------

%Captital Letters
\newcommand{\mfA}{\mathfrak{A}}	\newcommand{\mfB}{\mathfrak{B}}
\newcommand{\mfC}{\mathfrak{C}}	\newcommand{\mfD}{\mathfrak{D}}
\newcommand{\mfE}{\mathfrak{E}}	\newcommand{\mfF}{\mathfrak{F}}
\newcommand{\mfG}{\mathfrak{G}}	\newcommand{\mfH}{\mathfrak{H}}
\newcommand{\mfI}{\mathfrak{I}}	\newcommand{\mfJ}{\mathfrak{J}}
\newcommand{\mfK}{\mathfrak{K}}	\newcommand{\mfL}{\mathfrak{L}}
\newcommand{\mfM}{\mathfrak{M}}	\newcommand{\mfN}{\mathfrak{N}}
\newcommand{\mfO}{\mathfrak{O}}	\newcommand{\mfP}{\mathfrak{P}}
\newcommand{\mfQ}{\mathfrak{Q}}	\newcommand{\mfR}{\mathfrak{R}}
\newcommand{\mfS}{\mathfrak{S}}	\newcommand{\mfT}{\mathfrak{T}}
\newcommand{\mfU}{\mathfrak{U}}	\newcommand{\mfV}{\mathfrak{V}}
\newcommand{\mfW}{\mathfrak{W}}	\newcommand{\mfX}{\mathfrak{X}}
\newcommand{\mfY}{\mathfrak{Y}}	\newcommand{\mfZ}{\mathfrak{Z}}
%Small Letters
\newcommand{\mfa}{\mathfrak{a}}	\newcommand{\mfb}{\mathfrak{b}}
\newcommand{\mfc}{\mathfrak{c}}	\newcommand{\mfd}{\mathfrak{d}}
\newcommand{\mfe}{\mathfrak{e}}	\newcommand{\mff}{\mathfrak{f}}
\newcommand{\mfg}{\mathfrak{g}}	\newcommand{\mfh}{\mathfrak{h}}
\newcommand{\mfi}{\mathfrak{i}}	\newcommand{\mfj}{\mathfrak{j}}
\newcommand{\mfk}{\mathfrak{k}}	\newcommand{\mfl}{\mathfrak{l}}
\newcommand{\mfm}{\mathfrak{m}}	\newcommand{\mfn}{\mathfrak{n}}
\newcommand{\mfo}{\mathfrak{o}}	\newcommand{\mfp}{\mathfrak{p}}
\newcommand{\mfq}{\mathfrak{q}}	\newcommand{\mfr}{\mathfrak{r}}
\newcommand{\mfs}{\mathfrak{s}}	\newcommand{\mft}{\mathfrak{t}}
\newcommand{\mfu}{\mathfrak{u}}	\newcommand{\mfv}{\mathfrak{v}}
\newcommand{\mfw}{\mathfrak{w}}	\newcommand{\mfx}{\mathfrak{x}}
\newcommand{\mfy}{\mathfrak{y}}	\newcommand{\mfz}{\mathfrak{z}}

%---------------------------------------
% Bar
%---------------------------------------

%Captital Letters
\newcommand{\ovA}{\overline{A}}	\newcommand{\ovB}{\overline{B}}
\newcommand{\ovC}{\overline{C}}	\newcommand{\ovD}{\overline{D}}
\newcommand{\ovE}{\overline{E}}	\newcommand{\ovF}{\overline{F}}
\newcommand{\ovG}{\overline{G}}	\newcommand{\ovH}{\overline{H}}
\newcommand{\ovI}{\overline{I}}	\newcommand{\ovJ}{\overline{J}}
\newcommand{\ovK}{\overline{K}}	\newcommand{\ovL}{\overline{L}}
\newcommand{\ovM}{\overline{M}}	\newcommand{\ovN}{\overline{N}}
\newcommand{\ovO}{\overline{O}}	\newcommand{\ovP}{\overline{P}}
\newcommand{\ovQ}{\overline{Q}}	\newcommand{\ovR}{\overline{R}}
\newcommand{\ovS}{\overline{S}}	\newcommand{\ovT}{\overline{T}}
\newcommand{\ovU}{\overline{U}}	\newcommand{\ovV}{\overline{V}}
\newcommand{\ovW}{\overline{W}}	\newcommand{\ovX}{\overline{X}}
\newcommand{\ovY}{\overline{Y}}	\newcommand{\ovZ}{\overline{Z}}
%Small Letters
\newcommand{\ova}{\overline{a}}	\newcommand{\ovb}{\overline{b}}
\newcommand{\ovc}{\overline{c}}	\newcommand{\ovd}{\overline{d}}
\newcommand{\ove}{\overline{e}}	\newcommand{\ovf}{\overline{f}}
\newcommand{\ovg}{\overline{g}}	\newcommand{\ovh}{\overline{h}}
\newcommand{\ovi}{\overline{i}}	\newcommand{\ovj}{\overline{j}}
\newcommand{\ovk}{\overline{k}}	\newcommand{\ovl}{\overline{l}}
\newcommand{\ovm}{\overline{m}}	\newcommand{\ovn}{\overline{n}}
\newcommand{\ovo}{\overline{o}}	\newcommand{\ovp}{\overline{p}}
\newcommand{\ovq}{\overline{q}}	\newcommand{\ovr}{\overline{r}}
\newcommand{\ovs}{\overline{s}}	\newcommand{\ovt}{\overline{t}}
\newcommand{\ovu}{\overline{u}}	\newcommand{\ovv}{\overline{v}}
\newcommand{\ovw}{\overline{w}}	\newcommand{\ovx}{\overline{x}}
\newcommand{\ovy}{\overline{y}}	\newcommand{\ovz}{\overline{z}}

%---------------------------------------
% Tilde
%---------------------------------------

%Captital Letters
\newcommand{\tdA}{\tilde{A}}	\newcommand{\tdB}{\tilde{B}}
\newcommand{\tdC}{\tilde{C}}	\newcommand{\tdD}{\tilde{D}}
\newcommand{\tdE}{\tilde{E}}	\newcommand{\tdF}{\tilde{F}}
\newcommand{\tdG}{\tilde{G}}	\newcommand{\tdH}{\tilde{H}}
\newcommand{\tdI}{\tilde{I}}	\newcommand{\tdJ}{\tilde{J}}
\newcommand{\tdK}{\tilde{K}}	\newcommand{\tdL}{\tilde{L}}
\newcommand{\tdM}{\tilde{M}}	\newcommand{\tdN}{\tilde{N}}
\newcommand{\tdO}{\tilde{O}}	\newcommand{\tdP}{\tilde{P}}
\newcommand{\tdQ}{\tilde{Q}}	\newcommand{\tdR}{\tilde{R}}
\newcommand{\tdS}{\tilde{S}}	\newcommand{\tdT}{\tilde{T}}
\newcommand{\tdU}{\tilde{U}}	\newcommand{\tdV}{\tilde{V}}
\newcommand{\tdW}{\tilde{W}}	\newcommand{\tdX}{\tilde{X}}
\newcommand{\tdY}{\tilde{Y}}	\newcommand{\tdZ}{\tilde{Z}}
%Small Letters
\newcommand{\tda}{\tilde{a}}	\newcommand{\tdb}{\tilde{b}}
\newcommand{\tdc}{\tilde{c}}	\newcommand{\tdd}{\tilde{d}}
\newcommand{\tde}{\tilde{e}}	\newcommand{\tdf}{\tilde{f}}
\newcommand{\tdg}{\tilde{g}}	\newcommand{\tdh}{\tilde{h}}
\newcommand{\tdi}{\tilde{i}}	\newcommand{\tdj}{\tilde{j}}
\newcommand{\tdk}{\tilde{k}}	\newcommand{\tdl}{\tilde{l}}
\newcommand{\tdm}{\tilde{m}}	\newcommand{\tdn}{\tilde{n}}
\newcommand{\tdo}{\tilde{o}}	\newcommand{\tdp}{\tilde{p}}
\newcommand{\tdq}{\tilde{q}}	\newcommand{\tdr}{\tilde{r}}
\newcommand{\tds}{\tilde{s}}	\newcommand{\tdt}{\tilde{t}}
\newcommand{\tdu}{\tilde{u}}	\newcommand{\tdv}{\tilde{v}}
\newcommand{\tdw}{\tilde{w}}	\newcommand{\tdx}{\tilde{x}}
\newcommand{\tdy}{\tilde{y}}	\newcommand{\tdz}{\tilde{z}}

%---------------------------------------
% Vec
%---------------------------------------

%Captital Letters
\newcommand{\vcA}{\vec{A}}	\newcommand{\vcB}{\vec{B}}
\newcommand{\vcC}{\vec{C}}	\newcommand{\vcD}{\vec{D}}
\newcommand{\vcE}{\vec{E}}	\newcommand{\vcF}{\vec{F}}
\newcommand{\vcG}{\vec{G}}	\newcommand{\vcH}{\vec{H}}
\newcommand{\vcI}{\vec{I}}	\newcommand{\vcJ}{\vec{J}}
\newcommand{\vcK}{\vec{K}}	\newcommand{\vcL}{\vec{L}}
\newcommand{\vcM}{\vec{M}}	\newcommand{\vcN}{\vec{N}}
\newcommand{\vcO}{\vec{O}}	\newcommand{\vcP}{\vec{P}}
\newcommand{\vcQ}{\vec{Q}}	\newcommand{\vcR}{\vec{R}}
\newcommand{\vcS}{\vec{S}}	\newcommand{\vcT}{\vec{T}}
\newcommand{\vcU}{\vec{U}}	\newcommand{\vcV}{\vec{V}}
\newcommand{\vcW}{\vec{W}}	\newcommand{\vcX}{\vec{X}}
\newcommand{\vcY}{\vec{Y}}	\newcommand{\vcZ}{\vec{Z}}
%Small Letters
\newcommand{\vca}{\vec{a}}	\newcommand{\vcb}{\vec{b}}
\newcommand{\vcc}{\vec{c}}	\newcommand{\vcd}{\vec{d}}
\newcommand{\vce}{\vec{e}}	\newcommand{\vcf}{\vec{f}}
\newcommand{\vcg}{\vec{g}}	\newcommand{\vch}{\vec{h}}
\newcommand{\vci}{\vec{i}}	\newcommand{\vcj}{\vec{j}}
\newcommand{\vck}{\vec{k}}	\newcommand{\vcl}{\vec{l}}
\newcommand{\vcm}{\vec{m}}	\newcommand{\vcn}{\vec{n}}
\newcommand{\vco}{\vec{o}}	\newcommand{\vcp}{\vec{p}}
\newcommand{\vcq}{\vec{q}}	\newcommand{\vcr}{\vec{r}}
\newcommand{\vcs}{\vec{s}}	\newcommand{\vct}{\vec{t}}
\newcommand{\vcu}{\vec{u}}	\newcommand{\vcv}{\vec{v}}
%\newcommand{\vcw}{\vec{w}}	\newcommand{\vcx}{\vec{x}}
\newcommand{\vcy}{\vec{y}}	\newcommand{\vcz}{\vec{z}}

%---------------------------------------
% Greek Letters:-
%---------------------------------------
\newcommand{\eps}{\epsilon}
\newcommand{\veps}{\varepsilon}
\newcommand{\lm}{\lambda}
\newcommand{\Lm}{\Lambda}
\newcommand{\gm}{\gamma}
\newcommand{\Gm}{\Gamma}
\newcommand{\vph}{\varphi}
\newcommand{\ph}{\phi}
\newcommand{\om}{\omega}
\newcommand{\Om}{\Omega}
\newcommand{\sg}{\sigma}
\newcommand{\Sg}{\Sigma}

\newcommand{\Qed}{\begin{flushright}\qed\end{flushright}}
\newcommand{\parinn}{\setlength{\parindent}{1cm}}
\newcommand{\parinf}{\setlength{\parindent}{0cm}}
\newcommand{\del}[2]{\frac{\partial #1}{\partial #2}}
\newcommand{\Del}[3]{\frac{\partial^{#1} #2}{\partial^{#1} #3}}
\newcommand{\deld}[2]{\dfrac{\partial #1}{\partial #2}}
\newcommand{\Deld}[3]{\dfrac{\partial^{#1} #2}{\partial^{#1} #3}}
\newcommand{\uin}{\mathbin{\rotatebox[origin=c]{90}{$\in$}}}
\newcommand{\usubset}{\mathbin{\rotatebox[origin=c]{90}{$\subset$}}}
\newcommand{\lt}{\left}
\newcommand{\rt}{\right}
\newcommand{\exs}{\exists}
\newcommand{\st}{\strut}
\newcommand{\dps}[1]{\displaystyle{#1}}
\newcommand{\la}{\langle}
\newcommand{\ra}{\rangle}
\newcommand{\cls}[1]{\textsc{#1}}
\newcommand{\prb}[1]{\textsc{#1}}
\newcommand{\comb}[2]{\left(\begin{matrix}
		#1\\ #2
\end{matrix}\right)}
%\newcommand[2]{\quotient}{\faktor{#1}{#2}}
\newcommand\quotient[2]{
	\mathchoice
	{% \displaystyle
		\text{\raise1ex\hbox{$#1$}\Big/\lower1ex\hbox{$#2$}}%
	}
	{% \textstyle
		#1\,/\,#2
	}
	{% \scriptstyle
		#1\,/\,#2
	}
	{% \scriptscriptstyle  
		#1\,/\,#2
	}
}

\newcommand{\tensor}{\otimes}
\newcommand{\xor}{\oplus}

\newcommand{\sol}[1]{\begin{solution}#1\end{solution}}
\newcommand{\solve}[1]{\setlength{\parindent}{0cm}\textbf{\textit{Solution: }}\setlength{\parindent}{1cm}#1 \hfill $\blacksquare$}
\newcommand{\mat}[1]{\left[\begin{matrix}#1\end{matrix}\right]}
\newcommand{\matr}[1]{\begin{matrix}#1\end{matrix}}
\newcommand{\matp}[1]{\lt(\begin{matrix}#1\end{matrix}\rt)}
\newcommand{\detmat}[1]{\lt|\begin{matrix}#1\end{matrix}\rt|}
\newcommand\numberthis{\addtocounter{equation}{1}\tag{\theequation}}
\newcommand{\handout}[3]{
	\noindent
	\begin{center}
		\framebox{
			\vbox{
				\hbox to 6.5in { {\bf Complexity Theory I } \hfill Jan -- May, 2023 }
				\vspace{4mm}
				\hbox to 6.5in { {\Large \hfill #1  \hfill} }
				\vspace{2mm}
				\hbox to 6.5in { {\em #2 \hfill #3} }
			}
		}
	\end{center}
	\vspace*{4mm}
}

\newcommand{\lecture}[3]{\handout{Lecture #1}{Lecturer: #2}{Scribe:	#3}}

\let\marvosymLightning\Lightning
\newcommand{\ctr}{\text{\marvosymLightning}\hspace{0.5ex}} % Requires marvosym package

\newcommand{\ov}[1]{\overline{#1}}
\newcommand{\thmref}[1]{\hyperref[th:#1]{Theorem \ref{th:#1}}}
\newcommand{\propref}[1]{\hyperref[th:#1]{Proposition \ref{th:#1}}}
\newcommand{\lmref}[1]{\hyperref[th:#1]{Lemma \ref{th:#1}}}
\newcommand{\corref}[1]{\hyperref[th:#1]{Corollary \ref{th:#1}}}

\newcommand{\thrmref}[1]{\hyperref[#1]{Theorem \ref{#1}}}
\newcommand{\propnref}[1]{\hyperref[#1]{Proposition \ref{#1}}}
\newcommand{\lemref}[1]{\hyperref[#1]{Lemma \ref{#1}}}
\newcommand{\corrref}[1]{\hyperref[#1]{Corollary \ref{#1}}}

\DeclareMathOperator{\enc}{Enc}
\DeclareMathOperator{\res}{Res}
\DeclareMathOperator{\spec}{Spec}
\DeclareMathOperator{\cov}{Cov}
\DeclareMathOperator{\Var}{Var}
\DeclareMathOperator{\Rank}{rank}
\newcommand{\Tfae}{The following are equivalent:}
\newcommand{\tfae}{the following are equivalent:}
\newcommand{\sparsity}{\textit{sparsity}}

\newcommand{\uddots}{\reflectbox{$\ddots$}} 

\newenvironment{claimwidth}{\begin{center}\begin{adjustwidth}{0.05\textwidth}{0.05\textwidth}}{\end{adjustwidth}\end{center}}

\setlength{\parindent}{0pt}

%%%%%%%%%%%%%%%%%%%%%%%%%%%%%%%%%%%%%%%%%%%%%%%%%%%%%%%%%%%%%%%%%%%%%%%%%%%%%%%%%%%%%%%%%%%%%%%%%%%%%%%%%%%%%%%%%%%%%%%%%%%%%%%%%%%%%%%%

\begin{document}
	
	%%%%%%%%%%%%%%%%%%%%%%%%%%%%%%%%%%%%%%%%%%%%%%%%%%%%%%%%%%%%%%%%%%%%%%%%%%%%%%%%%%%%%%%%%%%%%%%%%%%%%%%%%%%%%%%%%%%%%%%%%%%%%%%%%%%%%%%%
	
	\textsf{\noindent \large\textbf{Soham Chatterjee} \hfill \textbf{Assignment - 1}\\
		Email: \href{sohamc@cmi.ac.in}{sohamc@cmi.ac.in} \hfill Roll: BMC202175\\
		\normalsize Course: Expander Graphs and Applications \hfill Date: March 5, 2024}
	
%%%%%%%%%%%%%%%%%%%%%%%%%%%%%%%%%%%%%%%%%%%%%%%%%%%%%%%%%%%%%%%%%%%%%%%%%%%%%%%%%%%%%%%%%%%%%%%%%%%%%%%%%%%%%%%%%%%%%%%%%%%%%%%%%%%%%%%%
% Problem 1
%%%%%%%%%%%%%%%%%%%%%%%%%%%%%%%%%%%%%%%%%%%%%%%%%%%%%%%%%%%%%%%%%%%%%%%%%%%%%%%%%%%%%%%%%%%%%%%%%%%%%%%%%%%%%%%%%%%%%%%%%%%%%%%%%%%%%%%%
	
\begin{problem}{%problem statement
		Problem 4.9 (The Replacement Product): Pseudorandomness By Salil Vadhan
	}{p1% problem reference text
}
Given a $D_1$-regular graph $G_1$ on $N_1$ vertices and a $D_2$-regular graph $G_2$ on $D_1$ vertices consider the following graph $G_1\textcircled{r}G_2$ on vertex set $[N_1]\times [D_1]$: vertex $(u,i)$ is connected to $(v,j)$ iff \begin{enumerate}[label=(\alph*)]
	\item $u=v$ and $(i,j)$ is an edge in $G_2$ or,
	\item $v$ is the $i$'th neighbour of $u$ in $G_1$ and $u$ is the $j$th neighbor of $v$. 
\end{enumerate}
That is, we ``replace" each vertex $v$ in $G_1$ with a copy of $G_2$, associating edge incident to $v$ with one vertex of $G_2$.
\begin{enumerate}
	\item Prove that there is a function $g$ such that if $G_1$ has spectral expansion $\gm_1>0$ and $G_2$ has spectral expansion $\gm_2>0$ (and both graphs are undirected) then $G_1\textcircled{r}G_2$ has spectral expansion $g(\gm_1,\gm_2,D_2)>0$.
	
	[Hint: Note that $(G_1\textcircled{r}G_2)^3$ has $G_1\textcircled{z}G_2$ as a subgraph]
	\item Show how to convert an explicit construction of constant degree (spectral) expanders into an explicit construction of degree 3 (spectral) expanders.
	\item Without using Theorem 4.14, prove an analogue of Part 1 for edge expansion. That is, there is a function $h$ such that if $G_1$ is an $\lt(\frac{N_1}2,\eps_1\rt)$ edge expander and $G_2$ is a $\lt(\frac{D_1}2,\eps_2\rt)$ edge expander then $G_1\textcircled{r}G_2$ is a $\lt(\frac{N_1D_1}{2},h(\eps_1,\eps_2,D_2)\rt)$ edge expander where $h(\eps_1,\eps_2,D_2)>0$ if $\eps_1,\eps_2>0$. 
	
	[Hint: Given any set $S$ of vertices of $G_1\textcircled{r}G_2$, partition $S$ into the clouds that are more than ``half-full" and those that are not]
	\item Prove that the functions $g(\gm_1,\gm_2,D_2)$ and $h(\eps_1,\eps_2,D_2)$ must depend on $D_2$ by showing that $G_1\textcircled{r}G_2$ cannot be a $\lt(\frac{N_1D_1}2,\eps\rt)$ edge expander if $\eps>\frac1{D_1+1}$ and $N_1\geq 2$
\end{enumerate}
\end{problem}
\solve{
\begin{enumerate}
	\item Let $A_1$ and $A_2$ denote the normalized adjacency matrices of $G_1$ and $G_2$ respectively. The degree of the new graph $G_1\textcircled{r}G_2$ is $D_2+1$. Now denote $B\triangleq I_{N_1}\otimes A_2$ and $A$ be a $N_1\cdot D_1\times N_1\cdot D_1$ matrix where $$A[(u,i),(v,j)]=\begin{cases}
		1 & \text{when $i$th neighbor of $u$ is $v$ and $j$th neighbor of $v$ is $u$ in $G_1$}\\
		0 & \text{otherwise}
	\end{cases}$$Therefore the adjacency matrix of the graph $G_1\textcircled{r}G_1$ is $A+D_2B$. Therefore the normalized adjacency matrix, $M$ $$M\triangleq \frac{A+D_2B}{D_2+1}$$
	
	Now notice the graph $(G_1\textcircled{r}G_2)^3$ contains the graph $G_1\textcircled{z}G_2$ as a subgraph. Hence $$M^3=\lt[\frac{A+D_2B}{D_2+1}\rt]^3=\frac{D_2^2}{(D_2+1)^3}BAB+\lt[1-\frac{D_2^2}{(D_2+1)^3}\rt]C$$ for some matrix $C$. Lets denote $p\coloneqq \frac{D_2^2}{(D_2+1)^3}$. Then $M^3=pBAB+(1-p)C$. Hence for any $v\perp u$ where $u$ is the uniform vector we have $$\|M^3v\|\leq p\|BABv\|+(1-p)\|Cv\|$$
	
	Now we can think as $C$ is a normalized adjacency matrix of an undirected graph. Hence for all $v\perp u$ we have $\|Cv\|\leq \|v\|$. Now we know for all $v\perp u$ $$\|BABv\|\leq (\lm_1+\lm_2+\lm_2^2)\|v\|$$ where $\lm_1=1-\gm_1$ and $\lm_2=1-\gm_2$. Hence $$\|M^3v\|\leq p(\lm_1+\lm_2+\lm_2^2)\|v\|+(1-p)\|v\|=[p(\lm_1+\lm_2+\lm_2^2)+(1-p)]\|v\|$$
	
	Now\begin{align*}
		1-[p(\lm_1+\lm_2+\lm_2^2)+(1-p)] & = 1-(1-p)-p(\lm_1+\lm_2+\lm_2^2)\\
		& =p-p(\lm_1+\lm_2+\lm_2^2)\\
		& =p[1-(\lm_1+\lm_2+\lm_2^2)]
	\end{align*}Now we know $$\lm_1+\lm_2+\lm_2^2<1\iff 0<1-(\lm_1+\lm_2+\lm_2^2)<1\qquad \text{and}\qquad 0<p<1$$Then $0<p[1-(\lm_1+\lm_2+\lm_2^2)]<1$. Hence $$0<p(\lm_1+\lm_2+\lm_2^2)+(1-p)<1$$ Therefore for all $v\perp u$, $$\|Mv\|\leq \lt[  p(\lm_1+\lm_2+\lm_2^2)+(1-p)\rt]^{\frac13}\|v\|$$ Now 
	\begin{align*}
		\lt[p(\lm_1+\lm_2+\lm_2^2)+(1-p)\rt]^{\frac13} & = \lt[1- p[1-(\lm_1+\lm_2+\lm_2^2)] \rt]^{\frac13}\\
		& \leq 1-\frac13 p[1-(\lm_1+\lm_2+\lm_2^2)]<1
	\end{align*}So $$g(\gm_1,\gm_2,D_2)=1-	\lt[p(\lm_1+\lm_2+\lm_2^2)+(1-p)\rt]^{\frac13} >0$$
	\item 
\end{enumerate}
}

%%%%%%%%%%%%%%%%%%%%%%%%%%%%%%%%%%%%%%%%%%%%%%%%%%%%%%%%%%%%%%%%%%%%%%%%%%%%%%%%%%%%%%%%%%%%%%%%%%%%%%%%%%%%%%%%%%%%%%%%%%%%%%%%%%%%%%%%
% Problem 2
%%%%%%%%%%%%%%%%%%%%%%%%%%%%%%%%%%%%%%%%%%%%%%%%%%%%%%%%%%%%%%%%%%%%%%%%%%%%%%%%%%%%%%%%%%%%%%%%%%%%%%%%%%%%%%%%%%%%%%%%%%%%%%%%%%%%%%%%

\begin{problem}{%problem statement
		Problem 4.10 (Unbalanced Vertex Expanders and Data Structures): Pseudorandomness By Salil Vadhan
	}{p2% problem reference text
	}
	Consider a $(K,(1-\eps)D)$ bipartite vertex expander $G$ with $N$ left vertices, $M$ right vertices and left degree $D$.
	\begin{enumerate}
		\item For a set $S$ of left vertices, a $y\in N(S)$ is called a \textit{unique} neighbor of $S$ if $y$ is incident to exactly one edge from $S$. Prove that every left-set $S$ of size at most $K$ has at least $(1-2\eps)D|S|$ unique neighbors.
		\item For a set $S$ of size at most $\frac{K}2$, prove that at most $\frac{|S|}2$ vertices outside $S$ have at least $\delta D$ neighbors in $N(S)$ for $\delta=O(\eps)$.
	\end{enumerate}
\end{problem}
\solve{
\begin{enumerate}
	\item Let $U$ be the set of unique neighbors in $N(S)$. Denote $T=\Gm(S)-U$. Then we have $|U\cup T|\geq (1-\eps)D|S|$. Now we will count the number of edges between $S$ and $\Gm(S)$. From each vertex in $S$ there are $D$ edges going out. Hence total $D|S|$ many edges are going out from $S$. Now in $\Gm(S)$ for each vertex in $U$ there is exactly one edge coming from $S$ and for each edge in $T$ there are at least 2 edges coming from $S$. Hence there are at least $|U|+2|T|$ many edges are coming towards $\Gm(S)$. Hence we have:
	
	\begin{multline*}
		|U|+2|T|\leq D|S|\iff |U|+2(|\Gm(S)|-|U|)\leq D|S|\\
		\iff |U|\geq 2|\Gm(S)|-D|S|\geq (1-\eps)D|S|-D|S|=(1-2\eps)D|S|
	\end{multline*}Hence there are at least  $(1-2\eps)D|S|$ unique neighbors.
	\item 
\end{enumerate}
}

%%%%%%%%%%%%%%%%%%%%%%%%%%%%%%%%%%%%%%%%%%%%%%%%%%%%%%%%%%%%%%%%%%%%%%%%%%%%%%%%%%%%%%%%%%%%%%%%%%%%%%%%%%%%%%%%%%%%%%%%%%%%%%%%%%%%%%%%
% Problem 3
%%%%%%%%%%%%%%%%%%%%%%%%%%%%%%%%%%%%%%%%%%%%%%%%%%%%%%%%%%%%%%%%%%%%%%%%%%%%%%%%%%%%%%%%%%%%%%%%%%%%%%%%%%%%%%%%%%%%%%%%%%%%%%%%%%%%%%%%

\begin{problem}{%problem statement
		Problem 5.5 (LDPC Codes): Pseudorandomness By Salil Vadhan
	}{p3% problem reference text
	}
	Given a $D_1$-regular graph $G_1$ on $N_1$ vertices and a $D_2$-regular graph $G_2$ on $D_1$ vertices consider the following graph $G_1\textcircled{r}G_2$ on vertex set $[N_1]\times [D_1]$: vertex $(u,i)$ is connected to $(v,j)$ iff \begin{enumerate}[label=(\alph*)]
		\item $u=v$ and $(i,j)$ is an edge in $G_2$ or,
		\item $v$ is the $i$'th neighbor of $u$ in $G_1$ and $u$ is the $j$th neighbor of $v$. 
	\end{enumerate}
	That is, we ``replace" each vertex $v$ in $G_1$ with a copy of $G_2$, associating edge incident to $v$ with one vertex of $G_2$.
	\begin{enumerate}
		\item Prove that there is a function $g$ such that if $G_1$ has spectral expansion $\gm_1>0$ and $G_2$ has spectral expansion $\gm_2>0$ (and both graphs are undirected) then $G_1\textcircled{r}G_2$ has spectral expansion $g(\gm_1,\gm_2,D_2)>0$.
		
		[Hint: Note that $(G_1\textcircled{r}G_2)^3$ has $G_1\textcircled{z}G_2$ as a subgraph]
		\item Show how to convert an explicit construction of constant degree (spectral) expanders into an explicit construction of degree 3 (spectral) expanders.
		\item Without using Theorem 4.14, prove an analogue of Part 1 for edge expansion. That is, there is a function $h$ such that if $G_1$ is an $\lt(\frac{N_1}2,\eps_1\rt)$ edge expander and $G_2$ is a $\lt(\frac{D_1}2,\eps_2\rt)$ edge expander then $G_1\textcircled{r}G_2$ is a $\lt(\frac{N_1D_1}{2},h(\eps_1,\eps_2,D_2)\rt)$ edge expander where $h(\eps_1,\eps_2,D_2)>0$ if $\eps_1,\eps_2>0$. 
		
		[Hint: Given any set $S$ of vertices of $G_1\textcircled{r}G_2$, partition $S$ into the clouds that are more than ``half-full" and those that are not]
		\item Prove that the functions $g(\gm_1,\gm_2,D_2)$ and $h(\eps_1,\eps_2,D_2)$ must depend on $D_2$ by showing that $G_1\textcircled{r}G_2$ cannot be a $\lt(\frac{N_1D_1}2,\eps\rt)$ edge expander if $\eps>\frac1{D_1+1}$ and $N_1\geq 2$
	\end{enumerate}
\end{problem}

%%%%%%%%%%%%%%%%%%%%%%%%%%%%%%%%%%%%%%%%%%%%%%%%%%%%%%%%%%%%%%%%%%%%%%%%%%%%%%%%%%%%%%%%%%%%%%%%%%%%%%%%%%%%%%%%%%%%%%%%%%%%%%%%%%%%%%%%
% Problem 1
%%%%%%%%%%%%%%%%%%%%%%%%%%%%%%%%%%%%%%%%%%%%%%%%%%%%%%%%%%%%%%%%%%%%%%%%%%%%%%%%%%%%%%%%%%%%%%%%%%%%%%%%%%%%%%%%%%%%%%%%%%%%%%%%%%%%%%%%

\begin{problem}{%problem statement
	}{p1% problem reference text
	}
	Given a $D_1$-regular graph $G_1$ on $N_1$ vertices and a $D_2$-regular graph $G_2$ on $D_1$ vertices consider the following graph $G_1\textcircled{r}G_2$ on vertex set $[N_1]\times [D_1]$: vertex $(u,i)$ is connected to $(v,j)$ iff \begin{enumerate}[label=(\alph*)]
		\item $u=v$ and $(i,j)$ is an edge in $G_2$ or,
		\item $v$ is the $i$'th neighbor of $u$ in $G_1$ and $u$ is the $j$th neighbour of $v$. 
	\end{enumerate}
	That is, we ``replace" each vertex $v$ in $G_1$ with a copy of $G_2$, associating edge incident to $v$ with one vertex of $G_2$.
	\begin{enumerate}
		\item Prove that there is a function $g$ such that if $G_1$ has spectral expansion $\gm_1>0$ and $G_2$ has spectral expansion $\gm_2>0$ (and both graphs are undirected) then $G_1\textcircled{r}G_2$ has spectral expansion $g(\gm_1,\gm_2,D_2)>0$.
		
		[Hint: Note that $(G_1\textcircled{r}G_2)^3$ has $G_1\textcircled{z}G_2$ as a subgraph]
		\item Show how to convert an explicit construction of constant degree (spectral) expanders into an explicit construction of degree 3 (spectral) expanders.
		\item Without using Theorem 4.14, prove an analogue of Part 1 for edge expansion. That is, there is a function $h$ such that if $G_1$ is an $\lt(\frac{N_1}2,\eps_1\rt)$ edge expander and $G_2$ is a $\lt(\frac{D_1}2,\eps_2\rt)$ edge expander then $G_1\textcircled{r}G_2$ is a $\lt(\frac{N_1D_1}{2},h(\eps_1,\eps_2,D_2)\rt)$ edge expander where $h(\eps_1,\eps_2,D_2)>0$ if $\eps_1,\eps_2>0$. 
		
		[Hint: Given any set $S$ of vertices of $G_1\textcircled{r}G_2$, partition $S$ into the clouds that are more than ``half-full" and those that are not]
		\item Prove that the functions $g(\gm_1,\gm_2,D_2)$ and $h(\eps_1,\eps_2,D_2)$ must depend on $D_2$ by showing that $G_1\textcircled{r}G_2$ cannot be a $\lt(\frac{N_1D_1}2,\eps\rt)$ edge expander if $\eps>\frac1{D_1+1}$ and $N_1\geq 2$
	\end{enumerate}
\end{problem}
\end{document}
