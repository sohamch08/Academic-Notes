\documentclass[a4paper, 11pt]{article}
\usepackage{comment} % enables the use of multi-line comments (\ifx \fi) 
\usepackage{fullpage} % changes the margin
\usepackage[a4paper, total={7in, 10in}]{geometry}
\usepackage{amsmath,mathtools,mathdots}
\usepackage{amssymb,amsthm}  % assumes amsmath package installed
\usepackage{float}
\usepackage{xcolor}
\usepackage{mdframed,stmaryrd}
\usepackage[shortlabels]{enumitem}
\usepackage{indentfirst}
\usepackage{hyperref}
\hypersetup{
	colorlinks=true,
	linkcolor=black,
	citecolor=myr,
	filecolor=myr,      
	urlcolor=black,
	pdftitle={Assignment}, %%%%%%%%%%%%%%%%   WRITE ASSIGNMENT PDF NAME  %%%%%%%%%%%%%%%%%%%%
}
\usepackage[most,many,breakable]{tcolorbox}
\usepackage{tikz}
\usepackage{caption}
%\usepackage{kpfonts}
%\usepackage{libertine}
\usepackage{physics}
\usepackage[ruled,vlined,linesnumbered]{algorithm2e}
\usepackage{mathrsfs}
\usepackage{tikz-cd}
\usepackage{float}
\usepackage{tfrupee}  
\usepackage{adjustbox}
\usepackage{wrapfig}
\newcommand{\mathbbm}[1]{\text{\usefont{U}{bbm}{m}{n}#1}}
\AfterEndEnvironment{wrapfigure}{\setlength{\intextsep}{0mm}}
\definecolor{mytheorembg}{HTML}{F2F2F9}
\definecolor{mytheoremfr}{HTML}{00007B}
\definecolor{doc}{RGB}{0,60,110}
\definecolor{myg}{RGB}{56, 140, 70}
\definecolor{myb}{RGB}{45, 111, 177}
\definecolor{myr}{RGB}{199, 68, 64}

\usetikzlibrary{decorations.pathreplacing,angles,quotes,patterns}
\definecolor{mytheorembg}{HTML}{F2F2F9}
\definecolor{mytheoremfr}{HTML}{00007B}
\definecolor{doc}{RGB}{0,60,110}
\definecolor{myg}{RGB}{56, 140, 70}
\definecolor{myb}{RGB}{45, 111, 177}
\definecolor{myr}{RGB}{199, 68, 64}
\newcounter{problem}
\tcbuselibrary{theorems,skins,hooks}
\newtcbtheorem[use counter=problem]{problem}{Problem}
{%
	enhanced,
	breakable,
	colback = white,
	frame hidden,
	boxrule = 0sp,
	borderline west = {2pt}{0pt}{black},
	arc=5pt,
	detach title,
	before upper = \tcbtitle\par\smallskip,
	coltitle = black,
	fonttitle = \bfseries,
	description font = \mdseries,
	separator sign none,
	segmentation style={solid, mytheoremfr},
}
{p}

\newtheorem{lemma}{Lemma}
\newtheorem*{definition*}{Definition}
\renewenvironment{proof}{\noindent{\it \textbf{Proof:}}\hspace*{1em}}{\hfill\qed\bigskip\\}
% To give references for any problem use like this
% suppose the problem number is p3 then 2 options either 
% \hyperref[p:p3]{<text you want to use to hyperlink> \ref{p:p3}}
%                  or directly 
%                   \ref{p:p3}



%---------------------------------------
% BlackBoard Math Fonts :-
%---------------------------------------

%Captital Letters
\newcommand{\bbA}{\mathbb{A}}	\newcommand{\bbB}{\mathbb{B}}
\newcommand{\bbC}{\mathbb{C}}	\newcommand{\bbD}{\mathbb{D}}
\newcommand{\bbE}{\mathbb{E}}	\newcommand{\bbF}{\mathbb{F}}
\newcommand{\bbG}{\mathbb{G}}	\newcommand{\bbH}{\mathbb{H}}
\newcommand{\bbI}{\mathbb{I}}	\newcommand{\bbJ}{\mathbb{J}}
\newcommand{\bbK}{\mathbb{K}}	\newcommand{\bbL}{\mathbb{L}}
\newcommand{\bbM}{\mathbb{M}}	\newcommand{\bbN}{\mathbb{N}}
\newcommand{\bbO}{\mathbb{O}}	\newcommand{\bbP}{\mathbb{P}}
\newcommand{\bbQ}{\mathbb{Q}}	\newcommand{\bbR}{\mathbb{R}}
\newcommand{\bbS}{\mathbb{S}}	\newcommand{\bbT}{\mathbb{T}}
\newcommand{\bbU}{\mathbb{U}}	\newcommand{\bbV}{\mathbb{V}}
\newcommand{\bbW}{\mathbb{W}}	\newcommand{\bbX}{\mathbb{X}}
\newcommand{\bbY}{\mathbb{Y}}	\newcommand{\bbZ}{\mathbb{Z}}

%---------------------------------------
% MathCal Fonts :-
%---------------------------------------

%Captital Letters
\newcommand{\mcA}{\mathcal{A}}	\newcommand{\mcB}{\mathcal{B}}
\newcommand{\mcC}{\mathcal{C}}	\newcommand{\mcD}{\mathcal{D}}
\newcommand{\mcE}{\mathcal{E}}	\newcommand{\mcF}{\mathcal{F}}
\newcommand{\mcG}{\mathcal{G}}	\newcommand{\mcH}{\mathcal{H}}
\newcommand{\mcI}{\mathcal{I}}	\newcommand{\mcJ}{\mathcal{J}}
\newcommand{\mcK}{\mathcal{K}}	\newcommand{\mcL}{\mathcal{L}}
\newcommand{\mcM}{\mathcal{M}}	\newcommand{\mcN}{\mathcal{N}}
\newcommand{\mcO}{\mathcal{O}}	\newcommand{\mcP}{\mathcal{P}}
\newcommand{\mcQ}{\mathcal{Q}}	\newcommand{\mcR}{\mathcal{R}}
\newcommand{\mcS}{\mathcal{S}}	\newcommand{\mcT}{\mathcal{T}}
\newcommand{\mcU}{\mathcal{U}}	\newcommand{\mcV}{\mathcal{V}}
\newcommand{\mcW}{\mathcal{W}}	\newcommand{\mcX}{\mathcal{X}}
\newcommand{\mcY}{\mathcal{Y}}	\newcommand{\mcZ}{\mathcal{Z}}



%---------------------------------------
% Bold Math Fonts :-
%---------------------------------------

%Captital Letters
\newcommand{\bmA}{\boldsymbol{A}}	\newcommand{\bmB}{\boldsymbol{B}}
\newcommand{\bmC}{\boldsymbol{C}}	\newcommand{\bmD}{\boldsymbol{D}}
\newcommand{\bmE}{\boldsymbol{E}}	\newcommand{\bmF}{\boldsymbol{F}}
\newcommand{\bmG}{\boldsymbol{G}}	\newcommand{\bmH}{\boldsymbol{H}}
\newcommand{\bmI}{\boldsymbol{I}}	\newcommand{\bmJ}{\boldsymbol{J}}
\newcommand{\bmK}{\boldsymbol{K}}	\newcommand{\bmL}{\boldsymbol{L}}
\newcommand{\bmM}{\boldsymbol{M}}	\newcommand{\bmN}{\boldsymbol{N}}
\newcommand{\bmO}{\boldsymbol{O}}	\newcommand{\bmP}{\boldsymbol{P}}
\newcommand{\bmQ}{\boldsymbol{Q}}	\newcommand{\bmR}{\boldsymbol{R}}
\newcommand{\bmS}{\boldsymbol{S}}	\newcommand{\bmT}{\boldsymbol{T}}
\newcommand{\bmU}{\boldsymbol{U}}	\newcommand{\bmV}{\boldsymbol{V}}
\newcommand{\bmW}{\boldsymbol{W}}	\newcommand{\bmX}{\boldsymbol{X}}
\newcommand{\bmY}{\boldsymbol{Y}}	\newcommand{\bmZ}{\boldsymbol{Z}}
%Small Letters
\newcommand{\bma}{\boldsymbol{a}}	\newcommand{\bmb}{\boldsymbol{b}}
\newcommand{\bmc}{\boldsymbol{c}}	\newcommand{\bmd}{\boldsymbol{d}}
\newcommand{\bme}{\boldsymbol{e}}	\newcommand{\bmf}{\boldsymbol{f}}
\newcommand{\bmg}{\boldsymbol{g}}	\newcommand{\bmh}{\boldsymbol{h}}
\newcommand{\bmi}{\boldsymbol{i}}	\newcommand{\bmj}{\boldsymbol{j}}
\newcommand{\bmk}{\boldsymbol{k}}	\newcommand{\bml}{\boldsymbol{l}}
\newcommand{\bmm}{\boldsymbol{m}}	\newcommand{\bmn}{\boldsymbol{n}}
\newcommand{\bmo}{\boldsymbol{o}}	\newcommand{\bmp}{\boldsymbol{p}}
\newcommand{\bmq}{\boldsymbol{q}}	\newcommand{\bmr}{\boldsymbol{r}}
\newcommand{\bms}{\boldsymbol{s}}	\newcommand{\bmt}{\boldsymbol{t}}
\newcommand{\bmu}{\boldsymbol{u}}	\newcommand{\bmv}{\boldsymbol{v}}
\newcommand{\bmw}{\boldsymbol{w}}	\newcommand{\bmx}{\boldsymbol{x}}
\newcommand{\bmy}{\boldsymbol{y}}	\newcommand{\bmz}{\boldsymbol{z}}


%---------------------------------------
% Scr Math Fonts :-
%---------------------------------------

\newcommand{\sA}{{\mathscr{A}}}   \newcommand{\sB}{{\mathscr{B}}}
\newcommand{\sC}{{\mathscr{C}}}   \newcommand{\sD}{{\mathscr{D}}}
\newcommand{\sE}{{\mathscr{E}}}   \newcommand{\sF}{{\mathscr{F}}}
\newcommand{\sG}{{\mathscr{G}}}   \newcommand{\sH}{{\mathscr{H}}}
\newcommand{\sI}{{\mathscr{I}}}   \newcommand{\sJ}{{\mathscr{J}}}
\newcommand{\sK}{{\mathscr{K}}}   \newcommand{\sL}{{\mathscr{L}}}
\newcommand{\sM}{{\mathscr{M}}}   \newcommand{\sN}{{\mathscr{N}}}
\newcommand{\sO}{{\mathscr{O}}}   \newcommand{\sP}{{\mathscr{P}}}
\newcommand{\sQ}{{\mathscr{Q}}}   \newcommand{\sR}{{\mathscr{R}}}
\newcommand{\sS}{{\mathscr{S}}}   \newcommand{\sT}{{\mathscr{T}}}
\newcommand{\sU}{{\mathscr{U}}}   \newcommand{\sV}{{\mathscr{V}}}
\newcommand{\sW}{{\mathscr{W}}}   \newcommand{\sX}{{\mathscr{X}}}
\newcommand{\sY}{{\mathscr{Y}}}   \newcommand{\sZ}{{\mathscr{Z}}}


%---------------------------------------
% Math Fraktur Font
%---------------------------------------

%Captital Letters
\newcommand{\mfA}{\mathfrak{A}}	\newcommand{\mfB}{\mathfrak{B}}
\newcommand{\mfC}{\mathfrak{C}}	\newcommand{\mfD}{\mathfrak{D}}
\newcommand{\mfE}{\mathfrak{E}}	\newcommand{\mfF}{\mathfrak{F}}
\newcommand{\mfG}{\mathfrak{G}}	\newcommand{\mfH}{\mathfrak{H}}
\newcommand{\mfI}{\mathfrak{I}}	\newcommand{\mfJ}{\mathfrak{J}}
\newcommand{\mfK}{\mathfrak{K}}	\newcommand{\mfL}{\mathfrak{L}}
\newcommand{\mfM}{\mathfrak{M}}	\newcommand{\mfN}{\mathfrak{N}}
\newcommand{\mfO}{\mathfrak{O}}	\newcommand{\mfP}{\mathfrak{P}}
\newcommand{\mfQ}{\mathfrak{Q}}	\newcommand{\mfR}{\mathfrak{R}}
\newcommand{\mfS}{\mathfrak{S}}	\newcommand{\mfT}{\mathfrak{T}}
\newcommand{\mfU}{\mathfrak{U}}	\newcommand{\mfV}{\mathfrak{V}}
\newcommand{\mfW}{\mathfrak{W}}	\newcommand{\mfX}{\mathfrak{X}}
\newcommand{\mfY}{\mathfrak{Y}}	\newcommand{\mfZ}{\mathfrak{Z}}
%Small Letters
\newcommand{\mfa}{\mathfrak{a}}	\newcommand{\mfb}{\mathfrak{b}}
\newcommand{\mfc}{\mathfrak{c}}	\newcommand{\mfd}{\mathfrak{d}}
\newcommand{\mfe}{\mathfrak{e}}	\newcommand{\mff}{\mathfrak{f}}
\newcommand{\mfg}{\mathfrak{g}}	\newcommand{\mfh}{\mathfrak{h}}
\newcommand{\mfi}{\mathfrak{i}}	\newcommand{\mfj}{\mathfrak{j}}
\newcommand{\mfk}{\mathfrak{k}}	\newcommand{\mfl}{\mathfrak{l}}
\newcommand{\mfm}{\mathfrak{m}}	\newcommand{\mfn}{\mathfrak{n}}
\newcommand{\mfo}{\mathfrak{o}}	\newcommand{\mfp}{\mathfrak{p}}
\newcommand{\mfq}{\mathfrak{q}}	\newcommand{\mfr}{\mathfrak{r}}
\newcommand{\mfs}{\mathfrak{s}}	\newcommand{\mft}{\mathfrak{t}}
\newcommand{\mfu}{\mathfrak{u}}	\newcommand{\mfv}{\mathfrak{v}}
\newcommand{\mfw}{\mathfrak{w}}	\newcommand{\mfx}{\mathfrak{x}}
\newcommand{\mfy}{\mathfrak{y}}	\newcommand{\mfz}{\mathfrak{z}}

%---------------------------------------
% Bar
%---------------------------------------

%Captital Letters
\newcommand{\ovA}{\overline{A}}	\newcommand{\ovB}{\overline{B}}
\newcommand{\ovC}{\overline{C}}	\newcommand{\ovD}{\overline{D}}
\newcommand{\ovE}{\overline{E}}	\newcommand{\ovF}{\overline{F}}
\newcommand{\ovG}{\overline{G}}	\newcommand{\ovH}{\overline{H}}
\newcommand{\ovI}{\overline{I}}	\newcommand{\ovJ}{\overline{J}}
\newcommand{\ovK}{\overline{K}}	\newcommand{\ovL}{\overline{L}}
\newcommand{\ovM}{\overline{M}}	\newcommand{\ovN}{\overline{N}}
\newcommand{\ovO}{\overline{O}}	\newcommand{\ovP}{\overline{P}}
\newcommand{\ovQ}{\overline{Q}}	\newcommand{\ovR}{\overline{R}}
\newcommand{\ovS}{\overline{S}}	\newcommand{\ovT}{\overline{T}}
\newcommand{\ovU}{\overline{U}}	\newcommand{\ovV}{\overline{V}}
\newcommand{\ovW}{\overline{W}}	\newcommand{\ovX}{\overline{X}}
\newcommand{\ovY}{\overline{Y}}	\newcommand{\ovZ}{\overline{Z}}
%Small Letters
\newcommand{\ova}{\overline{a}}	\newcommand{\ovb}{\overline{b}}
\newcommand{\ovc}{\overline{c}}	\newcommand{\ovd}{\overline{d}}
\newcommand{\ove}{\overline{e}}	\newcommand{\ovf}{\overline{f}}
\newcommand{\ovg}{\overline{g}}	\newcommand{\ovh}{\overline{h}}
\newcommand{\ovi}{\overline{i}}	\newcommand{\ovj}{\overline{j}}
\newcommand{\ovk}{\overline{k}}	\newcommand{\ovl}{\overline{l}}
\newcommand{\ovm}{\overline{m}}	\newcommand{\ovn}{\overline{n}}
\newcommand{\ovo}{\overline{o}}	\newcommand{\ovp}{\overline{p}}
\newcommand{\ovq}{\overline{q}}	\newcommand{\ovr}{\overline{r}}
\newcommand{\ovs}{\overline{s}}	\newcommand{\ovt}{\overline{t}}
\newcommand{\ovu}{\overline{u}}	\newcommand{\ovv}{\overline{v}}
\newcommand{\ovw}{\overline{w}}	\newcommand{\ovx}{\overline{x}}
\newcommand{\ovy}{\overline{y}}	\newcommand{\ovz}{\overline{z}}

%---------------------------------------
% Tilde
%---------------------------------------

%Captital Letters
\newcommand{\tdA}{\tilde{A}}	\newcommand{\tdB}{\tilde{B}}
\newcommand{\tdC}{\tilde{C}}	\newcommand{\tdD}{\tilde{D}}
\newcommand{\tdE}{\tilde{E}}	\newcommand{\tdF}{\tilde{F}}
\newcommand{\tdG}{\tilde{G}}	\newcommand{\tdH}{\tilde{H}}
\newcommand{\tdI}{\tilde{I}}	\newcommand{\tdJ}{\tilde{J}}
\newcommand{\tdK}{\tilde{K}}	\newcommand{\tdL}{\tilde{L}}
\newcommand{\tdM}{\tilde{M}}	\newcommand{\tdN}{\tilde{N}}
\newcommand{\tdO}{\tilde{O}}	\newcommand{\tdP}{\tilde{P}}
\newcommand{\tdQ}{\tilde{Q}}	\newcommand{\tdR}{\tilde{R}}
\newcommand{\tdS}{\tilde{S}}	\newcommand{\tdT}{\tilde{T}}
\newcommand{\tdU}{\tilde{U}}	\newcommand{\tdV}{\tilde{V}}
\newcommand{\tdW}{\tilde{W}}	\newcommand{\tdX}{\tilde{X}}
\newcommand{\tdY}{\tilde{Y}}	\newcommand{\tdZ}{\tilde{Z}}
%Small Letters
\newcommand{\tda}{\tilde{a}}	\newcommand{\tdb}{\tilde{b}}
\newcommand{\tdc}{\tilde{c}}	\newcommand{\tdd}{\tilde{d}}
\newcommand{\tde}{\tilde{e}}	\newcommand{\tdf}{\tilde{f}}
\newcommand{\tdg}{\tilde{g}}	\newcommand{\tdh}{\tilde{h}}
\newcommand{\tdi}{\tilde{i}}	\newcommand{\tdj}{\tilde{j}}
\newcommand{\tdk}{\tilde{k}}	\newcommand{\tdl}{\tilde{l}}
\newcommand{\tdm}{\tilde{m}}	\newcommand{\tdn}{\tilde{n}}
\newcommand{\tdo}{\tilde{o}}	\newcommand{\tdp}{\tilde{p}}
\newcommand{\tdq}{\tilde{q}}	\newcommand{\tdr}{\tilde{r}}
\newcommand{\tds}{\tilde{s}}	\newcommand{\tdt}{\tilde{t}}
\newcommand{\tdu}{\tilde{u}}	\newcommand{\tdv}{\tilde{v}}
\newcommand{\tdw}{\tilde{w}}	\newcommand{\tdx}{\tilde{x}}
\newcommand{\tdy}{\tilde{y}}	\newcommand{\tdz}{\tilde{z}}

%---------------------------------------
% Vec
%---------------------------------------

%Captital Letters
\newcommand{\vcA}{\vec{A}}	\newcommand{\vcB}{\vec{B}}
\newcommand{\vcC}{\vec{C}}	\newcommand{\vcD}{\vec{D}}
\newcommand{\vcE}{\vec{E}}	\newcommand{\vcF}{\vec{F}}
\newcommand{\vcG}{\vec{G}}	\newcommand{\vcH}{\vec{H}}
\newcommand{\vcI}{\vec{I}}	\newcommand{\vcJ}{\vec{J}}
\newcommand{\vcK}{\vec{K}}	\newcommand{\vcL}{\vec{L}}
\newcommand{\vcM}{\vec{M}}	\newcommand{\vcN}{\vec{N}}
\newcommand{\vcO}{\vec{O}}	\newcommand{\vcP}{\vec{P}}
\newcommand{\vcQ}{\vec{Q}}	\newcommand{\vcR}{\vec{R}}
\newcommand{\vcS}{\vec{S}}	\newcommand{\vcT}{\vec{T}}
\newcommand{\vcU}{\vec{U}}	\newcommand{\vcV}{\vec{V}}
\newcommand{\vcW}{\vec{W}}	\newcommand{\vcX}{\vec{X}}
\newcommand{\vcY}{\vec{Y}}	\newcommand{\vcZ}{\vec{Z}}
%Small Letters
\newcommand{\vca}{\vec{a}}	\newcommand{\vcb}{\vec{b}}
\newcommand{\vcc}{\vec{c}}	\newcommand{\vcd}{\vec{d}}
\newcommand{\vce}{\vec{e}}	\newcommand{\vcf}{\vec{f}}
\newcommand{\vcg}{\vec{g}}	\newcommand{\vch}{\vec{h}}
\newcommand{\vci}{\vec{i}}	\newcommand{\vcj}{\vec{j}}
\newcommand{\vck}{\vec{k}}	\newcommand{\vcl}{\vec{l}}
\newcommand{\vcm}{\vec{m}}	\newcommand{\vcn}{\vec{n}}
\newcommand{\vco}{\vec{o}}	\newcommand{\vcp}{\vec{p}}
\newcommand{\vcq}{\vec{q}}	\newcommand{\vcr}{\vec{r}}
\newcommand{\vcs}{\vec{s}}	\newcommand{\vct}{\vec{t}}
\newcommand{\vcu}{\vec{u}}	\newcommand{\vcv}{\vec{v}}
%\newcommand{\vcw}{\vec{w}}	\newcommand{\vcx}{\vec{x}}
\newcommand{\vcy}{\vec{y}}	\newcommand{\vcz}{\vec{z}}

%---------------------------------------
% Greek Letters:-
%---------------------------------------
\newcommand{\eps}{\epsilon}
\newcommand{\veps}{\varepsilon}
\newcommand{\lm}{\lambda}
\newcommand{\Lm}{\Lambda}
\newcommand{\gm}{\gamma}
\newcommand{\Gm}{\Gamma}
\newcommand{\vph}{\varphi}
\newcommand{\ph}{\phi}
\newcommand{\om}{\omega}
\newcommand{\Om}{\Omega}
\newcommand{\sg}{\sigma}
\newcommand{\Sg}{\Sigma}

\newcommand{\Qed}{\begin{flushright}\qed\end{flushright}}
\newcommand{\parinn}{\setlength{\parindent}{1cm}}
\newcommand{\parinf}{\setlength{\parindent}{0cm}}
\newcommand{\del}[2]{\frac{\partial #1}{\partial #2}}
\newcommand{\Del}[3]{\frac{\partial^{#1} #2}{\partial^{#1} #3}}
\newcommand{\deld}[2]{\dfrac{\partial #1}{\partial #2}}
\newcommand{\Deld}[3]{\dfrac{\partial^{#1} #2}{\partial^{#1} #3}}
\newcommand{\uin}{\mathbin{\rotatebox[origin=c]{90}{$\in$}}}
\newcommand{\usubset}{\mathbin{\rotatebox[origin=c]{90}{$\subset$}}}
\newcommand{\lt}{\left}
\newcommand{\rt}{\right}
\newcommand{\exs}{\exists}
\newcommand{\st}{\strut}
\newcommand{\dps}[1]{\displaystyle{#1}}
\newcommand{\la}{\langle}
\newcommand{\ra}{\rangle}
\newcommand{\cls}[1]{\textsc{#1}}
\newcommand{\prb}[1]{\textsc{#1}}
\newcommand{\comb}[2]{\left(\begin{matrix}
		#1\\ #2
\end{matrix}\right)}
%\newcommand[2]{\quotient}{\faktor{#1}{#2}}
\newcommand\quotient[2]{
	\mathchoice
	{% \displaystyle
		\text{\raise1ex\hbox{$#1$}\Big/\lower1ex\hbox{$#2$}}%
	}
	{% \textstyle
		#1\,/\,#2
	}
	{% \scriptstyle
		#1\,/\,#2
	}
	{% \scriptscriptstyle  
		#1\,/\,#2
	}
}

\newcommand{\tensor}{\otimes}
\newcommand{\xor}{\oplus}

\newcommand{\sol}[1]{\begin{solution}#1\end{solution}}
\newcommand{\solve}[1]{\setlength{\parindent}{0cm}\textbf{\textit{Solution: }}\setlength{\parindent}{1cm}#1 \hfill $\blacksquare$}
\newcommand{\mat}[1]{\left[\begin{matrix}#1\end{matrix}\right]}
\newcommand{\matr}[1]{\begin{matrix}#1\end{matrix}}
\newcommand{\matp}[1]{\lt(\begin{matrix}#1\end{matrix}\rt)}
\newcommand{\detmat}[1]{\lt|\begin{matrix}#1\end{matrix}\rt|}
\newcommand\numberthis{\addtocounter{equation}{1}\tag{\theequation}}
\newcommand{\handout}[3]{
	\noindent
	\begin{center}
		\framebox{
			\vbox{
				\hbox to 6.5in { {\bf Complexity Theory I } \hfill Jan -- May, 2023 }
				\vspace{4mm}
				\hbox to 6.5in { {\Large \hfill #1  \hfill} }
				\vspace{2mm}
				\hbox to 6.5in { {\em #2 \hfill #3} }
			}
		}
	\end{center}
	\vspace*{4mm}
}

\newcommand{\lecture}[3]{\handout{Lecture #1}{Lecturer: #2}{Scribe:	#3}}

\let\marvosymLightning\Lightning
\newcommand{\ctr}{\text{\marvosymLightning}\hspace{0.5ex}} % Requires marvosym package

\newcommand{\ov}[1]{\overline{#1}}
\newcommand{\thmref}[1]{\hyperref[th:#1]{Theorem \ref{th:#1}}}
\newcommand{\propref}[1]{\hyperref[th:#1]{Proposition \ref{th:#1}}}
\newcommand{\lmref}[1]{\hyperref[th:#1]{Lemma \ref{th:#1}}}
\newcommand{\corref}[1]{\hyperref[th:#1]{Corollary \ref{th:#1}}}

\newcommand{\thrmref}[1]{\hyperref[#1]{Theorem \ref{#1}}}
\newcommand{\propnref}[1]{\hyperref[#1]{Proposition \ref{#1}}}
\newcommand{\lemref}[1]{\hyperref[#1]{Lemma \ref{#1}}}
\newcommand{\corrref}[1]{\hyperref[#1]{Corollary \ref{#1}}}

\DeclareMathOperator{\enc}{Enc}
\DeclareMathOperator{\res}{Res}
\DeclareMathOperator{\spec}{Spec}
\DeclareMathOperator{\cov}{Cov}
\DeclareMathOperator{\Var}{Var}
\DeclareMathOperator{\Rank}{rank}
\newcommand{\Tfae}{The following are equivalent:}
\newcommand{\tfae}{the following are equivalent:}
\newcommand{\sparsity}{\textit{sparsity}}

\newcommand{\uddots}{\reflectbox{$\ddots$}} 

\newenvironment{claimwidth}{\begin{center}\begin{adjustwidth}{0.05\textwidth}{0.05\textwidth}}{\end{adjustwidth}\end{center}}

\setlength{\parindent}{0pt}


%%%%%%%%%%%%%%%%%%%%%%%%%%%%%%%%%%%%%%%%%%%%%%%%%%%%%%%%%%%%%%%%%%%%%%%%%%%%%%%%%%%%%%%%%%%%%%%%%%%%%%%%%%%%%%%%%%%%%%%%%%%%%%%%%%%%%%%%

\begin{document}
	
	%%%%%%%%%%%%%%%%%%%%%%%%%%%%%%%%%%%%%%%%%%%%%%%%%%%%%%%%%%%%%%%%%%%%%%%%%%%%%%%%%%%%%%%%%%%%%%%%%%%%%%%%%%%%%%%%%%%%%%%%%%%%%%%%%%%%%%%%
	
	{\noindent \large\textbf{Soham Chatterjee} \hfill \textbf{Assignment - 4}\\
		Email: \href{soham.chatterjee@tifr.res.in}{soham.chatterjee@tifr.res.in} \hfill Dept: STCS\\
		\normalsize Course: Mathematical Foundations for Computer Sciences \hfill October 29, 2024\\ 
		\noindent\rule{7in}{2.8pt}}
	
%%%%%%%%%%%%%%%%%%%%%%%%%%%%%%%%%%%%%%%%%%%%%%%%%%%%%%%%%%%%%%%%%%%%%%%%%%%%%%%%%%%%%%%%%%%%%%%%%%%%%%%%%%%%%%%%%%%%%%%%%%%%%%%%%%%%%%%%
% Problem 1
%%%%%%%%%%%%%%%%%%%%%%%%%%%%%%%%%%%%%%%%%%%%%%%%%%%%%%%%%%%%%%%%%%%%%%%%%%%%%%%%%%%%%%%%%%%%%%%%%%%%%%%%%%%%%%%%%%%%%%%%%%%%%%%%%%%%%%%%
	
\begin{problem}{%problem statement
	}{p1% problem reference text
}
Let $n>0$. Consider a directed graph on partitions of $n$ (with the parts sorted by size). There is a (directed) edge from a partition to another if the latter can be obtained from the former by moving a "unit" from a part to the one immediately after, while maintaining the order. For example, there is an edge from the partition $25=10+6+4+3+2$ to $25=10+5+5+3+2$ as a "unit" is moved from 6 to 4 and it does not change the order of the parts, but no edge vice-versa. Similarly, there is an edge from the partition $25=10+6+4+3+2$ to $25=10+6+4+3+1+1$. Characterize all the sink partitions (those without any outgoing edges) that can be by starting from the partition $n=n$ (with only one part) and following these edges.
\end{problem}
\solve{For any partition $n=k_1+\cdots +k_m$ where $k_1\geq \ddots \geq k_m$ we represent the partition by the tuple $(k_1,\dots, k_m)$. We call the alteration happened at $i$ on the partition $(k_1,\dots, k_m)$ if there is an edge from $(k_1,\dots, k_m)$ to $(k_1,\dots, k_{i-1},k_i-1,k_{i+1}+1,k_{i+2},\dots, k_m)$.
\begin{lemma}\label{lem11}
	A partition does not have any edge going out if and only if any two consecutive partitions have difference at most 1
\end{lemma}
\begin{proof}
	Consider a partition of $n$. Let ($k_1,\dots, k_m)$ is the partition of $n$. Now $\exs\ i\in[m-1]$ such that $k_i-k_{i+1}\geq 2\iff k_i-1\geq k_{i+1}+1\iff k_{i-1}\geq k_i>k_i-1\geq k_{i+1}+1>k_{i+3}\iff \exs\ $edge going out from the partition to the partition $(k_1,\dots, k_{i-1},k_i-1,k_{i+1}+1,k_{i+2},\dots, k_m)$.
\end{proof}
\begin{lemma}\label{lem12}
	A partition where there exists an element which appears more than 2 times is not reachable from $n$.
\end{lemma}
\begin{proof}
	Let there be a partition of $n$ which is reachable from $n$ and there exists an element which appears more than 2 times. Now in the path from $n$ to that partition consider the partition where first time when that element appeared more than 2 times. Let the partition be $(k_1,\dots, k_m)$. Then the parent of that partition didn't have all $k_i,k_{i+1},k_{i+2}$ equal. Then  alteration from the parent of that partition happened at any of the $\{i-1,i,i+1,i+2\}$ places. \begin{itemize}
		\item \textbf{Case 1:} If alteration happened at $i-1$ then before alteration the parent of the partition had $(k_1,\dots, k_{i-1}+1,k_i-1,k_{i+1},k_{i+2},\dots, k_m)$ and $k_{i}-1<k_{i+1}$. This is not possible. So this is case is not possible.
		\item \textbf{Case 2:} If alteration happened at $i$ then before alteration the parent of the partition had $(k_1,\dots, k_{i}+1,k_{i+1}-1,k_{i+2},\dots, k_m)$ and $k_{i+1}-1<k_{i+2}$. This is not possible. So this is case is not possible.
		\item \textbf{Case 3:} If alteration happened at $i+1$ then before alteration the parent of the partition had $(k_1,\dots,k_i,k_{i+1}+1,k_{i+2}-1,\dots, k_m)$ and $k_{i}<k_{i+1}+1$. This is not possible. So this is case is not possible.
		\item \textbf{Case 4:} If alteration happened at $i+2$ then before alteration the parent of the partition had $(k_1,\dots,k_i,k_{i+1},k_{i+2}+1,k_{i+3}-1,\dots, k_m)$ and $k_{i+1}<k_{i+2}+1$. This is not possible. So this is case is not possible.
	\end{itemize}Therefore none of the cases are possible. Hence the partition has no parent. Hence contradiction. Therefore we have the lemma.
\end{proof}

Hence we get that if a partition is reachable then no element in that partition is repeated more than 2 times. 
\begin{lemma}
	A sink partition reachable from $n$ can have at most one element which appears twice.
\end{lemma}
\begin{proof}
	Suppose that is not the case. Let $(k_1,\dots, k_m)$ be a reachable sink partition of $n$ such that there are at least two elements which appears twice in the partition. Let $i,j\in[m-1]$ such that $i\neq j$ and $i+1<j$ be the closest indexes such that $k_i=k_{i+1}$ and $k_j=k_{j+1}$. Let $k_i=a$ and $k_j=b$ Now by \lemref{lem11} $k_t-k_{t+1}\leq 1$ for all $t\in[m-1]$. Since $i,j$ are closest such that  $k_i=k_{i+1}$ and $k_j=k_{j+1}$ we have $$k_{i+2}=a+1,k_{i+3}=a+2,\dots,k_{j-1}=a+(j-i-2),k_j=a+(j-i-1)=b$$For brevity denote $j-i+1=l$. Then $j=l+1+i$. Certainly by \lemref{lem12} $k_{i-1}>k_i$ and $k_{j+2}<k_{j+1}$. Call $i,j$ as $i_1$ and $j_1$.
	
	Now since the partition is reachable from $n$ there is a path from $n$ to $(k_1,\dots, k_m)$. Consider the partition $(k_{1,1},\dots, k_{l_1,1})$ where first time the above event occurred i.e. $k_{t,1}=k_t$ for all $t\in\{i_1,\dots, j_1+1\}$. Here $l_1\leq m$ as with alterations number of partitions always stays same or increase. Since $(k_1,\dots, k_m)$ is a reachable sink partition $(k_{1,1},\dots, k_{l,1})$ is also reachable. Consider the parent of this partition. From the parent alteration can happen only on the positions $\{i_1-1,\dots, j_1+1\}$. Alteration at any of the positions in $\{i_1-1,\dots, j_1+1\}$ means the parent of the partition had 2 indexes $i_2,j_2$ with $i_2+1<j_2$ which were closer to each other than $i_1,j_1$ and also the elements at those two indexes are different and they appeared twice. So we get a new partition $(k_{1,1}',\dots, k_{l,1}')$ such that there exists indexes $i_2,j_2$ such that $i_2+1<j_2$ and $k_{i_2}'=k_{i_2+1}'$, $k_{j_2}'=k_{j_2+1}'$. 
	
	Again consider the partition $(k_{1,2},\dots, k_{l_2,2})$ where first time the above event occurred i.e. $k_{t,2}=k_{t,1}'$ for all $t\in\{i_2,\dots, j_2+1\}$.  Consider the parent of this partition. From the parent alteration can happen only on the positions $\{i_2-1,\dots, j_2+1\}$. Alteration at any of the positions in $\{i_2-1,\dots, j_2+1\}$ means the parent of the partition had 2 indexes $i_3,j_3$ with $i_3+1<j_3$ which were closer to each other than $i_2,j_2$ and also the elements at those two indexes are different and they appeared twice. So we get a new partition $(k_{1,2}',\dots, k_{l_3,2}')$ such that there exists indexes $i_3,j_3$ such that $i_3+1<j_3$ and $k_{i_3}'=k_{i_3+1}'$, $k_{j_3}'=k_{j_3+1}'$. 
	
	Continuing like this finally we get a partition $(k_{1,s-1}',\dots, k_{l_{s-1},s-1}')$ such that there exists $j\in [l_{s-1}]$ such that $$k_{j,s-1}'=k_{j+1,s-1}'=k_{j+2,s-1}'+1=k_{j+3,s-1}'+1$$And from this partition $(k_{1,s},\dots, k_{l_s,s})$ we take the partition where first time the above event occurred i.e. $k_{t,s}=k_{t,s-1}'$ for all $t\in\{j-1,j,j+1,j+2,j+3\}$. Now for brevity denote $(k_{1,s},\dots, k_{l_s,s})$ as $(k_1,\dots, k_l)$ where $k_{i}=k_{i+1}=k_{i+2}+1=k_{i+3}+1$ for $i\in [l]$. Now we will analyze case wise:
	\begin{itemize}
		\item \textbf{Case 1:} If alteration happened at $i-1$ then before alteration the parent of the partition had $(k_1,\dots, k_{i-1}+1,k_i-1,k_{i+1},k_{i+2},\dots, k_l)$ and $k_{i}-1<k_{i+1}$. This is not possible. So this is case is not possible.
		\item \textbf{Case 2:} If alteration happened at $i$ then before alteration the parent of the partition had $(k_1,\dots, k_{i}+1,k_{i+1}-1,k_{i+2},\dots, k_l)$ and $k_{i+1}-1=k_{i+2}=,k_{i+3}$. By \lemref{lem42} this partition should not be reachable. Hence contradition. This case is not possible.
		\item \textbf{Case 3:} If alteration happened at $i+1$ then before alteration the parent of the partition had $(k_1,\dots,k_i,k_{i+1}+1,k_{i+2}-1,\dots, k_l)$ and $k_{i}<k_{i+1}+1$. This is not possible. So this is case is not possible.
		\item \textbf{Case 4:} If alteration happened at $i+2$ then before alteration the parent of the partition had $(k_1,\dots,k_i,k_{i+1},k_{i+2}+1,k_{i+3}-1,\dots, k_l)$ and $k_i=k_{i+1}=k_{i+2}+1$. By \lemref{lem42} this partition should not be reachable. Hence contradition. This case is not possible.
		\item \textbf{Case 4:} If alteration happened at $i+3$ then before alteration the parent of the partition had $(k_1,\dots,k_i,k_{i+1},k_{i+2},k_{i+3}+1,k_{i+4}-1,\dots, k_l)$ and $k_{i+2}<k_{i+3}+1$. This is not possible. So this is case is not possible.
	\end{itemize}Therefore none of the cases is possible. Hence contradiction. Therefore $(k_1,\dots, k_m)$ is not a reachable sink partition of $n$. Therefore we have the lemma.	
\end{proof}

Now we will show every partition with at most one element appearing twice is reachable from $n$.
\begin{lemma}\label{lem14}
	For a partition $(r,r-1,\dots, r-k)$  is reachable from $n$ for any $r$ and $k$ which follows  $n=\sum\limits_{i=0}^k(r-i)$, $k<r$
\end{lemma}
\begin{proof}
	For any $r$ we will prove this inducting on $k$. For $k=0$ we have $n=r$ then we dont have to do any alteration. Hence the base case follows. Suppose for $k-1$ this is true. Now we have $n=\sum\limits_{i=0}^{k}(r-i)$. Now by inductive hypothesis we can reach $(r,r-1,\dots, r-(k-1))$ starting from $\sum\limits_{i=0}^{k-1}(r-i)$. We will do the same alterations to reach $(r,r-1,\dots, r-(k-1))$ but the starting number will be $\sum\limits_{i=0}^{k}(r-i)$. So after the same alterations we will reach $(2r-k,r-1,\dots, r-(k-1))$. Now we will do the following operations\begin{multline*}
		(2r-k,r-1,\dots, r-(k-1))\to (2r-k-1,r-1+1,\dots, r-(k-1))\to\\
		\cdots\to (2r-k-1,r-1,\dots, r-(k-1)+1) \to (2r-k-1,r-1,\dots, r-(k-1),1)
	\end{multline*}Now afterwards at any stage $i\in[r-k-1]$ we do the operation \begin{multline*}
		(2r-k-1-i,r-1,\dots, r-(k-1),i)\to (2r-k-1-i-1,r-1+1,\dots, r-(k-1),i)\to\\
		\cdots\to(2r-k-1-i-1,r-1,\dots, r-(k-1)+1,i)\to  (2r-k-1-i-1,r-1,\dots, r-(k-1),i+1)
	\end{multline*}We keep doing this operation for $r-k-1$ many times and finally we reach $(r,r-1,\dots, r-(k-1),r-k)$. Hence by mathematical induction this is true for all $k$, $k<r$. Hence we can reach  a partition $(r,r-1,\dots, r-k)$ from $n$ for any $r$ and $k$ which follows  $n=\sum\limits_{i=0}^k(r-i)$, $k<r$.
\end{proof}
\begin{lemma}\label{lem15}
	For a partition $(r,r-1,\dots, r-k,r-k)$  is reachable from $n$ for any $r$ and $k$ which follows  $n=\sum\limits_{i=0}^k(r-i)+(r-k)$, $k<r$
\end{lemma}
\begin{proof}
	We first apply the same operations as to reach $(r,r-1,\dots, r-k)$ but we start from $\sum\limits_{i=0}^k(r-i)+(r-k)$ as described in \lemref{lem14}. Then we reach the partition $(2r-k,r-1,\dots, r-k)$. Now again we do the same operations as in case of the inductive step of the proof of \lemref{lem14}. This propagates the $r-k$ by $1$ element at a time. This finally gives the partition $(r,r-1,\dots, r-k,r-k)$. Hence we can reach $(r,r-1,\dots, r-k,r-k)$ from $n$. 
\end{proof}
\begin{lemma}\label{lem16}
	For a partition $(r,r-1,\dots, r-l,r-l,\dots,r-k )$ is reachable from $n$ for any $r,k,l$  where $n=\sum\limits_{i=0}^k(r-i)+(r-l)$, $r>k> l$ 
\end{lemma}
\begin{proof}
	For a partition $(r+(k+1-l),r-1,\dots, r-l,r-l-1,\dots,r-k,r-k-1 )$ is reachable from $n$ for any $r,k,l$  where $n=\sum\limits_{i=0}^k(r-i)+(r-l)$, $r>k> l$ since $$\sum\limits_{i=0}^k(r-i)+(r-l)=\sum\limits_{i=0}^{k+1}(r-i)+(r-l)-(r-k-1)=\sum\limits_{i=0}^{k+1}(r-i)+(k+1-l)$$following the operations in the proof of \lemref{lem14}. Now we will do the following operation \begin{align*}
		&	(r+(k+1-l),r-1,\dots,r-k-1 )\to (r+(k+1-l)-1,r-1+1,\dots,r-k-1 )\to\\
		&	\cdots\to (r+(k+1-l)-1,r-1,\dots,r-k,r-k )\cdots \to  (r+(k+1-l)-2,r-1,\dots,r-k+1,r-k )\\
		&	\cdots\xrightarrow{i\text{ times}}(r+(k+1-l)-i,r-1,\dots,r-k+i-1,r-k+i-1\dots,r-k )\\
		&	\cdots\xrightarrow{(k+1-l)\text{ times}}(r,r-1,\dots,r-k+(k+1-l-1),r-k+(k+1-l-1)\dots,r-k )\\
		& =(r,\dots, r-l,r-l,\dots, r-k)
	\end{align*}Hence $(r+(k+1-l),r-1,\dots, r-l,r-l-1,\dots,r-k,r-k-1 )$ is reachable from $n$.
\end{proof}

Therefore with all these lemmas we obtain that if a partition is reachable from $n$ then the partition is of the form $(r,r-1,\dots, r-k)$ or $(r,r-1,\dots, r-l,r-l,r-l-1,\dots, r-k)$ where $r>k\geq l$. Notice if $r-k>1$ then we can also break $r-k$ into smaller partitions. Hence if a partition is sink partition then it must end with $1$. Therefore the sink partitions are of the form $(r,r-1,\dots, r-l,r-l,r-l-1,\dots, 1)$ where $n=\sum\limits_{i=0}^r-1(r-i)+(r-l)$ for some $r>l$. 
}\parinf

[I discussed solution of second part with Vivek]\parinn
\addtocounter{problem}{1}
%%%%%%%%%%%%%%%%%%%%%%%%%%%%%%%%%%%%%%%%%%%%%%%%%%%%%%%%%%%%%%%%%%%%%%%%%%%%%%%%%%%%%%%%%%%%%%%%%%%%%%%%%%%%%%%%%%%%%%%%%%%%%%%%%%%%%%%%
% Problem 2
%%%%%%%%%%%%%%%%%%%%%%%%%%%%%%%%%%%%%%%%%%%%%%%%%%%%%%%%%%%%%%%%%%%%%%%%%%%%%%%%%%%%%%%%%%%%%%%%%%%%%%%%%%%%%%%%%%%%%%%%%%%%%%%%%%%%%%%%

%\begin{problem}{%problem statement
%	}{p2% problem reference text
%	}
%Consider a tournament with $k$ teams. Each team gets a rank at the end of the tournament, possibly allowing for ties. For example, if there are 5 teams, team 3 may be first, teams 1, 4 maybe tied second, and teams 2, 5, may be tied third. Compute an explicit formula for the exponential generating function of $R(k)$ for the number of rankings, assuming that $R(0)=1$.
%\end{problem}
%\solve{	
%
%}




%%%%%%%%%%%%%%%%%%%%%%%%%%%%%%%%%%%%%%%%%%%%%%%%%%%%%%%%%%%%%%%%%%%%%%%%%%%%%%%%%%%%%%%%%%%%%%%%%%%%%%%%%%%%%%%%%%%%%%%%%%%%%%%%%%%%%%%%
% Problem 3
%%%%%%%%%%%%%%%%%%%%%%%%%%%%%%%%%%%%%%%%%%%%%%%%%%%%%%%%%%%%%%%%%%%%%%%%%%%%%%%%%%%%%%%%%%%%%%%%%%%%%%%%%%%%%%%%%%%%%%%%%%%%%%%%%%%%%%%%


\begin{problem}{%problem statement
	}{p3% problem reference text
	}
Let $(U, \leq)$ be poset and $M$ be a square matrix with rows and columns indexed by $U$ such that the $(x, y)$-th entry of $M$ is $\mathbbm{1}(x \leq y)$.\begin{itemize}[label=$\bullet$]
	\item Show that $M$ is invertible.
	\item Compute the inverse of $M$ when $(U, \leq)$ is the divisibility poset for positive integers at most $n$ (for some fixed $n$ ) and use it to derive the Möbius inversion formula.
\end{itemize}
\end{problem}
\solve{	
\begin{itemize}[label=$\bullet$]
	\item Consider the directed graph $G=(V,E)$ with vertices $V=U$ and for any $x,y\in U$, $x\neq y$ the directed edge $(x,y)\in E$ if $x\leq y$. Then the adjacency matrix of $G$ is the matrix $M-I$. Since $U$ is a poset the graph $G$ is a directed acyclic graph. Hence in $(M-I)^k$ for any $x,y\in U$, $x\neq y$ the entry $(M-I)^k(x,y)$ is the number of paths from $x\rightsquigarrow y$. Since  the maximum length of a path in $G$ is at most $|U|-1$ there is no path of length $|U|$ between any two vertices. Hence $(M-I)^{|U|}=0$.  Now $(M-I)^{|U|}=0$ implies that the minimal polynomial of $M$ is a divisor of $(x-1)^{|U|}$. Therefore the only eigenvalue of $M$ is $1$. Hence $\det(M)=1$. Therefore $M$ is invertible.
	\item Consider the $n\times n$ matrix $S$ where for all $d,m\in[n]$ such that  $$S(d,n)=\mathbbm{1}(d\mid m)\mu\lt(\frac{m}d\rt)$$We will show that $MS=I$. \parinn
	
	Now for any $m\in[n]$ we have $$MS(m,m)=\sum_{k=1}^n M(m,k)S(k,m)=\sum_{\substack{k\in[n]\\ m\mid k, k\mid m}}\mu\lt(\frac{m}k\rt)=\mu\lt(\frac{m}m\rt)=1$$Hence $MS$ has $1$'s on the diagonals. \parinn
	
	Now take any $m_1,m_2\in[n]$ such that $m_1\neq m_2$. We will show that $MS(m_1,m_2)=0$. Now if $m_1\nmid m_2$ then $\not\exists\ k\in [n]$ such that $m_1\mid k$ and $k\mid m_2$. Therefore $$MS(m_1,m_2)=\sum_{\substack{k\in[n]\\ m_1\mid k, k\mid m_2}}\mu\lt(\frac{m_2}k\rt)=0$$Now assume $m_1\mid m_2$. Now $m_1\neq m_2$ hence $m_2\neq 1$. Let $d=\frac{m_2}{m_1}$. Then we have $$MS(m_1,m_2)=\sum_{\substack{k\in[n]\\ m_1\mid k, k\mid m_2}}\mu\lt(\frac{m_2}k\rt)=\sum_{\substack{ k\mid d}}\mu\lt(\frac{d}{k}\rt)$$\begin{lemma}
		For any positive integer $n$ if $n\neq 1$ then $\sum\limits_{d\mid n}\mu\lt(\frac{n}d\rt)=0$
	\end{lemma}
\begin{proof}
	Let $n\neq 1$ and $n=\prod\limits_{i=1}^kp_i^{e_i}$ be the prime factorization of $n$ where for all $i\in[k]$, $e_i\neq 0$. Now for any $d$, where $d\mid n$, $$\mu\lt(\frac{n}{d}\rt)\neq0\iff\exs\ S\subseteq [k],\ \frac{n}{d}=\prod_{i\in S}p_i$$Now $\mu\lt(\frac{n}{d}\rt)=1$ iff $\exs\ S\subseteq [k],\ \frac{n}{d}=\prod\limits_{i\in S}p_i$ and $|S|\equiv 0\bmod 2$ and $\mu\lt(\frac{n}{d}\rt)=-1$ iff $\exs\ S\subseteq [k],\ \frac{n}{d}=\prod\limits_{i\in S}p_i$ and $|S|\equiv 1\bmod 2$. Therefore we have $$\sum\limits_{d\mid n}\mu\lt(\frac{n}d\rt)=\sum_{S\subseteq [k]}(-1)^{|S|}=\sum_{i=0}^k\binom{k}i(-1)^i=(1+(-1))^k=0$$So we have the lemma.
\end{proof}


Therefore from the lemma we have that $\sum\limits_{\substack{ k\mid d}}\mu\lt(\frac{d}{k}\rt)=0$. Therefore for $m_1\neq m_2\in [n]$ with $m_1\mid m_2$ we have $MS(m_1,m_2)=0$. Hence $MS=I$. Therefore $S$ is the inverse of $M$.

Now we will prove the M\"{o}bius Inversion Formula. Let $f:\bbN\to \bbR$ and $g:\bbN\to\bbR$ be two functions  such that $g(k)=\sum\limits_{d\mid k}f(d)$. Now define the column vectors $F$ and $G$ where for any $i\in[n]$, $F(i)=f(i)$ and $G(i)=g(i)$. Then for any $k\in[n]$ we have $$M^TF(k)=\sum_{d\mid k}f(d)=g(k)\implies M^TF=G$$Therefore $$F=(M^T)^{-1}G=S^TG$$Now $$S^TG(k)=\sum_{d\in[n]}S^T(k,d)g(d)=\sum_{d\in[n]}S(d,k)g(d)=\sum_{d\mid k}\mu\lt(\frac{k}d\rt)g(d)$$Therefore $\sum\limits_{d\mid k}\mu\lt(\frac{k}d\rt)g(d)=f(k)$ for all $k\in[n]$. Therefore we have the M\"{o}bius Formula.
\end{itemize}
}\parinf

[Me and Soumyadeep came up with the solution together and I discussed solution of second part with Aakash]\parinn


%%%%%%%%%%%%%%%%%%%%%%%%%%%%%%%%%%%%%%%%%%%%%%%%%%%%%%%%%%%%%%%%%%%%%%%%%%%%%%%%%%%%%%%%%%%%%%%%%%%%%%%%%%%%%%%%%%%%%%%%%%%%%%%%%%%%%%%%
% Problem 4
%%%%%%%%%%%%%%%%%%%%%%%%%%%%%%%%%%%%%%%%%%%%%%%%%%%%%%%%%%%%%%%%%%%%%%%%%%%%%%%%%%%%%%%%%%%%%%%%%%%%%%%%%%%%%%%%%%%%%%%%%%%%%%%%%%%%%%%%

\begin{problem}{%problem statement
	}{p4% problem reference text
	}
Let $k, p>0$ be integers where $p$ is prime. A seller has $k p$ marbles, where marble $i \in[k p]$ costs \rupee$i$, that he is trading. Call a trade "divisible" if both the cost and the number of marbles sold is a multiple of $p$ (selling no marbles is one such trade). How many divisible trades are there? Write your answer in terms of $p$-th roots of unity.
\end{problem}
\solve{
Consider the polynomial $f(t,c)=\sum\limits_{i=1}^{kp} (1+tc^i)$ where in $i^{th}$ term $(1+tc^i)$  the power of $c$ represent the cost of $i^{th}$ coin and $t$ represents if the $i^{th}$ is taken into consideration. Therefore for any $t^kc^l$ the $s(k,l)\coloneqq\textit{Coeff}(t^kc^l)$ represent the number of ways to pick $k$ coins such that the sum their costs is $l$. Hence $$f(t,c)=\sum\limits_{i=1}^{kp} (1+tc^i)=\sum_{k,l}s(k,l)t^kc^l$$Hence we have to find the coefficients $s(k,l)$ for which $p\mid k$ and $p\mid l$. In order to do that we will use the following lemma:

\begin{lemma}\label{lem41}
	Let $f(x)=\sum\limits_{i=0}^{n}a_ix^i$ where $a_i\in \bbR$ is a polynomial with $a_n\neq 0$. Then $$\frac1p\sum_{i=0}^{p-1}f(\zeta_p^i)=\sum_{k=0,\ p\mid k}^n a_k$$where $\zeta_p$ is the $p^{th}$ root of unity.
\end{lemma}
\begin{proof}
	For any $i\in \{0,1,\dots, p-1\}$, Therefore we have $$
		\sum_{i=0}^{p-1}f(\zeta_p^i)=\sum_{i=0}^{p-1}\sum_{k=0}^na_k\zeta_p^{ik}=\sum_{k=0}^{n}a_k\lt[\sum_{i=0}^{p-1}\zeta_p^{ki}\rt]$$Now for $k\equiv 0\bmod p$ we have $\sum\limits_{i=0}^{p-1}\zeta_p^{ki}=\sum\limits_{i=0}^{p-1}1=p$ and for $k\not\equiv 0\bmod p$ we have $$\sum\limits_{i=0}^{p-1}\zeta_p^{ki}=\sum_{k=0}^{p-1}\zeta_p^i=\frac{z^p-1}{z-1}=0$$Hence we have $$\sum_{i=0}^{p-1}f(\zeta_p^i)=\sum_{k=0}^{n}a_k\lt[\sum_{i=0}^{p-1}\zeta_p^{ki}\rt]=\sum_{k=0,\ p\mid k}^{n}pa_k=p\sum_{k=0,\ p\mid k}^{n}a_k$$Therefore $\frac1p\sum\limits_{i=0}^{p-1}f(\zeta_p^i)=\sum\limits_{k=0,\ p\mid k}^n a_k$.
\end{proof}

We will use this lemma to first find the sum of coefficients in $f(t,c)$ for which the power of $c$ is divisible by $p$ then that becomes a polynomial over $t$ where we take the sum of coefficient for which the power of $t$ is divisible by $p$.

So consider the polynomial $g(t)=\frac1p\sum\limits_{i=0}^{p-1}f(t,\zeta_p^i)$. We have another lemma
\begin{lemma}\label{lem42}
	$f(t,\zeta_p^i)=\lt[ \prod\limits_{i=0}^{p-1}(1+t\zeta_p^i)\rt]^k$
\end{lemma}
\begin{proof}
	We will show that $$\prod\limits_{i=0}^{p-1}\lt(1+t\zeta_p^{lp+i}\rt)= \prod\limits_{i=0}^{p-1}(1+t\zeta_p^i)$$ for any $l\in \{0,\dots, k-1\}$. We have $$\zeta_p^{lp+i}=\lt(\zeta_p^{p}\rt)^l\times \zeta_p^i=\zeta_p^i$$Therefore $$\prod\limits_{i=0}^{p-1}\lt(1+t\zeta_p^{lp+i}\rt)=\prod\limits_{i=0}^{p-1}\lt(1+t\zeta_p^{i}\rt)$$Hence we have $$f(t,\zeta_p^i)=\prod_{l=0}^{k-1}\prod\limits_{i=0}^{p-1}\lt(1+t\zeta_p^{lp+i}\rt)=\prod_{l=0}^{k-1}\prod\limits_{i=0}^{p-1}\lt(1+t\zeta_p^{i}\rt)=\lt[ \prod\limits_{i=0}^{p-1}(1+t\zeta_p^i)\rt]^k$$Hence we have the lemma.
\end{proof}

Hence we have $$g(t)=\frac1p\sum\limits_{i=0}^{p-1}f(t,\zeta_p^i)=\frac1p\lt[f(t,1)+\sum_{i=1}^{p-1}\lt[ \prod\limits_{i=0}^{p-1}(1+t\zeta_p^i)\rt]^k\rt]=\frac1p\lt[ (1+t)^{kp} +(p-1)\lt[ \prod\limits_{i=0}^{p-1}(1+t\zeta_p^i)\rt]^k\rt]$$By \lemref{lem41} $g(t)$ is actually independent of $\zeta_p$. Therefore in $g(t)$ the coefficient of $t^i$ is the sum of coefficients of $s(i,j)$ for all $j$ such that $p\mid j$. Hence $$g(t)=\sum_{i=0}^{kp}\lt[\sum_{j,\ p\mid j}s(i,j)\rt]t^i$$Now again by \lemref{lem41} the sum of the coefficients of the powers of $t$ that are multiple of $p$ is $\frac1p\sum\limits_{i=0}^{p-1}g\lt(\zeta_p^i\rt)$. \begin{align*}
	\frac1p\sum\limits_{i=0}^{p-1}g\lt(\zeta_p^i\rt) & = \frac1p\sum\limits_{i=0}^{p-1}\frac1p\lt[ (1+\zeta_p^{i})^{kp} +(p-1)\lt[ \prod\limits_{j=0}^{p-1}(1+\zeta_p^{i+j})\rt]^k\rt] & [\text{\lemref{lem42}}]\\
	& = \frac1{p^2}\sum\limits_{i=0}^{p-1}(1+\zeta_p^{i})^{kp}+\frac{p-1}{p^2}\sum\limits_{i=0}^{p-1}\lt[ \prod\limits_{j=0}^{p-1}(1+\zeta_p^{i+j})\rt]^k\\
	& = \frac1{p^2}\sum\limits_{i=0}^{p-1}(1+\zeta_p^{i})^{kp}+\frac{p-1}{p^2}\sum\limits_{i=0}^{p-1}\lt[ \prod\limits_{j=0}^{p-1}(1+\zeta_p^{j})\rt]^k\\
	& = \frac1{p^2}\sum\limits_{i=0}^{p-1}(1+\zeta_p^{i})^{kp}+\frac{p-1}{p^2}\;p\lt[ \prod\limits_{i=0}^{p-1}(1+\zeta_p^{i})\rt]^k\\
	& = \frac1{p^2}\sum\limits_{i=0}^{p-1}(1+\zeta_p^{i})^{kp}+\frac{p-1}{p}\lt[ \prod\limits_{i=0}^{p-1}(1+\zeta_p^{i})\rt]^k
\end{align*}Now we will analyze case wise:
\begin{itemize}
	\item \textbf{Case 1:} $p$ is odd prime. Then we have $$\lt[ \prod\limits_{i=0}^{p-1}(1+\zeta_p^{i})\rt]^k=2^k$$Hence we have $$\frac1{p^2}\sum\limits_{i=0}^{p-1}(1+\zeta_p^{i})^{kp}+\frac{p-1}{p}\lt[ \prod\limits_{i=0}^{p-1}(1+\zeta_p^{i})\rt]^k=\frac1{p^2}\sum\limits_{i=0}^{p-1}(1+\zeta_p^{i})^{kp}+\frac{p-1}{p}2^k$$
	\item \textbf{Case 2:} $p=2$. Then $\zeta_2=-1$. Hence $$\frac1{2^2}\sum\limits_{i=0}^{1}(1+(-1)^{i})^{2k}+\frac{1}{2}\lt[ \prod\limits_{i=0}^{1}(1+(-1)^{i})\rt]^k=\frac1{2^2}(1+(-1)^{0})^{2k}+\frac{1}{2}\lt[ \prod\limits_{i=0}^{1}(1+(-1)^{i})\rt]^{k}=2^{2k-2}+0=4^{k-1}$$
\end{itemize}Therefore the number of was to get both the number of marbles sold and the total cost of them divisible by $p$ is $4^{k-1}$ is $p=2$ and $\frac1{p^2}\sum\limits_{i=0}^{p-1}(1+\zeta_p^{i})^{kp}+\frac{p-1}{p}2^k$ if $p$ is an odd prime. 
}\parinf

[I discussed with Vivek]\parinn


%%%%%%%%%%%%%%%%%%%%%%%%%%%%%%%%%%%%%%%%%%%%%%%%%%%%%%%%%%%%%%%%%%%%%%%%%%%%%%%%%%%%%%%%%%%%%%%%%%%%%%%%%%%%%%%%%%%%%%%%%%%%%%%%%%%%%%%%
% Problem 5
%%%%%%%%%%%%%%%%%%%%%%%%%%%%%%%%%%%%%%%%%%%%%%%%%%%%%%%%%%%%%%%%%%%%%%%%%%%%%%%%%%%%%%%%%%%%%%%%%%%%%%%%%%%%%%%%%%%%%%%%%%%%%%%%%%%%%%%%

\begin{problem}{%problem statement
}{p5% problem reference text
}
A graph is a thunderstorm graph if every connected component is either a cycle (clouds) or a path (lightning bolts) or isolated vertices (raindrops). Compute an explicit formula, i.e., a formula without summation signs, for the exponential generating function of:
\begin{itemize}[label=$\bullet$]
	\item The number of connected thunderstorm graphs on the vertex set $[n]$.
	\item The number of thunderstorm graphs on the vertex set $[n]$.
\end{itemize}
\end{problem}
\solve{
\begin{itemize}[label=$\bullet$]
	\item Suppose $n\geq 3$. Since the graph is connected it can be either a path or a cycle. Now from a path with $[n]$ as vertices we get two permutations since if $\sg_1\to\sg_2\to\cdots \sg_n$ is a path then the we get the permutations $(\sg_1,\dots, \sg_n)$ and $(\sg_n,\dots, \sg_1)$. From each each path we get different sets of permutations. Now number of possible  permutations of $[n]$ is $n!$. Hence there are total $\frac{n!}2$ many paths with vertices $[n]$.\parinn
	
	 Now for each cycle of vertices $[n]$ gives $n$ distinct permutations since $\sg_1\to \sg_2\to \cdots\to \sg_n\to\sg_1$, cycle gives the permutations $(\sg_i,\sg_{i+1},\dots, \sg_n,\sg_1,\dots, \sg_{i-1})$ for all $i\in[n]$. For no two cycles we get same set of permutations.  Now if we take a mirror reflection of a cycle we get the same cycle in the graph but a different circular permutation. Hence total number of cycles is $\frac{n!}{2n}=\frac{(n-1)!}2$.  
	 
	 Hence there are total $\frac{n!}{2}+\frac{(n-1)!}2$ many thunderstorm graphs for $n\geq 3$. Now for $n=1,2$ there is only one graph is possible which is a path. Now define $a_n\coloneqq$ number of thunderstorm graphs on the vertex set $[n]$ for all $n\in\bbN$, $n\geq 3$. For $n=1,2$ we have $a_1=1=a_2$. For $n=0$ we have no graphs so $a_0=0$. So the exponential generating function, $g(x)$ is  
	\begin{align*}
\sum_{i=0}^{\infty}a_n\frac{x^n}{n!}&=\sum_{i=1}^{\infty}a_n\frac{x^n}{n!}\\
 &=x+\frac{x^2}{2!}+\sum_{n=3}^{\infty}\lt[\frac{n!}2+\frac{(n-1)!}{2}\rt]\frac{x^n}{n!}\\
 & =x+\frac{x^2}{2!}+\frac12\sum_{n=3}^{\infty}x^n+\frac12\sum_{n=3}^{\infty}\frac{x^n}{n}\\
 & = \frac{x}2-\frac12+\frac12\lt[\sum_{n=0}^{\infty}x^n\rt]+\frac12\lt[  \sum_{n=1}^{\infty}\frac{x^n}{n}   \rt]-\frac{x}2-\frac{x^2}{4}\\
 & = \frac1{2(1-x)}-\frac12\ln(1-x)-\frac{x^2}{4}-\frac12 & \lt[\int\frac{dx}{1-x}=\sum_{n=0}^{\infty}\int{x^n}dx=\sum_{n=1}^{\infty}\frac{x^n}{n}\rt]
	\end{align*}
	\item Fix $n$. The vertices are $[n]$. Suppose there are $k$ connected components. Now we will count the number of thunderstorm graphs with vertices $[n]$ with $k$ connected components. Suppose in an instance the size of the components are $c_1,\dots, c_k$. The number of ways to partition of $[n]$ into $k$ partitions of sizes $c_1,\dots, c_k$ is $\frac{n!}{k!\prod\limits_{i=1}^k c_i!}$. Now if we take $\frac{n!}{l!}   \textit{Coeff}_{x^n}\lt(g^l(x)\rt)$ this computes the number of thunder storm graphs with vertices $[n]$ and $k$ components since $g(x)$ has the number of connected thunderstorm graphs of size $i$ divided by $i!$.
	\begin{align*}
		\sum_{n=0}^{\infty}\frac1{n!}\lt[\sum_{l=1}^{n}\frac{n!}{l!}   \textit{Coeff}_{x^n}\lt(g^l(x)\rt)\rt]x^n&= 	\sum_{n=0}^{\infty}\lt[\sum_{l=1}^{n}\frac1{l!}   \textit{Coeff}_{x^n}\lt(g^l(x)\rt)\rt]x^n\\
	& = 	\sum_{n=0}^{\infty}\lt[\sum_{l=1}^{n} \textit{Coeff}_{x^n}\lt(\frac{g(x)^l}{l!}\rt)\rt]x^n\\
		& = 	\sum_{n=0}^{\infty} \textit{Coeff}_{x^n}\lt(\sum_{l=1}^{n}\frac{g(x)^l}{l!}\rt)x^n\\
		& = 	\sum_{n=0}^{\infty} \textit{Coeff}_{x^n}\lt(\sum_{l=1}^{\infty}\frac{g(x)^l}{l!}\rt)x^n\\
		& = 	\sum_{n=0}^{\infty} \textit{Coeff}_{x^n}\lt(e^{g(x)}-1\rt)x^n = e^{g(x)}-1
	\end{align*}Hence the exponential generating function is $$e^{g(x)}-1=e^{\frac1{2(1-x)}-\frac12\ln(1-x)-\frac{x^2}{4}-\frac12 }-1=\frac{e^{\frac1{2(1-x)}-\frac{x^2}{4}-\frac12 }}{\sqrt{1-x}}$$
\end{itemize}
}\parinf

[Me and Soumyadeep came up with the solution together]\parinn


\end{document}
