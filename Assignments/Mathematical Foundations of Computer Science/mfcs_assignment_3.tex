\documentclass[a4paper, 11pt]{article}
\usepackage{comment} % enables the use of multi-line comments (\ifx \fi) 
\usepackage{fullpage} % changes the margin
\usepackage[a4paper, total={7in, 10in}]{geometry}
\usepackage{amsmath,mathtools,mathdots}
\usepackage{amssymb,amsthm}  % assumes amsmath package installed
\usepackage{float}
\usepackage{xcolor}
\usepackage{mdframed,stmaryrd}
\usepackage[shortlabels]{enumitem}
\usepackage{indentfirst}
\usepackage{hyperref}
\hypersetup{
	colorlinks=true,
	linkcolor=black,
	citecolor=myr,
	filecolor=myr,      
	urlcolor=black,
	pdftitle={Assignment}, %%%%%%%%%%%%%%%%   WRITE ASSIGNMENT PDF NAME  %%%%%%%%%%%%%%%%%%%%
}
\usepackage[most,many,breakable]{tcolorbox}
\usepackage{tikz}
\usepackage{caption}
%\usepackage{kpfonts}
%\usepackage{libertine}
\usepackage{physics}
\usepackage[ruled,vlined,linesnumbered]{algorithm2e}
\usepackage{mathrsfs}
\usepackage{tikz-cd}
\usepackage{float}
\usepackage{tfrupee}  


\definecolor{mytheorembg}{HTML}{F2F2F9}
\definecolor{mytheoremfr}{HTML}{00007B}
\definecolor{doc}{RGB}{0,60,110}
\definecolor{myg}{RGB}{56, 140, 70}
\definecolor{myb}{RGB}{45, 111, 177}
\definecolor{myr}{RGB}{199, 68, 64}

\usetikzlibrary{decorations.pathreplacing,angles,quotes,patterns}
\definecolor{mytheorembg}{HTML}{F2F2F9}
\definecolor{mytheoremfr}{HTML}{00007B}
\definecolor{doc}{RGB}{0,60,110}
\definecolor{myg}{RGB}{56, 140, 70}
\definecolor{myb}{RGB}{45, 111, 177}
\definecolor{myr}{RGB}{199, 68, 64}

\tcbuselibrary{theorems,skins,hooks}
\newtcbtheorem{problem}{Problem}
{%
	enhanced,
	breakable,
	colback = white,
	frame hidden,
	boxrule = 0sp,
	borderline west = {2pt}{0pt}{black},
	arc=5pt,
	detach title,
	before upper = \tcbtitle\par\smallskip,
	coltitle = black,
	fonttitle = \bfseries,
	description font = \mdseries,
	separator sign none,
	segmentation style={solid, mytheoremfr},
}
{p}

\newtheorem{lemma}{Lemma}
\newtheorem*{definition*}{Definition}
\renewenvironment{proof}{\noindent{\it \textbf{Proof:}}\hspace*{1em}}{\hfill $\blacksquare$\bigskip\\}
% To give references for any problem use like this
% suppose the problem number is p3 then 2 options either 
% \hyperref[p:p3]{<text you want to use to hyperlink> \ref{p:p3}}
%                  or directly 
%                   \ref{p:p3}



\input{../../letterfonts}

\input{../../macros}

\setlength{\parindent}{0pt}


%%%%%%%%%%%%%%%%%%%%%%%%%%%%%%%%%%%%%%%%%%%%%%%%%%%%%%%%%%%%%%%%%%%%%%%%%%%%%%%%%%%%%%%%%%%%%%%%%%%%%%%%%%%%%%%%%%%%%%%%%%%%%%%%%%%%%%%%

\begin{document}
	
	%%%%%%%%%%%%%%%%%%%%%%%%%%%%%%%%%%%%%%%%%%%%%%%%%%%%%%%%%%%%%%%%%%%%%%%%%%%%%%%%%%%%%%%%%%%%%%%%%%%%%%%%%%%%%%%%%%%%%%%%%%%%%%%%%%%%%%%%
	
	{\noindent \large\textbf{Soham Chatterjee} \hfill \textbf{Assignment - 2}\\
		Email: \href{soham.chatterjee@tifr.res.in}{soham.chatterjee@tifr.res.in} \hfill Dept: STCS\\
		\normalsize Course: Mathematical Foundations for Computer Sciences \hfill \today\\ 
		\noindent\rule{7in}{2.8pt}}
	
%%%%%%%%%%%%%%%%%%%%%%%%%%%%%%%%%%%%%%%%%%%%%%%%%%%%%%%%%%%%%%%%%%%%%%%%%%%%%%%%%%%%%%%%%%%%%%%%%%%%%%%%%%%%%%%%%%%%%%%%%%%%%%%%%%%%%%%%
% Problem 1
%%%%%%%%%%%%%%%%%%%%%%%%%%%%%%%%%%%%%%%%%%%%%%%%%%%%%%%%%%%%%%%%%%%%%%%%%%%%%%%%%%%%%%%%%%%%%%%%%%%%%%%%%%%%%%%%%%%%%%%%%%%%%%%%%%%%%%%%
	
\begin{problem}{%problem statement
	}{p1% problem reference text
}
Let $m,n>0$ be given and let $S$ be a subset of $[m]\times [n]$. We say $S$ is downward closed if for all $i\leq i'\in [m]$ and $j\leq j'\in [n]$, we have $(i',j')\in S$ only if $(i,j)\in S$. How many downward closed sets are there?
\end{problem}
\solve{

}
%%%%%%%%%%%%%%%%%%%%%%%%%%%%%%%%%%%%%%%%%%%%%%%%%%%%%%%%%%%%%%%%%%%%%%%%%%%%%%%%%%%%%%%%%%%%%%%%%%%%%%%%%%%%%%%%%%%%%%%%%%%%%%%%%%%%%%%%
% Problem 2
%%%%%%%%%%%%%%%%%%%%%%%%%%%%%%%%%%%%%%%%%%%%%%%%%%%%%%%%%%%%%%%%%%%%%%%%%%%%%%%%%%%%%%%%%%%%%%%%%%%%%%%%%%%%%%%%%%%%%%%%%%%%%%%%%%%%%%%%

\begin{problem}{%problem statement
	}{p2% problem reference text
	}
Call an operator $\theta\in L(V)$ unitary if for all $v\in V$, we have $\|\theta(v)\|=\|v\|$ and positive if it is self-adjoint and for all $v\in V$, we have $\la\theta(v),v\ra\geq 0$.\begin{itemize}[label=$\bullet$]
	\item \textbf{Polar Decomposition.} Show that for all $\theta\in L(V)$, there exists a unitary $\mu\in L(V)$ and positive $\pi\in L(V)$ such that $\theta =\mu\circ\pi$.
	
	\textit{Hint: Start by showing $\theta^{\dagger}\circ\theta$ is positive and use the Spectral Theorems.}
	\item \textbf{Singular Value Decomposition.} Let $n=\dim V$. Show that, for all $\theta\in L(V)$, there exists two orthonormal basis $A=\{a_1,\dots, a_n\}$, $B=\{b_1,\dots, b_n\}$ of $V$ and ``singular values" $s_1,\dots, s_n$ such that, for all $v\in V$, we have:$$\theta(v)=\sum_{i=1}^n \la v,b_i\ra\cdot s_i\cdot a_i$$ 
\end{itemize}
\end{problem}
\solve{	
\begin{itemize}[label=$\bullet$]
	\item Let $\dim V=n$. We assume that $\theta$ is a nonzero operator. Since otherwise we can take $\mu$ to be the identity operator and $\pi$ to be the zero operator. Consider the operator $\theta^{\dagger}\circ\theta\in L(V)$. Now for any $v\in V$, $$\la\theta^{\dagger}\circ\theta(v)\ra=\lt\la\theta(v),\lt(\theta^{\dagger}\rt)^{\dagger}(v)\rt\ra=\la\theta(v),\theta(v)\ra=\la v,\theta^{\dagger}\circ(\theta(v))\ra=\la v,\theta^{\dagger}\circ\theta(v)\ra$$Hence $\theta^{\dagger}\circ\theta$ is self-adjoint. Now for any $v\in V$ we also have $$\la \theta^{\dagger}\circ\theta(v),v\ra=\la\theta(v),\theta(v)\ra\geq 0$$Hence $\theta^{\dagger}\circ\theta$ is also positive.  Therefore by spectral theorem there exists an orthonormal eigen basis $B=\{b_1,\dots, b_n\}$ with corresponding eigenvalues $\lm_1,\dots, \lm_n$ such that $\theta^{\dagger}\circ\theta(b_i)=\lm_ib_i$. Since $\theta^{\dagger}\circ\theta$ is positive all the eigenvalues are non-negative and since $\theta$ is nonzero operator not all eigenvalues are zero. \parinn
	
	Now take the set of vectors $B'=\lt\{\frac{1}{\sqrt{\lm_i}}\theta(b_i)\colon \lm_i\neq 0\rt\}$. This set is orthonormal since for $i,j\in [n]$ and $i\neq j$ and $\lm_i,\lm_j\neq 0$ we have $$\lt\la \frac{1}{\sqrt{\lm_i}}\theta(b_i),\frac{1}{\sqrt{\lm_j}}\theta(b_j)\rt\ra=\frac{1}{\sqrt{\lm_i\lm_j}}\la \theta(b_i),\theta(b_j)\ra=\frac{1}{\sqrt{\lm_i\lm_j}}\la \theta^{\dagger}\circ\theta(b_i),b_j\ra=0$$and for $i=j$ we have $$\lt\la \frac{1}{\sqrt{\lm_i}}\theta(b_i),\frac{1}{\sqrt{\lm_i}}\theta(b_i)\rt\ra=\frac{1}{\sqrt{\lm_i\lm_i}}\la \theta(b_i),\theta(b_i)\ra=\frac{1}{{\lm_i}}\la \theta^{\dagger}\circ\theta(b_i),b_i\ra=\frac1{\lm_i}\lm_i\la b_i,b_i\ra=1$$Now $B'$ can be extended to a orthonormal basis $B''=\{b''_i\colon i\in[n]\}$ of $V$ using Gram–Schmidt procedure. For simplicity let first $k$ many vectors of $B$ had nonzero eigenvalues and the vectors $b''_{k+1},\dots, b''_n$ are the new orthonormal added to $B'$ by Gram-Schmidt. Hence for $i\in[k]$ $b''_i=\frac{1}{\sqrt{\lm_i}}\theta(b_i)$. So we define the operator $\mu\in L(V)$ such that for any $i\in [n]$ $$\mu(b_i)=b''_i$$Now also define another operator $\pi\in L(V)$ where $\pi(b_i)=\sqrt{\lm_i} b_i$ for all $i\in[n]$. Both $\mu$ and $\pi$ are defined on basis so they are unique. 
	
	We claim $\theta=\mu\circ\pi$. If we show that for any $i\in[n]$ $\theta(b_i)=\mu\circ\pi(b_i)$ we are done since $B$ is a basis of $V$. Now if $\lm_i\neq 0$ then $$\mu\circ \pi(b_i)=\mu(\sqrt{\lm_i} b_i)=\sqrt
{\lm_i} \mu(b_i)=\sqrt
	{\lm_i} b''_i=\sqrt{\lm_i}\frac{1}{\sqrt{\lm_i}} \theta(b_i)=\theta(b_i)$$When $\lm_i=0$ then we have $$\mu\circ \pi(b_i)=\mu(\sqrt{\lm_i} b_i)=\sqrt
	{\lm_i} \mu(b_i)=0\cdot \mu(b_i)=0$$and on the other hand we have $$\la \theta(b_i),\theta(b_i)\ra=\la \theta^{\dagger}\circ\theta (b_i)\ra=\la \lm_i b_i,b_ir\ra=0$$Hence we have for all $i\in[n]$, $\theta(b_i)=\mu\circ\pi(b_i)$. Hence $\theta=\mu\circ\pi$. 
	
	Now we will show that $\mu$ is unitary and $\pi$ is positive. Now $\pi$ is diagonalizable with respect to an orthonormal eigen basis with all its eigenvalues are non-negative. Hence $\pi$ is positive. So only thing remains is to show that $\mu$ is unitary. Let for any $v\in V$, $v=\sum\limits_{i=1}^n a_ib_i$ where $a_i\in \bbC$. Then we have $$\lt\la \sum\limits_{i=1}^n a_ib_i,\sum\limits_{i=1}^n a_ib_i\rt\ra=\sum\limits_{i=1}^n |a_i|^2\la b_i,b_i\ra=\sum\limits_{i=1}^n |a_i|^2$$On the other hand we have $$\lt\la \mu\lt(\sum\limits_{i=1}^n a_ib_i\rt),\mu\lt(\sum\limits_{i=1}^n a_ib_i\rt)\rt\ra=\lt\la \sum\limits_{i=1}^n a_i\mu(b_i),\sum\limits_{i=1}^n a_i\mu(b_i)\rt\ra=\sum_{i=1}^n |a_i|^2\la b_i,b_i\ra=\sum_{i=1}^n |a_i|^2$$Hence $\mu$ is unitary. Therefore there exists an unitary operator $\mu\in L(V)$ and a positive operator $\pi\in L(V) $ such that $\theta=\mu\circ \pi$. 
	
	\item By the above proof of polar decomposition there exists an unitary operator $\mu\in L(V)$ and positive operator $\pi\in L(V)$ such that $\theta=\mu\circ \pi$. We also get an orthonormal eigenbasis $B=\{b_1,\dots, b_n\}$ of $\theta^{\dagger}\circ \theta$ with corresponding eigenvalues $\lm_1,\dots, \lm_n$ and another orthonormal basis $B''=\{b''_1,\dots, b''_n\}$ where $\mu(b_i)=b''_i$ and $\pi(b_i)=\sqrt{\lm_i}b_i$. Let $v\in V$. Then $v=\sum\limits_{i=1}^n \la v,b_i\ra b_i$. Then $$\theta(v)=\mu\circ\pi\lt(\sum\limits_{i=1}^n \la v,b_i\ra b_i\rt)=\sum\limits_{i=1}^n \la v,b_i\ra \mu(\sqrt{\lm_i}b_i)=\sum\limits_{i=1}^n \la v,b_i\ra\cdot\sqrt{\lm_i}\cdot\mu(b_i)=\sum\limits_{i=1}^n \la v,b_i\ra\cdot\sqrt{\lm_i} \cdot b''_i$$Hence here $A=B''$ and the singular values are eigenvalues of vectors in $B$.
\end{itemize}
}
%%%%%%%%%%%%%%%%%%%%%%%%%%%%%%%%%%%%%%%%%%%%%%%%%%%%%%%%%%%%%%%%%%%%%%%%%%%%%%%%%%%%%%%%%%%%%%%%%%%%%%%%%%%%%%%%%%%%%%%%%%%%%%%%%%%%%%%%
% Problem 3
%%%%%%%%%%%%%%%%%%%%%%%%%%%%%%%%%%%%%%%%%%%%%%%%%%%%%%%%%%%%%%%%%%%%%%%%%%%%%%%%%%%%%%%%%%%%%%%%%%%%%%%%%%%%%%%%%%%%%%%%%%%%%%%%%%%%%%%%
\newpage
\begin{problem}{%problem statement
	}{p3% problem reference text
	}
The following pattern is well known: $$A=\mat{1 & \\ 1 & 1&\\ 1&2 &1 & \\ 1 & 3 & 3 & 1 &\\ 1 & 4 & 6 & 4 & 1&\\ 1 & 5 & 10 & 10 & 5 &1 & \\ 1 & 6 & 15 & 20 & 15& 6 & 1 &\\ \vdots & \vdots & \vdots & \vdots & \vdots &\vdots & \vdots & \ddots }$$For all $n>0$, consider the $n$-th sub-triangle $B_n$ of $A$ defined as follows:\begin{align*}
	B_1=\mat{1} && B_2=\mat{1 &1\\ &2} && B_3=\mat{1 & 2& 1\\ & 3 &3\\ & & 6} && B_4 =\mat{1 &3 & 3 &1\\ & 4 & 6 & 4\\ & & 10 & 10\\ &&&20} && \cdots 
\end{align*}The triangles $B_n$ have the property that for all $i\leq j<n$, it holds that $(B_n)_{i,j}= (B_n)_{i+1,j+1}-(B_n)_{i,j+1}$. For all $n>0$, find the largest number of ones in a matrix of size $n$ that has entries in $\{0,1\}$ and satisfied the foregoing property modulo $2$.
\end{problem}
\solve{	For all $n>0$ we have $$(B_n)_{i,j}= (B_n)_{i+1,j+1}-(B_n)_{i,j+1}\iff (B_n)_{i,j}+(B_n)_{i,j+1}= (B_n)_{i+1,j+1}$$
	First we will prove an upper bound on the number of $1$'s in the triangle. [I got this bound statement from a reddit post\footnote{\url{https://www.reddit.com/r/mathriddles/comments/ojpqgg/binary_pascal_triangle/}}]
\begin{lemma}\label{p3lm1}
		The number of $1$'s in the resulting matrix of size $n>0$ for any $n\in\bbN$ is at most $\frac{n^2+n+1}3$
	\end{lemma}
\begin{proof}
	 We will prove this inductively. For base case $n=1$ we have the number of $1$'s is 1. and $\frac{1^2+1+1}3=1$. Hence the base case follows.
	 
	 Now suppose this is true for $n=1,\dots, k$. For $n=k+1$ we will consider two cases: the first row as either $\leq \frac{2k+2}3$ many $1$'s or $>\frac{2k+2}3$ many $1$'s.
	 
	 Suppose the first row has $\leq \frac{2k+2}{3}$ many $1$'s. Then from next row on wards there are $k$ rows and these $k$ rows can have at most $\frac{k^2+k+1}3$ many $1'$s by Induction Hypothesis. Therefore $$\#1'\text{s}=\frac{k^2+k+1}{3}+\frac{2k+2}{3}=\frac{k^2+3k+3}3=\frac{(k^2+2k+1)+(k+1)+1}3=\frac{(k+1)^2+(k+1)+1}{3}$$Therefore the statement is followed. 
	 
	 Suppose the first row has $>\frac{2k+2}{3}$ i.e. $\geq \frac{2k+3}3$ many $1$'s. Now in the second row each $1$ is originated from a $0$ and a $1$ in the first row. Each $0$ in the first row gives at most two $1$'s in the second row. Therefore $$\#1'\text{s in second row}\leq 2\times \#0'\text{s in first row}$$Hence \begin{align*}
	 	\#1'\text{s in first two rows}&=\#1'\text{s in first row}+\#1'\text{s in second row}\\
	 	& \leq \#1'\text{s in first row}+2\times \#0'\text{s in first row}\\
	 	& = 2(k+1)-\#1'\text{s in first row}\leq 2(k+1)-\frac{2k+3}3  =\frac{4k+3}3
	 \end{align*}Now from third row on wards there are $k-1$ rows and by inductive hypothesis there can be at most $\frac{(k-1)^2+(k-1)+1}3=\frac{k^2-k+1}3$ many $1'$s. Now if $3\mid k$ then from third row on wards there are at most $\frac{k^2-k}3$ many $1$'s are there. Therefore \begin{align*}
	 \#1'\text{s}&=\#1'\text{s from third row on wards}+\#1'\text{s in first two  row}\\
	 & \leq \frac{k^2-k}3+\frac{4k+3}3=\frac{k^2-k+4k+3}3\\
	 &=\frac{(k^2+2k+1)+(k+1)+1}3=\frac{(k+1)^2+(k+1)+1}3
 \end{align*}
If $3\nmid k$ then from third row on wards we keep the bound on the number of $1$'s to be $\frac{(k-1)^2+(k-1)+1}3=\frac{k^2-k+1}{3}$. But now $\frac{4k+3}3$ is not an integer. So the number of $1'$ in the first two rows is at most $\frac{4k+2}3$. Hence we have  \begin{align*}
	\#1'\text{s}&=\#1'\text{s from third row on wards}+\#1'\text{s in first two  row}\\
	& \leq \frac{k^2-k+1}3+\frac{4k+2}3=\frac{k^2-k+1+4k+2}3\\
	&=\frac{(k^2+2k+1)+(k+1)+1}3=\frac{(k+1)^2+(k+1)+1}3
\end{align*}Hence for both cases we have the total number of $1$'s is at most $\frac{(k+1)^2+(k+1)+1}{3}$. Hence by Mathematical Induction the number of $1$'s in the resulting matrix of size $n>0$ for any $n\in\bbN$ is at most $\frac{n^2+n+1}{3}$.
\end{proof}

Having this bound on the number of $1$'s we will now show an instance to achieve this number for any $n>0$. So we will show instances where for any $n>0$ from any $i^{th}$ row on wards the bound $\lt\lfloor \frac{i^2+i+1}{3}\rt\rfloor$ is achieved for all $i\in[n]$. 

Consider the sequence $\{0,1,1\}$. We put them in that order circularly. i.e. $$\begin{array}{lccccccccccc}
\text{Starting with 0: } &        0& 1& 1& 0& 1& 1& 0& 1& 1& \cdots\\
\text{Starting with first 1: } &  1& 1& 0& 1& 1& 0& 1& 1& 0& \cdots\\
\text{Starting with second 1: } & 1& 0& 1& 1& 0& 1& 1& 0& 1& \cdots
\end{array}$$Let $S_n^0$ denote the $n$-length sequence starting with $0$, $S_n^1$ denote $n$-length sequence starting with $1$ and $S_n^2$ denote $n$-length sequence starting with $1$. Now for any $j\in\bbF_3$ and $i\in[n]$, $S_n^j(i)$ denote the $i^{th}$ element in $S_n^j$. And in general for any $i>0$, $i\in\bbN$  the $i^{th}$ element of the sequence starting with $0$ is  by $S^0(i)$, for $i^{th}$ element of the sequence starting with first $1$ denoted by $S^1(i)$ and  for $i^{th}$ element of the sequence starting with second $1$ denoted by $S^2(i)$. Now we have the following relation

\begin{lemma}\label{p3lm2}
Then for any $j\in\bbF_3$ and for any $i>0$ and $i\in\bbN$ $$S^j(i)+S^j(i+1)\equiv S^{j+2}(i)\pmod 2$$Since $j\in\bbF_3$ we take $j+2\bmod 2$. 
\end{lemma}
\begin{proof}
	For $j=0$ we have \begin{align*}
		S^0(i)=\begin{cases}
			0 & \text{If $i\equiv 1\pmod 3$}\\
			1 & \text{Otherwise}
		\end{cases}, && S^1(i)=\begin{cases}
		0 & \text{If $i\equiv 0\pmod 3$}\\
		1 & \text{Otherwise}
	\end{cases}, &&S^2(i)=\begin{cases}
	0 & \text{If $i\equiv 2\pmod 3$}\\
	1 & \text{Otherwise}
\end{cases} 
	\end{align*}
Now we will analyze case wise:\begin{itemize}
	\item \textbf{Case 1:} $i\equiv 0\pmod 3$: Then $S^0(i)=1$, $ S^1(i)=0$ and $ S^{2}(i)=1$ Therefore $S^0(i+1)=0$, $S^1(i+1)=1,$ and $ S^{2}(i+1)=1$. Therefore we have $S^0(i)+S^1(i+1)=1+0=1=S^2(i)$, $S^1(i)+S^1(i+1)=0+1=1=S^0(i)$ and $S^2(i)+S^2(i+1)=1+1\equiv 0=S^1(i)\bmod 2$.
	\item \textbf{Case 2:} $i\equiv 1\pmod 3$: Then $S^0(i)=0$, $ S^1(i)=1$ and $ S^{2}(i)=1$ Therefore $S^0(i+1)=1$, $S^1(i+1)=1,$ and $ S^{2}(i+1)=0$. Therefore we have $S^0(i)+S^1(i+1)=0+1=1=S^2(i)$, $S^1(i)+S^1(i+1)=1+1=\equiv 0=S^0(i)\pmod 2$ and $S^2(i)+S^2(i+1)=1+0=1=S^1(i)$.
	\item \textbf{Case 3:} $i\equiv 2\pmod 3$: Then $S^0(i)=1$, $ S^1(i)=1$ and $ S^{2}(i)=0$ Therefore $S^0(i+1)=1$, $S^1(i+1)=0,$ and $ S^{2}(i+1)=1$. Therefore we have $S^0(i)+S^1(i+1)=1+1\equiv 0=S^2(i)\pmod 2$, $S^1(i)+S^1(i+1)=1+0=1=S^0(i)$ and $S^2(i)+S^2(i+1)=0+1= 0=S^1(i)$.
\end{itemize}Hence we have for all $i>0$ and $i\in\bbN$ and for all $j\in\bbF_3$ we have $S^j(i)+S^j(i+1)\equiv S^{j+2}(i)\pmod 2$. 
\end{proof}Now we will count the number of $1$'s in $S^j_n$ for any $j\in\bbF_3$. First we define the following function $f\bbF_3^2\to\bbF_3$ where we give the values of at all possible inputs by the table below:

\begin{center}
	\begin{tabular}{|c|c|c|c|}
		\hline
		$f(i,j)$ & $j=0$ & $j=1$ & $j=1$ \\\hline
		 $i=0$   &  $0$  &  $0$  &  $0$  \\\hline
		 $i=1$   &  $0$  &  $1$  &  $1$  \\\hline
		 $i=2$   &  $1$  &  $2$  &  $1$\\\hline
	\end{tabular}
\end{center}
\begin{lemma}\label{p3lm3}
	Let $n=3k+i$ where $i\in\{0,1,2\}$ and $k\in\bbN$. Then number of $1'$s in $S_n^j$ for any $j\in \bbF_3$ is $2k+f(i,j)$.
\end{lemma}
\begin{proof}
	For any $i>0$, $i\in\bbN$ and for any $j\in\bbF_3$ in the block $S^j(i), S^j(i+1)$, $S^j(i+2)$ there is exactly two $1$'s and one $0$ since in the sequence $0,1,1$ comes circularly again and again and any 3 consecutive element is just one time appearance of the sequence. Therefore for 3-block there are two $1$'s. Since $n=3k+i$, $S^j(3k)$ has $2k$ many $1$'s. Now we will analyze case wise:\begin{itemize}
		\item \textbf{Case 1} $i=0$: Then $n=3k$. Hence we already know we have $2k$ many $1$'s. And since $f(0,j)=0$ for all $j\in\bbF_3$ we have $2k+f(i,j)$ many $1$'s.
		\item \textbf{Case 2} $i=1$: We have $S^0(n)=0$ and $S^1(n)=S^2(n)=1$. Hence for $S^0$ we see no extra $1$ at $n^{th}$ position. Hence number of $1$'s in $S^0_n$ is $2k+1 =2k+f(1,0)$. For $S^1$ we see an extra $1$ at $n^{th}$ position. We also have $f(1,j)=1$ for $j=1,2$. Therefore number of $1$'s in $S^1_n$ or $S^2_n$ is $2k+1=2k+f(1,j)$ for $j=1,2$. Therefore for $i=1$ number of $1$'s in $S_n^j$ is $2k+f(1,j)$ for $j\in\bbF_3$.
		\item \textbf{Case 3} $i=2$: We have $S^2(n)=1$ and $S^0(n)=S^1(n)=1$. And by case 2 analysis we have $2k$ many $1$'s in $S_{n-1}^0$ and $2k+1$ many $1$'s in both $S_{n-1}^1$ and $S_{n-1}^2$. For $S^0$ there is  $1$ at $n^{th}$ position. Therefore we see an extra  $1$. Hence there are total $2k+1$ many $1$'s. We also have $f(2,0)=1$. Hence there are $2k+f(2,0)$ many $1$'s in $S_n^0$. For $S_n^1$ there is $1$ at $n^{th}$ position. Therefore we see an extra $1$. Therefore there are total $(2k+1)+1=2k+2$ many $1$'s in $S_n^1$. We also have $f(2,1)=2$. Therefore we have $2k+f(2,1)$ many $1$'s in $S_n^1$. Now for $S_n^2$ there is $0$ at $n^{th}$ position. Therefore we have no extra $1$. So the number of $1$'s in $S_n^2$ is same as $S_{n-1}^2$ which is $2k+1$. We have $f(2,2)=1$. So we have $2k+f(2,2)$ many $1$'s in $S_n^2$. Therefore we have for $i=2$ number of $1$'s in $S_n^j$ is $2k+f(2,j)$ for $j\in\bbF_3$. 
	\end{itemize}
Hence by analyzing all possible cases we get that for $n=3k+i$ where $k\in\bbN$ and $i\in\{0,1,2\}$ then number of $1$'s in $S_n^j$ is $2k+f(i,j)$.
\end{proof}

With all these setup for any $n>0$ and $n\in\bbN$ we define the $0-1$ matrix $M_n$ to be the following 
$$M_n=\mat{S_n^{l}\\
&S_{n-1}^{l-1}\\
&& S_{n-2}^{l-2}\\
&&&\ddots \\ &&&&S_1^1 }$$Where $l=n\bmod 3$ and we do the subtraction by 1 in modulo 3. So basically in $M_n$ the $k^{th}$ row  has $k-1$ leading $0$'s then $S_{n-k+1}^{n-k+1\bmod 3}$ for all $k\in[n]$.  Also observe that if we remove the first row and first column from $M_n$ we get $M_{n-1}$. Now by \lemref{p3lm2} $M_n$ follows the rule that $(M_n)_{i,j}+(M_n)_{i,j+1}= (M_n)_{i+1,j+1}$ Now we will show that the total number of $1$'s in $M_n$ is actually $\lt\lfloor \frac{n^2+n+1}{3}\rt\rfloor$. 
\begin{lemma}
	The total number of $1$'s in $M_n$ is $\lt\lfloor \frac{n^2+n+1}{3}\rt\rfloor$.
\end{lemma}
\begin{proof}
	We will prove this inductively on $n$. For $n=1$ we have $\lt\lfloor \frac{n^2+n+1}{3}\rt\rfloor=1$ which is true since $M_1=[S_1^1]=[1]$. Hence the base case follows. Let this is true for $n=1,\dots, l-1$. Now $n=l$ we will analyze case wise. 	Now we have $$\lt\lfloor \frac{l^2+l+1}{3}\rt\rfloor =\begin{cases}
		3k^2+k& \text{When $l=3k$}\\ 3(k^2+k)+1 & \text{When $l=3k+1$}\\ 3k^2+5k+2 & \text{When $l=3k+2$}
	\end{cases}$$Now if we ignore the first row and first column we have $M_{l-1}$. By inductive hypothesis $M_{l-1}$ has $\lt\lfloor \frac{(l-1)^2+(l-1)+1}{3}\rt\rfloor$ many $1$'s 
	\begin{itemize}[label=$\bullet$]
		\item \textbf{Case 1} $\boldsymbol{l=3k}$: Then $l-1=3(k-1)+2$. Then we have $$\lt\lfloor \frac{(l-1)^2+(l-1)+1}{3}\rt\rfloor=3(k-1)^2+5(k-1)+2=3(k^2-2k+1)+5k-5+2=3k^2-k$$And by \lemref{p3lm3} in $S_l^0$ there are $2k+f(0,0)=2k$. Hence total number of $1$'s is $$3k^2-k+2k=3k^2+k=\lt\lfloor \frac{l^2+l+1}{3}\rt\rfloor$$Hence this case follows.
		\item \textbf{Case 2} $\boldsymbol{l=3k+1}$: Then $l-1=3k$. Then we have $$\lt\lfloor \frac{(l-1)^2+(l-1)+1}{3}\rt\rfloor=3k^2+k$$And by \lemref{p3lm3} in $S_l^1$ there are $2k+f(1,1)=2k+1$ many $1$'s. Hence total number of $1$'s is $$3k^2+k+2k+1=3k^2+3k+1=\lt\lfloor \frac{l^2+l+1}{3}\rt\rfloor$$Hence this case follows.
		\item \textbf{Case 3} $\boldsymbol{l=3k+2}$: Then $l-1=3k+1$. Then we have $$\lt\lfloor \frac{(l-1)^2+(l-1)+1}{3}\rt\rfloor=3k^2+3k+1$$And by \lemref{p3lm3} in $S_l^2$ there are $2k+f(2,2)=2k+1$ many $1$'s. Hence total number of $1$'s is $$3k^2+3k+1+2k+1=3k^2+5k+2=\lt\lfloor \frac{l^2+l+1}{3}\rt\rfloor$$Hence this case follows.
	\end{itemize}
	Therefore in all cases $M_l$ has in total $\lt\lfloor \frac{l^2+l+1}{3}\rt\rfloor$ many $1$'s. Therefore by mathematical induction we have that for all $n>0$, $n\in\bbN$ the total number of $1$'s in $M_n$ is $\lt\lfloor \frac{n^2+n+1}{3}\rt\rfloor$.
\end{proof}
Since by \lemref{p3lm1} the maximum number of $1$'s we can achieve is $\lt\lfloor \frac{n^2+n+1}{3}\rt\rfloor$ for $n$ size matrix this sequence of matrices has the maximum number of $1$'s. 
}


%%%%%%%%%%%%%%%%%%%%%%%%%%%%%%%%%%%%%%%%%%%%%%%%%%%%%%%%%%%%%%%%%%%%%%%%%%%%%%%%%%%%%%%%%%%%%%%%%%%%%%%%%%%%%%%%%%%%%%%%%%%%%%%%%%%%%%%%
% Problem 4
%%%%%%%%%%%%%%%%%%%%%%%%%%%%%%%%%%%%%%%%%%%%%%%%%%%%%%%%%%%%%%%%%%%%%%%%%%%%%%%%%%%%%%%%%%%%%%%%%%%%%%%%%%%%%%%%%%%%%%%%%%%%%%%%%%%%%%%%

\begin{problem}{%problem statement
	}{p4% problem reference text
	}
Let $n>0$ be an integer. Count the number of subsets $S\subseteq [n]$ that:\begin{enumerate}[label=(\alph*)]
	\item satisfy $|S|\in S$.
	\item satisfy $|S|\in S$ and that for all $S'\subsetneq S$, we have $|S'|\notin S$.
\end{enumerate}
\end{problem}
\solve{
\begin{enumerate}[label=(\alph*)]
	\item Let $|S|=k$. Therefore $k\in S$. Now rest of the $k-1$ elements are from $[n]\setminus\{k\}$. So the rest $k-1$ elements can be chosen from $[n]\setminus\{k\}$ in $\binom{n-1}{k-1}$ ways. Therefore total number of sets $S\subseteq [n]$ that satisfy $|S|\in S$ is $$\sum_{k=1}^n\binom{n-1}{k-1}=\sum_{k=0}^{n-1}\binom{n-1}{k}=2^{n-1}$$Hence there are $2^{n-1}$ such sets are possible
	\item Let $|S|=k$. Now for all $S'\subsetneq S$, we have $|S'|\notin S$. Hence for all $m<k$, $m\notin S$. Therefore the rest of the $k-1$ elements of $S$ are from $[n]\setminus[k]$. For this to satisfy we should have $n-k\geq k-1\implies \frac{n+1}2\geq k$. For such $k$ the rest $k-1$ elements can be chosen from $[n]\setminus[k]$ in $\binom{n-k}{k-1}$ ways. Hence total number of sets $S\subseteq [n]$ that satisfy the given property is $$\sum_{k=1}^{\lt\lfloor\frac{n+1}2\rt\rfloor}\binom{n-k}{k-1}$$
\end{enumerate}
}

%%%%%%%%%%%%%%%%%%%%%%%%%%%%%%%%%%%%%%%%%%%%%%%%%%%%%%%%%%%%%%%%%%%%%%%%%%%%%%%%%%%%%%%%%%%%%%%%%%%%%%%%%%%%%%%%%%%%%%%%%%%%%%%%%%%%%%%%
% Problem 5
%%%%%%%%%%%%%%%%%%%%%%%%%%%%%%%%%%%%%%%%%%%%%%%%%%%%%%%%%%%%%%%%%%%%%%%%%%%%%%%%%%%%%%%%%%%%%%%%%%%%%%%%%%%%%%%%%%%%%%%%%%%%%%%%%%%%%%%%

\begin{problem}{%problem statement
}{p5% problem reference text
}
A triangulation of a polygon is a partition of its area into (disjoint) triangles with the same vertex set. \begin{itemize}[label=$\bullet$]
	\item Consider a regular polygon with $n$ sides. Show that any triangulation of this polygon has $n-2$ triangles. How many such triangulations are there?
	\item For what values of $n$ is there a triangulation into isosceles triangles? How many such triangulations are there?
\end{itemize}
Use the ideas above to show that a $d$-dimensional polytope that is the intersection of $n-$halfspaces can be partitioned into at most $n^d$ simplices.
\end{problem}
\solve{
}

\end{document}
