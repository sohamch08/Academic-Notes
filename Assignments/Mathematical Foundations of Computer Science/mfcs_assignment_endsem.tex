\documentclass[a4paper, 11pt]{article}
\usepackage{comment} % enables the use of multi-line comments (\ifx \fi) 
\usepackage{fullpage} % changes the margin
\usepackage[a4paper, total={7in, 10in}]{geometry}
\usepackage{amsmath,mathtools,mathdots}
\usepackage{amssymb,amsthm}  % assumes amsmath package installed
\usepackage{float}
\usepackage{xcolor}
\usepackage{mdframed,stmaryrd}
\usepackage[shortlabels]{enumitem}
\usepackage{indentfirst}
\usepackage{hyperref}
\hypersetup{
	colorlinks=true,
	linkcolor=black,
	citecolor=myr,
	filecolor=myr,      
	urlcolor=black,
	pdftitle={Assignment}, %%%%%%%%%%%%%%%%   WRITE ASSIGNMENT PDF NAME  %%%%%%%%%%%%%%%%%%%%
}
\usepackage[most,many,breakable]{tcolorbox}
\usepackage{tikz}
\usepackage{caption}
%\usepackage{kpfonts}
%\usepackage{libertine}
\usepackage{physics}
\usepackage[ruled,vlined,linesnumbered]{algorithm2e}
\usepackage{mathrsfs}
\usepackage{tikz-cd}
\usepackage{float}
\usepackage{tfrupee}  
\usepackage{adjustbox}
\usepackage{wrapfig}
\usepackage{listings}
\newcommand{\mathbbm}[1]{\text{\usefont{U}{bbm}{m}{n}#1}}
\AfterEndEnvironment{wrapfigure}{\setlength{\intextsep}{0mm}}
\definecolor{mytheorembg}{HTML}{F2F2F9}
\definecolor{mytheoremfr}{HTML}{00007B}
\definecolor{doc}{RGB}{0,60,110}
\definecolor{myg}{RGB}{56, 140, 70}
\definecolor{myb}{RGB}{45, 111, 177}
\definecolor{myr}{RGB}{199, 68, 64}

\usetikzlibrary{decorations.pathreplacing,angles,quotes,patterns}
\definecolor{mytheorembg}{HTML}{F2F2F9}
\definecolor{mytheoremfr}{HTML}{00007B}
\definecolor{doc}{RGB}{0,60,110}
\definecolor{myg}{RGB}{56, 140, 70}
\definecolor{myb}{RGB}{45, 111, 177}
\definecolor{myr}{RGB}{199, 68, 64}
\newcounter{problem}
\tcbuselibrary{theorems,skins,hooks}
\newtcbtheorem[use counter=problem]{problem}{Problem}
{%
	enhanced,
	breakable,
	colback = white,
	frame hidden,
	boxrule = 0sp,
	borderline west = {2pt}{0pt}{black},
	arc=5pt,
	detach title,
	before upper = \tcbtitle\par\smallskip,
	coltitle = black,
	fonttitle = \bfseries,
	description font = \mdseries,
	separator sign none,
	segmentation style={solid, mytheoremfr},
}
{p}

\newtheorem{lemma}{Lemma}
\newtheorem*{definition*}{Definition}
\renewenvironment{proof}{\noindent{\it \textbf{Proof:}}\hspace*{1em}}{\hfill\qed\bigskip\\}
% To give references for any problem use like this
% suppose the problem number is p3 then 2 options either 
% \hyperref[p:p3]{<text you want to use to hyperlink> \ref{p:p3}}
%                  or directly 
%                   \ref{p:p3}



%---------------------------------------
% BlackBoard Math Fonts :-
%---------------------------------------

%Captital Letters
\newcommand{\bbA}{\mathbb{A}}	\newcommand{\bbB}{\mathbb{B}}
\newcommand{\bbC}{\mathbb{C}}	\newcommand{\bbD}{\mathbb{D}}
\newcommand{\bbE}{\mathbb{E}}	\newcommand{\bbF}{\mathbb{F}}
\newcommand{\bbG}{\mathbb{G}}	\newcommand{\bbH}{\mathbb{H}}
\newcommand{\bbI}{\mathbb{I}}	\newcommand{\bbJ}{\mathbb{J}}
\newcommand{\bbK}{\mathbb{K}}	\newcommand{\bbL}{\mathbb{L}}
\newcommand{\bbM}{\mathbb{M}}	\newcommand{\bbN}{\mathbb{N}}
\newcommand{\bbO}{\mathbb{O}}	\newcommand{\bbP}{\mathbb{P}}
\newcommand{\bbQ}{\mathbb{Q}}	\newcommand{\bbR}{\mathbb{R}}
\newcommand{\bbS}{\mathbb{S}}	\newcommand{\bbT}{\mathbb{T}}
\newcommand{\bbU}{\mathbb{U}}	\newcommand{\bbV}{\mathbb{V}}
\newcommand{\bbW}{\mathbb{W}}	\newcommand{\bbX}{\mathbb{X}}
\newcommand{\bbY}{\mathbb{Y}}	\newcommand{\bbZ}{\mathbb{Z}}

%---------------------------------------
% MathCal Fonts :-
%---------------------------------------

%Captital Letters
\newcommand{\mcA}{\mathcal{A}}	\newcommand{\mcB}{\mathcal{B}}
\newcommand{\mcC}{\mathcal{C}}	\newcommand{\mcD}{\mathcal{D}}
\newcommand{\mcE}{\mathcal{E}}	\newcommand{\mcF}{\mathcal{F}}
\newcommand{\mcG}{\mathcal{G}}	\newcommand{\mcH}{\mathcal{H}}
\newcommand{\mcI}{\mathcal{I}}	\newcommand{\mcJ}{\mathcal{J}}
\newcommand{\mcK}{\mathcal{K}}	\newcommand{\mcL}{\mathcal{L}}
\newcommand{\mcM}{\mathcal{M}}	\newcommand{\mcN}{\mathcal{N}}
\newcommand{\mcO}{\mathcal{O}}	\newcommand{\mcP}{\mathcal{P}}
\newcommand{\mcQ}{\mathcal{Q}}	\newcommand{\mcR}{\mathcal{R}}
\newcommand{\mcS}{\mathcal{S}}	\newcommand{\mcT}{\mathcal{T}}
\newcommand{\mcU}{\mathcal{U}}	\newcommand{\mcV}{\mathcal{V}}
\newcommand{\mcW}{\mathcal{W}}	\newcommand{\mcX}{\mathcal{X}}
\newcommand{\mcY}{\mathcal{Y}}	\newcommand{\mcZ}{\mathcal{Z}}



%---------------------------------------
% Bold Math Fonts :-
%---------------------------------------

%Captital Letters
\newcommand{\bmA}{\boldsymbol{A}}	\newcommand{\bmB}{\boldsymbol{B}}
\newcommand{\bmC}{\boldsymbol{C}}	\newcommand{\bmD}{\boldsymbol{D}}
\newcommand{\bmE}{\boldsymbol{E}}	\newcommand{\bmF}{\boldsymbol{F}}
\newcommand{\bmG}{\boldsymbol{G}}	\newcommand{\bmH}{\boldsymbol{H}}
\newcommand{\bmI}{\boldsymbol{I}}	\newcommand{\bmJ}{\boldsymbol{J}}
\newcommand{\bmK}{\boldsymbol{K}}	\newcommand{\bmL}{\boldsymbol{L}}
\newcommand{\bmM}{\boldsymbol{M}}	\newcommand{\bmN}{\boldsymbol{N}}
\newcommand{\bmO}{\boldsymbol{O}}	\newcommand{\bmP}{\boldsymbol{P}}
\newcommand{\bmQ}{\boldsymbol{Q}}	\newcommand{\bmR}{\boldsymbol{R}}
\newcommand{\bmS}{\boldsymbol{S}}	\newcommand{\bmT}{\boldsymbol{T}}
\newcommand{\bmU}{\boldsymbol{U}}	\newcommand{\bmV}{\boldsymbol{V}}
\newcommand{\bmW}{\boldsymbol{W}}	\newcommand{\bmX}{\boldsymbol{X}}
\newcommand{\bmY}{\boldsymbol{Y}}	\newcommand{\bmZ}{\boldsymbol{Z}}
%Small Letters
\newcommand{\bma}{\boldsymbol{a}}	\newcommand{\bmb}{\boldsymbol{b}}
\newcommand{\bmc}{\boldsymbol{c}}	\newcommand{\bmd}{\boldsymbol{d}}
\newcommand{\bme}{\boldsymbol{e}}	\newcommand{\bmf}{\boldsymbol{f}}
\newcommand{\bmg}{\boldsymbol{g}}	\newcommand{\bmh}{\boldsymbol{h}}
\newcommand{\bmi}{\boldsymbol{i}}	\newcommand{\bmj}{\boldsymbol{j}}
\newcommand{\bmk}{\boldsymbol{k}}	\newcommand{\bml}{\boldsymbol{l}}
\newcommand{\bmm}{\boldsymbol{m}}	\newcommand{\bmn}{\boldsymbol{n}}
\newcommand{\bmo}{\boldsymbol{o}}	\newcommand{\bmp}{\boldsymbol{p}}
\newcommand{\bmq}{\boldsymbol{q}}	\newcommand{\bmr}{\boldsymbol{r}}
\newcommand{\bms}{\boldsymbol{s}}	\newcommand{\bmt}{\boldsymbol{t}}
\newcommand{\bmu}{\boldsymbol{u}}	\newcommand{\bmv}{\boldsymbol{v}}
\newcommand{\bmw}{\boldsymbol{w}}	\newcommand{\bmx}{\boldsymbol{x}}
\newcommand{\bmy}{\boldsymbol{y}}	\newcommand{\bmz}{\boldsymbol{z}}


%---------------------------------------
% Scr Math Fonts :-
%---------------------------------------

\newcommand{\sA}{{\mathscr{A}}}   \newcommand{\sB}{{\mathscr{B}}}
\newcommand{\sC}{{\mathscr{C}}}   \newcommand{\sD}{{\mathscr{D}}}
\newcommand{\sE}{{\mathscr{E}}}   \newcommand{\sF}{{\mathscr{F}}}
\newcommand{\sG}{{\mathscr{G}}}   \newcommand{\sH}{{\mathscr{H}}}
\newcommand{\sI}{{\mathscr{I}}}   \newcommand{\sJ}{{\mathscr{J}}}
\newcommand{\sK}{{\mathscr{K}}}   \newcommand{\sL}{{\mathscr{L}}}
\newcommand{\sM}{{\mathscr{M}}}   \newcommand{\sN}{{\mathscr{N}}}
\newcommand{\sO}{{\mathscr{O}}}   \newcommand{\sP}{{\mathscr{P}}}
\newcommand{\sQ}{{\mathscr{Q}}}   \newcommand{\sR}{{\mathscr{R}}}
\newcommand{\sS}{{\mathscr{S}}}   \newcommand{\sT}{{\mathscr{T}}}
\newcommand{\sU}{{\mathscr{U}}}   \newcommand{\sV}{{\mathscr{V}}}
\newcommand{\sW}{{\mathscr{W}}}   \newcommand{\sX}{{\mathscr{X}}}
\newcommand{\sY}{{\mathscr{Y}}}   \newcommand{\sZ}{{\mathscr{Z}}}


%---------------------------------------
% Math Fraktur Font
%---------------------------------------

%Captital Letters
\newcommand{\mfA}{\mathfrak{A}}	\newcommand{\mfB}{\mathfrak{B}}
\newcommand{\mfC}{\mathfrak{C}}	\newcommand{\mfD}{\mathfrak{D}}
\newcommand{\mfE}{\mathfrak{E}}	\newcommand{\mfF}{\mathfrak{F}}
\newcommand{\mfG}{\mathfrak{G}}	\newcommand{\mfH}{\mathfrak{H}}
\newcommand{\mfI}{\mathfrak{I}}	\newcommand{\mfJ}{\mathfrak{J}}
\newcommand{\mfK}{\mathfrak{K}}	\newcommand{\mfL}{\mathfrak{L}}
\newcommand{\mfM}{\mathfrak{M}}	\newcommand{\mfN}{\mathfrak{N}}
\newcommand{\mfO}{\mathfrak{O}}	\newcommand{\mfP}{\mathfrak{P}}
\newcommand{\mfQ}{\mathfrak{Q}}	\newcommand{\mfR}{\mathfrak{R}}
\newcommand{\mfS}{\mathfrak{S}}	\newcommand{\mfT}{\mathfrak{T}}
\newcommand{\mfU}{\mathfrak{U}}	\newcommand{\mfV}{\mathfrak{V}}
\newcommand{\mfW}{\mathfrak{W}}	\newcommand{\mfX}{\mathfrak{X}}
\newcommand{\mfY}{\mathfrak{Y}}	\newcommand{\mfZ}{\mathfrak{Z}}
%Small Letters
\newcommand{\mfa}{\mathfrak{a}}	\newcommand{\mfb}{\mathfrak{b}}
\newcommand{\mfc}{\mathfrak{c}}	\newcommand{\mfd}{\mathfrak{d}}
\newcommand{\mfe}{\mathfrak{e}}	\newcommand{\mff}{\mathfrak{f}}
\newcommand{\mfg}{\mathfrak{g}}	\newcommand{\mfh}{\mathfrak{h}}
\newcommand{\mfi}{\mathfrak{i}}	\newcommand{\mfj}{\mathfrak{j}}
\newcommand{\mfk}{\mathfrak{k}}	\newcommand{\mfl}{\mathfrak{l}}
\newcommand{\mfm}{\mathfrak{m}}	\newcommand{\mfn}{\mathfrak{n}}
\newcommand{\mfo}{\mathfrak{o}}	\newcommand{\mfp}{\mathfrak{p}}
\newcommand{\mfq}{\mathfrak{q}}	\newcommand{\mfr}{\mathfrak{r}}
\newcommand{\mfs}{\mathfrak{s}}	\newcommand{\mft}{\mathfrak{t}}
\newcommand{\mfu}{\mathfrak{u}}	\newcommand{\mfv}{\mathfrak{v}}
\newcommand{\mfw}{\mathfrak{w}}	\newcommand{\mfx}{\mathfrak{x}}
\newcommand{\mfy}{\mathfrak{y}}	\newcommand{\mfz}{\mathfrak{z}}

%---------------------------------------
% Bar
%---------------------------------------

%Captital Letters
\newcommand{\ovA}{\overline{A}}	\newcommand{\ovB}{\overline{B}}
\newcommand{\ovC}{\overline{C}}	\newcommand{\ovD}{\overline{D}}
\newcommand{\ovE}{\overline{E}}	\newcommand{\ovF}{\overline{F}}
\newcommand{\ovG}{\overline{G}}	\newcommand{\ovH}{\overline{H}}
\newcommand{\ovI}{\overline{I}}	\newcommand{\ovJ}{\overline{J}}
\newcommand{\ovK}{\overline{K}}	\newcommand{\ovL}{\overline{L}}
\newcommand{\ovM}{\overline{M}}	\newcommand{\ovN}{\overline{N}}
\newcommand{\ovO}{\overline{O}}	\newcommand{\ovP}{\overline{P}}
\newcommand{\ovQ}{\overline{Q}}	\newcommand{\ovR}{\overline{R}}
\newcommand{\ovS}{\overline{S}}	\newcommand{\ovT}{\overline{T}}
\newcommand{\ovU}{\overline{U}}	\newcommand{\ovV}{\overline{V}}
\newcommand{\ovW}{\overline{W}}	\newcommand{\ovX}{\overline{X}}
\newcommand{\ovY}{\overline{Y}}	\newcommand{\ovZ}{\overline{Z}}
%Small Letters
\newcommand{\ova}{\overline{a}}	\newcommand{\ovb}{\overline{b}}
\newcommand{\ovc}{\overline{c}}	\newcommand{\ovd}{\overline{d}}
\newcommand{\ove}{\overline{e}}	\newcommand{\ovf}{\overline{f}}
\newcommand{\ovg}{\overline{g}}	\newcommand{\ovh}{\overline{h}}
\newcommand{\ovi}{\overline{i}}	\newcommand{\ovj}{\overline{j}}
\newcommand{\ovk}{\overline{k}}	\newcommand{\ovl}{\overline{l}}
\newcommand{\ovm}{\overline{m}}	\newcommand{\ovn}{\overline{n}}
\newcommand{\ovo}{\overline{o}}	\newcommand{\ovp}{\overline{p}}
\newcommand{\ovq}{\overline{q}}	\newcommand{\ovr}{\overline{r}}
\newcommand{\ovs}{\overline{s}}	\newcommand{\ovt}{\overline{t}}
\newcommand{\ovu}{\overline{u}}	\newcommand{\ovv}{\overline{v}}
\newcommand{\ovw}{\overline{w}}	\newcommand{\ovx}{\overline{x}}
\newcommand{\ovy}{\overline{y}}	\newcommand{\ovz}{\overline{z}}

%---------------------------------------
% Tilde
%---------------------------------------

%Captital Letters
\newcommand{\tdA}{\tilde{A}}	\newcommand{\tdB}{\tilde{B}}
\newcommand{\tdC}{\tilde{C}}	\newcommand{\tdD}{\tilde{D}}
\newcommand{\tdE}{\tilde{E}}	\newcommand{\tdF}{\tilde{F}}
\newcommand{\tdG}{\tilde{G}}	\newcommand{\tdH}{\tilde{H}}
\newcommand{\tdI}{\tilde{I}}	\newcommand{\tdJ}{\tilde{J}}
\newcommand{\tdK}{\tilde{K}}	\newcommand{\tdL}{\tilde{L}}
\newcommand{\tdM}{\tilde{M}}	\newcommand{\tdN}{\tilde{N}}
\newcommand{\tdO}{\tilde{O}}	\newcommand{\tdP}{\tilde{P}}
\newcommand{\tdQ}{\tilde{Q}}	\newcommand{\tdR}{\tilde{R}}
\newcommand{\tdS}{\tilde{S}}	\newcommand{\tdT}{\tilde{T}}
\newcommand{\tdU}{\tilde{U}}	\newcommand{\tdV}{\tilde{V}}
\newcommand{\tdW}{\tilde{W}}	\newcommand{\tdX}{\tilde{X}}
\newcommand{\tdY}{\tilde{Y}}	\newcommand{\tdZ}{\tilde{Z}}
%Small Letters
\newcommand{\tda}{\tilde{a}}	\newcommand{\tdb}{\tilde{b}}
\newcommand{\tdc}{\tilde{c}}	\newcommand{\tdd}{\tilde{d}}
\newcommand{\tde}{\tilde{e}}	\newcommand{\tdf}{\tilde{f}}
\newcommand{\tdg}{\tilde{g}}	\newcommand{\tdh}{\tilde{h}}
\newcommand{\tdi}{\tilde{i}}	\newcommand{\tdj}{\tilde{j}}
\newcommand{\tdk}{\tilde{k}}	\newcommand{\tdl}{\tilde{l}}
\newcommand{\tdm}{\tilde{m}}	\newcommand{\tdn}{\tilde{n}}
\newcommand{\tdo}{\tilde{o}}	\newcommand{\tdp}{\tilde{p}}
\newcommand{\tdq}{\tilde{q}}	\newcommand{\tdr}{\tilde{r}}
\newcommand{\tds}{\tilde{s}}	\newcommand{\tdt}{\tilde{t}}
\newcommand{\tdu}{\tilde{u}}	\newcommand{\tdv}{\tilde{v}}
\newcommand{\tdw}{\tilde{w}}	\newcommand{\tdx}{\tilde{x}}
\newcommand{\tdy}{\tilde{y}}	\newcommand{\tdz}{\tilde{z}}

%---------------------------------------
% Vec
%---------------------------------------

%Captital Letters
\newcommand{\vcA}{\vec{A}}	\newcommand{\vcB}{\vec{B}}
\newcommand{\vcC}{\vec{C}}	\newcommand{\vcD}{\vec{D}}
\newcommand{\vcE}{\vec{E}}	\newcommand{\vcF}{\vec{F}}
\newcommand{\vcG}{\vec{G}}	\newcommand{\vcH}{\vec{H}}
\newcommand{\vcI}{\vec{I}}	\newcommand{\vcJ}{\vec{J}}
\newcommand{\vcK}{\vec{K}}	\newcommand{\vcL}{\vec{L}}
\newcommand{\vcM}{\vec{M}}	\newcommand{\vcN}{\vec{N}}
\newcommand{\vcO}{\vec{O}}	\newcommand{\vcP}{\vec{P}}
\newcommand{\vcQ}{\vec{Q}}	\newcommand{\vcR}{\vec{R}}
\newcommand{\vcS}{\vec{S}}	\newcommand{\vcT}{\vec{T}}
\newcommand{\vcU}{\vec{U}}	\newcommand{\vcV}{\vec{V}}
\newcommand{\vcW}{\vec{W}}	\newcommand{\vcX}{\vec{X}}
\newcommand{\vcY}{\vec{Y}}	\newcommand{\vcZ}{\vec{Z}}
%Small Letters
\newcommand{\vca}{\vec{a}}	\newcommand{\vcb}{\vec{b}}
\newcommand{\vcc}{\vec{c}}	\newcommand{\vcd}{\vec{d}}
\newcommand{\vce}{\vec{e}}	\newcommand{\vcf}{\vec{f}}
\newcommand{\vcg}{\vec{g}}	\newcommand{\vch}{\vec{h}}
\newcommand{\vci}{\vec{i}}	\newcommand{\vcj}{\vec{j}}
\newcommand{\vck}{\vec{k}}	\newcommand{\vcl}{\vec{l}}
\newcommand{\vcm}{\vec{m}}	\newcommand{\vcn}{\vec{n}}
\newcommand{\vco}{\vec{o}}	\newcommand{\vcp}{\vec{p}}
\newcommand{\vcq}{\vec{q}}	\newcommand{\vcr}{\vec{r}}
\newcommand{\vcs}{\vec{s}}	\newcommand{\vct}{\vec{t}}
\newcommand{\vcu}{\vec{u}}	\newcommand{\vcv}{\vec{v}}
%\newcommand{\vcw}{\vec{w}}	\newcommand{\vcx}{\vec{x}}
\newcommand{\vcy}{\vec{y}}	\newcommand{\vcz}{\vec{z}}

%---------------------------------------
% Greek Letters:-
%---------------------------------------
\newcommand{\eps}{\epsilon}
\newcommand{\veps}{\varepsilon}
\newcommand{\lm}{\lambda}
\newcommand{\Lm}{\Lambda}
\newcommand{\gm}{\gamma}
\newcommand{\Gm}{\Gamma}
\newcommand{\vph}{\varphi}
\newcommand{\ph}{\phi}
\newcommand{\om}{\omega}
\newcommand{\Om}{\Omega}
\newcommand{\sg}{\sigma}
\newcommand{\Sg}{\Sigma}

\newcommand{\Qed}{\begin{flushright}\qed\end{flushright}}
\newcommand{\parinn}{\setlength{\parindent}{1cm}}
\newcommand{\parinf}{\setlength{\parindent}{0cm}}
\newcommand{\del}[2]{\frac{\partial #1}{\partial #2}}
\newcommand{\Del}[3]{\frac{\partial^{#1} #2}{\partial^{#1} #3}}
\newcommand{\deld}[2]{\dfrac{\partial #1}{\partial #2}}
\newcommand{\Deld}[3]{\dfrac{\partial^{#1} #2}{\partial^{#1} #3}}
\newcommand{\uin}{\mathbin{\rotatebox[origin=c]{90}{$\in$}}}
\newcommand{\usubset}{\mathbin{\rotatebox[origin=c]{90}{$\subset$}}}
\newcommand{\lt}{\left}
\newcommand{\rt}{\right}
\newcommand{\exs}{\exists}
\newcommand{\st}{\strut}
\newcommand{\dps}[1]{\displaystyle{#1}}
\newcommand{\la}{\langle}
\newcommand{\ra}{\rangle}
\newcommand{\cls}[1]{\textsc{#1}}
\newcommand{\prb}[1]{\textsc{#1}}
\newcommand{\comb}[2]{\left(\begin{matrix}
		#1\\ #2
\end{matrix}\right)}
%\newcommand[2]{\quotient}{\faktor{#1}{#2}}
\newcommand\quotient[2]{
	\mathchoice
	{% \displaystyle
		\text{\raise1ex\hbox{$#1$}\Big/\lower1ex\hbox{$#2$}}%
	}
	{% \textstyle
		#1\,/\,#2
	}
	{% \scriptstyle
		#1\,/\,#2
	}
	{% \scriptscriptstyle  
		#1\,/\,#2
	}
}

\newcommand{\tensor}{\otimes}
\newcommand{\xor}{\oplus}

\newcommand{\sol}[1]{\begin{solution}#1\end{solution}}
\newcommand{\solve}[1]{\setlength{\parindent}{0cm}\textbf{\textit{Solution: }}\setlength{\parindent}{1cm}#1 \hfill $\blacksquare$}
\newcommand{\mat}[1]{\left[\begin{matrix}#1\end{matrix}\right]}
\newcommand{\matr}[1]{\begin{matrix}#1\end{matrix}}
\newcommand{\matp}[1]{\lt(\begin{matrix}#1\end{matrix}\rt)}
\newcommand{\detmat}[1]{\lt|\begin{matrix}#1\end{matrix}\rt|}
\newcommand\numberthis{\addtocounter{equation}{1}\tag{\theequation}}
\newcommand{\handout}[3]{
	\noindent
	\begin{center}
		\framebox{
			\vbox{
				\hbox to 6.5in { {\bf Complexity Theory I } \hfill Jan -- May, 2023 }
				\vspace{4mm}
				\hbox to 6.5in { {\Large \hfill #1  \hfill} }
				\vspace{2mm}
				\hbox to 6.5in { {\em #2 \hfill #3} }
			}
		}
	\end{center}
	\vspace*{4mm}
}

\newcommand{\lecture}[3]{\handout{Lecture #1}{Lecturer: #2}{Scribe:	#3}}

\let\marvosymLightning\Lightning
\newcommand{\ctr}{\text{\marvosymLightning}\hspace{0.5ex}} % Requires marvosym package

\newcommand{\ov}[1]{\overline{#1}}
\newcommand{\thmref}[1]{\hyperref[th:#1]{Theorem \ref{th:#1}}}
\newcommand{\propref}[1]{\hyperref[th:#1]{Proposition \ref{th:#1}}}
\newcommand{\lmref}[1]{\hyperref[th:#1]{Lemma \ref{th:#1}}}
\newcommand{\corref}[1]{\hyperref[th:#1]{Corollary \ref{th:#1}}}

\newcommand{\thrmref}[1]{\hyperref[#1]{Theorem \ref{#1}}}
\newcommand{\propnref}[1]{\hyperref[#1]{Proposition \ref{#1}}}
\newcommand{\lemref}[1]{\hyperref[#1]{Lemma \ref{#1}}}
\newcommand{\corrref}[1]{\hyperref[#1]{Corollary \ref{#1}}}

\DeclareMathOperator{\enc}{Enc}
\DeclareMathOperator{\res}{Res}
\DeclareMathOperator{\spec}{Spec}
\DeclareMathOperator{\cov}{Cov}
\DeclareMathOperator{\Var}{Var}
\DeclareMathOperator{\Rank}{rank}
\newcommand{\Tfae}{The following are equivalent:}
\newcommand{\tfae}{the following are equivalent:}
\newcommand{\sparsity}{\textit{sparsity}}

\newcommand{\uddots}{\reflectbox{$\ddots$}} 

\newenvironment{claimwidth}{\begin{center}\begin{adjustwidth}{0.05\textwidth}{0.05\textwidth}}{\end{adjustwidth}\end{center}}

\setlength{\parindent}{0pt}


%%%%%%%%%%%%%%%%%%%%%%%%%%%%%%%%%%%%%%%%%%%%%%%%%%%%%%%%%%%%%%%%%%%%%%%%%%%%%%%%%%%%%%%%%%%%%%%%%%%%%%%%%%%%%%%%%%%%%%%%%%%%%%%%%%%%%%%%

\begin{document}
	
	%%%%%%%%%%%%%%%%%%%%%%%%%%%%%%%%%%%%%%%%%%%%%%%%%%%%%%%%%%%%%%%%%%%%%%%%%%%%%%%%%%%%%%%%%%%%%%%%%%%%%%%%%%%%%%%%%%%%%%%%%%%%%%%%%%%%%%%%
	
	{\noindent \large\textbf{Soham Chatterjee} \hfill \textbf{Final Exam}\\
		Email: \href{soham.chatterjee@tifr.res.in}{soham.chatterjee@tifr.res.in} \hfill Dept: STCS\\
		\normalsize Course: Mathematical Foundations for Computer Sciences \hfill December 6, 2024\\ 
		\noindent\rule{7in}{2.8pt}}
	
%%%%%%%%%%%%%%%%%%%%%%%%%%%%%%%%%%%%%%%%%%%%%%%%%%%%%%%%%%%%%%%%%%%%%%%%%%%%%%%%%%%%%%%%%%%%%%%%%%%%%%%%%%%%%%%%%%%%%%%%%%%%%%%%%%%%%%%%
% Problem 1
%%%%%%%%%%%%%%%%%%%%%%%%%%%%%%%%%%%%%%%%%%%%%%%%%%%%%%%%%%%%%%%%%%%%%%%%%%%%%%%%%%%%%%%%%%%%%%%%%%%%%%%%%%%%%%%%%%%%%%%%%%%%%%%%%%%%%%%%
	
\begin{problem}{%problem statement
		Bollob\'{a}s Inequality
	}{p1% problem reference text
}
Let $m>0$ and $A_1,\dots, A_m$ and $B_1,\dots, B_m$ be finite sets. Suppose that for all $i,j\in[m]$, we have $A_i\cap B_j=\emptyset \iff i=j$. Then $$\sum_{i=1}^m \binom{|A_i|+|B_i|}{|A_i|}^{-1}\leq 1$$
\end{problem}
\solve{
Let $S=\lt(\bigcup\limits_{i=1}^n A_i\rt)\cup \lt(\bigcup\limits_{i=1}^n B_i\rt)$. Assume that $|S|=n$. Now $\binom{|A_i|+|B_i|}{|A_i|}=\frac{(|A_i|+|B_i|)!}{|A_i|!|B_i|!}$. Now $|A_i|!|B_i|!$ be the number of arrangements of the elements of $A_i\cup B_i$ so that first $|A_i|$ elements are form $A_i$ and the last $|B_i|$ elements are from $B_i$. And $(|A_i|+|B_i|)!$ is the number of arrangements of the elements of $A_i\cup B_i$. Let $E_i$ be the event that in a arrangement of the elements of $S$ has first $|A_i|$ elements from $A_i$ and last $|B_i|$ elements from $B_i$. Then $$\bbP[E_i]=\frac{|A_i|!(n-|A_i|-|B_i|)!|B_i|!}{n!}=\frac{|A_i|!|B_i|!}{(|A_i|+|B_i|)!}=\binom{|A_i|+|B_i|}{|A_i|}^{-1}\qquad \forall\ i\in[m]$$ If we can show that $E_i$ are pairwise disjoint events then we have $$\sum_{i=1}^m \binom{|A_i|+|B_i|}{|A_i|}^{-1}=\sum_{i=1}^n \bbP[E_i]=\bbP\lt[\bigcup_{i=1}^n E_i\rt]\leq 1$$Hence it suffices to show that the events are pairwise disjoint. Take $E_i$ and $E_j$ where $i\neq j$. Suppose both $E_i$ and $E_i$ events occurred in a arrangement. Then the first $|A_i|$ elements are from $A_i$ and the last $|B_i|$ elements are from $B_i$. Now $A_i\cap B_j\neq \emptyset$ and $A_j\cap B_i\neq \emptyset$. Then in the first $|A_i|$ elements there is an element from $B_j$. If $|A_i|\leq |A_j|$ then in the first $|A_j|$ elements there is an element from $B_j$ which is not possible if $E_j$ even occurred. So suppose $|A_i|> |A_j|$. In that case $A_j\cap B_i\neq \emptyset\implies $ there is an element of $B_i$ in the first $|A_i|$ elements which is not possible since  the $E_i$ event occurred. Hence contradiction. Both $E_i$ and $E_j$ events cannot occur simultaneously. Therefore $E_i$ and $E_j$ are pairwise disjoint. Therefore $\sum\limits_{i=1}^m \binom{|A_i|+|B_i|}{|A_i|}^{-1}\leq 1$.
}
%\parinf
%
%[Me and Soumyadeep solved this together.]\parinn
%%%%%%%%%%%%%%%%%%%%%%%%%%%%%%%%%%%%%%%%%%%%%%%%%%%%%%%%%%%%%%%%%%%%%%%%%%%%%%%%%%%%%%%%%%%%%%%%%%%%%%%%%%%%%%%%%%%%%%%%%%%%%%%%%%%%%%%%
% Problem 2
%%%%%%%%%%%%%%%%%%%%%%%%%%%%%%%%%%%%%%%%%%%%%%%%%%%%%%%%%%%%%%%%%%%%%%%%%%%%%%%%%%%%%%%%%%%%%%%%%%%%%%%%%%%%%%%%%%%%%%%%%%%%%%%%%%%%%%%%

\begin{problem}{%problem statement
	}{p2% problem reference text
	}
Solve the following parts.
\begin{itemize}
	\item Let $n>0$ and $S$ be a set of points on the $n$-dimensional hypercube such that $\operatorname{Pr}_{x \sim\{0,1\}^n}(x \in S)>\frac{2}{n}$. Show that $S$ contains three points that form an equilateral triangle.
\item Let $\mathrm{Sym}_n$ be the symmetric group of permutations over [n]. Let $1 \leq a<b \leq n$. Find the size of the group generated by the permutations $(123 \ldots n)$ and $(a b)$.
\item There is a coop run by two brothers, Arjun and Bheem. Initially, the coop has two hens, one owned by Arjun and one owned by Bheem. Every month, a uniformly random hen in the coop lays two eggs that grow up to two more hens. These new hens are owned by the brother who owned the mother hen (and the mother hen can continue to lay eggs in the future). For $n, k \geq 0$, what is the probability that after $n$ months, Arjun owns $2 k+1$ hens in the coop. By considering all possible values of $k$, derive an identity for $4^n$.
\end{itemize}
\end{problem}
\solve{	
\begin{itemize}
	\item Now two points in the hypercube are neighbors if they differ in one coordinate. Let $x,y$ are two distinct points in the boolean hypercube. Suppose they have a common neighbor. Then $x,y$ differ in two coordinates. Hence their distance is $\sqrt{2}$. Now 	 Given that $Pr_{x\sim \{0,1\}^n}[x\in S]>\frac2n$. Then $|S|>\frac2n\; 2^n=\frac{2^{n+1}}{n}$. 
	
	Now each point in the hypercube has $n$ neighbors. Consider the set $\{(x,u)\colon x\in S, u\text{ neighbor of $x$}\}$. We have $$|\{(x,u)\colon x\in S, u\text{ neighbor of $x$}\}|=|S|\times n>\frac{2^{n+1}}{n}\times n=2^{n+1}$$. But the hypercube has $2^n$ elements. Hence by pigeonhole principle there exists an element $z$ which is neighbor of at least $3$ elements of $S$. Let $x_1,x_2,x_3\in S$ such that $z$ is neighbor of $x_1,x_2,x_3$. Then distance between $x_i,x_j$ is  $\sqrt{2}$ for all $i,j\in [3]$ with $i\neq j$. Hence $x_1,x_2,x_3$ forms an equilateral triangle.
	\item Let $\phi=(12\cdots n)$ and $\pi=(ab)$. Therefore $\phi^n=1\implies \phi^{-1}=\phi^{n-1}$. Since $b>a$ suppose $d=b-a$. Let $G=\la \phi,\pi\ra$.\parinn
	\begin{lemma}
		$(1,1+kd)\in G$ for all $k\in \bbN$.
	\end{lemma}
\begin{proof}
	We will prove this using induction on $k$. For $k=1$, $(1,1+kd)=(1,1+b-a)$. Now $(\phi^{-1})^{a-1}\pi\phi^{a-1}\in G$. Now for any $i\in [n]$, $\phi^{a-1}(i)=i+a-1$ if $i+a-1\leq n$ and otherwise $\phi^{a-1}(i)=i+a-1-n$. Now if $i+a-1\neq a,b$ then $\pi(i+a-1)=i+a-1$ then $\pi(i+a-1-n)=i+a-1-n$. In both of these cases we have  $(\phi^{-1})^{a-1}\pi\phi^{a-1}(i)= (\phi^{-1})^{a-1}\phi^{a-1}(i)=i$. Now if $i+a-1=a$ then $i=1$ Then $\pi(a)=b$ Then $$(\phi^{-1})^{a-1}(b)=b-(a-1)=1+(b-a)=1+d$$ If $i+a-1=b$ then $i=1+d$. Then $\pi(b)=a$. Then $$(\phi^{-1})^{a-1}(a)=a-(a-1)=1$$Hence $(\phi^{-1})^{a-1}\pi\phi^{a-1}=(1,1+d)$. Hence the base case follows. 
	
	Suppose this is true for $1,\dots, k-1$. Then $(1,1+(k-1)d)\in G$. Then using \lemref{kincd} we have $(1+(k-1)d,1+kd)\in G$. Hence $$(1,1+kd)=(1,1+(k-1)d)(1+(k-1)d,1+kd)(1,1+(k-1)d)$$Hence $(1,1+kd)\in G$. By mathematical induction $(1,1+kd)\in G$ for all $k\in \bbN$
\end{proof}
\begin{lemma}\label{kincd}
	For any $k\in\bbN$, $$(1+kd,1+(k+1)d)=\phi^{kd}(1,1+d)(\phi^{-1})^{kd}$$ 
\end{lemma}
\begin{proof}
	 For any $i\in [n]$, $(\phi^{-1})^{kd}(i)=i-kd$. Now if $i-kd\neq 1,1+d$ then $\pi(i-kd)=i-kd$. Then $\phi^{kd}\pi(\phi^{-1})^{kd}(i)=\phi^{kd}(\phi^{-1})^{kd}(i)=i$. Now suppose $i-kd=1$. Then $i=1+kd$. Then $\phi(i-kd)=1+kd$. Then $\phi^{kd}(1+kd)=1+kd-kd=1$. If $i-kd=1+d$ then $i=1+(k+1)d$. Then $\pi(1+d)=1$. Then $\phi^{kd}(1)=1+kd$. Therefore  $\phi^{kd}(1,1+d)(\phi^{-1})^{kd}=(1+kd,1+(k+1)d)$
\end{proof}

Now if $gcd(n,d)=1$ then there exists $a,b\in \bbZ$ where $a>0$ such that  $ad+bn=1$. Therefore $bn=1-ad$. Therefore $(1,1+ad)=(1,2)$. And for any $k\in [n]$, $(1,k)=(1,1+(k-1)ad)$. Hence for all $k\in[n]$, $(1,k)\in G$. Hence for any $i,j\in[n]$, $i\neq j$, $(i,j)=(1,i)(1,j)(1,i)$. Hence all transpositions are in $G$. Hence $G$ contains all the permutations of $Sym_n$. Hence $Sym_n$ is generated by $(12\cdots n)$ and $(ab)$.

Now if $gcd(n,d)\neq 1$. Suppose $gcd(n,d)=l$. Then there exists $a,b\in\bbZ$ with $a>0$ such that $ad+bn=l$. Then  for any $r\in [l]$, $(r,r+kl)\in G$ for any $k\in \bbN$.  Therefore any permutation of the elements 
\end{itemize}
}
\parinf

[I discussed the solution with Aakash and Shubham]\parinn\textbf{}



%%%%%%%%%%%%%%%%%%%%%%%%%%%%%%%%%%%%%%%%%%%%%%%%%%%%%%%%%%%%%%%%%%%%%%%%%%%%%%%%%%%%%%%%%%%%%%%%%%%%%%%%%%%%%%%%%%%%%%%%%%%%%%%%%%%%%%%%
% Problem 3
%%%%%%%%%%%%%%%%%%%%%%%%%%%%%%%%%%%%%%%%%%%%%%%%%%%%%%%%%%%%%%%%%%%%%%%%%%%%%%%%%%%%%%%%%%%%%%%%%%%%%%%%%%%%%%%%%%%%%%%%%%%%%%%%%%%%%%%%
%\addtocounter{problem}{1}

\begin{problem}{%problem statement
	}{p3% problem reference text
	}
In this question, we will show that $(n-1)!\equiv -1\pmod{n}$ if and only if $n$ is  prime
\begin{itemize}
	\item Show that, if $G$ is a finite Abelian group, then the order of any element in $G$ divides the maximal order of all elements.
	\item For all $n>0$, define the set $G_n=\{i\in [n]\mid GCD(i,n)=1\}$. Show that $G_n$ together with the multiplication operator modulo $n$ forms a finite Abelian Group.
	\item If $n$ is prime, then show that $\prod_{0<i<j<n}(j-i)\not\equiv 0\pmod{n}$. Using a Vandermonde matrix, conclude that $G_n$ must be cyclic.
	\item Derive that $(n-1)!\equiv -1\pmod{n}$ if and only if $n$ is prime.
\end{itemize}
\end{problem}
\solve{	
\begin{itemize}
	\item Let $g$ be the element of $G$ which has the maximum order of all elements. Let $r_g$ be the order of $g$. Now let $h$ be any element. Assume $r_h$ be the order of $h$. Suppose $r_h\nmid r_g$. Let $r$ is the lcm of $r_g$ and $r_h$. Then consider the element $gh$. Now for any $n\in \bbN$, $(gh)^n=g^nh^n$. For any $n<r$, either $n\not\equiv 0\pmod{r_g}$ or $n\not\equiv 0\pmod{r_h}$. Hence in both cases $g^nh^n\neq e$ where $e$ is the identity element of $G$. But $(gh)^r=g^rh^r=e\cdot e=e$. Therefore the order of the element is $r$. Since $r_h\nmid r_g$, $r>r_g,r_h$. So we get a new element which has order more than $r_g$ which contradicts that $g$ has the maximum order. Hence contradiction. $r_h\mid r_g$. Since $h$ is an arbitrary element of $G$, the order of any element of $G$ divides the maximal order of all elements in $G$.
	\item \begin{itemize}
		\item \textbf{Closure:} Let $a,b\in G_n$. Then $gcd(a,n)=1$ and $gcd(b,n)=1$. Let $gcd(ab,n)=d$. If $d\neq 1$ then there exists $d_a$ and $d_b$ such that $d_a\mid a$ and $d_b\mid b$ nand $d_ad_b=d$. Since $d\neq 1$ both $d_a$ and $d_b$ can not be equal to 1. Hence one of them is at least not equal to 1. WLOG suppose $d_a\neq 1$. Then $d_a\mid a $ and $d_a\mid n$ which contradicts $gcd(a,n)=1$. Hence $gcd(ab,n)=1$. So $ab\in G_n$.
		\item \textbf{Associativity:} Let $a,b,c\in G_n$. Now multiplication of integers follows associativity. Therefore $(a\cdot b)\cdot c=a\cdot (b\cdot c)$. Therefore then going modulo $n$ we get the same values i.e. $(a\cdot b)\cdot c\pmod{n}=a\cdot (b\cdot c)\pmod{n}$. Therefore $(a\cdot b)\cdot c\equiv a\cdot (b\cdot c) \pmod n$. Hence the group operation follows associativity.
		\item \textbf{Identity:} Consider the element $1$. Now for any $a\in [n]$ we have $1\cdot a=a\cdot 1=a$. Therefore $1\cdot a\equiv a\cdot 1\equiv a\pmod n$. 
		\item \textbf{Inverse:} Let $a\in G_n$. Now consider the set elements $S=\{k\cdot a\mid k\in G_n\}$. Now in this set $S$ there are $\phi(n)$ elements where $\phi$ is the Euler Totient Function. Now consider the set $S_{\bmod n}=\{x\pmod n\mid x\in S\}$. Now for any $n\in \bbN$, $n\cdot a=\sum\limits_{i=1}^na$. Since $G_n$ is closed under operation, $S_{\bmod n}\subseteq G_n$. Now $S_{\bmod n}$ has at most $\phi(n)$ elements. Now let there exists two $i,j\in [n-1]$ where $i\neq j$ and $j>i$ such that $i\cdot a\equiv j\cdot a\pmod n\implies (i-j)\cdot a\equiv 0\pmod n$. Therefore $(i-j)\cdot a\mid n$. Hence $a\mid n$. But since $a\in G_n$ we know $gcd(a,n)=1$. Hence contradiction. Therefore such $i,j$ does not exists. Hence $|S_{\bmod n}|=\phi(n)$. Therefore $S_{\bmod n}=G_n$. Therefore there exists $k\in G_n$ such that $k\cdot a\equiv 1\pmod n$. Therefore there exists inverse of $a$. 
		\item \textbf{Commutativity:} Let $a,b\in G_n$. Now we know multiplication of integers follows commutativity. Therefore $a\cdot b=b\cdot a$. Therefore going modulo $n$ we have $a\cdot b\bmod n=b\cdot a\bmod n\implies a\cdot b\equiv b\cdot a\pmod n$.
	\end{itemize}
Therefore $G_n$ is a finite Abelian group.
\item Since $n$ is prime all the positive integers less than $n$ are in $G_n$. Hence for all $i,j\in [n-1]$ where $i<j$ and $i\neq j-1$ we have $j-i\nmid n$ otherwise $j-i$ will be a nontrivial factor of $n$. Therefore we have $\prod\limits_{0<i<j<n}(j-i)=\prod\limits_{0<i<j<n, i<j-1}(j-i)(j-i)\not\equiv 0\pmod n$. \parinn

Since $n$ is prime all the positive integers less than $n$ are in $G_n$ So consider the matrix $M=(m_{i,j})_{1\leq i,j\leq n-1}$ where $m_{i,j}=i^{j-1}$. 
\begin{lemma}
If $$M=\mat{1&1&\cdots & 1 &1\\a_1&a_2&\cdots &a_{n-2}&a_{n-1}\\\vdots&\vdots&\ddots&\vdots&\vdots\\a_1^{n-3}&a_2^{n-3}&\cdots&a_{n-2}^{n-3}& a_{n-1}^{n-3}\\a_1^{n-2}&a_2^{n-2}&\cdots&a_{n-2}^{n-2}&a_{n-1}^{n-2}}\implies \det M=\prod\limits_{0<i<j<n} (a_j-a_i)$$
\end{lemma}
\begin{proof}
	We will prove this using induction on $n$. For $n=1$ this is true. So the base case follows. Suppose this is true for $n-1$. We will show for $n$. Now consider the matrix $M'$ where er subtracted the $i-1$ times the $j^{th}$ row from $(j+1)^{th}$ row and we $$M'=\mat{1&1&\cdots & 1 &1\\   0&a_2-a_1&\cdots &a_{n-2}-a_1&a_{n-1}-a_1\\   \vdots&\vdots&\ddots&\vdots&\vdots\\   0&a_2^{n-2}-a_1a_{2}^{n-3}&\cdots&a_{n-2}^n-a_1a_{n-1}^{n-3}& a_{n-1}^{n-2}-a_1a_{n-1}^{n-3}}$$Therefore $$\det M= \det M'=\prod_{j=2}^{n-1} (a_j-a_1)\det \mat{1& 1&\cdots&1\\a_2&a_3&\cdots&a_{n-1}\\ \vdots&\vdots&\ddots&\vdots\\a_2^{n-2}&a_3^{n-2}&\cdots&a_{n-1}^{n-2}}$$ Hence by inductive hypothesis we have 
	$$\det M=\prod_{j=2}^{n-1} (a_j-a_1) \prod_{1<i<j<n}(a_j-a_i)=\prod\limits_{0<i<j<n} (a_j-a_i)$$Hence by mathematical induction this is true for all $n$.
\end{proof}

Therefore using the lemma we have $\det M=\prod\limits_{0<i<j<n} (j-i)$. Now we know $\det M\neq 0\pmod n$ if $n$ is prime. Therefore $M$ is invertible.

By the first part we know order of any element of $G_n$ divides the maximal order  of all elements. Now let $g$ be the element in $G_n$ which has the maximal order. Consider the vector $\hat{g}=\mat{1 & g & g^2 & \cdots & g^{n-2}}$.  Now $\hat{g}$ is a column of $M$. Since $M$ is invertible $1,g,\dots, g^{n-2}$ are linearly independent. Therefore order of $g$ is $n-1$. Hence $1,g,\dots, g^{n-2}$ are all elements of $G_n$. But $G_n$ has $n-1$ elements. Therefore $g$ is the generator of $G_n$. Hence $G_n$ is cyclic.
\item Let $n$ is a prime. Then $(n-1)!=\prod\limits_{g\in G_n} g$. Now for all elements $g\in G_n$ there exists $g^{-1}\in G_n$. If $g\neq g^{-1}$ then their product in $\prod\limits_{g\in G_n} g$ vanishes. Therefore only the elements whose product actually contributes to the value of $\prod\limits_{g\in G_n} g$ are $g$'s in $G_n$ such that $g=g^{-1}\iff g^2=1$.  Therefore $$\prod\limits_{g\in G_n} g=\prod\limits_{\substack{g\in G_n\\ g^2=1}} g$$Now for any element $g\in G_n$ $$g^2=1\implies g^2-1\equiv 0\pmod n\implies  (g+1)(g-1)\equiv 0\pmod n$$Therefore either $g+1\equiv 0\pmod n\implies g=n-1$ or $g-1\equiv 0 \pmod n\implies g=1$. Therefore $$\prod\limits_{g\in G_n} g=(n-1)\cdot 1=n-1\equiv -1\pmod n\implies (n-1)!\equiv -1\pmod n$$\parinn

Now suppose $n$ is not prime. Then there exists $1<a,b<n$ such that $a\cdot b=n$. Then $$(n-1)!=a\cdot b\cdot \prod_{i=1, i\neq a,b}^{n-1}i=n\prod_{i=1, i\neq a,b}^{n-1}i$$Hence $n\mid (n-1)!\implies (n-1)!\equiv 0\pmod n$.

Therefore $(n-1)!\equiv -1\pmod n$ if and only if $n$ is a prime.
\end{itemize}

}
%\parinf
%
%[Me and Soumyadeep solved this together.]\parinn


%%%%%%%%%%%%%%%%%%%%%%%%%%%%%%%%%%%%%%%%%%%%%%%%%%%%%%%%%%%%%%%%%%%%%%%%%%%%%%%%%%%%%%%%%%%%%%%%%%%%%%%%%%%%%%%%%%%%%%%%%%%%%%%%%%%%%%%%
% Problem 4
%%%%%%%%%%%%%%%%%%%%%%%%%%%%%%%%%%%%%%%%%%%%%%%%%%%%%%%%%%%%%%%%%%%%%%%%%%%%%%%%%%%%%%%%%%%%%%%%%%%%%%%%%%%%%%%%%%%%%%%%%%%%%%%%%%%%%%%%
\addtocounter{problem}{1}
%\begin{problem}{%problem statement
%	}{p4% problem reference text
%	}
%
%\end{problem}
%
%
%
%[Me and Soumyadeep solved this together.]\parinn	


%%%%%%%%%%%%%%%%%%%%%%%%%%%%%%%%%%%%%%%%%%%%%%%%%%%%%%%%%%%%%%%%%%%%%%%%%%%%%%%%%%%%%%%%%%%%%%%%%%%%%%%%%%%%%%%%%%%%%%%%%%%%%%%%%%%%%%%%
% Problem 5
%%%%%%%%%%%%%%%%%%%%%%%%%%%%%%%%%%%%%%%%%%%%%%%%%%%%%%%%%%%%%%%%%%%%%%%%%%%%%%%%%%%%%%%%%%%%%%%%%%%%%%%%%%%%%%%%%%%%%%%%%%%%%%%%%%%%%%%%
%\newpage
\begin{problem}{%problem statement
}{p5% problem reference text
}
Let $\Omega$ be a set and $\operatorname{Sym}_{\Omega}$ be the set of permutations on $\Omega$. We say that $G$ acts on $\Omega$ if there is a homomorphism $\theta$ from $G$ to $\operatorname{Sym}_{\Omega}$. For all $x \in \Omega$ and an action $\theta$, we define the orbit of $x$ under $\theta$ to be the set $\operatorname{orb}(x)=\{\theta(g)(x) \mid g \in G\}$ and the stabilizer of $x$ under $\theta$ to be $\operatorname{stab}(x)=\{g \in G \mid \theta(g)(x)=x\}$. In the following, assume that $G$ and $\Omega$ are finite.
\begin{itemize}
	 \item Show that $|\operatorname{orb}(x)| \cdot|\operatorname{stab}(x)|=|G|$.
\item Let $x, y \in \Omega$ be related if $y \in \operatorname{orb}(x)$. Show that this is an equivalence relation.
\item Let $\operatorname{fix}(g)=|\{x \in \Omega \mid g \in \operatorname{stab}(x)\}|$. Show that the number of different orbits is $\mathbb{E}_{g \sim G}[|\operatorname{fix}(g)|]$.
\item Using the above, compute the number of ways to color the vertices of a regular tetrahedron using $k$ colors.
\end{itemize}
\end{problem}
\solve{
\begin{itemize}
	\item We will first prove that $\operatorname{stab}(x)$ is a subgroup of $G$ for any $x\in \Om$. Now first of all $\operatorname{stab}(x)$ contains the identity of element of $G$. Therefore $\operatorname{stab}(x)\neq \emptyset$. Now let $g,h\in \operatorname{stab}(x)$. Then $\theta(g)(x)=x$ and $\theta(h)(x)=x$. Then $$\theta(gh)(x)=\theta(g)\cdot \theta(h)(x)=\theta(g)\lt(\theta(h)(x)\rt)=\theta(g)(x)=x$$Therefore $gh\in \operatorname{stab}(x)$. Hence $\operatorname{stab}(x)$ is a subgroup of $G$. Now consider the set of cosets of $\operatorname{stab}(x)$ in $G$ be   $G/\operatorname{stab}(x)$. By Lagrange Theorem $|G/\operatorname{stab}(x)|=\frac{|G|}{|\operatorname{stab}(x)|}$. Hence it is enough to show that $|G/\operatorname{stab}(x)|=|\operatorname{orb}(x)|$. Now let $g\cdot \operatorname{stab}(x)$ be a coset of $G/\operatorname{stab}(x)$. Then any element of $g\cdot \operatorname{stab}(x)$ is of the form $gh$ where $h\in \operatorname{stab}(x)$. Now consider the function $f:G/\operatorname{stab}(x)\to \operatorname{orb}(x)$ where $f(g\cdot \operatorname{stab}(x))=\theta(g)(x)$. Now for all $\theta(gh)(x)=\theta(g)\theta(h)(x)=\theta(g)(x)$.  Therefore for any two elements  $h,h'\in g\cdot \operatorname{stab}(x)$, $g(g\cdot \operatorname{stab}(x))=\theta(g)(x)=\theta(h)(x)=\theta(h')(x)$. So $f$ is well defined. Now the for all $y\in \operatorname{orb}(x)$, $\exs \ g\in G$ such that $\theta(g)(x)=y$. So in $G/\operatorname{stab}(x)$ consider the coset $g\cdot \operatorname{stab}(x)$. Then $f(g\cdot \operatorname{stab}(x))=y$. Therefore $f$ is surjective. Now let there exists $g\cdot \operatorname{stab}(x), g'\cdot \operatorname{stab}(x)\in G/\operatorname{stab}(x)$ such that $f(g\cdot \operatorname{stab}(x))=f(g'\cdot \operatorname{stab}(x))=y$ for some $y\in \operatorname{orb}(x)$. In that case we have $$\theta(g^{-1}\cdot g')(x)=\theta(g^{-1})\theta(g')(x)=\theta(g^{-1})(y)=x$$Hence $g^{-1}\cdot g\in \operatorname{stab}(x)$. Therefore $g\cdot (g^{-1}\cdot g')=g'$. Hence $g'\in g\cdot \operatorname{stab}(x)$. Therefore all the elements of $G$ for which sends $x$ to same destination are in same equivalence class in $G/\operatorname{stab}(x)$. Therefore $f$ is injective. Hence $f$ is a bijection. Therefore $|G/\operatorname{stab}(x)|=|\operatorname{orb}(x)|$. Hence we get $|\operatorname{orb}(x)|\cdot |\operatorname{stab}(x)|=|G|$. 
	\item $x,y\in \Om$ are related if $y\in \operatorname{orb}(x)$ i.e. $\exs\ g\in G$ such that $\theta(g)(x)=y$. Now for any $x\in \Om, $  $x$ is related to $x$ since the identity element of $G$, $\theta(1)(x)=x$. Hence the relation is reflexive. Now let $x.y\in Om$ are related. then $\exs\ $ $g\in G$ such that $\theta(g)(x)=y$. Then $\theta(g^{-1})(y)=x$. Therefore the relation is symmetric. Now suppose $x,y,z\in \Om$ and $x,y$ are related, $y,z$ are related. Hence there exists $g,h\in G$ such that $\theta(g)(x)=y$ and $\theta(h)(y)=z$. Then $\theta(hg)(x)=\theta(h)\theta(g)(x)=\theta(h)(y)=z$. Hence $x,z$ are also related. Therefore the relation is transitive. Hence the relation is an equivalence relation.
	\item Let $\Gm$ denote the set of different orbits. \begin{align*}
		\frac{1}{|G|} \sum_{g \in G}\left|\operatorname{fix}(g)\right| & =\frac{1}{|G|} \sum_{g \in G}|\{x \in \Om: g x=x\}|                                                                        \\
		& =\frac1{|G|} |\{(g,x)\colon g\in G, x\in \Om, g\cdot x=x\}|\\
		                                             & =\frac{1}{|G|} \sum_{x \in \Om}|\{g \in G: g x=x\}|                                                                        \\
		                                             & =\frac{1}{|G|} \sum_{x \in \Om}|\operatorname{stab}(x)|                                                                    \\
		                                             & =\frac{1}{|G|} \sum_{x \in \Om} \frac{|G|}{|\operatorname{orb}(x)|}                                                        \\
		                                             & =\sum_{x \in \Om} \frac{1}{|\operatorname{orb}(x)|}                                                                        \\
		                                             & =\sum_{\operatorname{orb}(x) \in \Gm}\left(\sum_{y \in \operatorname{orb}(y)} \frac{1}{|\operatorname{orb}(x)|}\right) \\
		                                             & =\sum_{\operatorname{orb}(x) \in \Gm}\left(\sum_{y \in \operatorname{orb}(x)} \frac{1}{|\operatorname{orb}(x)|}\right) \\
		                                             & =\sum_{\operatorname{orb}(x) \in \Gm} 1                                                                                \\
		                                             & =|\Gm|
	\end{align*} 
\item The rotational symmetries of a regular tetrahedrons are:
\begin{itemize}
	\item Identity Rotation: It leaves all the vertices fixed. So there is only one such symmetry.
	\item $120^{\circ}$ or $240^{\circ}$ rotation about an axis which passes through a vertex and the center of the opposite face. Each axis has 2 rotation choices. And there are $4$ choices for the vertex. Hence there are $8$ such symmetries.
	\item $180^{\circ}$ rotation about an axis through midpoints of opposite edges. For each axis there is only one rotation. There are 3 such axes. Hence there are $3$ such symmetries.
\end{itemize}Hence total number of symmetries $1+8+3=12$. $A_4$ be the set of permutations of $S_4$ which contains even permutations. Since half of the permutations of $S_4$ are even permutations we have $|A_4|=12$. Let $g,h\in A_4$. Then $g$ and $h$ can be broken into even number product of transpositions. Let $g$ can be written as product of $2k$ transpositions and $h$ can be written as product of $2l$ transpositions.  Therefore $gh$ is also even permutation since $gh$ can be written as product $2(k+l)$ transpositions where the first $2k$ are transpositions from $g$ and the last $2l$ are transpositions from  $h$. Hence $gh\in A_4$. Therefore $A_4$ is a subgroup of $S_4$.\parinn

\begin{lemma}
	Symmetry group of a regular tetrahedron is $A_4$
\end{lemma}
\begin{proof}
	We first enumerate the vertices of the regular tetrahedron by $[4]$. The identity rotation is basically fixing all the vertices. Therefore the identity element of $A_4$  corresponds to the identity rotation.
	
	Now consider the $120^{\circ}$ or $240^{\circ}$ rotations about an axis which passes through a vertex and the center of the opposite face. Now in $A_4$ all the 3-length cycles corresponds to such rotations. Because for any 3-cycle we first fix an element. WLOG suppose we fix 4. Then we permute the rest of the elements. So the only options are $(123)$ and $(132)$. Now the permutation $(123)$ corresponds to $120^{\circ}$ rotation about the axis passing through  the vertex 4 and middle point of the triangle generated by the vertices 1,2,3. Similarly $(123)$ corresponds to $240^{\circ}$ rotation about the axis passing through  the vertex 4 and middle point of the triangle generated by the vertices 1,2,3. Hence each of the  $120^{\circ}$ or $240^{\circ}$ rotations symmetries corresponds to each of the 3-cycles of $A_4$. 
	
	For $180^{\circ}$ rotation about an axis through midpoints of opposite edges if one of the edge is between $a,b\in [4]$, $a\neq b$ then the other edge is between $[4]-\{a,b\}$ since they are opposite edge. Let $\{c,d\}=[4]-\{a,b\}$. Now rotating $180-^{\circ}$ about the axis through the middle points of the edges $ab$ and $cd$ is basically permuting $\{a,b\}$ and $\{c,d\}$. Therefore $180-^{\circ}$ about the axis through the middle points of the edges $ab$ and $cd$ corresponds to the permutation $(ab)(cd)$. There are only one such symmetry for the axis through the middle point of edges $ab$ and $cd$. Now there are $3$ such permutations in $A_4$. And each of them corresponds to $180^{\circ}$ rotation about an axis through midpoints of opposite edges. 
	
	Hence there is a one-one correspondence between rotational symmetries of regular tetrahedron and the group $A_4$. Hence the symmetry group of a regular tetrahedron is $A_4$.  
\end{proof}

Now we will calculate $|\text{fix}(g)|$ for each $g\in A_4$. There are $k$ colors.
\begin{itemize}
	\item Identity: Since this fixes all the vertices all $k^4$ coloring are fixed under the identity.
	\item $120^{\circ}$ or $240^{\circ}$ rotation about an axis which passes through a vertex and the center of the opposite face: One vertex is fixed and the other 3 vertices are cycled. A coloring is invariant if the 3 cycled vertices are of same color and the $4^{th}$ vertex can be any color. Hence the number of fixed colorings is $k^2$.
	\item $180^{\circ}$ rotation about an axis through midpoints of opposite edges: Here it swaps two disjoint pairs of vertices. Hence a coloring is invariant if each pair has the same color. Therefore the number of fixed colorings is $k^2$>.
\end{itemize}

Now applying result in previous part we have the number of distinct colorings is $\frac1{12}(k^4+8k^2+3k^2)=\frac{k^4+11k^2}{12}$
\end{itemize}
}


%%%%%%%%%%%%%%%%%%%%%%%%%%%%%%%%%%%%%%%%%%%%%%%%%%%%%%%%%%%%%%%%%%%%%%%%%%%%%%%%%%%%%%%%%%%%%%%%%%%%%%%%%%%%%%%%%%%%%%%%%%%%%%%%%%%%%%%%
% Problem 6
%%%%%%%%%%%%%%%%%%%%%%%%%%%%%%%%%%%%%%%%%%%%%%%%%%%%%%%%%%%%%%%%%%%%%%%%%%%%%%%%%%%%%%%%%%%%%%%%%%%%%%%%%%%%%%%%%%%%%%%%%%%%%%%%%%%%%%%%
\addtocounter{problem}{1}
%\begin{problem}{%problem statement
%	}{p6% problem reference text
%	}
%6. Let $n>0$ and $f:\{-1,1\}^n \rightarrow\{-1,1\}$. For $w \in[0,1]$, we say that a family $\mathcal{S}$ of subsets of $[n]$ is $w$-heavy if it holds that:
%$$
%\sum_{\substack{S \subseteq[n] \\ S \in \mathcal{S}}} \hat{f}(S)^2 \geq w
%$$
%
%Show that:
%\begin{itemize}
%	\item For all $\epsilon>0$, the family of all sets of size $\leq \mathrm{I}(f) / \epsilon$ is $(1-\epsilon)$-heavy.
%\item For all $\rho \in(0,1)$, the family of all sets of size $\leq \frac{2}{1-\rho}$ is $\left(1.2 \cdot \operatorname{Stab}_\rho(f)-0.2\right)$-heavy.
%\item Let $\epsilon>0$ and $\mathcal{S}$ be a family that is $(1-\epsilon / 2)$-heavy. Show that $f$ is PAC-learnable. Namely, show that there is a randomized algorithm that given access to $n, \epsilon$, and $\mathcal{S}$ takes time poly $\left(|\mathcal{S}|, n, \epsilon^{-1}\right)$ and with probability at least 0.99 , outputs a function $h:\{-1,1\}^n \rightarrow\{-1,1\}$ that agrees with $f$ on all but a $1-\epsilon$ fraction of inputs. The algorithm has access to random queries of $f$, i.e., it can ask for samples of the form $(x, f(x))$ for uniformly random $x$, but cannot query $f$ in any other way. 
%
%\textit{Hint: PAC stands for probably approximately correct. It may be helpful to read about PAC-learnability, but is not necessary to solve the question.}
%\end{itemize}
%\end{problem}
%\solve{
%	\begin{itemize}
%		\item 
%	\end{itemize}
%}

%%%%%%%%%%%%%%%%%%%%%%%%%%%%%%%%%%%%%%%%%%%%%%%%%%%%%%%%%%%%%%%%%%%%%%%%%%%%%%%%%%%%%%%%%%%%%%%%%%%%%%%%%%%%%%%%%%%%%%%%%%%%%%%%%%%%%%%%
% Problem 7
%%%%%%%%%%%%%%%%%%%%%%%%%%%%%%%%%%%%%%%%%%%%%%%%%%%%%%%%%%%%%%%%%%%%%%%%%%%%%%%%%%%%%%%%%%%%%%%%%%%%%%%%%%%%%%%%%%%%%%%%%%%%%%%%%%%%%%%%

\begin{problem}{%problem statement
	}{p7% problem reference text
	}
Let $n>0$ and $M$ be an $n\times n$ symmetric matrix with positive entries. Let $\lm$ be the largest eigenvalue of $M$. Show that 
	\begin{itemize}
		\item $\lm>0$.
		\item There exists an eigenvector for $\lm$ with positive entries.
		\item $\lm$ has multiplicity $1$.
		\item For all other eigenvalues $\lm'$, we have $\lm>|\lm'|$.
	\end{itemize}
\end{problem}
\solve{
\begin{itemize}
\item Entries of $M$ are positive. Hence the trace of $M$, $tr(M)>0$. Now $tr(M)$ is the sum of eigenvalues of $M$. Hence there must exists an eigenvalue of $M$ which is positive since $tr(M)>0$. Hence the $\lm>0$ since $\lm$ is the largest eigenvalue.
\item Since $M$ is a real symmetric matrix. It is diagonalizable over $\bbR$. Therefore all the eigenvalues  of $M$ are real and the corresponding eigenvectors of $M$ are real. Let $x,y$ are  eigenvectors for distinct eigenvalues $\lm_1$ and $\lm_2$. Then $$\lm_2y^Tx= (\lm_2 y)^Tx= (My)^T x=y^TMx=y^T(\lm _1 x)= \lm _1 y^Tx\implies (\lm_1-\lm_2)y^Tx=0\implies y^Tx=0$$Hence eigenvectors for distinct eigenvalues are orthogonal to each other. Therefore using Gram Schmidt we obtain orthonormal eigenbasis within same eigenspace. Therefore we get a real orthornormal eigenbasis for real symmetric matrix. \parinn

Therefore there is an real orthonormal eigenbasis $\{v_i\colon i\in[n]\}$ with real eigenvalues $\{\lm_i\colon i\in[n]\}$ such that $M=\sum\limits_{i=1}^n \lm_iv_iv_i^T$. Let $v\in \bbR^n$ where all the entries of $v$ are nonnegative and $\|v\|=1$. Then there exists unique $b_i\in \bbR$ such that $v=\sum_{i=1}^n b_iv_i$. Then we have $$v^{T}Mv = v^{T}\lt(\sum_{i=1}^n \lm_i v_iv_i^T\rt)\lt(\sum_{i=1}^n b_iv_i\rt)=\lt(\sum_{i=1}^n b_i^*v_i^T\rt)\sum_{i=1}^n \lm_ib_i v_i=\sum_{i=1}^n \lm_i |b_i|^2$$Therefore $$|v^TMv|\leq \sum_{i=1}^n |\lm_i||b_i|^2\leq \lm \sum_{i=1}^n |b_i|^2=\lm$$Now if $v$ is the eigenvector with eigenvalue $\lm$ then $v^{T}M=\lm v^Tv=\lm$. Hence the equality holds. Let $v$ is not a eigenvector with eigenvalue $\lm$. Suppose the first $k$ eigenvectors are of eigenvalue $\lm$.  Then $|b_i|^2<1$ for all $i\in[k]$. Since each $|b_i|^2\in[0,1]$, $\sum_{i=1}^n \lm_i |b_i|^2$ is a convex combination of the eigenvalues of $M$. Since $\lm$ is the largest eigenvalue the largest value the convex combinations of the eigenvalues can take is $\lm$. Since $|b_1|^2<1$ the value taken by convex combination is less than $\lm$. Hence $v^{T}Mv<\lm$. Therefore $v^{T}Mv=\lm$ if and only if $v$ is an eigenvector for eigenvalue $\lm$. 

Since $\lm$ is an eigenvalue of $M$ let $w$ be the normalized real eigenvector of $M$ with corresponding eigen value $\lm$. Let $M=(m_{i,j})_{1\leq i,j\leq n}$. Now define the vector $x$, where $x_i=|w_i|$. Now $$0<\lm = \sum_{i,j}m_{i,j}w_iw_j=\lt| \sum_{i,j}m_{i,j}w_iw_j\rt|\leq \sum_{i,j}m_{i,j}|w_i||w_j|=\sum_{i,j}m_{i,j}x_ix_j\leq \lm$$ Therefore $x$ is an eigenvector with eigenvalue $\lm$. Hence we have $Mx=\lm x$ and $x$ has all entries nonnegative. Now if for any $i\in[n]$, $x_i\neq 0$ since $\lm x_i=\sum_{j=1}^n a_{i,j}x_j$ and at least one $x_j\neq 0$. Hence all the entries of $x$ are positive.
\item Suppose $\lm$ has multiplicity more than $1$. Then there exists two real orthonormal eigenvectors vectors $u,v$ for eigenvalue $\lm$. Let $\exs\ i\in[n]$ such that $u_i<0$. Then $$0=\lm (u_i+|u_i|)=\sum_{j=1}^n a_{i,j}(u_j+|u_j|)$$Since $u_j+|u_j|\geq 0$ for all $j\in[n]$ and $a_{i,j}>0$ for all $i,j\in[n]$ we conclude $u_j+|u_j|=0$ for all $j\in [n]$. Therefore wither $u_j=|u_j|$ for all $j\in[n]$ or $u_j=-|u_j|$ for all $j\in [n]$. Similarly we obtain the same for $v$ and get $v_j=|v_j|$ for all $j\in [n]$ or $v_j=-|v_j|$ for all $j\in [n]$. Then $$v^Tu=\sum_{j=1}^n v_ju_j=\pm \sum_{j=1}^n |v_ju_j|\neq 0$$Therefore $u,v$ are not orthogonal. Hence contradiction. Therefore $\lm$ has multiplicity 1.
\item Let $\lm'$ be any other eigenvalue. So $\lm>\lm'$. Let $u$ be the normalized eigenvector of $\lm'$. Then we have $\lm>|u^TMu|=|\lm' u^Tu|=|\lm'|$. Hence $\lm>|\lm'|$ for any other eigenvalue. 
\end{itemize}
}\parinf

[I discussed with Shubham and Soumyadeep and Aakash]\parinn

%%%%%%%%%%%%%%%%%%%%%%%%%%%%%%%%%%%%%%%%%%%%%%%%%%%%%%%%%%%%%%%%%%%%%%%%%%%%%%%%%%%%%%%%%%%%%%%%%%%%%%%%%%%%%%%%%%%%%%%%%%%%%%%%%%%%%%%%
% Problem 8
%%%%%%%%%%%%%%%%%%%%%%%%%%%%%%%%%%%%%%%%%%%%%%%%%%%%%%%%%%%%%%%%%%%%%%%%%%%%%%%%%%%%%%%%%%%%%%%%%%%%%%%%%%%%%%%%%%%%%%%%%%%%%%%%%%%%%%%%

\begin{problem}{%problem statement
	}{p8% problem reference text
	}
Let $0<n \ll k$ and $x_1, \ldots, x_k$ be points in $\mathbb{R}^n$ and $v_1, \ldots, v_k \in \mathbb{R}$ be corresponding values. In many learning settings, the goal is to find the smallest simple function, say a low degree polynomial, that best approximates the given values. For example, one can try to find the degree 3 polynomial $p(\cdot)$ (over $n$ variables) that minimizes the "loss function":
$$
\sum_{i=1}^k\left(p\left(x_i\right)-v_i\right)^2
$$
\begin{itemize}
	\item Show that the set of all polynomials with degree at most 3 with standard operations is a vector space $V$ and, if all the values are 0 , the set of polynomials for which the loss function is 0 is a subspace of $V$.
\item Show that $x_1, \ldots, x_k$ define a linear transformation $\theta$ from $V$ to $\mathbb{R}^k$. Use this to show an inner product function for which the above loss minimization problem is just a problem of computing the projection of a given vector.
\item Solve the learning problem using the adjoint transformation $\theta^{\dagger}$. You may assume that $\theta$ is one-to-one.
\end{itemize}
\end{problem}
\solve{
\begin{itemize}
	\item Consider the set $V$ of all polynomials with degree $3$ at most. \begin{itemize}
		\item \textbf{Closure:} For all $p,q\in S$, $p+q$ has degree at most 3. Therefore $p+q\in V$. So $V$ is closed under addition
		\item \textbf{Associativity} For all $p,q,r\in V$, $\forall\ x\in\bbR$, \begin{multline*}
			((p+q)+r)(x)=(p+q)(x)+r(x)=(p(x)+q(x))+r(x)=p(x)+(q(x)+r(x))\\
			=p(x)+(q+r)(x)=(p+(q+r))(x)
		\end{multline*}Hence $(p+q)+r=p+(q+r)$. So $V$ follows associativity.
		\item \textbf{Identity:} The identity element of $V$ is the $0$ polynomial. It has degree $0$ so $0\in V$. And for any $x\in \bbR$, and $p\in V$, $p(x)+0(x)=0(x)+p(x)+p(x)$. Hence $p+0+0+p=p$.
		\item \textbf{Inverse:} For any $p\in V$, $p\neq 0$, the inverse of $p$ is $-p$ where the coefficients of $-p$ are the negative of the coefficients of $p$. Then for all $x\in \bbR$, $$(p+(-p))(x)=p(x)+(-p)(x)=p(x)-p(x)=0\quad ((-p)+p)(x)=(-p)(x)+p(x)=-p(x)+p(x)=0$$Therefore $p+(-p)=(-p)+p=0$. 
		\item \textbf{Commutativity:} Let $p,q\in V$. Then for any $x\in \bbR$, $(p+q)(x)=p(x)+q(x)=q(x)+p(x)=(q+p)(x)$. Therefore $V$ is commutative.
		\item \textbf{Scaler Multiplication:} For any $a,b\in \bbR$, and $\forall\ p\in V$ $$\forall\ x\in \bbR\ (1\cdot p)(x)=p(x)\implies 1\cdot p=p$$ $$\forall\ x\in\bbR\ (a\cdot (b\cdot p))(x)=a\cdot (b\cdot p)(x)=a\cdot (b\cdot p(x))=(a\cdot b)p(x)\implies a\cdot (b\cdot p)=(a\cdot b)p$$So $V$ is closed under scaler multiplication.
		\item \textbf{Distributivity:} For all $p,q\in V$ and $a,b\in \bbR$ \begin{multline*}
			\forall\ x\in \bbR\ (a\cdot (p+q))(x)=a\cdot (p+q)(x)=a\cdot (p(x)+q(x))=a\cdot p(x)+a\cdot q(x)=(a\cdot p+a\cdot q)(x)\\
			\implies a\cdot (p+q)=a\cdot p+a\cdot q
		\end{multline*}And \begin{multline*}
			\forall\ x\in \bbR\ ((a+b)\cdot p)(x)=(a+b)\cdot p(x)=a\cdot p(x)+b\cdot p(x)=(a\cdot p+b\cdot p)(x)\implies (a+b)\cdot p=a\cdot p+b\cdot p
	\end{multline*}Hence $V$ follows distributivity property
	\end{itemize}Therefore $V$ is a vector space over $\bbR$\parinn


Now for any $a\in \bbR$, consider the map $v_a: V\to \bbR$ where $v_a(p)=p(a)$. Now for any $p,q\in V$, $v_a(p+q)=(p+q)(a)=p(a)+q(a)=v_a(p)+v_a(q)$. Therefore $v_a$ is a linear map. Now the consider the map $\theta:V\to\bbR^k$ where $$\theta(p)=(v_{x_i}(p)\colon i\in [k])$$Therefore $\theta$ is also a linear map between the vector spaces $V$ and $\bbR^k$.

 Suppose all the values are 0. Then the set of polynomials of $V$ for which the loss function is 0 is  exactly the polynomials whose evaluations at $x_i$ is 0 for all $i\in[k]$. Hence the set of polynomials in $V$ for which the loss function is 0 is $\ker \theta$. Since $\ker\theta$ is a subspace of $V$
 \item As we explained before for any  $a\in \bbR$, $v_a:V\to \bbR$ where $v_a(p)=p(a)$ is a linear transformation. Therefore the map $\theta=(v_{x_i}\colon i\in [k])$ where for any $p\in V$, $\theta(p)=(p(x_1),p(x_2),\dots, p(x_k))$ is also linear map. Let $W=\Im \theta$.\parinn
 
 We will take the standard inner product on $\bbR^k$. Then $\bbR^k=W\oplus W^{\perp}$ where $W^{\perp}$ is orthogonal to $W$. Let $v\in \bbR$. Then $v$ can be written uniquely as $v=w+w^{\perp}$ where $w\in W$ and $w^{\perp}\in W^{\perp}$. Then we have $$\|v-\theta(p)\|^2=\|w-\theta(p)\|^2+\|w^{\perp}\|^2$$Hence $p\in V$ be such that  $\|v-\theta(p)\|$ is minimum then $\|w-\theta(p)\|^2$ is minimum which can be zero. Therefore $p\in\theta^{-1}(w)$. Therefore the loss is minimum when we pick a polynomial from the $\theta^{-1}(v-w^{\perp})$ where $w^{\perp}$ is the  projection or $v$ orthogonal to the space $\Im\theta$. 
\end{itemize}
}
%%%%%%%%%%%%%%%%%%%%%%%%%%%%%%%%%%%%%%%%%%%%%%%%%%%%%%%%%%%%%%%%%%%%%%%%%%%%%%%%%%%%%%%%%%%%%%%%%%%%%%%%%%%%%%%%%%%%%%%%%%%%%%%%%%%%%%%%
% Problem 9
%%%%%%%%%%%%%%%%%%%%%%%%%%%%%%%%%%%%%%%%%%%%%%%%%%%%%%%%%%%%%%%%%%%%%%%%%%%%%%%%%%%%%%%%%%%%%%%%%%%%%%%%%%%%%%%%%%%%%%%%%%%%%%%%%%%%%%%%

\begin{problem}{%problem statement
	}{p9% problem reference text
	}
Let $n>0$. Call an operator $\theta \in L\left(\mathbb{R}^n\right)$ a "rotation" if for all $v \in \mathbb{R}^n$, the (Euclidean) norm of $v$ equals the (Euclidean) norm of $\theta(v)$. As an example, the transformation $\theta \in L\left(\mathbb{R}^2\right)$ mapping $(x, y) \rightarrow(y, x)$ is a rotation. For what values of $n>0$ does there exists a rotation that does not have any non-zero fixed points, i.e., for all $v \neq 0 \in \mathbb{R}^n$, we have $\theta(v) \neq v$.
\end{problem}
\solve{
For all $n>0$, consider the operator $\theta\in L(\bbR^n)$ where for any $v\in \bbR^n$, $\theta(v)=-v$. Now $\|v\|=\|-v\|$. Hence $\theta$ is also a rotation operator. Then this operator has only one fixed point which is the $0$ vector. Therefore for all $n>0$ there exists a rotation that does not have any non-zero fixed point. 
}\parinf

[Me and Soumyadeep solved this together.]\parinn
%%%%%%%%%%%%%%%%%%%%%%%%%%%%%%%%%%%%%%%%%%%%%%%%%%%%%%%%%%%%%%%%%%%%%%%%%%%%%%%%%%%%%%%%%%%%%%%%%%%%%%%%%%%%%%%%%%%%%%%%%%%%%%%%%%%%%%%%
% Problem 10
%%%%%%%%%%%%%%%%%%%%%%%%%%%%%%%%%%%%%%%%%%%%%%%%%%%%%%%%%%%%%%%%%%%%%%%%%%%%%%%%%%%%%%%%%%%%%%%%%%%%%%%%%%%%%%%%%%%%%%%%%%%%%%%%%%%%%%%%

\begin{problem}{%problem statement
	}{p10% problem reference text
	}
Fields with only 1 element do not exist. Nonetheless, we can still use them in problem sets. Let $q>0$ and $V$ be a vector space over a field $\mathbb{F}$ of size $q$. Define $n=\operatorname{dim}(V)$.
\begin{itemize}
	\item For $0 \leq k \leq n$, compute the number of subspaces of $V$ of dimension $k$. Call this number $\binom{n}{k}_q$.
	\item Prove the identities:
	$$
	\begin{aligned}
		\binom{n}{k}_q & =q^k \cdot\binom{n-1}{k}_q+\binom{n-1}{k-1}_q=\binom{n-1}{k}_q+q^{n-k} \cdot\binom{n-1}{k-1}_q \\
		\binom{2 n}{n}_q & =\sum_{k=0}^n q^{k^2} \cdot\binom{n}{k}_q^2
	\end{aligned}
	$$
\end{itemize}

Setting $q=1$, observe how a vector space over a field of size 1 behaves like a set.
\end{problem}
\solve{
\begin{itemize}
\item For $k=0$ we have $\binom{n}{0}_q=1$. So now we will assume $k\geq 1$. We will first calculate the number of ways to choose $k$ linearly independent vectors in $V$ which forms as a basis of a $k$-dimensional vector space. \begin{lemma}
	Let $W$ be an $n$-dimensional vector space over $\mathbb{F}_q$, the finite field with $q$ elements, and let $0 \leq k \leq n$. Then there exist
	$$
	\frac{(q^n-1)(q^n-q)(q^n-q^2) \dotsm (q^n-q^{k-1})}{k!}
	$$
	many linearly independent subsets of $W$ consisting of $k$ elements.
\end{lemma}
\begin{proof}
	$W$ has $q^n$ elements. So number of ways to select a nonzero vector from $W$ is $q^n-1$. Now we have chosen the first linearly independent vector $v_1$. Now the second linearly independent vector must not by in the space $\la v_1\ra$. So number ways to choose the second vector is $q^n-q$. So now we have chosen $v_2$. Continuing like this suppose we have chosen $v_i$. Then the next vector should not belong to the subspace $\la v_1,\dots, v_i\ra$. Therefore there are $q^n-q^i$ choices for the next vector so that it is linearly independent of all the other vectors chosen before. So there are $q^n-q^i$ choices for $v_{i+1}$. Therefore for $i=k-1$ there are $q^n-q^{k-1}$ choices for choosing $v_k$. Hence there are $\prod\limits_{i=0}^{k}(q^n-q^i)$ many set of linearly independent $k$-vector tuples $(v_1,\dots, v_k)$. Now we can pick the same set of linearly independent vectors $\{v_1,\dots, v_k\}$ in any order. So the number of ways to choose $k$ linearly independent vectors from $W$ is $\frac1{k!}\prod\limits_{i=0}^{k-1}(q^n-q^i)$. 
\end{proof}\parinn 

Now for any $k$-dimensional subspace of $V$ is spanned by $k$ linearly independent vectors of $V$. But same $k$-dimensional subspace can be generated by multiple set of $k$ linearly independent vectors. So we have to divide the total number of ways to chose $k$ linearly independent subsets from $V$ by total number of ways to chose $k$ linearly independent subsets from a $k$-dimensional vector space. So number of different $k$-dimensional subspaces of $V$ is $\frac{\frac1{k!}\prod\limits_{i=0}^{k-1}(q^n-q^i)}{\frac1{k!}\prod\limits_{i=0}^{k-1}(q^k-q^i)}=\frac{\prod\limits_{i=0}^{k-1}(q^n-q^i)}{\prod\limits_{i=0}^{k-1}(q^k-q^i)}$. Therefore $\binom{n}{k}_q=\prod\limits_{i=0}^{k-1}\frac{q^n-q^i}{q^k-q^i}=\prod\limits_{i=0}^{k-1}\frac{q^{n-i}-1}{q^{k-i}-1}$. 
\item \begin{itemize}
	\item  \begin{align*}
		q^k\binom{n-1}{k}_q+\binom{n-1}{k-1}_q & = q^k\prod\limits_{i=0}^{k-1}\frac{q^{n-1-i}-1}{q^{k-i}-1} + \prod\limits_{i=0}^{k-2}\frac{q^{n-1-i}-1}{q^{k-1-i}-1} \\
		                                       & = \lt[\frac{q^k(q^{n-k}-1)}{q^{k}-1}+1\rt]\prod_{i=0}^{k-2}\frac{q^{n-1-i}-1}{q^{k-1-i}-1}                           \\
		                                       & = \frac{q^k(q^{n-k}-1)- (q^k-1)}{q^{k}-1}\prod_{i=0}^{k-2}\frac{q^{n-1-i}-1}{q^{k-1-i}-1}                            \\
		                                       & = \frac{q^n-1}{q^k-1}\prod_{i=0}^{k-2}\frac{q^{n-1-i}-1}{q^{k-1-i}-1}                                                \\
		                                       & = \frac{q^n-1}{q^k-1}\prod_{i=1}^{k-1}\frac{q^{n-i}-1}{q^{k-i}-1}                                                    \\
		                                       & =\prod_{i=0}^{k-1}\frac{q^{n-i}-1}{q^{k-i}-1}= \binom{n}{k}_q
	\end{align*}
\item \begin{align*}
	\binom{n-1}{k}_q+q^{n-k} \binom{n-1}{k-1}_q & = \prod\limits_{i=0}^{k-1}\frac{q^{n-1-i}-1}{q^{k-i}-1} + q^{n-k}\prod\limits_{i=0}^{k-2}\frac{q^{n-1-i}-1}{q^{k-1-i}-1} \\
	                                            & = \lt[\frac{q^{n-k}-1}{q^{k}-1}+q^{n-k}\rt]\prod_{i=0}^{k-2}\frac{q^{n-1-i}-1}{q^{k-1-i}-1}                              \\
	                                            & = \frac{q^{n-k}-1+ (q^k-1)q^{n-k}}{q^{k}-1}\prod_{i=0}^{k-2}\frac{q^{n-1-i}-1}{q^{k-1-i}-1}                              \\
	                                            & = \frac{q^n-1}{q^k-1}\prod_{i=0}^{k-2}\frac{q^{n-1-i}-1}{q^{k-1-i}-1}                                                    \\
	                                            & = \frac{q^n-1}{q^k-1}\prod_{i=1}^{k-1}\frac{q^{n-i}-1}{q^{k-i}-1}                                                        \\
	                                            & = \prod_{i=0}^{k-1}\frac{q^{n-i}-1}{q^{k-i}-1}= \binom{n}{k}_q
\end{align*}
\item \parinn 
Now we have \begin{align*}
	\binom{n}{k}_q & =\prod_{i=0}^{k-1}\frac{q^{n-i}-1}{q^{k-i}-1}                                                                           \\
	               & =\prod_{i=0}^{k-1}\frac{q^{n-i}-1}{q^{k-i}-1}\prod_{i=0}^{n-k-1}\frac{q^{n-k-i}-1}{q^{n-k-i}-1}                         \\
	               & =\frac{\prod\limits_{i=0}^{n-1}(q^{n-i}-1)}{\prod\limits_{i=0}^{k-1}(q^{k-i}-1)\prod\limits_{i=0}^{n-k-1}(q^{n-k-i}-1)} \\
	               & =\prod_{i=0}^{n-k-1}\frac{q^{n-i}-1}{q^{n-k-i}-1}=\binom{n}{n-k}_q
\end{align*}
First we will prove an identity on the generating function of $\binom{n}{k}_q$. 
\begin{lemma}\label{qbinomgen}
	$\dps{\prod\limits_{k=0}^{n-1}(1+q^kx)=\sum\limits_{k=0}^{n}q^{\frac{k(k-1)}2}\binom{n}{k}_qx^k}$ for any $n\in \bbN$.
\end{lemma}
\begin{proof}
	We will prove this using induction on $n$. For $n=1$ we have $\prod\limits_{k=0}^{0}(1+q^kx)=1+x$. Now $$\sum_{k=0}^{1}q^{\frac{k(k-1)}2}\binom{n}{k}_qx^k=\binom{n}{0}_q+\binom{1}{1}_qx=1+x$$Hence the base case follows. Suppose this is true for $n-1$. Then \begin{align*}
		\prod_{k=0}^n (1+q^k x) & = (1+q^n)\sum_{k=0}^n q^{\frac{k(k-1)}2}\binom{n}{k}_qx^k\\
		& = \sum_{k=0}^n q^{\frac{k(k-1)}2}\binom{n}{k}_qx^k + \sum_{k=0}^n q^nq^{\frac{k(k-1)}2}\binom{n}{k}_qx^{k+1}\\
		& = 1+ \sum_{k=1}^n q^{\frac{k(k-1)}2}\binom{n}{k}_qx^k + \sum_{k=0}^{n-1} q^n q^{\frac{k(k-1)}2}\binom{n}{k}_qx^{k+1}+q^{n+\frac{n(n-1)}{2}}\binom{n}{n}_q x^{n+1}\\
		& =1+ \sum_{k=0}^{n-1} q^{\frac{(k+1)k}2}\binom{n}{k+1}_qx^{k+1} + \sum_{k=0}^{n-1} q^n q^{\frac{k(k-1)}2}\binom{n}{k}_qx^{k+1}+q^{n+\frac{n(n-1)}{2}}\binom{n}{n}_q x^{n+1}\\
		& = 1+ \sum_{k=0}^{n-1}\lt(  q^{\frac{(k+1)k}2}\binom{n}{k+1}_q + q^{n+\frac{k(k-1)}2}\binom{n}{k}_q  \rt)x^{k+1}+ q^{n+\frac{n(n-1)}{2}}\binom{n}{n}_q x^{n+1}\\
		& = 1+ \sum_{k=0}^{n-1}\lt(  q^{\frac{(k-1)k}2}\binom{n}{k+1}_q + q^{n+\frac{k(k-1)}2+k}\binom{n}{k}_q  \rt)x^{k+1}+ q^{n+\frac{n(n-1)}{2}}\binom{n}{n}_q x^{n+1}\\
			& = 1+ \sum_{k=0}^{n-1}  q^kq^{\frac{(k-1)k}2}\lt(\binom{n}{k+1}_q + q^{n-k}\binom{n}{k}_q  \rt)x^{k+1}+ q^{n+\frac{n(n-1)}{2}}\binom{n}{n}_q x^{n+1}\\
		& = 1+ \sum_{k=0}^{n-1}  q^{\frac{(k+1)k}2}\binom{n+1}{k+1}_qx^{k+1}+ q^{n+\frac{n(n-1)}{2}}\binom{n}{n}_q x^{n+1}\\
		& = 1+ \sum_{k=1}^{n}  q^{\frac{(k-1)k}2}\binom{n+1}{k}_qx^{k}+ q^{n+\frac{n(n-1)}{2}}\binom{n}{n}_q x^{n+1}\\
		& = \sum_{k=0}^{n+1}  q^{\frac{(k-1)k}2}\binom{n+1}{k}_qx^{k}
	\end{align*} Hence by mathematical induction this is true for all $n\in\bbN$. 
\end{proof}
\begin{lemma}\label{breakqprod}
	$\dps{\prod_{k=0}^{m+n-1}(1+q^kx)=\lt(  \prod_{k=0}^{m-1}(1+q^kx)\rt)\lt( \prod_{k=0}^{n-1}(1+q^{k+m}x) \rt)}$ for any $m,n\in\bbN$.
\end{lemma}
\begin{proof}
	First we fix any $m$. We will show this using induction on $n$. For $n=1$ we have $$\lt(  \prod_{k=0}^{m-1}(1+q^kx)\rt)\lt( \prod_{k=0}^{1-1}(1+q^{k+m}x)\rt) =\lt(  \prod_{k=0}^{m-1}(1+q^kx)\rt)(1+q^m)=\prod_{k=0}^{m}(1+q^k x)$$Hence the base case follows. Now suppose this is true for $n$. Then \begin{multline*}
		\prod_{k=0}^{m+n}(1+q^kx) = (1+q^{m+n}x)\lt(  \prod_{k=0}^{m-1}(1+q^kx)\rt)\lt( \prod_{k=0}^{n-1}(1+q^{k+m}x) \rt)\\
		=\lt(  \prod_{k=0}^{m-1}(1+q^kx)\rt)\lt( \prod_{k=0}^{n}(1+q^{k+m}x) \rt)
	\end{multline*}Therefore by mathematical induction this is true for $n$. 
\end{proof}
\begin{lemma}\label{qvandermonde}
	${\displaystyle {\binom {m+n}{k}}_{\!\!q}=\sum _{j=0}^k{\binom {m}{k-j}}_{\!\!q}{\binom {n}{j}}_{\!\!q}q^{j(m-k+j)}}$.
\end{lemma}
\begin{proof}
	Using the \lemref{qbinomgen} for any $m,n\in \bbN$ we have $$\prod\limits_{k=0}^{m+n-1}(1+q^kx)=\sum\limits_{k=0}^{m+n}q^{\frac{k(k-1)}2}\binom{m+n}{k}_qx^k$$Now we also have using \lemref{breakqprod} \begin{align*}
		\prod_{k=0}^{m+n-1}(1+q^kx)& =\lt(  \prod_{k=0}^{m-1}(1+q^kx)\rt)\lt( \prod_{k=0}^{n-1}(1+q^{k}(q^m x)) \rt)\\
		& = \lt(\sum\limits_{k=0}^{m}q^{\frac{k(k-1)}2}\binom{m}{k}_qx^k\rt)\lt(\sum\limits_{k=0}^{n}q^{\frac{k(k-1)}2}\binom{m}{k}_q(q^m x)^k\rt)\\
		& = \lt(\sum\limits_{k=0}^{m}q^{\frac{k(k-1)}2}\binom{m}{k}_qx^k\rt)\lt(\sum\limits_{k=0}^{n}q^{mk+\frac{k(k-1)}2}\binom{m}{k}_qx^k\rt)\\
		& = \sum_{k=0}^{m+n}\lt(\sum_{l=0}^k q^{\frac{(k-l)(k-l-1)}2}\binom{m}{k-l}_qq^{ml+\frac{l(l-1)}{2}}\binom{n}{l}_q  \rt)x^k\\
		& = \sum_{k=0}^{m+n}\lt(\sum_{l=0}^k q^{\frac{k(k+1)}2}q^{l(m+l-k)}\binom{m}{k-l}_q\binom{n}{l}_q \rt)x^k
	\end{align*}
Hence by equating coefficients we have $\binom {m+n}{k}_{\!\!q}=\sum\limits _{j=0}^k{\binom {m}{k-j}}_{\!\!q}{\binom {n}{j}}_{\!\!q}q^{j(m-k+j)}$
\end{proof}

Using \lemref{qvandermonde} for $m=n=k$ we get $$\binom {2n}{n}_{\!\!q}=\sum\limits _{j=0}^n{\binom {n}{n-j}}_{\!\!q}{\binom {n}{j}}_{\!\!q}q^{j(n-n+j)}=\sum\limits _{j=0}^n{\binom {n}{n-j}}_{\!\!q}{\binom {n}{j}}_{\!\!q}q^{j^2}=\sum\limits _{j=0}^n{\binom {n}{j}}_{\!\!q}^2q^{j^2}$$
\end{itemize}
\end{itemize}
}\parinf

[Me and Soumyadeep solved this together. I discussed with Aakash and Shubham]\parinn
\end{document}
