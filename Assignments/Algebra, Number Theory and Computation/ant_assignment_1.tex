\documentclass[a4paper, 11pt]{article}
\usepackage{comment} % enables the use of multi-line comments (\ifx \fi) 
\usepackage{fullpage} % changes the margin
\usepackage[a4paper, total={7in, 10in}]{geometry}
\usepackage{amsmath,mathtools,mathdots,graphicx,adjustbox,xcolor}
\usepackage{amssymb,amsthm}  % assumes amsmath package installed
\usepackage{float}
\usepackage{xcolor}
\usepackage{mdframed}
\usepackage[shortlabels]{enumitem}
\usepackage{indentfirst}
\usepackage{hyperref}
\hypersetup{
	colorlinks=true,
	linkcolor=doc!80,
	citecolor=myr,
	filecolor=myr,      
	urlcolor=doc!80,
	pdftitle={Assignment}, %%%%%%%%%%%%%%%%   WRITE ASSIGNMENT PDF NAME  %%%%%%%%%%%%%%%%%%%%
}
\usepackage[most,many,breakable]{tcolorbox}
\usepackage{tikz}
\usepackage{caption}
\usepackage{kpfonts}
\usepackage{libertine}
\usepackage{physics}
\usepackage[ruled,vlined,linesnumbered]{algorithm2e}
\usepackage{mathrsfs}
\usepackage{tikz-cd}
\usepackage{float,marvosym}

\definecolor{mytheorembg}{HTML}{F2F2F9}
\definecolor{mytheoremfr}{HTML}{00007B}
\definecolor{doc}{RGB}{0,60,110}
\definecolor{myg}{RGB}{56, 140, 70}
\definecolor{myb}{RGB}{45, 111, 177}
\definecolor{myr}{RGB}{199, 68, 64}

\usetikzlibrary{decorations.pathreplacing,angles,quotes,patterns}
\definecolor{mytheorembg}{HTML}{F2F2F9}
\definecolor{mytheoremfr}{HTML}{00007B}
\definecolor{doc}{RGB}{0,60,110}
\definecolor{myg}{RGB}{56, 140, 70}
\definecolor{myb}{RGB}{45, 111, 177}
\definecolor{myr}{RGB}{199, 68, 64}
\newcounter{problem}
\tcbuselibrary{theorems,skins,hooks}
\newtcbtheorem[use counter=problem]{problem}{Problem}
{%
	enhanced,
	breakable,
	colback = mytheorembg,
	frame hidden,
	boxrule = 0sp,
	borderline west = {2pt}{0pt}{mytheoremfr},
	arc=5pt,
	detach title,
	before upper = \tcbtitle\par\smallskip,
	coltitle = mytheoremfr,
	fonttitle = \bfseries\sffamily,
	description font = \mdseries,
	separator sign none,
	segmentation style={solid, mytheoremfr},
}
{p}
\DeclareMathOperator{\nae}{NAE}
\newtheorem{lemma}{Lemma}
\renewenvironment{proof}{\noindent{\it \textbf{Proof:}}\hspace*{1em}}{\qed\bigskip\\}
% To give references for any problem use like this
% suppose the problem number is p3 then 2 options either 
% \hyperref[p:p3]{<text you want to use to hyperlink> \ref{p:p3}}
%                  or directly 
%                   \ref{p:p3}

\DeclareMathOperator{\trac}{\textsf{Trace}}
\DeclareMathOperator{\range}{Range}
\DeclareMathOperator{\mult}{Mult}


%---------------------------------------
% BlackBoard Math Fonts :-
%---------------------------------------

%Captital Letters
\newcommand{\bbA}{\mathbb{A}}	\newcommand{\bbB}{\mathbb{B}}
\newcommand{\bbC}{\mathbb{C}}	\newcommand{\bbD}{\mathbb{D}}
\newcommand{\bbE}{\mathbb{E}}	\newcommand{\bbF}{\mathbb{F}}
\newcommand{\bbG}{\mathbb{G}}	\newcommand{\bbH}{\mathbb{H}}
\newcommand{\bbI}{\mathbb{I}}	\newcommand{\bbJ}{\mathbb{J}}
\newcommand{\bbK}{\mathbb{K}}	\newcommand{\bbL}{\mathbb{L}}
\newcommand{\bbM}{\mathbb{M}}	\newcommand{\bbN}{\mathbb{N}}
\newcommand{\bbO}{\mathbb{O}}	\newcommand{\bbP}{\mathbb{P}}
\newcommand{\bbQ}{\mathbb{Q}}	\newcommand{\bbR}{\mathbb{R}}
\newcommand{\bbS}{\mathbb{S}}	\newcommand{\bbT}{\mathbb{T}}
\newcommand{\bbU}{\mathbb{U}}	\newcommand{\bbV}{\mathbb{V}}
\newcommand{\bbW}{\mathbb{W}}	\newcommand{\bbX}{\mathbb{X}}
\newcommand{\bbY}{\mathbb{Y}}	\newcommand{\bbZ}{\mathbb{Z}}

%---------------------------------------
% MathCal Fonts :-
%---------------------------------------

%Captital Letters
\newcommand{\mcA}{\mathcal{A}}	\newcommand{\mcB}{\mathcal{B}}
\newcommand{\mcC}{\mathcal{C}}	\newcommand{\mcD}{\mathcal{D}}
\newcommand{\mcE}{\mathcal{E}}	\newcommand{\mcF}{\mathcal{F}}
\newcommand{\mcG}{\mathcal{G}}	\newcommand{\mcH}{\mathcal{H}}
\newcommand{\mcI}{\mathcal{I}}	\newcommand{\mcJ}{\mathcal{J}}
\newcommand{\mcK}{\mathcal{K}}	\newcommand{\mcL}{\mathcal{L}}
\newcommand{\mcM}{\mathcal{M}}	\newcommand{\mcN}{\mathcal{N}}
\newcommand{\mcO}{\mathcal{O}}	\newcommand{\mcP}{\mathcal{P}}
\newcommand{\mcQ}{\mathcal{Q}}	\newcommand{\mcR}{\mathcal{R}}
\newcommand{\mcS}{\mathcal{S}}	\newcommand{\mcT}{\mathcal{T}}
\newcommand{\mcU}{\mathcal{U}}	\newcommand{\mcV}{\mathcal{V}}
\newcommand{\mcW}{\mathcal{W}}	\newcommand{\mcX}{\mathcal{X}}
\newcommand{\mcY}{\mathcal{Y}}	\newcommand{\mcZ}{\mathcal{Z}}



%---------------------------------------
% Bold Math Fonts :-
%---------------------------------------

%Captital Letters
\newcommand{\bmA}{\boldsymbol{A}}	\newcommand{\bmB}{\boldsymbol{B}}
\newcommand{\bmC}{\boldsymbol{C}}	\newcommand{\bmD}{\boldsymbol{D}}
\newcommand{\bmE}{\boldsymbol{E}}	\newcommand{\bmF}{\boldsymbol{F}}
\newcommand{\bmG}{\boldsymbol{G}}	\newcommand{\bmH}{\boldsymbol{H}}
\newcommand{\bmI}{\boldsymbol{I}}	\newcommand{\bmJ}{\boldsymbol{J}}
\newcommand{\bmK}{\boldsymbol{K}}	\newcommand{\bmL}{\boldsymbol{L}}
\newcommand{\bmM}{\boldsymbol{M}}	\newcommand{\bmN}{\boldsymbol{N}}
\newcommand{\bmO}{\boldsymbol{O}}	\newcommand{\bmP}{\boldsymbol{P}}
\newcommand{\bmQ}{\boldsymbol{Q}}	\newcommand{\bmR}{\boldsymbol{R}}
\newcommand{\bmS}{\boldsymbol{S}}	\newcommand{\bmT}{\boldsymbol{T}}
\newcommand{\bmU}{\boldsymbol{U}}	\newcommand{\bmV}{\boldsymbol{V}}
\newcommand{\bmW}{\boldsymbol{W}}	\newcommand{\bmX}{\boldsymbol{X}}
\newcommand{\bmY}{\boldsymbol{Y}}	\newcommand{\bmZ}{\boldsymbol{Z}}
%Small Letters
\newcommand{\bma}{\boldsymbol{a}}	\newcommand{\bmb}{\boldsymbol{b}}
\newcommand{\bmc}{\boldsymbol{c}}	\newcommand{\bmd}{\boldsymbol{d}}
\newcommand{\bme}{\boldsymbol{e}}	\newcommand{\bmf}{\boldsymbol{f}}
\newcommand{\bmg}{\boldsymbol{g}}	\newcommand{\bmh}{\boldsymbol{h}}
\newcommand{\bmi}{\boldsymbol{i}}	\newcommand{\bmj}{\boldsymbol{j}}
\newcommand{\bmk}{\boldsymbol{k}}	\newcommand{\bml}{\boldsymbol{l}}
\newcommand{\bmm}{\boldsymbol{m}}	\newcommand{\bmn}{\boldsymbol{n}}
\newcommand{\bmo}{\boldsymbol{o}}	\newcommand{\bmp}{\boldsymbol{p}}
\newcommand{\bmq}{\boldsymbol{q}}	\newcommand{\bmr}{\boldsymbol{r}}
\newcommand{\bms}{\boldsymbol{s}}	\newcommand{\bmt}{\boldsymbol{t}}
\newcommand{\bmu}{\boldsymbol{u}}	\newcommand{\bmv}{\boldsymbol{v}}
\newcommand{\bmw}{\boldsymbol{w}}	\newcommand{\bmx}{\boldsymbol{x}}
\newcommand{\bmy}{\boldsymbol{y}}	\newcommand{\bmz}{\boldsymbol{z}}


%---------------------------------------
% Scr Math Fonts :-
%---------------------------------------

\newcommand{\sA}{{\mathscr{A}}}   \newcommand{\sB}{{\mathscr{B}}}
\newcommand{\sC}{{\mathscr{C}}}   \newcommand{\sD}{{\mathscr{D}}}
\newcommand{\sE}{{\mathscr{E}}}   \newcommand{\sF}{{\mathscr{F}}}
\newcommand{\sG}{{\mathscr{G}}}   \newcommand{\sH}{{\mathscr{H}}}
\newcommand{\sI}{{\mathscr{I}}}   \newcommand{\sJ}{{\mathscr{J}}}
\newcommand{\sK}{{\mathscr{K}}}   \newcommand{\sL}{{\mathscr{L}}}
\newcommand{\sM}{{\mathscr{M}}}   \newcommand{\sN}{{\mathscr{N}}}
\newcommand{\sO}{{\mathscr{O}}}   \newcommand{\sP}{{\mathscr{P}}}
\newcommand{\sQ}{{\mathscr{Q}}}   \newcommand{\sR}{{\mathscr{R}}}
\newcommand{\sS}{{\mathscr{S}}}   \newcommand{\sT}{{\mathscr{T}}}
\newcommand{\sU}{{\mathscr{U}}}   \newcommand{\sV}{{\mathscr{V}}}
\newcommand{\sW}{{\mathscr{W}}}   \newcommand{\sX}{{\mathscr{X}}}
\newcommand{\sY}{{\mathscr{Y}}}   \newcommand{\sZ}{{\mathscr{Z}}}


%---------------------------------------
% Math Fraktur Font
%---------------------------------------

%Captital Letters
\newcommand{\mfA}{\mathfrak{A}}	\newcommand{\mfB}{\mathfrak{B}}
\newcommand{\mfC}{\mathfrak{C}}	\newcommand{\mfD}{\mathfrak{D}}
\newcommand{\mfE}{\mathfrak{E}}	\newcommand{\mfF}{\mathfrak{F}}
\newcommand{\mfG}{\mathfrak{G}}	\newcommand{\mfH}{\mathfrak{H}}
\newcommand{\mfI}{\mathfrak{I}}	\newcommand{\mfJ}{\mathfrak{J}}
\newcommand{\mfK}{\mathfrak{K}}	\newcommand{\mfL}{\mathfrak{L}}
\newcommand{\mfM}{\mathfrak{M}}	\newcommand{\mfN}{\mathfrak{N}}
\newcommand{\mfO}{\mathfrak{O}}	\newcommand{\mfP}{\mathfrak{P}}
\newcommand{\mfQ}{\mathfrak{Q}}	\newcommand{\mfR}{\mathfrak{R}}
\newcommand{\mfS}{\mathfrak{S}}	\newcommand{\mfT}{\mathfrak{T}}
\newcommand{\mfU}{\mathfrak{U}}	\newcommand{\mfV}{\mathfrak{V}}
\newcommand{\mfW}{\mathfrak{W}}	\newcommand{\mfX}{\mathfrak{X}}
\newcommand{\mfY}{\mathfrak{Y}}	\newcommand{\mfZ}{\mathfrak{Z}}
%Small Letters
\newcommand{\mfa}{\mathfrak{a}}	\newcommand{\mfb}{\mathfrak{b}}
\newcommand{\mfc}{\mathfrak{c}}	\newcommand{\mfd}{\mathfrak{d}}
\newcommand{\mfe}{\mathfrak{e}}	\newcommand{\mff}{\mathfrak{f}}
\newcommand{\mfg}{\mathfrak{g}}	\newcommand{\mfh}{\mathfrak{h}}
\newcommand{\mfi}{\mathfrak{i}}	\newcommand{\mfj}{\mathfrak{j}}
\newcommand{\mfk}{\mathfrak{k}}	\newcommand{\mfl}{\mathfrak{l}}
\newcommand{\mfm}{\mathfrak{m}}	\newcommand{\mfn}{\mathfrak{n}}
\newcommand{\mfo}{\mathfrak{o}}	\newcommand{\mfp}{\mathfrak{p}}
\newcommand{\mfq}{\mathfrak{q}}	\newcommand{\mfr}{\mathfrak{r}}
\newcommand{\mfs}{\mathfrak{s}}	\newcommand{\mft}{\mathfrak{t}}
\newcommand{\mfu}{\mathfrak{u}}	\newcommand{\mfv}{\mathfrak{v}}
\newcommand{\mfw}{\mathfrak{w}}	\newcommand{\mfx}{\mathfrak{x}}
\newcommand{\mfy}{\mathfrak{y}}	\newcommand{\mfz}{\mathfrak{z}}

%---------------------------------------
% Bar
%---------------------------------------

%Captital Letters
\newcommand{\ovA}{\overline{A}}	\newcommand{\ovB}{\overline{B}}
\newcommand{\ovC}{\overline{C}}	\newcommand{\ovD}{\overline{D}}
\newcommand{\ovE}{\overline{E}}	\newcommand{\ovF}{\overline{F}}
\newcommand{\ovG}{\overline{G}}	\newcommand{\ovH}{\overline{H}}
\newcommand{\ovI}{\overline{I}}	\newcommand{\ovJ}{\overline{J}}
\newcommand{\ovK}{\overline{K}}	\newcommand{\ovL}{\overline{L}}
\newcommand{\ovM}{\overline{M}}	\newcommand{\ovN}{\overline{N}}
\newcommand{\ovO}{\overline{O}}	\newcommand{\ovP}{\overline{P}}
\newcommand{\ovQ}{\overline{Q}}	\newcommand{\ovR}{\overline{R}}
\newcommand{\ovS}{\overline{S}}	\newcommand{\ovT}{\overline{T}}
\newcommand{\ovU}{\overline{U}}	\newcommand{\ovV}{\overline{V}}
\newcommand{\ovW}{\overline{W}}	\newcommand{\ovX}{\overline{X}}
\newcommand{\ovY}{\overline{Y}}	\newcommand{\ovZ}{\overline{Z}}
%Small Letters
\newcommand{\ova}{\overline{a}}	\newcommand{\ovb}{\overline{b}}
\newcommand{\ovc}{\overline{c}}	\newcommand{\ovd}{\overline{d}}
\newcommand{\ove}{\overline{e}}	\newcommand{\ovf}{\overline{f}}
\newcommand{\ovg}{\overline{g}}	\newcommand{\ovh}{\overline{h}}
\newcommand{\ovi}{\overline{i}}	\newcommand{\ovj}{\overline{j}}
\newcommand{\ovk}{\overline{k}}	\newcommand{\ovl}{\overline{l}}
\newcommand{\ovm}{\overline{m}}	\newcommand{\ovn}{\overline{n}}
\newcommand{\ovo}{\overline{o}}	\newcommand{\ovp}{\overline{p}}
\newcommand{\ovq}{\overline{q}}	\newcommand{\ovr}{\overline{r}}
\newcommand{\ovs}{\overline{s}}	\newcommand{\ovt}{\overline{t}}
\newcommand{\ovu}{\overline{u}}	\newcommand{\ovv}{\overline{v}}
\newcommand{\ovw}{\overline{w}}	\newcommand{\ovx}{\overline{x}}
\newcommand{\ovy}{\overline{y}}	\newcommand{\ovz}{\overline{z}}

%---------------------------------------
% Tilde
%---------------------------------------

%Captital Letters
\newcommand{\tdA}{\tilde{A}}	\newcommand{\tdB}{\tilde{B}}
\newcommand{\tdC}{\tilde{C}}	\newcommand{\tdD}{\tilde{D}}
\newcommand{\tdE}{\tilde{E}}	\newcommand{\tdF}{\tilde{F}}
\newcommand{\tdG}{\tilde{G}}	\newcommand{\tdH}{\tilde{H}}
\newcommand{\tdI}{\tilde{I}}	\newcommand{\tdJ}{\tilde{J}}
\newcommand{\tdK}{\tilde{K}}	\newcommand{\tdL}{\tilde{L}}
\newcommand{\tdM}{\tilde{M}}	\newcommand{\tdN}{\tilde{N}}
\newcommand{\tdO}{\tilde{O}}	\newcommand{\tdP}{\tilde{P}}
\newcommand{\tdQ}{\tilde{Q}}	\newcommand{\tdR}{\tilde{R}}
\newcommand{\tdS}{\tilde{S}}	\newcommand{\tdT}{\tilde{T}}
\newcommand{\tdU}{\tilde{U}}	\newcommand{\tdV}{\tilde{V}}
\newcommand{\tdW}{\tilde{W}}	\newcommand{\tdX}{\tilde{X}}
\newcommand{\tdY}{\tilde{Y}}	\newcommand{\tdZ}{\tilde{Z}}
%Small Letters
\newcommand{\tda}{\tilde{a}}	\newcommand{\tdb}{\tilde{b}}
\newcommand{\tdc}{\tilde{c}}	\newcommand{\tdd}{\tilde{d}}
\newcommand{\tde}{\tilde{e}}	\newcommand{\tdf}{\tilde{f}}
\newcommand{\tdg}{\tilde{g}}	\newcommand{\tdh}{\tilde{h}}
\newcommand{\tdi}{\tilde{i}}	\newcommand{\tdj}{\tilde{j}}
\newcommand{\tdk}{\tilde{k}}	\newcommand{\tdl}{\tilde{l}}
\newcommand{\tdm}{\tilde{m}}	\newcommand{\tdn}{\tilde{n}}
\newcommand{\tdo}{\tilde{o}}	\newcommand{\tdp}{\tilde{p}}
\newcommand{\tdq}{\tilde{q}}	\newcommand{\tdr}{\tilde{r}}
\newcommand{\tds}{\tilde{s}}	\newcommand{\tdt}{\tilde{t}}
\newcommand{\tdu}{\tilde{u}}	\newcommand{\tdv}{\tilde{v}}
\newcommand{\tdw}{\tilde{w}}	\newcommand{\tdx}{\tilde{x}}
\newcommand{\tdy}{\tilde{y}}	\newcommand{\tdz}{\tilde{z}}

%---------------------------------------
% Vec
%---------------------------------------

%Captital Letters
\newcommand{\vcA}{\vec{A}}	\newcommand{\vcB}{\vec{B}}
\newcommand{\vcC}{\vec{C}}	\newcommand{\vcD}{\vec{D}}
\newcommand{\vcE}{\vec{E}}	\newcommand{\vcF}{\vec{F}}
\newcommand{\vcG}{\vec{G}}	\newcommand{\vcH}{\vec{H}}
\newcommand{\vcI}{\vec{I}}	\newcommand{\vcJ}{\vec{J}}
\newcommand{\vcK}{\vec{K}}	\newcommand{\vcL}{\vec{L}}
\newcommand{\vcM}{\vec{M}}	\newcommand{\vcN}{\vec{N}}
\newcommand{\vcO}{\vec{O}}	\newcommand{\vcP}{\vec{P}}
\newcommand{\vcQ}{\vec{Q}}	\newcommand{\vcR}{\vec{R}}
\newcommand{\vcS}{\vec{S}}	\newcommand{\vcT}{\vec{T}}
\newcommand{\vcU}{\vec{U}}	\newcommand{\vcV}{\vec{V}}
\newcommand{\vcW}{\vec{W}}	\newcommand{\vcX}{\vec{X}}
\newcommand{\vcY}{\vec{Y}}	\newcommand{\vcZ}{\vec{Z}}
%Small Letters
\newcommand{\vca}{\vec{a}}	\newcommand{\vcb}{\vec{b}}
\newcommand{\vcc}{\vec{c}}	\newcommand{\vcd}{\vec{d}}
\newcommand{\vce}{\vec{e}}	\newcommand{\vcf}{\vec{f}}
\newcommand{\vcg}{\vec{g}}	\newcommand{\vch}{\vec{h}}
\newcommand{\vci}{\vec{i}}	\newcommand{\vcj}{\vec{j}}
\newcommand{\vck}{\vec{k}}	\newcommand{\vcl}{\vec{l}}
\newcommand{\vcm}{\vec{m}}	\newcommand{\vcn}{\vec{n}}
\newcommand{\vco}{\vec{o}}	\newcommand{\vcp}{\vec{p}}
\newcommand{\vcq}{\vec{q}}	\newcommand{\vcr}{\vec{r}}
\newcommand{\vcs}{\vec{s}}	\newcommand{\vct}{\vec{t}}
\newcommand{\vcu}{\vec{u}}	\newcommand{\vcv}{\vec{v}}
%\newcommand{\vcw}{\vec{w}}	\newcommand{\vcx}{\vec{x}}
\newcommand{\vcy}{\vec{y}}	\newcommand{\vcz}{\vec{z}}

%---------------------------------------
% Greek Letters:-
%---------------------------------------
\newcommand{\eps}{\epsilon}
\newcommand{\veps}{\varepsilon}
\newcommand{\lm}{\lambda}
\newcommand{\Lm}{\Lambda}
\newcommand{\gm}{\gamma}
\newcommand{\Gm}{\Gamma}
\newcommand{\vph}{\varphi}
\newcommand{\ph}{\phi}
\newcommand{\om}{\omega}
\newcommand{\Om}{\Omega}
\newcommand{\sg}{\sigma}
\newcommand{\Sg}{\Sigma}

\newcommand{\Qed}{\begin{flushright}\qed\end{flushright}}
\newcommand{\parinn}{\setlength{\parindent}{1cm}}
\newcommand{\parinf}{\setlength{\parindent}{0cm}}
\newcommand{\del}[2]{\frac{\partial #1}{\partial #2}}
\newcommand{\Del}[3]{\frac{\partial^{#1} #2}{\partial^{#1} #3}}
\newcommand{\deld}[2]{\dfrac{\partial #1}{\partial #2}}
\newcommand{\Deld}[3]{\dfrac{\partial^{#1} #2}{\partial^{#1} #3}}
\newcommand{\uin}{\mathbin{\rotatebox[origin=c]{90}{$\in$}}}
\newcommand{\usubset}{\mathbin{\rotatebox[origin=c]{90}{$\subset$}}}
\newcommand{\lt}{\left}
\newcommand{\rt}{\right}
\newcommand{\exs}{\exists}
\newcommand{\st}{\strut}
\newcommand{\dps}[1]{\displaystyle{#1}}
\newcommand{\la}{\langle}
\newcommand{\ra}{\rangle}
\newcommand{\cls}[1]{\textsc{#1}}
\newcommand{\prb}[1]{\textsc{#1}}
\newcommand{\comb}[2]{\left(\begin{matrix}
		#1\\ #2
\end{matrix}\right)}
%\newcommand[2]{\quotient}{\faktor{#1}{#2}}
\newcommand\quotient[2]{
	\mathchoice
	{% \displaystyle
		\text{\raise1ex\hbox{$#1$}\Big/\lower1ex\hbox{$#2$}}%
	}
	{% \textstyle
		#1\,/\,#2
	}
	{% \scriptstyle
		#1\,/\,#2
	}
	{% \scriptscriptstyle  
		#1\,/\,#2
	}
}

\newcommand{\tensor}{\otimes}
\newcommand{\xor}{\oplus}

\newcommand{\sol}[1]{\begin{solution}#1\end{solution}}
\newcommand{\solve}[1]{\setlength{\parindent}{0cm}\textbf{\textit{Solution: }}\setlength{\parindent}{1cm}#1 \hfill $\blacksquare$}
\newcommand{\mat}[1]{\left[\begin{matrix}#1\end{matrix}\right]}
\newcommand{\matr}[1]{\begin{matrix}#1\end{matrix}}
\newcommand{\matp}[1]{\lt(\begin{matrix}#1\end{matrix}\rt)}
\newcommand{\detmat}[1]{\lt|\begin{matrix}#1\end{matrix}\rt|}
\newcommand\numberthis{\addtocounter{equation}{1}\tag{\theequation}}
\newcommand{\handout}[3]{
	\noindent
	\begin{center}
		\framebox{
			\vbox{
				\hbox to 6.5in { {\bf Complexity Theory I } \hfill Jan -- May, 2023 }
				\vspace{4mm}
				\hbox to 6.5in { {\Large \hfill #1  \hfill} }
				\vspace{2mm}
				\hbox to 6.5in { {\em #2 \hfill #3} }
			}
		}
	\end{center}
	\vspace*{4mm}
}

\newcommand{\lecture}[3]{\handout{Lecture #1}{Lecturer: #2}{Scribe:	#3}}

\let\marvosymLightning\Lightning
\newcommand{\ctr}{\text{\marvosymLightning}\hspace{0.5ex}} % Requires marvosym package

\newcommand{\ov}[1]{\overline{#1}}
\newcommand{\thmref}[1]{\hyperref[th:#1]{Theorem \ref{th:#1}}}
\newcommand{\propref}[1]{\hyperref[th:#1]{Proposition \ref{th:#1}}}
\newcommand{\lmref}[1]{\hyperref[th:#1]{Lemma \ref{th:#1}}}
\newcommand{\corref}[1]{\hyperref[th:#1]{Corollary \ref{th:#1}}}

\newcommand{\thrmref}[1]{\hyperref[#1]{Theorem \ref{#1}}}
\newcommand{\propnref}[1]{\hyperref[#1]{Proposition \ref{#1}}}
\newcommand{\lemref}[1]{\hyperref[#1]{Lemma \ref{#1}}}
\newcommand{\corrref}[1]{\hyperref[#1]{Corollary \ref{#1}}}

\DeclareMathOperator{\enc}{Enc}
\DeclareMathOperator{\res}{Res}
\DeclareMathOperator{\spec}{Spec}
\DeclareMathOperator{\cov}{Cov}
\DeclareMathOperator{\Var}{Var}
\DeclareMathOperator{\Rank}{rank}
\newcommand{\Tfae}{The following are equivalent:}
\newcommand{\tfae}{the following are equivalent:}
\newcommand{\sparsity}{\textit{sparsity}}

\newcommand{\uddots}{\reflectbox{$\ddots$}} 

\newenvironment{claimwidth}{\begin{center}\begin{adjustwidth}{0.05\textwidth}{0.05\textwidth}}{\end{adjustwidth}\end{center}}

\setlength{\parindent}{0pt}

%%%%%%%%%%%%%%%%%%%%%%%%%%%%%%%%%%%%%%%%%%%%%%%%%%%%%%%%%%%%%%%%%%%%%%%%%%%%%%%%%%%%%%%%%%%%%%%%%%%%%%%%%%%%%%%%%%%%%%%%%%%%%%%%%%%%%%%%

\begin{document}
	
	%%%%%%%%%%%%%%%%%%%%%%%%%%%%%%%%%%%%%%%%%%%%%%%%%%%%%%%%%%%%%%%%%%%%%%%%%%%%%%%%%%%%%%%%%%%%%%%%%%%%%%%%%%%%%%%%%%%%%%%%%%%%%%%%%%%%%%%%
	
	\textsf{\noindent \large\textbf{Soham Chatterjee} \hfill \textbf{Assignment - 1}\\
		Email: \href{soham.chatterjee@tifr.res.in}{soham.chatterjee@tifr.res.in} \hfill Dept: STCS, TIFR\\
		\normalsize Course: Algebra, Number Theory and Computation \hfill Date: \today}
	
%%%%%%%%%%%%%%%%%%%%%%%%%%%%%%%%%%%%%%%%%%%%%%%%%%%%%%%%%%%%%%%%%%%%%%%%%%%%%%%%%%%%%%%%%%%%%%%%%%%%%%%%%%%%%%%%%%%%%%%%%%%%%%%%%%%%%%%%
% Problem 1
%%%%%%%%%%%%%%%%%%%%%%%%%%%%%%%%%%%%%%%%%%%%%%%%%%%%%%%%%%%%%%%%%%%%%%%%%%%%%%%%%%%%%%%%%%%%%%%%%%%%%%%%%%%%%%%%%%%%%%%%%%%%%%%%%%%%%%%%
	
\begin{problem}{%problem statement
		\hfill 5 Points
	}{p1% problem reference text
}
Let $\bbF$ be a field of characteristic equal to $p$. Then, show that over the polynomial ring $\bbF[x,y]$, $(x+y)^p=x^p+y^p$
\end{problem}
\solve{	
\begin{lemma}
	$p\mid \binom{p}{k}\iff 0<k<p$
\end{lemma}
\begin{proof}
	Let $0<k<p$. Then $\binom{p}{k}=\frac{p!}{k!(p-k)!}$. As $0<k<p$, $0<p-k<p$. Since $p$ is a prime none of numbers from $0$ to $\max\{k,p-k\}$ divides $p$. Therefore $p$ never gets canceled out in $\binom{p}{k}$. Hence $p\mid\binom{p}{k}$. 
	
	Now suppose $p\mid \binom{p}{k}$. Now $$\binom{p}{k}=\frac{p!}{k!}{(p-k)!}=\frac{\prod\limits_{i=1}^k(p-k+i)}{k!}=\frac{\prod\limits_{i=1}^{p-k}(k+i)}{(p-k)!}$$Now the highest power of $p$ that divides $\prod\limits_{i=1}^k(p-k+i)$ and $\prod\limits_{i=1}^{p-k}(k+i)$ is 1. Therefore $p\nmid k!$ and $p\nmid (p-k)!$. Therefore $k<p$ and $p-k<p$. Hence we have $0<k<p$.
\end{proof}

So now using the lemma we have $(x+y)^p=x^p+y^p+\sum\limits_{i=1}^{p-1}\binom{p}{i}x^{p-i}y^i=x^p+y^p+p\cdot C$ where $p\cdot  C=\sum\limits_{i=1}^{p-1}\binom{p}{i}x^{p-i}y^i$. Since the characteristic of the field is $p$ we have $p\cdot C=0$. Hence $(x+y)^p=x^p+y^p$.
}
%%%%%%%%%%%%%%%%%%%%%%%%%%%%%%%%%%%%%%%%%%%%%%%%%%%%%%%%%%%%%%%%%%%%%%%%%%%%%%%%%%%%%%%%%%%%%%%%%%%%%%%%%%%%%%%%%%%%%%%%%%%%%%%%%%%%%%%%
% Problem 2
%%%%%%%%%%%%%%%%%%%%%%%%%%%%%%%%%%%%%%%%%%%%%%%%%%%%%%%%%%%%%%%%%%%%%%%%%%%%%%%%%%%%%%%%%%%%%%%%%%%%%%%%%%%%%%%%%%%%%%%%%%%%%%%%%%%%%%%%
%\newpage

\begin{problem}{%problem statement
	\hfill 20 Points
	}{p2% problem reference text
	}
Let $q$ be a prime power and let $k>0$ be a natural number. The polynomial $\trac(x)$ is defined as $$\trac(x)=x+x^q+x^{q^2}+\cdots +x^{q^{k-1}}$$
\begin{enumerate}[label=(\alph*)]
	\item \textbf{(5 points)} Show that for every $\alpha\in\bbF_{q^k}$, $\trac(\alpha)\in\bbF_q$.
	\item \textbf{(5 points)} Show that when viewed as a map from the vector space $\bbF_{q^k}$ to $\bbF_q$. $\trac$ is $\bbF_q$-linear.
	\item \textbf{(10 points)} Using the properties of $\trac$, conclude that for \textit{every}  $\bbF_q$ linear map $L$ from $\bbF_{q^k}$ to $\bbF_q$, there is an $\alpha_L\in\bbF_{q^k}$ such that  for all $\beta\in\bbF_{q^k}$, $L(\beta)=\trac(\alpha_L\cdot \beta)$.
\end{enumerate}
\end{problem}
\solve{	
\begin{enumerate}[label=(\alph*)]
 \item The Frobenius  map $\vph:\bbF_{q^k}\to\bbF_{q^k}$, where $\vph(x)=x^q$ is an automorphism and it is $\bbF_q$-linear and the only elements for which $\vph(x)=x$ are the elements of $\bbF_q$.
 \begin{lemma}
	The maps $\trac$ and $\vph$ commutes over $\bbF_{q^k}$.
\end{lemma}
\begin{proof}
	\begin{align*}
		\trac\circ\vph(x)=\trac(x^q)&=x^q+(x^q)^{q}+(x^q)^{q^2}+\cdots+(x^q)^{q^{k-1}}\\
		& = x^q+\lt(x^{q^2}\rt)^q+\lt(x^{q^3}\rt)^q+\cdots +\lt(x^{q^{k-1}}\rt)^q\\
		& = \lt(x+x^q+x^{q^2}+\cdots +x^{q^{k-1}}\rt)^q=\lt(\trac(x)\rt)^q=\vph\circ\trac(x)
	\end{align*}
Hence two maps commutes.
\end{proof}

Now notice that for any $\alpha\in\bbF_{q^k}$ $$\trac(\alpha)^q=\trac(\alpha^q)=\sum\limits_{i=0}^{k-1}(\alpha^q)^{q^i}=\sum\limits_{i=0}^{k-1}\alpha^{q^{i+1}}=\sum\limits_{i=1}^{k}\alpha^{q^i}=\sum\limits_{i=0}^{k-1}\alpha^{q^i}=\trac(\alpha)$$ 
The third from the last inequality is true is because $\alpha^{q^k}=\alpha$ for all $\alpha\in\bbF_{q^k}$. Hence for all $\alpha\in\range(\trac)$. $\vph(\alpha)=\alpha$. Now the only elements which remains same under the Frobenius map are the elements of $\bbF_q$. Therefore $\range(\trac)\subseteq \bbF_q$. So the for all $\alpha\in\bbF_{q^k}$, $\trac(\alpha)\in\bbF$.
\item Suppose $a,b\in\bbF_{q^k}$ and $\alpha\in \bbF_q$. Then we have \begin{multline*}
	\trac(\alpha\cdot a+b)=\sum\limits_{i=0}^{k-1}(\alpha\cdot a+b)^{q^i}=\sum\limits_{i=0}^{k-1}(\alpha\cdot a )^{q^i}+b^{q^i}=\sum\limits_{i=0}^{k-1} \lt(\alpha\cdot a^{q^i}+b^{q^i}\rt)\\=\alpha\lt(\sum\limits_{i=0}^{k-1}a^{q_i}\rt)+\lt(\sum\limits_{i=0}^{k-1}b^{q_i}\rt)=\alpha\trac(a)+\trac(b)
\end{multline*}
Therefore $\trac(x)$ is a $\bbF_q$-linear map.
\item Let $S=\{ L:\bbF_{q^k}\to \bbF_q\mid L\text{ is } \bbF_{q}-\text{linear}\}$. As $\bbF_{q^k}$ forms a vector space over $\bbF_q$ the set $S$ also forms a vector space over $\bbF_q$ and actually called the dual of $\bbF_q$. Since dimension of the vector space $\bbF_{q^k}$ over $\bbF_q$ is $k$ we have the dimension of $S$ over $\bbF_q$ is also $k$. \parinn

Now since dimension of $\bbF_{q^k}$ is $k$ over $\bbF_q$ there exists $k $ elements of $\bbF_{q^k}$, $\{\beta_1,\dots, \beta_k\}\subseteq \bbF_{q^k}$ such that they form a basis of $\bbF_{q^k}$ over $\bbF_{q}$. Then consider the collection of maps $\{\trac(\beta_i\cdot x)\mid i\in[k]\}$. We will show that these maps are linearly independent. And since they are $\bbF_q$-linear they are in $S$. Since they form a $k$ size collection of linearly independent $\bbF_q$-linear maps they span the set $S$.
\begin{lemma}
	$\{\trac(\beta_i\cdot x)\mid i\in[k]\}$ are linearly independent.
\end{lemma}
\begin{proof}
	Suppose they are linearly dependent. Let there exists $\gm_i\in\bbF_q$ for all $i\in[k]$ not all zero such that $\sum\limits_{i=1}^k\gm_i\trac(\beta_i\cdot x)\equiv 0$. Then we have $$\sum\limits_{i=1}^k\gm_i\trac(\beta_i\cdot x)=\sum_{i=1}^k\trac((\gm_i\beta_i)\cdot x)=\trac\lt(\lt(\sum_{i=1}^k\gm_i\beta_i\rt)x\rt)$$Therefore $\trac\lt(\lt(\sum\limits_{i=1}^k\gm_i\beta_i\rt)\alpha\rt)=0$ for all $\alpha\in\bbF_{q^k}$. Since $\beta_i's$ are linearly independent $\sum\limits_{i=1}^k\gm_i\beta_i\neq 0$. Let $\dl\coloneqq \sum\limits_{i=1}^k\gm_i\beta_i$. Then $\trac(\dl\cdot \alpha)= 0$ for all $\alpha\in\bbF_{q^k}$. But that means every element of $\bbF_{q^k}$ is a root of $\trac(x)$ but that is not possible since $\deg\trac(x)=q^{k-1}$. Hence contradiction. Therefore $\{\trac(\beta_i\cdot x)\mid i\in[k]\}$ are linearly independent.
\end{proof}

Therefore the set $\{\trac(\beta_i\cdot x)\mid i\in[k]\}$ spans the set of $\bbF_q$-linear maps over $\bbF_{q^k}$. Now let $L\in S$. Then there exists $\gm_i\in\bbF$ for all $i\in[k]$ such that $L=\sum\limits_{i=1}^k \gm_i\trac(\beta_i\cdot x)=\sum\limits_{i=1}^k \trac(\gm_i\beta_i\cdot x)=\trac\lt(\lt(\sum\limits_{i=1}^k \gm_i\beta_i\rt)x\rt)=L(\alpha_L\cdot x)$ where $\alpha_L=\sum\limits_{i=1}^k \gm_i\beta_i$.
\end{enumerate}
}
%\newpage

%%%%%%%%%%%%%%%%%%%%%%%%%%%%%%%%%%%%%%%%%%%%%%%%%%%%%%%%%%%%%%%%%%%%%%%%%%%%%%%%%%%%%%%%%%%%%%%%%%%%%%%%%%%%%%%%%%%%%%%%%%%%%%%%%%%%%%%%
% Problem 3
%%%%%%%%%%%%%%%%%%%%%%%%%%%%%%%%%%%%%%%%%%%%%%%%%%%%%%%%%%%%%%%%%%%%%%%%%%%%%%%%%%%%%%%%%%%%%%%%%%%%%%%%%%%%%%%%%%%%%%%%%%%%%%%%%%%%%%%%

\begin{problem}{%problem statement
		\hfill 10 Points
	}{p3% problem reference text
	}
Let $q$ be a prime power, $k>0$ be a natural number and let $S\subset \bbF_{q^k}$ be a subspace of $\bbF_{q^k}$ of dimension $s$, when we view $\bbF_{q^k}$ as a $k$ dimensional linear space over $\bbF_q$. Consider the polynomial $P_S(x)$ defined as $$P_S(x)=\prod_{\alpha\in S}(x-\alpha )$$Show that there exist $\beta_1,\beta_2,\dots, \beta_s\in\bbF_{q^k}$ such that $$P_S(x)=x_{q^s}+\beta_1x^{q^{s-1}}+\beta_2x^{q^{s-2}}+\cdots +\beta_sx$$

\end{problem}
\solve{
The dimension of $S$ over $\bbF_q$ in $\bbF_{q^k}$ is $s$. Therefore there exists $\gm_1,\dots, \gm_s$ which forms a basis of $S$ over $\bbF_q$. Denote the followings:
$$M(x)\coloneqq\mat{
	\gm_1 & \gm_1^q & \gm_1^{q^2} & \cdots & \gm_1^{q^s}\\
	\gm_2 & \gm_2^q & \gm_2^{q^2} & \cdots & \gm_2^{q^s}\\
	\vdots& \vdots  & \vdots   	  & \ddots & \vdots 	\\
	\gm_s & \gm_s^q & \gm_s^{q^2} & \cdots & \gm_s^{q^s}\\
	x & x^q & x^{q^2} & \cdots & x^{q^s}}\qquad  \dl\coloneqq\det\matp{	\gm_1 & \gm_1^q & \gm_1^{q^2} & \cdots & \gm_1^{q^{s-1}}\\
	\gm_2 & \gm_2^q & \gm_2^{q^2} & \cdots & \gm_2^{q^{s-1}}\\
	\vdots& \vdots  & \vdots   	  & \ddots & \vdots 	\\
	\gm_s & \gm_s^q & \gm_s^{q^2} & \cdots & \gm_s^{q^{s-1}} }$$
Then consider the polynomial $f(x)=\det(M(x))$. Clearly we have $$f(x)=\dl x^{q^s}+f_1x^{q^{s-1}}+f_2x^{q^{s-2}}+\cdots +f_s x$$ for some $f_i\in\bbF_{q^k}$ for all $i\in[s]$.  Now if $\dl=0$ then  matrix $\mat{	\gm_1 & \gm_1^q & \gm_1^{q^2} & \cdots & \gm_1^{q^{s-1}}\\
	\gm_2 & \gm_2^q & \gm_2^{q^2} & \cdots & \gm_2^{q^{s-1}}\\
	\vdots& \vdots  & \vdots   	  & \ddots & \vdots 	\\
	\gm_s & \gm_s^q & \gm_s^{q^2} & \cdots & \gm_s^{q^{s-1}}}$ is not full rank i.e. the rows of the matrix are not linearly independent. Hence $\gm_i$'s are not linearly independent which is not possible. Therefore $\dl\neq 0$. Hence the polynomial $f(x)$ has degree $x^{q^s}$. Now consider the modified polynomial $\tdf(x)=x^{q^s}+\sum\limits_{i=1}^s\tdf_ix^{q^i}$ where $\tdf_i=\frac{f_i}{\dl}$
\begin{lemma}
$\rank(M(\alpha))<n\iff \alpha\in S$
\end{lemma}
\begin{proof}
	Let $\alpha\in S$. Then there exists $c_i\in\bbF_q$ such that $\alpha=\sum\limits_{i=1}^k c_i\beta_i$. Then $\alpha^{q^j}=\sum\limits_{i=1}^k c_j\beta_i^{q^j}$ for all $j\in\bbZ_{\geq 0}$. There for the rows of $M(\alpha)$ are not linearly independent. Hence $\rank(M(x))<n$.
	
	Now suppose $\rank(M(\alpha))<n$ for some $\alpha\in \bbF_{q^k}$. Then the rows of $M(\alpha)$ are not linearly independent. Hence there exists $c_i\in \bbF_q$ for all $i\in[k]$ such that $\sum\limits_{i=1}^kc_i\gm_i=\alpha$. Hence $\alpha\in S$.
\end{proof}

Hence with the lemma we get that $$\det(M(\alpha))=0\iff \rank(M(\alpha))<n\iff \alpha\in S$$Hence the roots of $\tdf$ are all the elements of $S$. 

Now both $\tdf$ and $P_S$ are nonzero, monic, has degree $x^{q^s}$ and they have the same set of roots. Therefore $\tdf(x)=P_S(x)$. Therefore we can express $P_S(x)$ as $$P_S(x)=x^{q^s}+\tdf_1x^{q^[s-1]}+\tdf_2x^{q^{s-2}}+\cdots +\tdf_sx$$

}
\newpage

%%%%%%%%%%%%%%%%%%%%%%%%%%%%%%%%%%%%%%%%%%%%%%%%%%%%%%%%%%%%%%%%%%%%%%%%%%%%%%%%%%%%%%%%%%%%%%%%%%%%%%%%%%%%%%%%%%%%%%%%%%%%%%%%%%%%%%%%
% Problem 4
%%%%%%%%%%%%%%%%%%%%%%%%%%%%%%%%%%%%%%%%%%%%%%%%%%%%%%%%%%%%%%%%%%%%%%%%%%%%%%%%%%%%%%%%%%%%%%%%%%%%%%%%%%%%%%%%%%%%%%%%%%%%%%%%%%%%%%%%

\begin{problem}{%problem statement
\hfill 20 Points
	}{p4% problem reference text
	}
Let $\alpha_1,\alpha_2,\dots, \alpha_n$ distinct elements of some field $\bbF$. And, let $V(\alpha_1,\alpha_2,\dots, \alpha_n)$ be the $n\times n$ matrix whose $(i,j)$ entry equals $\alpha_i^{j-1}$.
\begin{enumerate}[label=(\alph*)]
\item \textbf{(5 points)} Show that $V$ has rank equal to $n$.
\item \textbf{(10 points)} Show that the determinant of $V$ equals $\prod\limits_{i<j}(\alpha_j-\alpha_i)$
\item \textbf{(5 points)} For every $\beta_1,\beta_2\dots, \beta_n\in \bbF$, show that there is a unique polynomial $f\in\bbF[x]$ of degree at most $n-1$ such that for every $i\in\{1,2,\dots, n\}$, $f(\alpha_i)=\beta_i$.
\end{enumerate}
\end{problem}
\solve{
\begin{enumerate}[label=(\alph*)]
\item Suppose the rank of $V$ is less than $n$. Then the columns of $V$ are linearly dependent. Then there exists $\beta_j\in\bbF$ for all $j\in[n]$ not all zero such that for all $i\in[n]$ $\sum\limits_{j=1}^n \beta_j\cdot \alpha_i^{j-1}=0$. Then consider the polynomial $f\in\bbF[x]$ where $f(x)=\sum\limits_{i=1}^n\beta_i x^{i-1}$. Then we conclude that $f(\alpha_i)=0$ for all $i\in[n]$. Therefore roots of $f$ are $\alpha_1,\alpha_2,\dots, \alpha_n$. But $\deg f\leq n-1$. Hence $f$ cannot have more than $n-1$ roots. Hence contradiction. Therefore rank of $V$ is $n$.

\item We will prove this using induction on $n$. For base case $n=1$. $V(\alpha)$ contains only one element $1$. Hence this is true. Suppose this is true for $n-1$. Now 
\begin{align*}
	\det\matp{	
	1&\alpha_1 & \alpha_1^2 & \alpha_1^{3} & \cdots & \alpha_1^{{n-1}}\\
	1&\alpha_2 & \alpha_2^2 & \alpha_2^{3} & \cdots & \alpha_2^{{n-1}}\\
	\vdots& \vdots  & \vdots   	  & \ddots & \vdots 	\\
	1&\alpha_n & \alpha_n^2 & \alpha_n^{3} & \cdots & \alpha_n^{{n-1}}}& =\det\matp{	
	1& 0& 0 & 0 & \cdots & 0\\
	1&\alpha_2-\alpha_1 & \alpha_2^2-\alpha_1\alpha_2  & \alpha_2^{3}-\alpha_1\alpha_2^{2} & \cdots & \alpha_2^{{n-1}}-\alpha_1\alpha_2^{{n-2}}\\
	\vdots& \vdots  & \vdots   	  & \ddots & \vdots 	\\
	1&\alpha_n-\alpha_1 & \alpha_n^2-\alpha_1\alpha_n & \alpha_n^{3}-\alpha_1\alpha_n^{2} & \cdots & \alpha_n^{{n-1}}-\alpha_1\alpha_n^{{n-2}}}\\[2mm]
	& = \prod_{i=2}^n(\alpha_i-\alpha_1)\det\matp{	
		1 & \alpha_2 & \alpha_2^{2} & \cdots & \alpha_2^{{n-2}}\\
		\vdots& \vdots  & \vdots   	  & \ddots & \vdots 	\\
		1 & \alpha_n & \alpha_n^{2} & \cdots &\alpha_n^{{n-2}}}
\end{align*}By inductive hypothesis we have $$\det(V(\alpha_2,\dots, \alpha_n))=
\det\matp{	
1 & \alpha_2 & \alpha_2^{2} & \cdots & \alpha_2^{{n-2}}\\
\vdots& \vdots  & \vdots   	  & \ddots & \vdots 	\\
1 & \alpha_n & \alpha_n^{2} & \cdots &\alpha_n^{{n-2}}}=\prod_{2\leq i<j\leq n}(\alpha_j-\alpha_i)$$Therefore $$\det(V(\alpha_1,\dots,\alpha_n))=\prod\limits_{1\leq i<j\leq n}(\alpha_j-\alpha_i)$$ Therefore by mathematical induction this is true for all $n\in\bbN$.
\item Consider the vector $\hat{f}=\mat{f_0 & f_1 & \cdots & f_{n-1}}^T$ where $f_i$'s denote the coefficients of the polynomial $f(x)=\sum\limits_{i=0}^n f_ix^i$ for which $f(\alpha_i)=\beta_i$ and the vector $b=\mat{\beta_1 & \beta_2 & \cdots & \beta_n}^T$. Now such a polynomial $f$ exists if and only if the equation $V\hat{f}=b$ is satisfied. Since $V$ has full rank $V$ is invertible. Hence we get a $\hat{f}=V^{-1}b$. Therefore the equation has a unique solution. Hence there exists an unique polynomial $f$ such that $f(\alpha_i)=\beta_i$.
\end{enumerate}
}
%%%%%%%%%%%%%%%%%%%%%%%%%%%%%%%%%%%%%%%%%%%%%%%%%%%%%%%%%%%%%%%%%%%%%%%%%%%%%%%%%%%%%%%%%%%%%%%%%%%%%%%%%%%%%%%%%%%%%%%%%%%%%%%%%%%%%%%%
% Problem 5
%%%%%%%%%%%%%%%%%%%%%%%%%%%%%%%%%%%%%%%%%%%%%%%%%%%%%%%%%%%%%%%%%%%%%%%%%%%%%%%%%%%%%%%%%%%%%%%%%%%%%%%%%%%%%%%%%%%%%%%%%%%%%%%%%%%%%%%%

\begin{problem}{%problem statement
\hfill 20 Points	}{p5% problem reference text
	}
Let $\bbF$ be any field. $\alpha\in\bbF$ is said to be a zero (or root) of multiplicity at least $k$ of a non-zero polynomial $f(x)\in\bbF[x]$ if $f(\alpha)=\del{f}{x}(\alpha)=\cdots=\del{^{k-1}f}{x^{k-1}}(\alpha)=0$ and $\del{^kf}{x^k}(\alpha)\neq 0$.\begin{enumerate}[label=(\alph*)]
	\item (\textbf{10 points}) Show that $\alpha$ is a zero of multiplicity at least $k$ of $f$ if and only if $(x-\alpha)^k$ divides $f(x)$.
	\item(\textbf{10 points}) If $\alpha_1,\alpha_2,\dots,\alpha_t$ are distinct elements of $\bbC$, then show that $$\sum\limits_{i=1}^t(\mult(f,\alpha_i))\leq \text{Degree}(f)$$where $\mult(f,\alpha_i)$ denotes the multiplicity of $f$ at $\alpha_i$.
\end{enumerate}
\end{problem}
\solve{
\begin{enumerate}[label=(\alph*)]
\item We will denote $f^{(i)}(x)=\del{^if}{x^i}(x)$ where $f^{(0)}(x)=f(x)$. \parinn

$(\Leftarrow):$ We will prove this using induction on $k$> For base case $k=1$. Then $(x-\alpha)\mid f(x)$. Hence $\alpha$ is a root of $f$. Therefore $\alpha$ is a zero of $f$ with multiplicity at least $1$. Suppose this is true for $k-1$. Now we will show for $k$. Let $(x-\alpha)^k\mid f(x)$. Since $\alpha$ is a root of $f$ we have  $f(x)=(x-\alpha)g(x)$ for some $g(x)\in\bbF[x]$. Now $(x-\alpha)^{k-1}\mid g(x)$. Therefore by inductive hypothesis $\alpha$ is a zero of $g$ with multiplicity at least $k-1$ i.e. $g^{(i)}(\alpha)=0$ for all $i\in\{0,\dots, k-2\}$ and since $g$ is not a zero polynomial there exists $l>k-2$ such that $g^{(l)}(\alpha)\neq 0$. 
\begin{lemma}
	$f^{(i)}(x)=ig^{(i-1)}(x)+(x-\alpha)g^{(i)}(x)$
\end{lemma}
\begin{proof}
	We will prove this using induction on $i$. For base case $i=1$. Then $f^{(1)}(x)=g(x)+(x-\alpha)g^{(1)}(x)$. So base case is true. Let this is true for $i-1$. Now \begin{multline*}
		f^{(i-1)}=(i-1)g^{(i-2)}(x)+(x-\alpha)g^{(i-1)}(x)\implies\\
		f^{(i)}(x)=(i-1)g^{(i-1)}(x)+g^{(i-1)}(x)+(x-\alpha)g^{(i)}=ig^{(i-1)}(x)+(x-\alpha)g^{(i)}(x)
	\end{multline*}
	Hence by mathematical induction this is true.
\end{proof}

Therefore $f^{(i)}(\alpha)=ig^{(i-1)}(\alpha)=0$ for all $i\in[k-1]$ and $f^{(l+1)}(\alpha)=(l+1)g^{(l)}(\alpha)\neq 0$ where $l>k-2$. Therefore $f^{(i)}(\alpha)=0$ for all $i\in\{0,\dots, k-1\}$ and $f^{(l+1)}(\alpha)\neq 0$ where $l+1>k-1$. Therefore $\alpha$ is a zero of $f$ with multiplicity at least $k$.


$(\Rightarrow):$ We will do induction on $k$. For base case $k=1$. Then $f(\alpha)=0$ and $\del{f}{x}(\alpha)\neq 0$. Therefore $(x-\alpha)\mid f$. Hence the base case follows. Now suppose this is true for $k-1$. 

We will prove for $k$. Now $f^{(i)}(\alpha)=0$ for all $i\in\{0,\dots, k-1\}$ and there exists $l>k-1$ such that $f^{(l)}(\alpha)\neq 0$. Therefore $f(x)=(x-\alpha)g(x)$ for some $g\in\bbF[x]$. Then $f^{(i)}(x)=ig^{(i-1)}(x)+(x-\alpha)g^{(i)}(x)$.
Now consider the polynomial $g(x)$. We have $g^{(i)}(\alpha)=0$ for all $i\in\{0,\dots,k-2\}$ and $g^{(l-1)}(\alpha)\neq 0$. Hence $\alpha$ is a zero of $g$ with multiplicity at least $k-1$. Therefore by inductive hypothesis we have $(x-\alpha)^{k-1}\mid g(x)$.  Hence $(x-\alpha)^k\mid f(x)$. Therefore by mathematical induction this is true.
\item Since $f$ is over $\bbC$, $f$ completely splits over $\bbC$. Now for any $\alpha\in\bbC$ we have by the above part that $(x-\alpha)^{\mult(f,\alpha)}\mid f(x)$. \parinn

We will prove this by induction on $n$. For base case $n=1$ then for $\alpha_1$ we have $$(x-\alpha_1)^{\mult(f,\alpha_1)}\mid f(x)\implies (x-\alpha_1)^{\mult(f,\alpha_1)}g_1(x)=f(x)\implies \mult(f,\alpha_1)\leq \text{Degree}(f)$$So base case follows. Suppose this is true for $n-1$. We will prove for $n$ now. Now if $\mult(f,\alpha_i)=0$ for any $i\in[n]$ then $$\sum_{j=1}^n\mult(f,\alpha_j)=\sum_{j=1,j\neq i}^n\mult(f,\alpha_i)$$Therefore by inductive hypothesis this is true. So assume $\mult(f,\alpha_i)>0$ for all $i\in[n]$. Then $f(x)=(x-\alpha_1)^{\mult(f,\alpha_1)}g(x)$ for some $g\in\bbC[x]$. Hence $\deg (f)=\mult(f,\alpha_1)+\deg (g)$. Now $(x-\alpha_i)$'s are relatively coprime with each other. Therefore $(x-\alpha_i)^{\mult(f,\alpha_i)}$'s are also relatively coprime with each other. Hence $(x-\alpha_i)^{\mult(f,\alpha_i)}\mid g(x)$ for all $i\in\{2,\dots, n\}$. Now by inductive hypothesis we have $\sum\limits_{i=2}^n\mult(f,\alpha_i)\leq \text{Degree}(g)$. Therefore we have $\sum\limits_{i=1}^n\mult(f,\alpha_i)\leq \text{Degree}(f)$. Hence this is true for all $n$.
\end{enumerate}
}
\end{document}
