\documentclass[a4paper, 11pt]{article}
\usepackage{comment} % enables the use of multi-line comments (\ifx \fi) 
\usepackage{fullpage} % changes the margin
\usepackage[a4paper, total={7in, 10in}]{geometry}
\usepackage{amsmath,mathtools}
\usepackage{amssymb,amsthm}  % assumes amsmath package installed
\usepackage{float}
\usepackage{xcolor}
\usepackage{mdframed}
\usepackage[shortlabels]{enumitem}
\usepackage{indentfirst}
\usepackage{hyperref}
\hypersetup{
	colorlinks=true,
	linkcolor=blue,
	filecolor=magenta,      
	urlcolor=blue!70!red,
	pdftitle={Assignment}, %%%%%%%%%%%%%%%%   WRITE ASSIGNMENT PDF NAME  %%%%%%%%%%%%%%%%%%%%
}
\usepackage[most,many,breakable]{tcolorbox}



\definecolor{mytheorembg}{HTML}{F2F2F9}
\definecolor{mytheoremfr}{HTML}{00007B}


\tcbuselibrary{theorems,skins,hooks}
\newtcbtheorem{problem}{Problem}
{%
	enhanced,
	breakable,
	colback = mytheorembg,
	frame hidden,
	boxrule = 0sp,
	borderline west = {2pt}{0pt}{mytheoremfr},
	sharp corners,
	detach title,
	before upper = \tcbtitle\par\smallskip,
	coltitle = mytheoremfr,
	fonttitle = \bfseries\sffamily,
	description font = \mdseries,
	separator sign none,
	segmentation style={solid, mytheoremfr},
}
{p}

% To give references for any problem use like this
% suppose the problem number is p3 then 2 options either 
% \hyperref[p:p3]{<text you want to use to hyperlink> \ref{p:p3}}
%                  or directly 
%                   \ref{p:p3}



%---------------------------------------
% BlackBoard Math Fonts :-
%---------------------------------------

%Captital Letters
\newcommand{\bbA}{\mathbb{A}}	\newcommand{\bbB}{\mathbb{B}}
\newcommand{\bbC}{\mathbb{C}}	\newcommand{\bbD}{\mathbb{D}}
\newcommand{\bbE}{\mathbb{E}}	\newcommand{\bbF}{\mathbb{F}}
\newcommand{\bbG}{\mathbb{G}}	\newcommand{\bbH}{\mathbb{H}}
\newcommand{\bbI}{\mathbb{I}}	\newcommand{\bbJ}{\mathbb{J}}
\newcommand{\bbK}{\mathbb{K}}	\newcommand{\bbL}{\mathbb{L}}
\newcommand{\bbM}{\mathbb{M}}	\newcommand{\bbN}{\mathbb{N}}
\newcommand{\bbO}{\mathbb{O}}	\newcommand{\bbP}{\mathbb{P}}
\newcommand{\bbQ}{\mathbb{Q}}	\newcommand{\bbR}{\mathbb{R}}
\newcommand{\bbS}{\mathbb{S}}	\newcommand{\bbT}{\mathbb{T}}
\newcommand{\bbU}{\mathbb{U}}	\newcommand{\bbV}{\mathbb{V}}
\newcommand{\bbW}{\mathbb{W}}	\newcommand{\bbX}{\mathbb{X}}
\newcommand{\bbY}{\mathbb{Y}}	\newcommand{\bbZ}{\mathbb{Z}}

%---------------------------------------
% MathCal Fonts :-
%---------------------------------------

%Captital Letters
\newcommand{\mcA}{\mathcal{A}}	\newcommand{\mcB}{\mathcal{B}}
\newcommand{\mcC}{\mathcal{C}}	\newcommand{\mcD}{\mathcal{D}}
\newcommand{\mcE}{\mathcal{E}}	\newcommand{\mcF}{\mathcal{F}}
\newcommand{\mcG}{\mathcal{G}}	\newcommand{\mcH}{\mathcal{H}}
\newcommand{\mcI}{\mathcal{I}}	\newcommand{\mcJ}{\mathcal{J}}
\newcommand{\mcK}{\mathcal{K}}	\newcommand{\mcL}{\mathcal{L}}
\newcommand{\mcM}{\mathcal{M}}	\newcommand{\mcN}{\mathcal{N}}
\newcommand{\mcO}{\mathcal{O}}	\newcommand{\mcP}{\mathcal{P}}
\newcommand{\mcQ}{\mathcal{Q}}	\newcommand{\mcR}{\mathcal{R}}
\newcommand{\mcS}{\mathcal{S}}	\newcommand{\mcT}{\mathcal{T}}
\newcommand{\mcU}{\mathcal{U}}	\newcommand{\mcV}{\mathcal{V}}
\newcommand{\mcW}{\mathcal{W}}	\newcommand{\mcX}{\mathcal{X}}
\newcommand{\mcY}{\mathcal{Y}}	\newcommand{\mcZ}{\mathcal{Z}}



%---------------------------------------
% Bold Math Fonts :-
%---------------------------------------

%Captital Letters
\newcommand{\bmA}{\boldsymbol{A}}	\newcommand{\bmB}{\boldsymbol{B}}
\newcommand{\bmC}{\boldsymbol{C}}	\newcommand{\bmD}{\boldsymbol{D}}
\newcommand{\bmE}{\boldsymbol{E}}	\newcommand{\bmF}{\boldsymbol{F}}
\newcommand{\bmG}{\boldsymbol{G}}	\newcommand{\bmH}{\boldsymbol{H}}
\newcommand{\bmI}{\boldsymbol{I}}	\newcommand{\bmJ}{\boldsymbol{J}}
\newcommand{\bmK}{\boldsymbol{K}}	\newcommand{\bmL}{\boldsymbol{L}}
\newcommand{\bmM}{\boldsymbol{M}}	\newcommand{\bmN}{\boldsymbol{N}}
\newcommand{\bmO}{\boldsymbol{O}}	\newcommand{\bmP}{\boldsymbol{P}}
\newcommand{\bmQ}{\boldsymbol{Q}}	\newcommand{\bmR}{\boldsymbol{R}}
\newcommand{\bmS}{\boldsymbol{S}}	\newcommand{\bmT}{\boldsymbol{T}}
\newcommand{\bmU}{\boldsymbol{U}}	\newcommand{\bmV}{\boldsymbol{V}}
\newcommand{\bmW}{\boldsymbol{W}}	\newcommand{\bmX}{\boldsymbol{X}}
\newcommand{\bmY}{\boldsymbol{Y}}	\newcommand{\bmZ}{\boldsymbol{Z}}
%Small Letters
\newcommand{\bma}{\boldsymbol{a}}	\newcommand{\bmb}{\boldsymbol{b}}
\newcommand{\bmc}{\boldsymbol{c}}	\newcommand{\bmd}{\boldsymbol{d}}
\newcommand{\bme}{\boldsymbol{e}}	\newcommand{\bmf}{\boldsymbol{f}}
\newcommand{\bmg}{\boldsymbol{g}}	\newcommand{\bmh}{\boldsymbol{h}}
\newcommand{\bmi}{\boldsymbol{i}}	\newcommand{\bmj}{\boldsymbol{j}}
\newcommand{\bmk}{\boldsymbol{k}}	\newcommand{\bml}{\boldsymbol{l}}
\newcommand{\bmm}{\boldsymbol{m}}	\newcommand{\bmn}{\boldsymbol{n}}
\newcommand{\bmo}{\boldsymbol{o}}	\newcommand{\bmp}{\boldsymbol{p}}
\newcommand{\bmq}{\boldsymbol{q}}	\newcommand{\bmr}{\boldsymbol{r}}
\newcommand{\bms}{\boldsymbol{s}}	\newcommand{\bmt}{\boldsymbol{t}}
\newcommand{\bmu}{\boldsymbol{u}}	\newcommand{\bmv}{\boldsymbol{v}}
\newcommand{\bmw}{\boldsymbol{w}}	\newcommand{\bmx}{\boldsymbol{x}}
\newcommand{\bmy}{\boldsymbol{y}}	\newcommand{\bmz}{\boldsymbol{z}}


%---------------------------------------
% Scr Math Fonts :-
%---------------------------------------

\newcommand{\sA}{{\mathscr{A}}}   \newcommand{\sB}{{\mathscr{B}}}
\newcommand{\sC}{{\mathscr{C}}}   \newcommand{\sD}{{\mathscr{D}}}
\newcommand{\sE}{{\mathscr{E}}}   \newcommand{\sF}{{\mathscr{F}}}
\newcommand{\sG}{{\mathscr{G}}}   \newcommand{\sH}{{\mathscr{H}}}
\newcommand{\sI}{{\mathscr{I}}}   \newcommand{\sJ}{{\mathscr{J}}}
\newcommand{\sK}{{\mathscr{K}}}   \newcommand{\sL}{{\mathscr{L}}}
\newcommand{\sM}{{\mathscr{M}}}   \newcommand{\sN}{{\mathscr{N}}}
\newcommand{\sO}{{\mathscr{O}}}   \newcommand{\sP}{{\mathscr{P}}}
\newcommand{\sQ}{{\mathscr{Q}}}   \newcommand{\sR}{{\mathscr{R}}}
\newcommand{\sS}{{\mathscr{S}}}   \newcommand{\sT}{{\mathscr{T}}}
\newcommand{\sU}{{\mathscr{U}}}   \newcommand{\sV}{{\mathscr{V}}}
\newcommand{\sW}{{\mathscr{W}}}   \newcommand{\sX}{{\mathscr{X}}}
\newcommand{\sY}{{\mathscr{Y}}}   \newcommand{\sZ}{{\mathscr{Z}}}


%---------------------------------------
% Math Fraktur Font
%---------------------------------------

%Captital Letters
\newcommand{\mfA}{\mathfrak{A}}	\newcommand{\mfB}{\mathfrak{B}}
\newcommand{\mfC}{\mathfrak{C}}	\newcommand{\mfD}{\mathfrak{D}}
\newcommand{\mfE}{\mathfrak{E}}	\newcommand{\mfF}{\mathfrak{F}}
\newcommand{\mfG}{\mathfrak{G}}	\newcommand{\mfH}{\mathfrak{H}}
\newcommand{\mfI}{\mathfrak{I}}	\newcommand{\mfJ}{\mathfrak{J}}
\newcommand{\mfK}{\mathfrak{K}}	\newcommand{\mfL}{\mathfrak{L}}
\newcommand{\mfM}{\mathfrak{M}}	\newcommand{\mfN}{\mathfrak{N}}
\newcommand{\mfO}{\mathfrak{O}}	\newcommand{\mfP}{\mathfrak{P}}
\newcommand{\mfQ}{\mathfrak{Q}}	\newcommand{\mfR}{\mathfrak{R}}
\newcommand{\mfS}{\mathfrak{S}}	\newcommand{\mfT}{\mathfrak{T}}
\newcommand{\mfU}{\mathfrak{U}}	\newcommand{\mfV}{\mathfrak{V}}
\newcommand{\mfW}{\mathfrak{W}}	\newcommand{\mfX}{\mathfrak{X}}
\newcommand{\mfY}{\mathfrak{Y}}	\newcommand{\mfZ}{\mathfrak{Z}}
%Small Letters
\newcommand{\mfa}{\mathfrak{a}}	\newcommand{\mfb}{\mathfrak{b}}
\newcommand{\mfc}{\mathfrak{c}}	\newcommand{\mfd}{\mathfrak{d}}
\newcommand{\mfe}{\mathfrak{e}}	\newcommand{\mff}{\mathfrak{f}}
\newcommand{\mfg}{\mathfrak{g}}	\newcommand{\mfh}{\mathfrak{h}}
\newcommand{\mfi}{\mathfrak{i}}	\newcommand{\mfj}{\mathfrak{j}}
\newcommand{\mfk}{\mathfrak{k}}	\newcommand{\mfl}{\mathfrak{l}}
\newcommand{\mfm}{\mathfrak{m}}	\newcommand{\mfn}{\mathfrak{n}}
\newcommand{\mfo}{\mathfrak{o}}	\newcommand{\mfp}{\mathfrak{p}}
\newcommand{\mfq}{\mathfrak{q}}	\newcommand{\mfr}{\mathfrak{r}}
\newcommand{\mfs}{\mathfrak{s}}	\newcommand{\mft}{\mathfrak{t}}
\newcommand{\mfu}{\mathfrak{u}}	\newcommand{\mfv}{\mathfrak{v}}
\newcommand{\mfw}{\mathfrak{w}}	\newcommand{\mfx}{\mathfrak{x}}
\newcommand{\mfy}{\mathfrak{y}}	\newcommand{\mfz}{\mathfrak{z}}

%---------------------------------------
% Bar
%---------------------------------------

%Captital Letters
\newcommand{\ovA}{\overline{A}}	\newcommand{\ovB}{\overline{B}}
\newcommand{\ovC}{\overline{C}}	\newcommand{\ovD}{\overline{D}}
\newcommand{\ovE}{\overline{E}}	\newcommand{\ovF}{\overline{F}}
\newcommand{\ovG}{\overline{G}}	\newcommand{\ovH}{\overline{H}}
\newcommand{\ovI}{\overline{I}}	\newcommand{\ovJ}{\overline{J}}
\newcommand{\ovK}{\overline{K}}	\newcommand{\ovL}{\overline{L}}
\newcommand{\ovM}{\overline{M}}	\newcommand{\ovN}{\overline{N}}
\newcommand{\ovO}{\overline{O}}	\newcommand{\ovP}{\overline{P}}
\newcommand{\ovQ}{\overline{Q}}	\newcommand{\ovR}{\overline{R}}
\newcommand{\ovS}{\overline{S}}	\newcommand{\ovT}{\overline{T}}
\newcommand{\ovU}{\overline{U}}	\newcommand{\ovV}{\overline{V}}
\newcommand{\ovW}{\overline{W}}	\newcommand{\ovX}{\overline{X}}
\newcommand{\ovY}{\overline{Y}}	\newcommand{\ovZ}{\overline{Z}}
%Small Letters
\newcommand{\ova}{\overline{a}}	\newcommand{\ovb}{\overline{b}}
\newcommand{\ovc}{\overline{c}}	\newcommand{\ovd}{\overline{d}}
\newcommand{\ove}{\overline{e}}	\newcommand{\ovf}{\overline{f}}
\newcommand{\ovg}{\overline{g}}	\newcommand{\ovh}{\overline{h}}
\newcommand{\ovi}{\overline{i}}	\newcommand{\ovj}{\overline{j}}
\newcommand{\ovk}{\overline{k}}	\newcommand{\ovl}{\overline{l}}
\newcommand{\ovm}{\overline{m}}	\newcommand{\ovn}{\overline{n}}
\newcommand{\ovo}{\overline{o}}	\newcommand{\ovp}{\overline{p}}
\newcommand{\ovq}{\overline{q}}	\newcommand{\ovr}{\overline{r}}
\newcommand{\ovs}{\overline{s}}	\newcommand{\ovt}{\overline{t}}
\newcommand{\ovu}{\overline{u}}	\newcommand{\ovv}{\overline{v}}
\newcommand{\ovw}{\overline{w}}	\newcommand{\ovx}{\overline{x}}
\newcommand{\ovy}{\overline{y}}	\newcommand{\ovz}{\overline{z}}

%---------------------------------------
% Tilde
%---------------------------------------

%Captital Letters
\newcommand{\tdA}{\tilde{A}}	\newcommand{\tdB}{\tilde{B}}
\newcommand{\tdC}{\tilde{C}}	\newcommand{\tdD}{\tilde{D}}
\newcommand{\tdE}{\tilde{E}}	\newcommand{\tdF}{\tilde{F}}
\newcommand{\tdG}{\tilde{G}}	\newcommand{\tdH}{\tilde{H}}
\newcommand{\tdI}{\tilde{I}}	\newcommand{\tdJ}{\tilde{J}}
\newcommand{\tdK}{\tilde{K}}	\newcommand{\tdL}{\tilde{L}}
\newcommand{\tdM}{\tilde{M}}	\newcommand{\tdN}{\tilde{N}}
\newcommand{\tdO}{\tilde{O}}	\newcommand{\tdP}{\tilde{P}}
\newcommand{\tdQ}{\tilde{Q}}	\newcommand{\tdR}{\tilde{R}}
\newcommand{\tdS}{\tilde{S}}	\newcommand{\tdT}{\tilde{T}}
\newcommand{\tdU}{\tilde{U}}	\newcommand{\tdV}{\tilde{V}}
\newcommand{\tdW}{\tilde{W}}	\newcommand{\tdX}{\tilde{X}}
\newcommand{\tdY}{\tilde{Y}}	\newcommand{\tdZ}{\tilde{Z}}
%Small Letters
\newcommand{\tda}{\tilde{a}}	\newcommand{\tdb}{\tilde{b}}
\newcommand{\tdc}{\tilde{c}}	\newcommand{\tdd}{\tilde{d}}
\newcommand{\tde}{\tilde{e}}	\newcommand{\tdf}{\tilde{f}}
\newcommand{\tdg}{\tilde{g}}	\newcommand{\tdh}{\tilde{h}}
\newcommand{\tdi}{\tilde{i}}	\newcommand{\tdj}{\tilde{j}}
\newcommand{\tdk}{\tilde{k}}	\newcommand{\tdl}{\tilde{l}}
\newcommand{\tdm}{\tilde{m}}	\newcommand{\tdn}{\tilde{n}}
\newcommand{\tdo}{\tilde{o}}	\newcommand{\tdp}{\tilde{p}}
\newcommand{\tdq}{\tilde{q}}	\newcommand{\tdr}{\tilde{r}}
\newcommand{\tds}{\tilde{s}}	\newcommand{\tdt}{\tilde{t}}
\newcommand{\tdu}{\tilde{u}}	\newcommand{\tdv}{\tilde{v}}
\newcommand{\tdw}{\tilde{w}}	\newcommand{\tdx}{\tilde{x}}
\newcommand{\tdy}{\tilde{y}}	\newcommand{\tdz}{\tilde{z}}

%---------------------------------------
% Vec
%---------------------------------------

%Captital Letters
\newcommand{\vcA}{\vec{A}}	\newcommand{\vcB}{\vec{B}}
\newcommand{\vcC}{\vec{C}}	\newcommand{\vcD}{\vec{D}}
\newcommand{\vcE}{\vec{E}}	\newcommand{\vcF}{\vec{F}}
\newcommand{\vcG}{\vec{G}}	\newcommand{\vcH}{\vec{H}}
\newcommand{\vcI}{\vec{I}}	\newcommand{\vcJ}{\vec{J}}
\newcommand{\vcK}{\vec{K}}	\newcommand{\vcL}{\vec{L}}
\newcommand{\vcM}{\vec{M}}	\newcommand{\vcN}{\vec{N}}
\newcommand{\vcO}{\vec{O}}	\newcommand{\vcP}{\vec{P}}
\newcommand{\vcQ}{\vec{Q}}	\newcommand{\vcR}{\vec{R}}
\newcommand{\vcS}{\vec{S}}	\newcommand{\vcT}{\vec{T}}
\newcommand{\vcU}{\vec{U}}	\newcommand{\vcV}{\vec{V}}
\newcommand{\vcW}{\vec{W}}	\newcommand{\vcX}{\vec{X}}
\newcommand{\vcY}{\vec{Y}}	\newcommand{\vcZ}{\vec{Z}}
%Small Letters
\newcommand{\vca}{\vec{a}}	\newcommand{\vcb}{\vec{b}}
\newcommand{\vcc}{\vec{c}}	\newcommand{\vcd}{\vec{d}}
\newcommand{\vce}{\vec{e}}	\newcommand{\vcf}{\vec{f}}
\newcommand{\vcg}{\vec{g}}	\newcommand{\vch}{\vec{h}}
\newcommand{\vci}{\vec{i}}	\newcommand{\vcj}{\vec{j}}
\newcommand{\vck}{\vec{k}}	\newcommand{\vcl}{\vec{l}}
\newcommand{\vcm}{\vec{m}}	\newcommand{\vcn}{\vec{n}}
\newcommand{\vco}{\vec{o}}	\newcommand{\vcp}{\vec{p}}
\newcommand{\vcq}{\vec{q}}	\newcommand{\vcr}{\vec{r}}
\newcommand{\vcs}{\vec{s}}	\newcommand{\vct}{\vec{t}}
\newcommand{\vcu}{\vec{u}}	\newcommand{\vcv}{\vec{v}}
%\newcommand{\vcw}{\vec{w}}	\newcommand{\vcx}{\vec{x}}
\newcommand{\vcy}{\vec{y}}	\newcommand{\vcz}{\vec{z}}

%---------------------------------------
% Greek Letters:-
%---------------------------------------
\newcommand{\eps}{\epsilon}
\newcommand{\veps}{\varepsilon}
\newcommand{\lm}{\lambda}
\newcommand{\Lm}{\Lambda}
\newcommand{\gm}{\gamma}
\newcommand{\Gm}{\Gamma}
\newcommand{\vph}{\varphi}
\newcommand{\ph}{\phi}
\newcommand{\om}{\omega}
\newcommand{\Om}{\Omega}
\newcommand{\sg}{\sigma}
\newcommand{\Sg}{\Sigma}

\newcommand{\Qed}{\begin{flushright}\qed\end{flushright}}
\newcommand{\parinn}{\setlength{\parindent}{1cm}}
\newcommand{\parinf}{\setlength{\parindent}{0cm}}
\newcommand{\del}[2]{\frac{\partial #1}{\partial #2}}
\newcommand{\Del}[3]{\frac{\partial^{#1} #2}{\partial^{#1} #3}}
\newcommand{\deld}[2]{\dfrac{\partial #1}{\partial #2}}
\newcommand{\Deld}[3]{\dfrac{\partial^{#1} #2}{\partial^{#1} #3}}
\newcommand{\uin}{\mathbin{\rotatebox[origin=c]{90}{$\in$}}}
\newcommand{\usubset}{\mathbin{\rotatebox[origin=c]{90}{$\subset$}}}
\newcommand{\lt}{\left}
\newcommand{\rt}{\right}
\newcommand{\exs}{\exists}
\newcommand{\st}{\strut}
\newcommand{\dps}[1]{\displaystyle{#1}}
\newcommand{\la}{\langle}
\newcommand{\ra}{\rangle}
\newcommand{\cls}[1]{\textsc{#1}}
\newcommand{\prb}[1]{\textsc{#1}}
\newcommand{\comb}[2]{\left(\begin{matrix}
		#1\\ #2
\end{matrix}\right)}
%\newcommand[2]{\quotient}{\faktor{#1}{#2}}
\newcommand\quotient[2]{
	\mathchoice
	{% \displaystyle
		\text{\raise1ex\hbox{$#1$}\Big/\lower1ex\hbox{$#2$}}%
	}
	{% \textstyle
		#1\,/\,#2
	}
	{% \scriptstyle
		#1\,/\,#2
	}
	{% \scriptscriptstyle  
		#1\,/\,#2
	}
}

\newcommand{\tensor}{\otimes}
\newcommand{\xor}{\oplus}

\newcommand{\sol}[1]{\begin{solution}#1\end{solution}}
\newcommand{\solve}[1]{\setlength{\parindent}{0cm}\textbf{\textit{Solution: }}\setlength{\parindent}{1cm}#1 \hfill $\blacksquare$}
\newcommand{\mat}[1]{\left[\begin{matrix}#1\end{matrix}\right]}
\newcommand{\matr}[1]{\begin{matrix}#1\end{matrix}}
\newcommand{\matp}[1]{\lt(\begin{matrix}#1\end{matrix}\rt)}
\newcommand{\detmat}[1]{\lt|\begin{matrix}#1\end{matrix}\rt|}
\newcommand\numberthis{\addtocounter{equation}{1}\tag{\theequation}}
\newcommand{\handout}[3]{
	\noindent
	\begin{center}
		\framebox{
			\vbox{
				\hbox to 6.5in { {\bf Complexity Theory I } \hfill Jan -- May, 2023 }
				\vspace{4mm}
				\hbox to 6.5in { {\Large \hfill #1  \hfill} }
				\vspace{2mm}
				\hbox to 6.5in { {\em #2 \hfill #3} }
			}
		}
	\end{center}
	\vspace*{4mm}
}

\newcommand{\lecture}[3]{\handout{Lecture #1}{Lecturer: #2}{Scribe:	#3}}

\let\marvosymLightning\Lightning
\newcommand{\ctr}{\text{\marvosymLightning}\hspace{0.5ex}} % Requires marvosym package

\newcommand{\ov}[1]{\overline{#1}}
\newcommand{\thmref}[1]{\hyperref[th:#1]{Theorem \ref{th:#1}}}
\newcommand{\propref}[1]{\hyperref[th:#1]{Proposition \ref{th:#1}}}
\newcommand{\lmref}[1]{\hyperref[th:#1]{Lemma \ref{th:#1}}}
\newcommand{\corref}[1]{\hyperref[th:#1]{Corollary \ref{th:#1}}}

\newcommand{\thrmref}[1]{\hyperref[#1]{Theorem \ref{#1}}}
\newcommand{\propnref}[1]{\hyperref[#1]{Proposition \ref{#1}}}
\newcommand{\lemref}[1]{\hyperref[#1]{Lemma \ref{#1}}}
\newcommand{\corrref}[1]{\hyperref[#1]{Corollary \ref{#1}}}

\DeclareMathOperator{\enc}{Enc}
\DeclareMathOperator{\res}{Res}
\DeclareMathOperator{\spec}{Spec}
\DeclareMathOperator{\cov}{Cov}
\DeclareMathOperator{\Var}{Var}
\DeclareMathOperator{\Rank}{rank}
\newcommand{\Tfae}{The following are equivalent:}
\newcommand{\tfae}{the following are equivalent:}
\newcommand{\sparsity}{\textit{sparsity}}

\newcommand{\uddots}{\reflectbox{$\ddots$}} 

\newenvironment{claimwidth}{\begin{center}\begin{adjustwidth}{0.05\textwidth}{0.05\textwidth}}{\end{adjustwidth}\end{center}}

\setlength{\parindent}{0pt}

%%%%%%%%%%%%%%%%%%%%%%%%%%%%%%%%%%%%%%%%%%%%%%%%%%%%%%%%%%%%%%%%%%%%%%%%%%%%%%%%%%%%%%%%%%%%%%%%%%%%%%%%%%%%%%%%%%%%%%%%%%%%%%%%%%%%%%%%

\begin{document}
	
	%%%%%%%%%%%%%%%%%%%%%%%%%%%%%%%%%%%%%%%%%%%%%%%%%%%%%%%%%%%%%%%%%%%%%%%%%%%%%%%%%%%%%%%%%%%%%%%%%%%%%%%%%%%%%%%%%%%%%%%%%%%%%%%%%%%%%%%%
	
	\textsf{\noindent \large\textbf{Soham Chatterjee} \hfill \textbf{Assignment - 1}\\
		Email: \href{sohamc@cmi.ac.in}{sohamc@cmi.ac.in} \hfill Roll: BMC202175\\
		\normalsize Course: Complex Analysis \hfill Date: January 13, 2023}
	
	%%%%%%%%%%%%%%%%%%%%%%%%%%%%%%%%%%%%%%%%%%%%%%%%%%%%%%%%%%%%%%%%%%%%%%%%%%%%%%%%%%%%%%%%%%%%%%%%%%%%%%%%%%%%%%%%%%%%%%%%%%%%%%%%%%%%%%%%
	% Problem 1
	%%%%%%%%%%%%%%%%%%%%%%%%%%%%%%%%%%%%%%%%%%%%%%%%%%%%%%%%%%%%%%%%%%%%%%%%%%%%%%%%%%%%%%%%%%%%%%%%%%%%%%%%%%%%%%%%%%%%%%%%%%%%%%%%%%%%%%%%
	
	\begin{problem}{%problem statement
		}{p1% problem reference text
		}
				Find the general analytic function $f=u+iv$, such that $u=x^2-y^2$.
	
		%Problem		
	\end{problem}
	
	\solve{
		%Solution
		Given that $u=x^2-y^2$. Then $\del{u}{x}=2x$ and $\del{u}{y}=-2y$. Since the function is analytic $u,v$ follows the Cauchy Riemann Equations. Hence $$\del{u}{x}=\del{v}{y}\qquad \del{u}{y}=-\del{v}{x}$$Hence $\del{v}{x}=2x$ and $\del{v}{y}=2y$. Now since $\del{v}{x}=2y$ we can assume $v=2xy+g(y)$ where $g$ is some real valued function. But then $\del{v}{y}=2x$ implies that $g'(y)=0$ hence $g$ is some constant function. Hence $v=2xy+c$ where $c\in \bbR$ some constant. Hence $$f(x,y)=x^2-y^2+i(2xy+c)==x^2-y^2+2ixy+ic=(x+iy)^2+ic\iff f(z)=z^2+ic$$
	}
	
	
	%%%%%%%%%%%%%%%%%%%%%%%%%%%%%%%%%%%%%%%%%%%%%%%%%%%%%%%%%%%%%%%%%%%%%%%%%
	% Problem 2
	%%%%%%%%%%%%%%%%%%%%%%%%%%%%%%%%%%%%%%%%%%%%%%%%%%%%%%%%%%%%%%%%%%%%%%%%%
	
	\begin{problem}{%problem statement
		}{p2% problem reference text
		}
		%Problem		
		Let $f(x,y)=(u(x,y),v(x,y)) $be a function defined on an open set $U$ of the plane, and taking values in the plane. We may think of $f$ as a complex valued function defined on the open subset $U$ of the complex numbers in the obvious way. Show that $f$ is differentiable as a complex valued function at a point $p$ of $U$ if and only if the total derivative of $f$ commutes with multiplication by complex numbers $i$. $T(ab)=bT(a)$
	\end{problem}
	
	\solve{
		%Solution
	Total derivative of $f$ at $p$ will be the matrix $$T=\mat{\del{u}{x} & \del{u}{y}\\ \del{v}{x} & \del{v}{y}}$$ Let $M=\mat{0 & -1\\ 1 & 0}$. Since $M^2=T_2$ one can treat this matrix $M$ as the $\sqrt{-1}=i$. Now we define a ring homomorphism $\varphi : \bbC\to M_2(\bbR)$ where $$\varphi (a)=aI_2 \ \forall\ a\in \bbR \qquad \varphi  (i)=M$$Therefore for any $a+ib\in \bbC$ $$\varphi (a+ib)=\varphi(a)+\varphi(ib)=aI_2+\varphi(i)\varphi(b)=aI_2+MbI_2=aI_2+bM=\mat{a & -b\\ b & a}$$Now if $a+ib\in \ker(\varphi)$ then $$\varphi(a+ib)=0\implies \mat{a & -b\\ b & a}=0\implies a=b=0$$Hence $\ker{\varphi}=\{0\}$. By First Isomorphism Theorem $\bbC\cong \Im (\varphi)$		
	
	Let $f$ is differentiable at $p$. Hence $f$ follows the Cauchy Riemann Equations $$\del{u}{x}=\del{v}{y}\qquad \del{u}{y}=-\del{v}{x}$$. Hence 
	Hence total derivative of $f$ commutes with complex multiplication is equivalent to showing $T$ commutes with $M$. $$TM=\mat{\del{u}{x} & \del{u}{y}\\ \del{v}{x} & \del{v}{y}}\mat{0 & -1\\ 1 & 0}=\mat{\del{u}{y} & -\del{u}{x}\\ \del{v}{y} & -\del{v}{x}}$$ and $$MT=\mat{0 & -1\\ 1 & 0}\mat{\del{u}{x} & \del{u}{y}\\ \del{v}{x} & \del{v}{y}}= \mat{-\del{v}{x} & -\del{v}{y} \\ \del{u}{x} & \del{u}{y}  }$$ Therefore $TM=MT$. 
	
	Let total derivative of $f$ commutes with complex multiplication. Hence $TM=MT$. Therefore $$\del{u}{x}=\del{v}{y}\qquad \del{u}{y}=-\del{v}{x}$$Hence $f$ is differentiable at $p$ and it follows the Cauchy Riemann Equations. Hence $f$ is complex differentiable at $p$.
mat	}
	
	
	%%%%%%%%%%%%%%%%%%%%%%%%%%%%%%%%%%%%%%%%%%%%%%%%%%%%%%%%%%%%%%%%%%%%%%%%%
	% Problem 3
	%%%%%%%%%%%%%%%%%%%%%%%%%%%%%%%%%%%%%%%%%%%%%%%%%%%%%%%%%%%%%%%%%%%%%%%%%
	
	\begin{problem}{%problem statement
		}{p3% problem reference text
		}
		%Problem
					Prove Cauchy's inequality: Let $a=(a_1,...,a_n)$, $b=(b_1,...,b_n)$ be two complex vectors, then $$( a\cdot b)^2 \leq \|a\|^2  \|b\|^2$$ where $(a\cdot b)$ is the scalar product of vectors.
				
	\end{problem}
	
	\solve{
		%Solution
		Consider the vector $a+\lm b$. Now since for any vector $v$, $\sqrt{(v\cdot v)}=\|v\|\geq 0$ we have $\sqrt{((a+\lm b)\cdot (a+\lm b))}\geq 0$. Now \begin{align*}
			((a+\lm b)\cdot (a+\lm b))  & = (a\cdot (a+\lm b))+\lm (b\cdot (a+\lm b))\\
			& = (a\cdot a)+\lm (a\cdot b)+\lm (b\cdot a)+\lm^2(b\cdot b)\\
			& \|a\|^2+2 \lm (a\cdot b) + \lm^2\|b\|^2
		\end{align*}
	Since $	((a+\lm b)\cdot (a+\lm b))  \geq 0$ the discriminant of the polynomial, $p(\lm)=\|a\|^2+2 \lm (a\cdot b) + \lm^2\|b\|^2$ is non-positive. Hence $$ 4(a\cdot b) ^2\leq 4\|a\|^2\|b\|^2\iff (a\cdot b) ^2\leq \|a\|^2\|b\|^2$$
	}
	
	
	%%%%%%%%%%%%%%%%%%%%%%%%%%%%%%%%%%%%%%%%%%%%%%%%%%%%%%%%%%%%%%%%%%%%%%%%%
	% Problem 4
	%%%%%%%%%%%%%%%%%%%%%%%%%%%%%%%%%%%%%%%%%%%%%%%%%%%%%%%%%%%%%%%%%%%%%%%%%
	
	\begin{problem}{%problem statement
			Ahlfors Exercise 2.1 Problem 1
		}{p4% problem reference text
		}
		%Problem		
		If $g(w)$ and $f(z)$ are analytic functions, show that $g(f(z))$ is also analytic.
	\end{problem}
	
	\solve{
		%Solution
		Since $f$ is analytic $$\lim_{h\to 0}\frac{|f(z+h)-f(z)-f'(z)h|}{|h|}=0$$and Let $f(z)=b$ then $g'(f(z))=g'(b)$ then $$ \lim_{h\to 0}\frac{|g(b+k)-g(b)-g'(b)k|}{|k|}=0$$Let \begin{align*}
		\alpha(h) & = f(z+h)-f(a)-f'(z)h & \eps(h) & = \frac{|\alpha(h)|}{|h|}\to 0\text{ as }h\to 0  \\[3mm]
		\beta(k)  & = g(b+k)-g(b)-g'(b)k & \eta(k) & = \begin{cases*}
			\dfrac{|\beta(k)|}{|k|}\to0  \text{ as }k\to 0 \\
			0      \text{ when }                         k=0
		\end{cases*}
	\end{align*}
	$\eta$ is continuous at $k=0$. We want to show that \[\lim_{h\to 0}\frac{|g(f(z+h))-g(f(z))-g'(b)f'(z)h|}{|h|}=0\iff \lim_{k\to 0}\frac{|g(b+k)-g(b)-g'(b)f'(z)h|}{|h|}=0\]where $f(z+h)=b+k\iff k=f(z+h)-f(z)$. We have taken a specific value of $k$ depending on $h$. g'(b)o now $k$ is a function of $h$. Hence $f'(z)h=f(z+h)-f(z)-\alpha(h)=k-\alpha(h)$
	\begin{align*}
		& g(b+k) -g(b) -g'(b)f'(z)h           \\
		= & g(b+k) -g(b) -g'(b)(k-\alpha(h))  \\
		= & g(b+k)-g(b)-g'(b)k+g'(b)\alpha(h)
	\end{align*}Therefore\[ \frac{|g(b+k)-g(b)-g'(b)f'(z)h|}{|h|}\leq \overbrace{\frac{|g(b+k)-g(b)-g'(b)k|}{|h|}}^{\substack{\beta(k)\\ \parallel}}+\frac{|g'(b)\alpha(h)|}{|h|}\] want to bound each of these separately
	\[\frac{|g'(b)\alpha(h)|}{|h|}= |g'(b)|\frac{|\alpha(h)|}{|h|}\to 0\]Now how to bound the first term. In the first term $\frac{|\beta(k)|}{|h|}=|\eta(k)|\frac{|k|}{|h|}$. Now \begin{align*}
		& k=f'(z)h+\alpha(h)                                                                                                                                                        \\
		\implies & |k|= |f'(z)h|+|\alpha(h)|                                                                                                                                        \\
		\implies & \frac{|k|}{|h|}\leq \frac{|f'(z)h|}{|h|}+\frac{|\alpha(h)|}{|h|}= \frac{|f'(z)||h|}{|h|}+\frac{|\alpha(h)|}{|h|} = |f'(z)|+\frac{|\alpha(h)|}{|h|}
	\end{align*}Hence $$\frac{|\beta(k)|}{|k|}=|\eta(k)|\frac{|k|}{|h|}\leq \eta(k)\lt[|f'(z)|+\frac{|\alpha(h)|}{|h|}\rt]$$As $h\to 0$ $|f'(z)|+\frac{|\alpha(h)|}{|h|}\to |f'(z)|+0$ which is finite. And as $h\to 0$, $k\to 0\implies |\eta(k)|\to 0$ because $\eta$ is continuous at 0. Hence $g\circ f$ is differentiable at $z$ $\forall \ z\in \bbC$. Therefore $g\circ f$ is  analytic.
	}
	
	
	%%%%%%%%%%%%%%%%%%%%%%%%%%%%%%%%%%%%%%%%%%%%%%%%%%%%%%%%%%%%%%%%%%%%%%%%%
	% Problem 5
	%%%%%%%%%%%%%%%%%%%%%%%%%%%%%%%%%%%%%%%%%%%%%%%%%%%%%%%%%%%%%%%%%%%%%%%%%
	
	\begin{problem}{%problem statement
			Ahlfors Exercise 2.1 Problem 2
		}{p5% problem reference text
		}
		%Problem		
		Verify Cauchy-Riemann’s equations for the functions $z^2$ and $z^3$
	\end{problem}
	
	\solve{
		%Solution
		\begin{align*}
			 z^2&=(x+i y)^2=\underbrace{x^2-y^2}_{u_1}+\underbrace{2 x y }_{v_1}i \\
			 z^3&=(x+i y)^3=\left(x^2-y^2+2 x y i\right)(x+y i)=\left[\underbrace{x^3-3 x y^2}_{u_2}\right]+\left[\underbrace{3 x^2 y-y^3}_{v_2}\right] i
		\end{align*}For $z^2$
	$$\del{u_1}{x}=2x=\del{v_1}{y}\qquad \del{u_1}{y}=-2y=-\del{v_1}{y}$$and for $z^3$ $$\del{u_2}{x}=3x^2-3y^2=\del{v_2}{Y}\qquad \del{u_2}{y}=-6xy=-\del{v_2}{x}$$Hence both the functions follow the Cauchy-Riemann's Equations
	}
	
		%%%%%%%%%%%%%%%%%%%%%%%%%%%%%%%%%%%%%%%%%%%%%%%%%%%%%%%%%%%%%%%%%%%%%%%%%
	% Problem 6
	%%%%%%%%%%%%%%%%%%%%%%%%%%%%%%%%%%%%%%%%%%%%%%%%%%%%%%%%%%%%%%%%%%%%%%%%%
	
	\begin{problem}{%problem statement
			Ahlfors Exercise 2.1 Problem 3
		}{p5% problem reference text
		}
		%Problem		
Find the most general harmonic polynomial of the form $ax^3+bx^2y+cxy^2+dy^3$. Determine the conjugate harmonic function and the corresponding analytic function by integration and by the formal method.
	\end{problem}
	
	\solve{
		%Solution
		Solution: Let $u(x, y)=a x^3+b x^2 y+c x y^2+d y^3$. Then,
		$$
		\begin{aligned}
			\frac{\partial u}{\partial x} & =3 a x^2+2 b x y+c y^2 & \frac{\partial u}{\partial y} & =b x^2+2 c x y+3 d y^2 \\
			\frac{\partial^2 u}{\partial x^2} & =6 a x+2 b y & \frac{\partial^2 u}{\partial y^2} & =2 c x+6 d y
		\end{aligned}
		$$
		So, for $u$ to be harmonic, it must be that
		$$
		\frac{\partial^2 u}{\partial x^2}+\dfrac{\partial^2 u}{\partial y^2}=2(3 a+c) x+2(b+3 d) y=0
		$$
		evidently, this occurs if and only if $c=-3 a$ and $b=-3 d$. Hence $$u(x,y)=ax^3-3dx^y-3axy^2+dy^3$$Now, to find the conjugate harmonic polynomial $v$ it must be that $\deld{v}{x}=-\deld{u}{y}$ and $\deld{v}{y}=\deld{u}{x}$. Using the first relation, we find that
		$$
		v=-\int b x^2+2 c x y+3 d y^2 d x=\int 3 d x^2+6 a x y-3 d y^2 d x=d x^3+3 a x^2 y-3 d x y^2+\varphi(y)
		$$
		where we have substituted the values of $c$ and $b$ found, and $\varphi(y)$ is some function of $y$. Differentiating this with respect to $y$ gives
		$$
		3 a x^2-6 d x y+\varphi^{\prime}(y)=\dfrac{\partial v}{\partial y}=\dfrac{\partial u}{\partial x}=3 a x^2+2 b x y+c y^2=3 a x^2-6d x y-3 a y^2 .
		$$
		It follows that $\varphi^{\prime}(y)=-3 a y^2$ so that $\varphi(y)=-a y^3+C$. In total,
		$$
		v=d x^3+3 a x^2 y-3 d x y^2-a y^3+C
		$$
		for a real constant $C$. The corresponding analytic function is
		$$
		f(z)=\left[a x^3-3 d x^2 y-3 a x y^2+d y^2\right]+i\left[d x^3+3 a x^2 y-3 d x y^2-a y^3+C\right]
		$$
		The formal method states that we can construct $f(z)$ by
		$$
				f(z)=2 u\lt(\frac{z}{2},\frac{z}{2i}\rt)-\overline{f}(0,0)=2 u\lt(\frac{z}{2},\frac{z}{2i}\rt)-u(0,0)+C i
		$$where C is a real constant. Reducing this gives
		\begin{align*}
			f(z) & =2\left(a\left(\frac{z}{2}\right)^3-3 d\left(\frac{z}{2}\right)^2\left(\frac{z}{2 i}\right)-3 a\left(\frac{z}{2}\right)\left(\frac{z}{2 i}\right)^2+d\left(\frac{z}{2 i}\right)^3\right)+C i \\
			& =\frac{1}{4}\left(a z^3-3 d z^2\left(\frac{z}{i}\right)-3 a z\left(\frac{z}{i}\right)^2+d\left(\frac{z}{i}\right)^3\right)+C i \\
			& =(a+d i) z^3+C i .
		\end{align*}
	Expanding out the last expression recovers our formula for $f(z)$ deduced from integration
	}
		%%%%%%%%%%%%%%%%%%%%%%%%%%%%%%%%%%%%%%%%%%%%%%%%%%%%%%%%%%%%%%%%%%%%%%%%%
	% Problem 7
	%%%%%%%%%%%%%%%%%%%%%%%%%%%%%%%%%%%%%%%%%%%%%%%%%%%%%%%%%%%%%%%%%%%%%%%%%
	
	\begin{problem}{%problem statement
			Ahlfors Exercise 2.1 Problem 4
		}{p5% problem reference text
		}
		%Problem		
Show that an analytic function cannot have a constant absolute value without reducing to a constant.
	\end{problem}
	
	\solve{
		%Solution
		If $|f(z)|=0$ $\forall\ z\in\bbC$ then we have $f(z)=0$. Now let $|f(z)|=c>0$ $\forall \ z\in \bbC$. Let $f=u+iv$ then $|f(z)|=\sqrt{u^2(x,y)+v^2(x,y)}=c$. Therefore \begin{align*}
			\del{}{x}(u^2+v^2)=2u\del{u}{x}+2v\del{v}{x}=u\del{u}{x}-v\del{u}{y}=0\\
			\del{}{y}(u^2+v^2)=2u\del{u}{y}+2v\del{v}{y}=u\del{u}{y}+v\del{u}{x}=0
		\end{align*}We can write this in matrix form that $$\mat{u & -v\\ v & u}\mat{\del{u}{x} \\ \del{u}{y}}=\mat{0\\ 0}$$Now if $\mat{u & -v\\ v & u}$ is not invertible then determinant=$u^2+v^2=0$ which is not possible then $\mat{\del{u}{x} \\ \del{u}{y}}=\mat{0\\ 0}$Hence all the partial derivatives of $f$ is 0. So $f'(z)=0$ or $f(z)=c$
	}
	
		%%%%%%%%%%%%%%%%%%%%%%%%%%%%%%%%%%%%%%%%%%%%%%%%%%%%%%%%%%%%%%%%%%%%%%%%%
	% Problem 8
	%%%%%%%%%%%%%%%%%%%%%%%%%%%%%%%%%%%%%%%%%%%%%%%%%%%%%%%%%%%%%%%%%%%%%%%%%
	
	\begin{problem}{%problem statement
			Ahlfors Exercise 2.1 Problem 5
		}{p5% problem reference text
		}
		%Problem		
		Prove rigorously that the functions $f(z)$ and $\overline{f(z)}$ are simultaneously analytic.
	\end{problem}
	
	\solve{
		%Solution
		Suppose first that $f(z)$ is analytic. Let $g(z)=\overline{f(\bar{z})}$. Let $u_f, v_f$ and $u_g, v_g$ denote the real and imaginary parts of $f$ and $g$ respectively. Then, these are related by
		$$
		u_g(x, y)=u_f(x,-y) \quad v_g(x, y)=-v_f(x,-y) .
		$$
		Since $f$ is analytic, $u_f$ and $v_f$ have continuous partial derivatives - so does $u_g, v_g$ have.  Next we verify the Cauchy-Riemann equations:
		\begin{align*}
			&\frac{\partial u_g}{\partial x}=\frac{\partial u_f}{\partial x} & & \frac{\partial v_g}{\partial x}=-\frac{\partial v_f}{\partial x} \\
			& \frac{\partial u_g}{\partial y}=-\frac{\partial u_f}{\partial y}  &&  \frac{\partial v_g}{\partial y}=\frac{\partial v_f}{\partial y}
		\end{align*}
		Since $f$ is analytic, $u_f$ and $v_f$ satisfy the Cauchy-Riemann equations. Thus,
		$$
		\begin{aligned}
			& \frac{\partial u_g}{\partial x}=\frac{\partial u_f}{\partial x}=\frac{\partial v_f}{\partial y}=\frac{\partial v_g}{\partial y} \\
			& \frac{\partial u_g}{\partial y}=-\frac{\partial u_f}{\partial y}=\frac{\partial v_f}{\partial x}=-\frac{\partial v_g}{\partial y}
		\end{aligned}
		$$
		Thus, $u_g$ and $v_g$ satisfy the Cauchy-Riemann equations. It follows that $g(z)=\overline{f(\bar{z})}$ is analytic. Finally, observe that $\overline{g(\bar{z})}=f(z)$ so that the above proof works to show the converse, with the roles of $f$ and $g$ reversed.
	}
		%%%%%%%%%%%%%%%%%%%%%%%%%%%%%%%%%%%%%%%%%%%%%%%%%%%%%%%%%%%%%%%%%%%%%%%%%
	% Problem 9
	%%%%%%%%%%%%%%%%%%%%%%%%%%%%%%%%%%%%%%%%%%%%%%%%%%%%%%%%%%%%%%%%%%%%%%%%%
	\pagebreak
	\begin{problem}{%problem statement
			Ahlfors Exercise 2.1 Problem 6
		}{p5% problem reference text
		}
		%Problem		
Prove that the functions $u(z)$ and $u(\overline{z})$ are simultaneously harmonic.
	\end{problem}
	
	\solve{
		%Solution
		Since $u$ is the real part of $f(z), u(z)=u(x, y)$ where $z=x+i y$. Suppose $u(z)$ is harmonic. Then $u(z)$ satisfies Laplace equation.
		$$
		\Delta u(z)=\del{^2u}{x^2}+\del{^2u}{y^2}=0
		$$
		Now, $u(\bar{z})=u(x,-y)$ where $\dfrac{\partial^2}{\partial x^2} u(\bar{z})=\deld{^2u}{x^2}$ and $\dfrac{\partial^2}{\partial y^2} u(\overline{z})=\deld{^2u}{y^2}$ so
		$$
		\Delta u(\overline{z})=\del{^2u}{x^2}+\del{^2u}{y^2}=0
		$$
		Since $u(z)$ is harmonic, $\Delta u(\overline{z})=0$ so it follows that $u(\overline{z})$ is harmonic as well. Now observe that $u(z)=u(\overline{\overline{z}})$. Hence above proof works to show the converse. 
	}
	

	
\end{document}
