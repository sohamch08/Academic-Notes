\documentclass[a4paper, 11pt]{article}
\usepackage{comment} % enables the use of multi-line comments (\ifx \fi) 
\usepackage{fullpage} % changes the margin
\usepackage[a4paper, total={7in, 10in}]{geometry}
\usepackage{amsmath,mathtools}
\usepackage{amssymb,amsthm}  % assumes amsmath package installed
\usepackage{float}
\usepackage{graphicx}
\graphicspath{{./images/}}
\usepackage{xcolor}
\usepackage{mdframed}
\usepackage[shortlabels]{enumitem}
\usepackage{indentfirst}
\usepackage{hyperref}
\hypersetup{
	colorlinks=true,
	linkcolor=blue,
	filecolor=magenta,      
	urlcolor=blue!70!red,
	pdftitle={Assignment}, %%%%%%%%%%%%%%%%   WRITE ASSIGNMENT PDF NAME  %%%%%%%%%%%%%%%%%%%%
}
\usepackage[most,many,breakable]{tcolorbox}



\definecolor{mytheorembg}{HTML}{F2F2F9}
\definecolor{mytheoremfr}{HTML}{00007B}


\tcbuselibrary{theorems,skins,hooks}
\newtcbtheorem{problem}{Problem}
{%
	enhanced,
	breakable,
	colback = mytheorembg,
	frame hidden,
	boxrule = 0sp,
	borderline west = {2pt}{0pt}{mytheoremfr},
	sharp corners,
	detach title,
	before upper = \tcbtitle\par\smallskip,
	coltitle = mytheoremfr,
	fonttitle = \bfseries\sffamily,
	description font = \mdseries,
	separator sign none,
	segmentation style={solid, mytheoremfr},
}
{p}

% To give references for any problem use like this
% suppose the problem number is p3 then 2 options either 
% \hyperref[p:p3]{<text you want to use to hyperlink> \ref{p:p3}}
%                  or directly 
%                   \ref{p:p3}


%---------------------------------------
% BlackBoard Math Fonts :-
%---------------------------------------

%Captital Letters
\newcommand{\bbA}{\mathbb{A}}	\newcommand{\bbB}{\mathbb{B}}
\newcommand{\bbC}{\mathbb{C}}	\newcommand{\bbD}{\mathbb{D}}
\newcommand{\bbE}{\mathbb{E}}	\newcommand{\bbF}{\mathbb{F}}
\newcommand{\bbG}{\mathbb{G}}	\newcommand{\bbH}{\mathbb{H}}
\newcommand{\bbI}{\mathbb{I}}	\newcommand{\bbJ}{\mathbb{J}}
\newcommand{\bbK}{\mathbb{K}}	\newcommand{\bbL}{\mathbb{L}}
\newcommand{\bbM}{\mathbb{M}}	\newcommand{\bbN}{\mathbb{N}}
\newcommand{\bbO}{\mathbb{O}}	\newcommand{\bbP}{\mathbb{P}}
\newcommand{\bbQ}{\mathbb{Q}}	\newcommand{\bbR}{\mathbb{R}}
\newcommand{\bbS}{\mathbb{S}}	\newcommand{\bbT}{\mathbb{T}}
\newcommand{\bbU}{\mathbb{U}}	\newcommand{\bbV}{\mathbb{V}}
\newcommand{\bbW}{\mathbb{W}}	\newcommand{\bbX}{\mathbb{X}}
\newcommand{\bbY}{\mathbb{Y}}	\newcommand{\bbZ}{\mathbb{Z}}

%---------------------------------------
% MathCal Fonts :-
%---------------------------------------

%Captital Letters
\newcommand{\mcA}{\mathcal{A}}	\newcommand{\mcB}{\mathcal{B}}
\newcommand{\mcC}{\mathcal{C}}	\newcommand{\mcD}{\mathcal{D}}
\newcommand{\mcE}{\mathcal{E}}	\newcommand{\mcF}{\mathcal{F}}
\newcommand{\mcG}{\mathcal{G}}	\newcommand{\mcH}{\mathcal{H}}
\newcommand{\mcI}{\mathcal{I}}	\newcommand{\mcJ}{\mathcal{J}}
\newcommand{\mcK}{\mathcal{K}}	\newcommand{\mcL}{\mathcal{L}}
\newcommand{\mcM}{\mathcal{M}}	\newcommand{\mcN}{\mathcal{N}}
\newcommand{\mcO}{\mathcal{O}}	\newcommand{\mcP}{\mathcal{P}}
\newcommand{\mcQ}{\mathcal{Q}}	\newcommand{\mcR}{\mathcal{R}}
\newcommand{\mcS}{\mathcal{S}}	\newcommand{\mcT}{\mathcal{T}}
\newcommand{\mcU}{\mathcal{U}}	\newcommand{\mcV}{\mathcal{V}}
\newcommand{\mcW}{\mathcal{W}}	\newcommand{\mcX}{\mathcal{X}}
\newcommand{\mcY}{\mathcal{Y}}	\newcommand{\mcZ}{\mathcal{Z}}



%---------------------------------------
% Bold Math Fonts :-
%---------------------------------------

%Captital Letters
\newcommand{\bmA}{\boldsymbol{A}}	\newcommand{\bmB}{\boldsymbol{B}}
\newcommand{\bmC}{\boldsymbol{C}}	\newcommand{\bmD}{\boldsymbol{D}}
\newcommand{\bmE}{\boldsymbol{E}}	\newcommand{\bmF}{\boldsymbol{F}}
\newcommand{\bmG}{\boldsymbol{G}}	\newcommand{\bmH}{\boldsymbol{H}}
\newcommand{\bmI}{\boldsymbol{I}}	\newcommand{\bmJ}{\boldsymbol{J}}
\newcommand{\bmK}{\boldsymbol{K}}	\newcommand{\bmL}{\boldsymbol{L}}
\newcommand{\bmM}{\boldsymbol{M}}	\newcommand{\bmN}{\boldsymbol{N}}
\newcommand{\bmO}{\boldsymbol{O}}	\newcommand{\bmP}{\boldsymbol{P}}
\newcommand{\bmQ}{\boldsymbol{Q}}	\newcommand{\bmR}{\boldsymbol{R}}
\newcommand{\bmS}{\boldsymbol{S}}	\newcommand{\bmT}{\boldsymbol{T}}
\newcommand{\bmU}{\boldsymbol{U}}	\newcommand{\bmV}{\boldsymbol{V}}
\newcommand{\bmW}{\boldsymbol{W}}	\newcommand{\bmX}{\boldsymbol{X}}
\newcommand{\bmY}{\boldsymbol{Y}}	\newcommand{\bmZ}{\boldsymbol{Z}}
%Small Letters
\newcommand{\bma}{\boldsymbol{a}}	\newcommand{\bmb}{\boldsymbol{b}}
\newcommand{\bmc}{\boldsymbol{c}}	\newcommand{\bmd}{\boldsymbol{d}}
\newcommand{\bme}{\boldsymbol{e}}	\newcommand{\bmf}{\boldsymbol{f}}
\newcommand{\bmg}{\boldsymbol{g}}	\newcommand{\bmh}{\boldsymbol{h}}
\newcommand{\bmi}{\boldsymbol{i}}	\newcommand{\bmj}{\boldsymbol{j}}
\newcommand{\bmk}{\boldsymbol{k}}	\newcommand{\bml}{\boldsymbol{l}}
\newcommand{\bmm}{\boldsymbol{m}}	\newcommand{\bmn}{\boldsymbol{n}}
\newcommand{\bmo}{\boldsymbol{o}}	\newcommand{\bmp}{\boldsymbol{p}}
\newcommand{\bmq}{\boldsymbol{q}}	\newcommand{\bmr}{\boldsymbol{r}}
\newcommand{\bms}{\boldsymbol{s}}	\newcommand{\bmt}{\boldsymbol{t}}
\newcommand{\bmu}{\boldsymbol{u}}	\newcommand{\bmv}{\boldsymbol{v}}
\newcommand{\bmw}{\boldsymbol{w}}	\newcommand{\bmx}{\boldsymbol{x}}
\newcommand{\bmy}{\boldsymbol{y}}	\newcommand{\bmz}{\boldsymbol{z}}


%---------------------------------------
% Scr Math Fonts :-
%---------------------------------------

\newcommand{\sA}{{\mathscr{A}}}   \newcommand{\sB}{{\mathscr{B}}}
\newcommand{\sC}{{\mathscr{C}}}   \newcommand{\sD}{{\mathscr{D}}}
\newcommand{\sE}{{\mathscr{E}}}   \newcommand{\sF}{{\mathscr{F}}}
\newcommand{\sG}{{\mathscr{G}}}   \newcommand{\sH}{{\mathscr{H}}}
\newcommand{\sI}{{\mathscr{I}}}   \newcommand{\sJ}{{\mathscr{J}}}
\newcommand{\sK}{{\mathscr{K}}}   \newcommand{\sL}{{\mathscr{L}}}
\newcommand{\sM}{{\mathscr{M}}}   \newcommand{\sN}{{\mathscr{N}}}
\newcommand{\sO}{{\mathscr{O}}}   \newcommand{\sP}{{\mathscr{P}}}
\newcommand{\sQ}{{\mathscr{Q}}}   \newcommand{\sR}{{\mathscr{R}}}
\newcommand{\sS}{{\mathscr{S}}}   \newcommand{\sT}{{\mathscr{T}}}
\newcommand{\sU}{{\mathscr{U}}}   \newcommand{\sV}{{\mathscr{V}}}
\newcommand{\sW}{{\mathscr{W}}}   \newcommand{\sX}{{\mathscr{X}}}
\newcommand{\sY}{{\mathscr{Y}}}   \newcommand{\sZ}{{\mathscr{Z}}}


%---------------------------------------
% Math Fraktur Font
%---------------------------------------

%Captital Letters
\newcommand{\mfA}{\mathfrak{A}}	\newcommand{\mfB}{\mathfrak{B}}
\newcommand{\mfC}{\mathfrak{C}}	\newcommand{\mfD}{\mathfrak{D}}
\newcommand{\mfE}{\mathfrak{E}}	\newcommand{\mfF}{\mathfrak{F}}
\newcommand{\mfG}{\mathfrak{G}}	\newcommand{\mfH}{\mathfrak{H}}
\newcommand{\mfI}{\mathfrak{I}}	\newcommand{\mfJ}{\mathfrak{J}}
\newcommand{\mfK}{\mathfrak{K}}	\newcommand{\mfL}{\mathfrak{L}}
\newcommand{\mfM}{\mathfrak{M}}	\newcommand{\mfN}{\mathfrak{N}}
\newcommand{\mfO}{\mathfrak{O}}	\newcommand{\mfP}{\mathfrak{P}}
\newcommand{\mfQ}{\mathfrak{Q}}	\newcommand{\mfR}{\mathfrak{R}}
\newcommand{\mfS}{\mathfrak{S}}	\newcommand{\mfT}{\mathfrak{T}}
\newcommand{\mfU}{\mathfrak{U}}	\newcommand{\mfV}{\mathfrak{V}}
\newcommand{\mfW}{\mathfrak{W}}	\newcommand{\mfX}{\mathfrak{X}}
\newcommand{\mfY}{\mathfrak{Y}}	\newcommand{\mfZ}{\mathfrak{Z}}
%Small Letters
\newcommand{\mfa}{\mathfrak{a}}	\newcommand{\mfb}{\mathfrak{b}}
\newcommand{\mfc}{\mathfrak{c}}	\newcommand{\mfd}{\mathfrak{d}}
\newcommand{\mfe}{\mathfrak{e}}	\newcommand{\mff}{\mathfrak{f}}
\newcommand{\mfg}{\mathfrak{g}}	\newcommand{\mfh}{\mathfrak{h}}
\newcommand{\mfi}{\mathfrak{i}}	\newcommand{\mfj}{\mathfrak{j}}
\newcommand{\mfk}{\mathfrak{k}}	\newcommand{\mfl}{\mathfrak{l}}
\newcommand{\mfm}{\mathfrak{m}}	\newcommand{\mfn}{\mathfrak{n}}
\newcommand{\mfo}{\mathfrak{o}}	\newcommand{\mfp}{\mathfrak{p}}
\newcommand{\mfq}{\mathfrak{q}}	\newcommand{\mfr}{\mathfrak{r}}
\newcommand{\mfs}{\mathfrak{s}}	\newcommand{\mft}{\mathfrak{t}}
\newcommand{\mfu}{\mathfrak{u}}	\newcommand{\mfv}{\mathfrak{v}}
\newcommand{\mfw}{\mathfrak{w}}	\newcommand{\mfx}{\mathfrak{x}}
\newcommand{\mfy}{\mathfrak{y}}	\newcommand{\mfz}{\mathfrak{z}}

%---------------------------------------
% Bar
%---------------------------------------

%Captital Letters
\newcommand{\ovA}{\overline{A}}	\newcommand{\ovB}{\overline{B}}
\newcommand{\ovC}{\overline{C}}	\newcommand{\ovD}{\overline{D}}
\newcommand{\ovE}{\overline{E}}	\newcommand{\ovF}{\overline{F}}
\newcommand{\ovG}{\overline{G}}	\newcommand{\ovH}{\overline{H}}
\newcommand{\ovI}{\overline{I}}	\newcommand{\ovJ}{\overline{J}}
\newcommand{\ovK}{\overline{K}}	\newcommand{\ovL}{\overline{L}}
\newcommand{\ovM}{\overline{M}}	\newcommand{\ovN}{\overline{N}}
\newcommand{\ovO}{\overline{O}}	\newcommand{\ovP}{\overline{P}}
\newcommand{\ovQ}{\overline{Q}}	\newcommand{\ovR}{\overline{R}}
\newcommand{\ovS}{\overline{S}}	\newcommand{\ovT}{\overline{T}}
\newcommand{\ovU}{\overline{U}}	\newcommand{\ovV}{\overline{V}}
\newcommand{\ovW}{\overline{W}}	\newcommand{\ovX}{\overline{X}}
\newcommand{\ovY}{\overline{Y}}	\newcommand{\ovZ}{\overline{Z}}
%Small Letters
\newcommand{\ova}{\overline{a}}	\newcommand{\ovb}{\overline{b}}
\newcommand{\ovc}{\overline{c}}	\newcommand{\ovd}{\overline{d}}
\newcommand{\ove}{\overline{e}}	\newcommand{\ovf}{\overline{f}}
\newcommand{\ovg}{\overline{g}}	\newcommand{\ovh}{\overline{h}}
\newcommand{\ovi}{\overline{i}}	\newcommand{\ovj}{\overline{j}}
\newcommand{\ovk}{\overline{k}}	\newcommand{\ovl}{\overline{l}}
\newcommand{\ovm}{\overline{m}}	\newcommand{\ovn}{\overline{n}}
\newcommand{\ovo}{\overline{o}}	\newcommand{\ovp}{\overline{p}}
\newcommand{\ovq}{\overline{q}}	\newcommand{\ovr}{\overline{r}}
\newcommand{\ovs}{\overline{s}}	\newcommand{\ovt}{\overline{t}}
\newcommand{\ovu}{\overline{u}}	\newcommand{\ovv}{\overline{v}}
\newcommand{\ovw}{\overline{w}}	\newcommand{\ovx}{\overline{x}}
\newcommand{\ovy}{\overline{y}}	\newcommand{\ovz}{\overline{z}}

%---------------------------------------
% Tilde
%---------------------------------------

%Captital Letters
\newcommand{\tdA}{\tilde{A}}	\newcommand{\tdB}{\tilde{B}}
\newcommand{\tdC}{\tilde{C}}	\newcommand{\tdD}{\tilde{D}}
\newcommand{\tdE}{\tilde{E}}	\newcommand{\tdF}{\tilde{F}}
\newcommand{\tdG}{\tilde{G}}	\newcommand{\tdH}{\tilde{H}}
\newcommand{\tdI}{\tilde{I}}	\newcommand{\tdJ}{\tilde{J}}
\newcommand{\tdK}{\tilde{K}}	\newcommand{\tdL}{\tilde{L}}
\newcommand{\tdM}{\tilde{M}}	\newcommand{\tdN}{\tilde{N}}
\newcommand{\tdO}{\tilde{O}}	\newcommand{\tdP}{\tilde{P}}
\newcommand{\tdQ}{\tilde{Q}}	\newcommand{\tdR}{\tilde{R}}
\newcommand{\tdS}{\tilde{S}}	\newcommand{\tdT}{\tilde{T}}
\newcommand{\tdU}{\tilde{U}}	\newcommand{\tdV}{\tilde{V}}
\newcommand{\tdW}{\tilde{W}}	\newcommand{\tdX}{\tilde{X}}
\newcommand{\tdY}{\tilde{Y}}	\newcommand{\tdZ}{\tilde{Z}}
%Small Letters
\newcommand{\tda}{\tilde{a}}	\newcommand{\tdb}{\tilde{b}}
\newcommand{\tdc}{\tilde{c}}	\newcommand{\tdd}{\tilde{d}}
\newcommand{\tde}{\tilde{e}}	\newcommand{\tdf}{\tilde{f}}
\newcommand{\tdg}{\tilde{g}}	\newcommand{\tdh}{\tilde{h}}
\newcommand{\tdi}{\tilde{i}}	\newcommand{\tdj}{\tilde{j}}
\newcommand{\tdk}{\tilde{k}}	\newcommand{\tdl}{\tilde{l}}
\newcommand{\tdm}{\tilde{m}}	\newcommand{\tdn}{\tilde{n}}
\newcommand{\tdo}{\tilde{o}}	\newcommand{\tdp}{\tilde{p}}
\newcommand{\tdq}{\tilde{q}}	\newcommand{\tdr}{\tilde{r}}
\newcommand{\tds}{\tilde{s}}	\newcommand{\tdt}{\tilde{t}}
\newcommand{\tdu}{\tilde{u}}	\newcommand{\tdv}{\tilde{v}}
\newcommand{\tdw}{\tilde{w}}	\newcommand{\tdx}{\tilde{x}}
\newcommand{\tdy}{\tilde{y}}	\newcommand{\tdz}{\tilde{z}}

%---------------------------------------
% Vec
%---------------------------------------

%Captital Letters
\newcommand{\vcA}{\vec{A}}	\newcommand{\vcB}{\vec{B}}
\newcommand{\vcC}{\vec{C}}	\newcommand{\vcD}{\vec{D}}
\newcommand{\vcE}{\vec{E}}	\newcommand{\vcF}{\vec{F}}
\newcommand{\vcG}{\vec{G}}	\newcommand{\vcH}{\vec{H}}
\newcommand{\vcI}{\vec{I}}	\newcommand{\vcJ}{\vec{J}}
\newcommand{\vcK}{\vec{K}}	\newcommand{\vcL}{\vec{L}}
\newcommand{\vcM}{\vec{M}}	\newcommand{\vcN}{\vec{N}}
\newcommand{\vcO}{\vec{O}}	\newcommand{\vcP}{\vec{P}}
\newcommand{\vcQ}{\vec{Q}}	\newcommand{\vcR}{\vec{R}}
\newcommand{\vcS}{\vec{S}}	\newcommand{\vcT}{\vec{T}}
\newcommand{\vcU}{\vec{U}}	\newcommand{\vcV}{\vec{V}}
\newcommand{\vcW}{\vec{W}}	\newcommand{\vcX}{\vec{X}}
\newcommand{\vcY}{\vec{Y}}	\newcommand{\vcZ}{\vec{Z}}
%Small Letters
\newcommand{\vca}{\vec{a}}	\newcommand{\vcb}{\vec{b}}
\newcommand{\vcc}{\vec{c}}	\newcommand{\vcd}{\vec{d}}
\newcommand{\vce}{\vec{e}}	\newcommand{\vcf}{\vec{f}}
\newcommand{\vcg}{\vec{g}}	\newcommand{\vch}{\vec{h}}
\newcommand{\vci}{\vec{i}}	\newcommand{\vcj}{\vec{j}}
\newcommand{\vck}{\vec{k}}	\newcommand{\vcl}{\vec{l}}
\newcommand{\vcm}{\vec{m}}	\newcommand{\vcn}{\vec{n}}
\newcommand{\vco}{\vec{o}}	\newcommand{\vcp}{\vec{p}}
\newcommand{\vcq}{\vec{q}}	\newcommand{\vcr}{\vec{r}}
\newcommand{\vcs}{\vec{s}}	\newcommand{\vct}{\vec{t}}
\newcommand{\vcu}{\vec{u}}	\newcommand{\vcv}{\vec{v}}
%\newcommand{\vcw}{\vec{w}}	\newcommand{\vcx}{\vec{x}}
\newcommand{\vcy}{\vec{y}}	\newcommand{\vcz}{\vec{z}}

%---------------------------------------
% Greek Letters:-
%---------------------------------------
\newcommand{\eps}{\epsilon}
\newcommand{\veps}{\varepsilon}
\newcommand{\lm}{\lambda}
\newcommand{\Lm}{\Lambda}
\newcommand{\gm}{\gamma}
\newcommand{\Gm}{\Gamma}
\newcommand{\vph}{\varphi}
\newcommand{\ph}{\phi}
\newcommand{\om}{\omega}
\newcommand{\Om}{\Omega}
\newcommand{\sg}{\sigma}
\newcommand{\Sg}{\Sigma}

\newcommand{\Qed}{\begin{flushright}\qed\end{flushright}}
\newcommand{\parinn}{\setlength{\parindent}{1cm}}
\newcommand{\parinf}{\setlength{\parindent}{0cm}}
\newcommand{\del}[2]{\frac{\partial #1}{\partial #2}}
\newcommand{\Del}[3]{\frac{\partial^{#1} #2}{\partial^{#1} #3}}
\newcommand{\deld}[2]{\dfrac{\partial #1}{\partial #2}}
\newcommand{\Deld}[3]{\dfrac{\partial^{#1} #2}{\partial^{#1} #3}}
\newcommand{\uin}{\mathbin{\rotatebox[origin=c]{90}{$\in$}}}
\newcommand{\usubset}{\mathbin{\rotatebox[origin=c]{90}{$\subset$}}}
\newcommand{\lt}{\left}
\newcommand{\rt}{\right}
\newcommand{\exs}{\exists}
\newcommand{\st}{\strut}
\newcommand{\dps}[1]{\displaystyle{#1}}
\newcommand{\la}{\langle}
\newcommand{\ra}{\rangle}
\newcommand{\cls}[1]{\textsc{#1}}
\newcommand{\prb}[1]{\textsc{#1}}
\newcommand{\comb}[2]{\left(\begin{matrix}
		#1\\ #2
\end{matrix}\right)}
%\newcommand[2]{\quotient}{\faktor{#1}{#2}}
\newcommand\quotient[2]{
	\mathchoice
	{% \displaystyle
		\text{\raise1ex\hbox{$#1$}\Big/\lower1ex\hbox{$#2$}}%
	}
	{% \textstyle
		#1\,/\,#2
	}
	{% \scriptstyle
		#1\,/\,#2
	}
	{% \scriptscriptstyle  
		#1\,/\,#2
	}
}

\newcommand{\tensor}{\otimes}
\newcommand{\xor}{\oplus}

\newcommand{\sol}[1]{\begin{solution}#1\end{solution}}
\newcommand{\solve}[1]{\setlength{\parindent}{0cm}\textbf{\textit{Solution: }}\setlength{\parindent}{1cm}#1 \hfill $\blacksquare$}
\newcommand{\mat}[1]{\left[\begin{matrix}#1\end{matrix}\right]}
\newcommand{\matr}[1]{\begin{matrix}#1\end{matrix}}
\newcommand{\matp}[1]{\lt(\begin{matrix}#1\end{matrix}\rt)}
\newcommand{\detmat}[1]{\lt|\begin{matrix}#1\end{matrix}\rt|}
\newcommand\numberthis{\addtocounter{equation}{1}\tag{\theequation}}
\newcommand{\handout}[3]{
	\noindent
	\begin{center}
		\framebox{
			\vbox{
				\hbox to 6.5in { {\bf Complexity Theory I } \hfill Jan -- May, 2023 }
				\vspace{4mm}
				\hbox to 6.5in { {\Large \hfill #1  \hfill} }
				\vspace{2mm}
				\hbox to 6.5in { {\em #2 \hfill #3} }
			}
		}
	\end{center}
	\vspace*{4mm}
}

\newcommand{\lecture}[3]{\handout{Lecture #1}{Lecturer: #2}{Scribe:	#3}}

\let\marvosymLightning\Lightning
\newcommand{\ctr}{\text{\marvosymLightning}\hspace{0.5ex}} % Requires marvosym package

\newcommand{\ov}[1]{\overline{#1}}
\newcommand{\thmref}[1]{\hyperref[th:#1]{Theorem \ref{th:#1}}}
\newcommand{\propref}[1]{\hyperref[th:#1]{Proposition \ref{th:#1}}}
\newcommand{\lmref}[1]{\hyperref[th:#1]{Lemma \ref{th:#1}}}
\newcommand{\corref}[1]{\hyperref[th:#1]{Corollary \ref{th:#1}}}

\newcommand{\thrmref}[1]{\hyperref[#1]{Theorem \ref{#1}}}
\newcommand{\propnref}[1]{\hyperref[#1]{Proposition \ref{#1}}}
\newcommand{\lemref}[1]{\hyperref[#1]{Lemma \ref{#1}}}
\newcommand{\corrref}[1]{\hyperref[#1]{Corollary \ref{#1}}}

\DeclareMathOperator{\enc}{Enc}
\DeclareMathOperator{\res}{Res}
\DeclareMathOperator{\spec}{Spec}
\DeclareMathOperator{\cov}{Cov}
\DeclareMathOperator{\Var}{Var}
\DeclareMathOperator{\Rank}{rank}
\newcommand{\Tfae}{The following are equivalent:}
\newcommand{\tfae}{the following are equivalent:}
\newcommand{\sparsity}{\textit{sparsity}}

\newcommand{\uddots}{\reflectbox{$\ddots$}} 

\newenvironment{claimwidth}{\begin{center}\begin{adjustwidth}{0.05\textwidth}{0.05\textwidth}}{\end{adjustwidth}\end{center}}

\setlength{\parindent}{0pt}

%%%%%%%%%%%%%%%%%%%%%%%%%%%%%%%%%%%%%%%%%%%%%%%%%%%%%%%%%%%%%%%%%%%%%%%%%%%%%%%%%%%%%%%%%%%%%%%%%%%%%%%%%%%%%%%%%%%%%%%%%%%%%%%%%%%%%%%%

\begin{document}
	
	%%%%%%%%%%%%%%%%%%%%%%%%%%%%%%%%%%%%%%%%%%%%%%%%%%%%%%%%%%%%%%%%%%%%%%%%%%%%%%%%%%%%%%%%%%%%%%%%%%%%%%%%%%%%%%%%%%%%%%%%%%%%%%%%%%%%%%%%
	
	\textsf{\noindent \large\textbf{Soham Chatterjee} \hfill \textbf{Assignment - 4}\\
		Email: \href{sohamc@cmi.ac.in}{sohamc@cmi.ac.in} \hfill Roll: BMC202175\\
		\normalsize Course: Complex Analysis \hfill Date: April 14, 2023}
	
	%%%%%%%%%%%%%%%%%%%%%%%%%%%%%%%%%%%%%%%%%%%%%%%%%%%%%%%%%%%%%%%%%%%%%%%%%%%%%%%%%%%%%%%%%%%%%%%%%%%%%%%%%%%%%%%%%%%%%%%%%%%%%%%%%%%%%%%%
	% Problem 1
	%%%%%%%%%%%%%%%%%%%%%%%%%%%%%%%%%%%%%%%%%%%%%%%%%%%%%%%%%%%%%%%%%%%%%%%%%%%%%%%%%%%%%%%%%%%%%%%%%%%%%%%%%%%%%%%%%%%%%%%%%%%%%%%%%%%%%%%%
	
	\begin{problem}{%problem statement
			Ahlfors Page 130: Problem 5
		}{p1% problem reference text
		}
Prove that an isolated singularity of $f(z)$ is removable as soon as either $\Re (f(z))$ or $\Im (f(z))$ is bounded above or below. \textit{Hint:} Apply a fractional linear transformation.
		%Problem		
	\end{problem}
	
	\solve{
		%Solution
\begin{align*}
	\Re>c& & \Im>c\\ 
	\Re<c && \Im<c
\end{align*}
 Note that the fractional linear transformation $T(z) = \frac{z - 1}{z + 1}$ maps right half plane ($\Re > 0$)
to the unit disc. Consequently, the map $T_c(z)=\frac{z-1-c}{z+1-c}$ maps the region $\Re>c$ onto the unit disc. 

If $\Im>c, \Re<c, \Im<c$, we can always rotate these domains onto $\Re>c$. Hence we can map these regions to the unit disc via
$$
T_{c, n}(z)=\frac{i^n z-1-c}{i^n z+1-c}
$$
for $n=0,1,2,3$ (corresponding to the four cases, in order). 

Hence, without loss of generality we may assume $\Re>0$. Consider now the composite map
$$
g(z)=\frac{f(z)-1}{f(z)+1}
$$
If $g$ has a removable singularity at zero, then
$$
\lim_{z \rightarrow 0} z g(z)=0
$$
and consequently,
$$
\lim _{z \rightarrow 0} z[f(z)-1]=\lim_{z \rightarrow 0} z f(z)=0
$$}
	
	
	%%%%%%%%%%%%%%%%%%%%%%%%%%%%%%%%%%%%%%%%%%%%%%%%%%%%%%%%%%%%%%%%%%%%%%%%%
	% Problem 2
	%%%%%%%%%%%%%%%%%%%%%%%%%%%%%%%%%%%%%%%%%%%%%%%%%%%%%%%%%%%%%%%%%%%%%%%%%
	
	\begin{problem}{%problem statement
			Ahlfors Page 130: Problem 6
		}{p2% problem reference text
		}
		%Problem		
Show that an isolated singularity of $f(z)$ cannot be a pole of $\exp f(z)$. \textit{Hint:} $f$ and $e^f$ cannot have a common pole (why?). Now apply Theorem 9.		
	\end{problem}                               
	
	\solve{
		%Solution
		Let $z_0$ is an isolated singularity of $f$. Now if $z_0$ is a pole of $f$ then $h(z)=f^{-1}(z)$ can be made analytic in some open disk $D$  around $z_0$ with $h(z_0)=0$. 
		
		Now $h(z)$ maps the open desk $D$ to some open neighborhood $U$ of $0$. Hence there is an open disk $D_1$ containing 0 inside $U$. Now let $G_1=h^{-1}(D_1)$. $G_1$ is open and contains $z_0$. So take a open disk $D'$ inside $G_1$ and again continue like we did for $D$ and repeat again and again to get $G_n$ and $D_n$. Thus we get a chain of open sets each contained in the previous one $$G_1\supseteq G_2\supseteq G_3\supseteq \cdots \supseteq G_n\supseteq \cdots$$where $h(G_k)=D_k$. Let $r_{d-k}$ denote the radius of $D_k$. Hence $f(G_k)=\{z\mid |z|\geq r_{D_k}\}$. Therefore $f(G_k)\subseteq f(G_{k+1})$. 
		
		Now if we take $D_k$'s in such way that $r_{D_k}>2R_{D_{k+1}}$ and take the sequence $x_k\in G_k\setminus  G_{k+1}$. Since each $D_k$ contains 0 we take $x_i$ such that $\Re(f(x_i))=0$ then $$\lim_{n\to \infty}x_n=z_0$$Therefore $$\lim_{n\to \infty}|e^{f(x_n)}|=\lim_{n\to \infty}|e^{\Re(f(x_n))}|=\lim_{n\to \infty}|e^{0}|=1$$Hence $z_0$ is not a pole of $e^{f(z)}$
		
		Now if $z_0$ is an removable singularity it is also a removable singularity for $e^{f(z)}$
		
		If $z_0$ is an essential singularity then  consider any non-zero $c \in \mathbb{C}$. By the Theorem 9, there is a sequence $z_n \rightarrow z_0$ such that $f\left(z_n\right) \rightarrow \log (c)$. So $\exp \left(f\left(z_n\right)\right) \rightarrow c$. Since this is true for all non-zero $c$, $\exp (f(z))$ must have an essential singularity at $z_0$.
		
		Hence $z_0$ is not a pole of $e^{f(z)}$.
	}
	
	
	%%%%%%%%%%%%%%%%%%%%%%%%%%%%%%%%%%%%%%%%%%%%%%%%%%%%%%%%%%%%%%%%%%%%%%%%%
	% Problem 3
	%%%%%%%%%%%%%%%%%%%%%%%%%%%%%%%%%%%%%%%%%%%%%%%%%%%%%%%%%%%%%%%%%%%%%%%%%
	
	\begin{problem}{%problem statement
			Ahlfors Page 133: Problem 3
		}{p3% problem reference text
		}
		%Problem
	Apply the representation $f(z) = w_o+\zeta(z)^n$ to $\cos z$ with $z_o = 0$.	Determine $\zeta(z)$ explicitly.
	\end{problem}
	
	\solve{
		%Solution
		Take $g(z)=\cos z-1$. Now $g'(z)=-\sin z$, $g''(z)=-\cos z$ and $g(0)=0=g'(0)$, $g''(0)=-1$. Therefore $n=2$. Now $$\cos z-1=-2\sin^2\lt(\frac{z}2\rt)\iff \cos z=1-2\sin^2\lt(\frac{z}2\rt)\quad \forall\ z\in\bbC$$Hence $$\cos z-1=\lt(\pm \sqrt{2}i\sin\frac{z}{2}  \rt)^2$$Hence $\zeta(z)=\sqrt{2}i\sin\frac{z}{2}$ (we are taking the positive branch)
	}
	
	
	%%%%%%%%%%%%%%%%%%%%%%%%%%%%%%%%%%%%%%%%%%%%%%%%%%%%%%%%%%%%%%%%%%%%%%%%%
	% Problem 4
	%%%%%%%%%%%%%%%%%%%%%%%%%%%%%%%%%%%%%%%%%%%%%%%%%%%%%%%%%%%%%%%%%%%%%%%%%

	
	\begin{problem}{%problem statement
			Ahlfors Page 133: Problem 4
		}{p4% problem reference text
		}
		%Problem		
 If $f(z)$ is analytic at the origin and $f'(0) \neq 0$, prove the existence of an analytic $g(z)$ such that $f(z^n) = f(0)+g(z)^n$ in a neighborhood of 0.
	\end{problem}
	
	\solve{
		%Solution
		Take $h(z)=f(z^n)$. Then \begin{align*}
			h'(z)&=nz^{n-1}f'(z^n)\\
			h''(z)&=n(n-1)z^{n-2}f'(z^n)+n^2z^{2(n-1)}f''(z^n)=n(n-1)z^{n-2}f'(z^n)+\tdg_2(z) \\
			h^{(3)}(z)& = n(n-1)n(2)z^{n-3}f'(z^n)+\tdg_3(z)\\
			\vdots\quad & \qquad\qquad\qquad\vdots\\
			h^{(n)}(z)& = n!f'(z^n)+\tdg_n(z)
		\end{align*}Now forall $k\in \{1,2,\dots,n\}$ $\tdg_k(0)=0$ and forall $k\in \{1,2,\dots,n-1\}$, $h^{(k)}(0)=0$ and $g^{(n)}(0)=n!f'(0)$. Since given that $f'(0)\neq 0$ we have $g^{(n)}(0)\neq 0$. Hence there exists a function $h(z)$ such that $h(0)\neq 0$ and $$g(z)-f(0)=(z-0)^nh(z)$$Now since $H()z)$ is continuous for $\eps=|h(z_0)|$ there exists $\delta$  such that $$\forall\ |z-z_0|<\delta\implies |h(z)-h(z_0)|<|h(z_0)$$Hence we have a single valued branch of $\sqrt[n]{h(z)}$, $\zeta(z)=(z-z_0)\sqrt[n]{h(a)}$. Then we have $$h(z)-f(0)=\zeta^n(z)\iff f(z^n)=f(0)+\zeta^n(z)$$
	}
	

	%%%%%%%%%%%%%%%%%%%%%%%%%%%%%%%%%%%%%%%%%%%%%%%%%%%%%%%%%%%%%%%%%%%%%%%%%
	% Problem 5
	%%%%%%%%%%%%%%%%%%%%%%%%%%%%%%%%%%%%%%%%%%%%%%%%%%%%%%%%%%%%%%%%%%%%%%%%%
	
	\begin{problem}{%problem statement
Ahlfors Page 136: Problem 1
		}{p5% problem reference text
		}
		%Problem		
	Show by use of (36), or directly, that $|f(z)| \leq 1$ for $|z| \leq 1$ implies
	$$
	\frac{\left|f^{\prime}(z)\right|}{\left(1-|f(z)|^2\right)} \leq \frac{1}{1-|z|^2}
	$$
	\end{problem}
	
	\solve{
		%Solution
		Here the given function $f$, $|f(z)|\leq 1$ for $|z||\leq 1$. Then choose any $z_0\in \overline{B(0,1)}$ and let $w_0=f(z_0)$. Hence we have 
		\begin{align*}
			& \lt|\frac{f(z)-w_0}{1-\overline{w_0}f(z)}  \rt| \leq \lt|\frac{z-z_0}{1-\overline{z_0}z} \rt|                   &                        \\
			\implies & \frac{\lt|\frac{f(z)-f(z_0)}{z-z_0}\rt|}{\lt|1-\overline{w_0}f(z) \rt|}\leq \lt|\frac{1}{1-\overline{z_0}z}\rt| &                        \\
			\implies & \frac{\lt| f'(z_0)\rt|}{\lt|1-\overline{f(z_0)}f(z_0)\rt|}\leq \frac{1}{\lt| 1-\overline{z_0}z_0\rt|}           & [\text{As $z\to z_0$}] \\
			\implies & \frac{|f'(z_0)|}{1-|f(z_0)|^2}\leq \frac1{|1-|z_0|^2}                                                           &
		\end{align*}Since $z_0$ is arbitrary we have 
	$$
	\frac{\left|f^{\prime}(z)\right|}{\left(1-|f(z)|^2\right)} \leq \frac{1}{1-|z|^2}
	$$
	}
	
		%%%%%%%%%%%%%%%%%%%%%%%%%%%%%%%%%%%%%%%%%%%%%%%%%%%%%%%%%%%%%%%%%%%%%%%%%
	% Problem 6
	%%%%%%%%%%%%%%%%%%%%%%%%%%%%%%%%%%%%%%%%%%%%%%%%%%%%%%%%%%%%%%%%%%%%%%%%%
	
	\begin{problem}{%problem statement
			Ahlfors Page 136: Problem 2
		}{p6% problem reference text
		}
		%Problem		
If $f(z)$ is analytic and $\Im f(z) \geq 0$ for $\Im z>0$, show that
$$
\frac{\left|f(z)-f\left(z_0\right)\right|}{\left|f(z)-\overline{f\left(z_0\right)}\right|} \leq \frac{\left|z-z_0\right|}{\left|z-\bar{z}_0\right|}
$$
and
$$
\frac{\left|f^{\prime}(z)\right|}{\Im f(z)} \leq \frac{1}{y} \quad(z=x+i y)
$$
	\end{problem}
	
	\solve{
		%Solution
		Let $z_0$ be any complex number and $w_0=f(z_0)$. Take $F(\zeta)=Sf(T^{-1}\zeta)$ like in the proof of the inequality in $(36)$ in Ahlfors for some linear transformation $T,S$. Now define $$Tz=\frac{z-z_0}{z-\overline{z_0}}\quad \text{and}\quad Sz=\frac{z-w_0}{z-\overline{w_0}}$$Now if we can show that $F(\zeta)$ satisfies the conditions of Schwarz Lemma then we have $|F(\zeta)|\leq |\zeta|$ by setting $Tz=\zeta$. 
	
	First we have to show $F(0)=0$. Now $Tz_0=0$ hence $T^{-1}0=z_0$. Similarly $Sw_0=0$. Since $f(z_0)=w_0$ we have $F(0)=0$
	
	Now we have to show $|F(\zeta)|\leq 1$. Now $Tz$ maps the upper half plane to inside of the unit disk. Hence im $\Im z\geq 0 $ then $|\zeta|<1$. Similarly $S$ also maps the upper half plane to inside of the unit disk. Therefore $T^{-1}(\zeta)$ maps the inside of the unit disk to upper half plane. And given that if $\Im z\geq 0$ then $\Im f(z)\geq 0$. Hence $|F(\zeta)|\leq 1$ for $|\zeta|<1$. 
	
	Since $F$ satisfies the conditions of Schwarz Lemma we have \begin{multline*}
		 |F(\zeta)|\leq |\zeta|\implies  |Sf(T^{-1}\zeta)|\leq |\zeta|	 	\implies  |Sf(z)|\leq |Tz|\implies\\
		  \lt|\frac{f(z)-w_0}{f(z)-\overline{w_0}}\rt|\leq \lt|\frac{z-z_0}{z-\overline{z_0}}\rt|	\implies \lt|\frac{f(z)-f(z_0)}{f(z)-\overline{f(z_0)}}\rt|\leq \lt|\frac{z-z_0}{z-\overline{z_0}}\rt|
	\end{multline*}		
Now $$\lt|\frac{f(z)-f(z_0)}{f(z)-\overline{f(z_0)}}\rt|\leq \lt|\frac{z-z_0}{z-\overline{z_0}}\rt|\implies \lt|\frac{f(z)-f(z_0)}{z-z_0}\rt|\leq \lt|\frac{f(z)-\overline{f(z_0)}}{z-\overline{z_0}}\rt|$$By taking $z\downarrow z_0$ we have $$\lim_{z\downarrow z_0}\lt|\frac{f(z)-f(z_0)}{z-z_0}\rt|\leq\lim_{z\downarrow z_0} \lt|\frac{f(z)-\overline{f(z_0)}}{z-\overline{z_0}}\rt|\implies |f'(z_0)|\leq  \frac{2(\Im f(z_0))}{2(\Im z_0)}\implies \frac{|f'(z_0)|}{\Im f(z_0)}\leq \frac1{\Im z_0}\text{ for $\Im z_0\geq 0$}$$ Since $z_0$ is arbitrary in the upper half plane, hence for any $z=x+iy$, with $y\geq 0$ we have $$\frac{\left|f^{\prime}(z)\right|}{\Im f(z)} \leq \frac{1}{y} $$
		
	}
		%%%%%%%%%%%%%%%%%%%%%%%%%%%%%%%%%%%%%%%%%%%%%%%%%%%%%%%%%%%%%%%%%%%%%%%%%
	% Problem 7
	%%%%%%%%%%%%%%%%%%%%%%%%%%%%%%%%%%%%%%%%%%%%%%%%%%%%%%%%%%%%%%%%%%%%%%%%%
	
	\begin{problem}{%problem statement
			Ahlfors Page 136: Problem 3
		}{p7% problem reference text
		}
		%Problem
In \hyperref[p:p5]{Problem 5} and \hyperref[p:p6]{Problem 6}, prove that equality implies that $f(z)$ is a linear transformation.
	\end{problem}
	
	\solve{
		%Solution
		Take $F(\zeta)=Sf(T^{-1}\zeta)$ like in the proof of the inequality in $(36)$ in Ahlfors for some linear transformation $T,S$. Now $|F(\zeta)|\leq 1$ and $F(0)=0$. Hence if $|F(\zeta)|=|\zeta|$ then $F(z)=cz$ for some constant $c\in \bbC$ by Schwarz Lemma. Hence $$Sf(T^{-1}\zeta)=cs\iff f(z)=S^{-1}(cT(z))\quad [\text{take $z=T^{-1}(\zeta)$}]$$Therefore $f(z)$ is a linear combination
	}

	
		%%%%%%%%%%%%%%%%%%%%%%%%%%%%%%%%%%%%%%%%%%%%%%%%%%%%%%%%%%%%%%%%%%%%%%%%%
	% Problem 8
	%%%%%%%%%%%%%%%%%%%%%%%%%%%%%%%%%%%%%%%%%%%%%%%%%%%%%%%%%%%%%%%%%%%%%%%%%
	
	\begin{problem}{%problem statement
			Ahlfors Page 148: Problem 2
		}{p8% problem reference text
		}
		%Problem		
		Prove that the region obtained from a simply connected region by removing $m$ points has the connectivity $m+1$, and find a homology basis.
	\end{problem}
	
	\solve{
		%Solution
		\begin{itemize}
			\item Let the region $\Om$ is obtained from the simply connected region $S$ by removing $m$ points. Let the points are $x_1,\dots, x_m$. Hence $\bbC-\Om=(\bbC-S)\cup \{x_1,\dots,x_m\}$. 
		
		$\bbC-S$ is connected in the extended plane by definition. Hence $\bbC-S$ is counted as one connected component. Since the $m$ points are chosen from the region $S$, the singleton sets $\{x_i\}$ form connected components for all $i\in\{1,2,\dots,m\}$. Hence there are $m+1$ connected components in $\bbC-\Om$ which are $\bbC-S$ and $\{x_i\}$ $\forall\ i\in \{1,2,\dots,m\}$
			\item Since the points are distinct and finite we can take the circles as the curves around each  point choose their radius in such a way that no two curves intersect each other and no curv pass through any of those $m$ points. These $m$ many circles will be the homology basis.
		\end{itemize}
	}
		%%%%%%%%%%%%%%%%%%%%%%%%%%%%%%%%%%%%%%%%%%%%%%%%%%%%%%%%%%%%%%%%%%%%%%%%%
	% Problem 9
	%%%%%%%%%%%%%%%%%%%%%%%%%%%%%%%%%%%%%%%%%%%%%%%%%%%%%%%%%%%%%%%%%%%%%%%%%
	\pagebreak
	\begin{problem}{%problem statement
			Ahlfors Page 48: Problem 5
		}{p9% problem reference text
		}
		%Problem		
 Show that a single-valued analytic branch of $\sqrt{1-z^2}$ can be defined in any region such that the points $\pm 1$ are in the same component of the complement. What are the possible values of $$\int\frac{dz}{\sqrt{1-z^2}}$$ over a  closed  curve in the region?
	\end{problem}
	
	\solve{
		%Solution
	\begin{itemize}
		\item \parinn	Let $\Omega$ be any region. Now given that in $\bbC-\Om$, 1 and $-1$ lie in same component.  Now we have to define a single valued analytic branch of $\sqrt{1-z^2}$.
		
		First we will define an single valued analytic branch of $\log \lt( \frac{1-z}{1+z} \rt)$ in $\Om$. Let $\gm$ be any closed curve in $\Om$. Then $$\int_{\gm}\lt( \frac1{z-1}-\frac1{z+1}\rt)dz=n(\gm;1)-n(\gm;-1)=0$$ since 1, $-1$ lie in same component. Hence it has an anti-derivative function $f(z)$ which is analytic defined in the region $\Om$. Now we have $$\frac{d}{dz}\lt(\frac{z+1}{z-1}\rt)e^f(z)=e^z\lt(\frac{(z-1)-(z+1)}{(z-1)^2}+\frac{z+1}{z-1}\lt(\frac1{z-1}-\frac1{z+1}  \rt)  \rt)=0$$Hence $\frac{z+1}{z-1}e^f(z)$ is a constant function. Hence $\frac{z+1}{z-1}e^{f(z)}=cz$ for some $c\in \bbC$. Let $e^a=c$ then let $g(z)=f(z)-a$ then $$\frac{z+1}{z-1}e^{f(z)}=cz\implies e^{-a}e^{f(z)}=\frac{z-1}{z+1}\implies e^{f(z)-a}=\frac{z-1}{z+1}\implies e^{g(z)}=\frac{z-1}{z+1}$$ Hence we can define $\log \lt(\frac{z-1}{z+1}\rt)=g(z)=f(z)-a$. Hence now we have an analytic function $g(z)$ defined in $\Om$. 
		
		Now \begin{align*}
			& e^{g(z)}=\frac{z-1}{z+1}=-\frac{1-z^2}{(1+z)^2} \implies -(z+1)^2e^{g(z)}=1-z^2 \implies \Big(i(z+1)e^{\frac{g(z)}{2} } \Big)=1-z^2
		\end{align*}
	Let $h(z)=i(z+1)e^{\frac{g(z)}{2}}$ then $h(z)$ defines an analytic branch or $\sqrt{1-z^2}$. 
	
	$\sqrt{1-z^2}$ vanishes at both 1 and $-1$. If $\gm$ doesn't pass through $[-1,1]$ interval, removing the branch cut, we can pick an analytic branch of $\sqrt{1-z^2}$ in $\bbC-[-1,1]$. For our region we can extend  the branch of $\sqrt{1-z^2}$ to $\bbC-[-1,1]$ in this case. So we can replace the integrand function with the extended branch and assume that it is analytic in $\bbC-[-1,1]$ and $\Om=\bbC-[-1,1]$. And WLOG we can also assume that the component in which 1,$-1$ lies, is the line segment $[-1,1]$. Now we can take any circle of radius $r$ where $r>1$ to be the homology basis  for the component of the line segment $[-1,1]$.
	
	Now suppose $\gm$ passes through $[-1,1]$. Now there exists another path $p$ which is not cut by $\gm$. Now we extend the branch to $\bbC-p$  and take circle of radius $r$  such that path $p$ lies in side the disk of radius $r$. This circle will form a homology basis  corresponding to $p$.
	\item Since $\lim\limits_{z\to \infty}\frac{1-z^2}{z^2}=-1$ we have $\lim\limits_{z\to \infty}\frac{\sqrt{1-z^2}}{z}=i\text{ or }-i$ Hence for the first case $$\frac{1}{\sqrt{1-z^2}}-\frac{1}{iz}=\frac{iz-\sqrt{1-z^2}}{iz\sqrt{1-z^2}}=O(z^{-3})$$Similarly in the second case $$\frac{1}{\sqrt{1-z^2}}+\frac{1}{iz}=\frac{iz+\sqrt{1-z^2}}{iz\sqrt{1-z^2}}=O(z^{-3})$$ Hence we have $\frac1{\sqrt{1-z^2}}=\pm\frac1{iz}+O(z^{-3})$. Hence $$\int_{\gm}\frac{dz}{\sqrt{1-z^2}}=n(\gm;0)\int_{|z|=R}\frac{dz}{\sqrt{1-z^2}}=n(\gm;0)\lim_{R\to \infty}\int_{|z|=R}\lt( \pm\frac1{iz}+O(z^3) \rt)dz=n(\gm;0)\lt(\underbrace{\pm\int_{|z|=R}\frac{dz}{iz}}_{\pm 2\pi}\rt)$$Hence $\int\limits_{\gm}\frac{dz}{\sqrt{1-z^2}}$ is integral multiple of $2\pi$
	
	\end{itemize}
}
		

	
\end{document}
