\documentclass[a4paper, 11pt]{article}
\usepackage{comment} % enables the use of multi-line comments (\ifx \fi) 
\usepackage{fullpage} % changes the margin
\usepackage[a4paper, total={7in, 10in}]{geometry}
\usepackage{amsmath,mathtools}
\usepackage{amssymb,amsthm}  % assumes amsmath package installed
\usepackage{float}
\usepackage{xcolor}
\usepackage{mdframed}
\usepackage[shortlabels]{enumitem}
\usepackage{indentfirst}
\usepackage{hyperref}
\hypersetup{
	colorlinks=true,
	linkcolor=blue,
	filecolor=magenta,      
	urlcolor=blue!70!red,
	pdftitle={Complexity Assignment}, %%%%%%%%%%%%%%%%   WRITE ASSIGNMENT PDF NAME  %%%%%%%%%%%%%%%%%%%%
}
\usepackage[most,many,breakable]{tcolorbox}



\definecolor{mytheorembg}{HTML}{F2F2F9}
\definecolor{mytheoremfr}{HTML}{00007B}


\tcbuselibrary{theorems,skins,hooks}
\newtcbtheorem{problem}{Problem}
{%
	enhanced,
	breakable,
	colback = mytheorembg,
	frame hidden,
	boxrule = 0sp,
	borderline west = {2pt}{0pt}{mytheoremfr},
	sharp corners,
	detach title,
	before upper = \tcbtitle\par\smallskip,
	coltitle = mytheoremfr,
	fonttitle = \bfseries\sffamily,
	description font = \mdseries,
	separator sign none,
	segmentation style={solid, mytheoremfr},
}
{p}

% To give references for any problem use like this
% suppose the problem number is p3 then 2 options either 
% \hyperref[p:p3]{<text you want to use to hyperlink> \ref{p:p3}}
%                  or directly 
%                   \ref{p:p3}



%---------------------------------------
% BlackBoard Math Fonts :-
%---------------------------------------

%Captital Letters
\newcommand{\bbA}{\mathbb{A}}	\newcommand{\bbB}{\mathbb{B}}
\newcommand{\bbC}{\mathbb{C}}	\newcommand{\bbD}{\mathbb{D}}
\newcommand{\bbE}{\mathbb{E}}	\newcommand{\bbF}{\mathbb{F}}
\newcommand{\bbG}{\mathbb{G}}	\newcommand{\bbH}{\mathbb{H}}
\newcommand{\bbI}{\mathbb{I}}	\newcommand{\bbJ}{\mathbb{J}}
\newcommand{\bbK}{\mathbb{K}}	\newcommand{\bbL}{\mathbb{L}}
\newcommand{\bbM}{\mathbb{M}}	\newcommand{\bbN}{\mathbb{N}}
\newcommand{\bbO}{\mathbb{O}}	\newcommand{\bbP}{\mathbb{P}}
\newcommand{\bbQ}{\mathbb{Q}}	\newcommand{\bbR}{\mathbb{R}}
\newcommand{\bbS}{\mathbb{S}}	\newcommand{\bbT}{\mathbb{T}}
\newcommand{\bbU}{\mathbb{U}}	\newcommand{\bbV}{\mathbb{V}}
\newcommand{\bbW}{\mathbb{W}}	\newcommand{\bbX}{\mathbb{X}}
\newcommand{\bbY}{\mathbb{Y}}	\newcommand{\bbZ}{\mathbb{Z}}

%---------------------------------------
% MathCal Fonts :-
%---------------------------------------

%Captital Letters
\newcommand{\mcA}{\mathcal{A}}	\newcommand{\mcB}{\mathcal{B}}
\newcommand{\mcC}{\mathcal{C}}	\newcommand{\mcD}{\mathcal{D}}
\newcommand{\mcE}{\mathcal{E}}	\newcommand{\mcF}{\mathcal{F}}
\newcommand{\mcG}{\mathcal{G}}	\newcommand{\mcH}{\mathcal{H}}
\newcommand{\mcI}{\mathcal{I}}	\newcommand{\mcJ}{\mathcal{J}}
\newcommand{\mcK}{\mathcal{K}}	\newcommand{\mcL}{\mathcal{L}}
\newcommand{\mcM}{\mathcal{M}}	\newcommand{\mcN}{\mathcal{N}}
\newcommand{\mcO}{\mathcal{O}}	\newcommand{\mcP}{\mathcal{P}}
\newcommand{\mcQ}{\mathcal{Q}}	\newcommand{\mcR}{\mathcal{R}}
\newcommand{\mcS}{\mathcal{S}}	\newcommand{\mcT}{\mathcal{T}}
\newcommand{\mcU}{\mathcal{U}}	\newcommand{\mcV}{\mathcal{V}}
\newcommand{\mcW}{\mathcal{W}}	\newcommand{\mcX}{\mathcal{X}}
\newcommand{\mcY}{\mathcal{Y}}	\newcommand{\mcZ}{\mathcal{Z}}



%---------------------------------------
% Bold Math Fonts :-
%---------------------------------------

%Captital Letters
\newcommand{\bmA}{\boldsymbol{A}}	\newcommand{\bmB}{\boldsymbol{B}}
\newcommand{\bmC}{\boldsymbol{C}}	\newcommand{\bmD}{\boldsymbol{D}}
\newcommand{\bmE}{\boldsymbol{E}}	\newcommand{\bmF}{\boldsymbol{F}}
\newcommand{\bmG}{\boldsymbol{G}}	\newcommand{\bmH}{\boldsymbol{H}}
\newcommand{\bmI}{\boldsymbol{I}}	\newcommand{\bmJ}{\boldsymbol{J}}
\newcommand{\bmK}{\boldsymbol{K}}	\newcommand{\bmL}{\boldsymbol{L}}
\newcommand{\bmM}{\boldsymbol{M}}	\newcommand{\bmN}{\boldsymbol{N}}
\newcommand{\bmO}{\boldsymbol{O}}	\newcommand{\bmP}{\boldsymbol{P}}
\newcommand{\bmQ}{\boldsymbol{Q}}	\newcommand{\bmR}{\boldsymbol{R}}
\newcommand{\bmS}{\boldsymbol{S}}	\newcommand{\bmT}{\boldsymbol{T}}
\newcommand{\bmU}{\boldsymbol{U}}	\newcommand{\bmV}{\boldsymbol{V}}
\newcommand{\bmW}{\boldsymbol{W}}	\newcommand{\bmX}{\boldsymbol{X}}
\newcommand{\bmY}{\boldsymbol{Y}}	\newcommand{\bmZ}{\boldsymbol{Z}}
%Small Letters
\newcommand{\bma}{\boldsymbol{a}}	\newcommand{\bmb}{\boldsymbol{b}}
\newcommand{\bmc}{\boldsymbol{c}}	\newcommand{\bmd}{\boldsymbol{d}}
\newcommand{\bme}{\boldsymbol{e}}	\newcommand{\bmf}{\boldsymbol{f}}
\newcommand{\bmg}{\boldsymbol{g}}	\newcommand{\bmh}{\boldsymbol{h}}
\newcommand{\bmi}{\boldsymbol{i}}	\newcommand{\bmj}{\boldsymbol{j}}
\newcommand{\bmk}{\boldsymbol{k}}	\newcommand{\bml}{\boldsymbol{l}}
\newcommand{\bmm}{\boldsymbol{m}}	\newcommand{\bmn}{\boldsymbol{n}}
\newcommand{\bmo}{\boldsymbol{o}}	\newcommand{\bmp}{\boldsymbol{p}}
\newcommand{\bmq}{\boldsymbol{q}}	\newcommand{\bmr}{\boldsymbol{r}}
\newcommand{\bms}{\boldsymbol{s}}	\newcommand{\bmt}{\boldsymbol{t}}
\newcommand{\bmu}{\boldsymbol{u}}	\newcommand{\bmv}{\boldsymbol{v}}
\newcommand{\bmw}{\boldsymbol{w}}	\newcommand{\bmx}{\boldsymbol{x}}
\newcommand{\bmy}{\boldsymbol{y}}	\newcommand{\bmz}{\boldsymbol{z}}


%---------------------------------------
% Scr Math Fonts :-
%---------------------------------------

\newcommand{\sA}{{\mathscr{A}}}   \newcommand{\sB}{{\mathscr{B}}}
\newcommand{\sC}{{\mathscr{C}}}   \newcommand{\sD}{{\mathscr{D}}}
\newcommand{\sE}{{\mathscr{E}}}   \newcommand{\sF}{{\mathscr{F}}}
\newcommand{\sG}{{\mathscr{G}}}   \newcommand{\sH}{{\mathscr{H}}}
\newcommand{\sI}{{\mathscr{I}}}   \newcommand{\sJ}{{\mathscr{J}}}
\newcommand{\sK}{{\mathscr{K}}}   \newcommand{\sL}{{\mathscr{L}}}
\newcommand{\sM}{{\mathscr{M}}}   \newcommand{\sN}{{\mathscr{N}}}
\newcommand{\sO}{{\mathscr{O}}}   \newcommand{\sP}{{\mathscr{P}}}
\newcommand{\sQ}{{\mathscr{Q}}}   \newcommand{\sR}{{\mathscr{R}}}
\newcommand{\sS}{{\mathscr{S}}}   \newcommand{\sT}{{\mathscr{T}}}
\newcommand{\sU}{{\mathscr{U}}}   \newcommand{\sV}{{\mathscr{V}}}
\newcommand{\sW}{{\mathscr{W}}}   \newcommand{\sX}{{\mathscr{X}}}
\newcommand{\sY}{{\mathscr{Y}}}   \newcommand{\sZ}{{\mathscr{Z}}}


%---------------------------------------
% Math Fraktur Font
%---------------------------------------

%Captital Letters
\newcommand{\mfA}{\mathfrak{A}}	\newcommand{\mfB}{\mathfrak{B}}
\newcommand{\mfC}{\mathfrak{C}}	\newcommand{\mfD}{\mathfrak{D}}
\newcommand{\mfE}{\mathfrak{E}}	\newcommand{\mfF}{\mathfrak{F}}
\newcommand{\mfG}{\mathfrak{G}}	\newcommand{\mfH}{\mathfrak{H}}
\newcommand{\mfI}{\mathfrak{I}}	\newcommand{\mfJ}{\mathfrak{J}}
\newcommand{\mfK}{\mathfrak{K}}	\newcommand{\mfL}{\mathfrak{L}}
\newcommand{\mfM}{\mathfrak{M}}	\newcommand{\mfN}{\mathfrak{N}}
\newcommand{\mfO}{\mathfrak{O}}	\newcommand{\mfP}{\mathfrak{P}}
\newcommand{\mfQ}{\mathfrak{Q}}	\newcommand{\mfR}{\mathfrak{R}}
\newcommand{\mfS}{\mathfrak{S}}	\newcommand{\mfT}{\mathfrak{T}}
\newcommand{\mfU}{\mathfrak{U}}	\newcommand{\mfV}{\mathfrak{V}}
\newcommand{\mfW}{\mathfrak{W}}	\newcommand{\mfX}{\mathfrak{X}}
\newcommand{\mfY}{\mathfrak{Y}}	\newcommand{\mfZ}{\mathfrak{Z}}
%Small Letters
\newcommand{\mfa}{\mathfrak{a}}	\newcommand{\mfb}{\mathfrak{b}}
\newcommand{\mfc}{\mathfrak{c}}	\newcommand{\mfd}{\mathfrak{d}}
\newcommand{\mfe}{\mathfrak{e}}	\newcommand{\mff}{\mathfrak{f}}
\newcommand{\mfg}{\mathfrak{g}}	\newcommand{\mfh}{\mathfrak{h}}
\newcommand{\mfi}{\mathfrak{i}}	\newcommand{\mfj}{\mathfrak{j}}
\newcommand{\mfk}{\mathfrak{k}}	\newcommand{\mfl}{\mathfrak{l}}
\newcommand{\mfm}{\mathfrak{m}}	\newcommand{\mfn}{\mathfrak{n}}
\newcommand{\mfo}{\mathfrak{o}}	\newcommand{\mfp}{\mathfrak{p}}
\newcommand{\mfq}{\mathfrak{q}}	\newcommand{\mfr}{\mathfrak{r}}
\newcommand{\mfs}{\mathfrak{s}}	\newcommand{\mft}{\mathfrak{t}}
\newcommand{\mfu}{\mathfrak{u}}	\newcommand{\mfv}{\mathfrak{v}}
\newcommand{\mfw}{\mathfrak{w}}	\newcommand{\mfx}{\mathfrak{x}}
\newcommand{\mfy}{\mathfrak{y}}	\newcommand{\mfz}{\mathfrak{z}}

%---------------------------------------
% Bar
%---------------------------------------

%Captital Letters
\newcommand{\ovA}{\overline{A}}	\newcommand{\ovB}{\overline{B}}
\newcommand{\ovC}{\overline{C}}	\newcommand{\ovD}{\overline{D}}
\newcommand{\ovE}{\overline{E}}	\newcommand{\ovF}{\overline{F}}
\newcommand{\ovG}{\overline{G}}	\newcommand{\ovH}{\overline{H}}
\newcommand{\ovI}{\overline{I}}	\newcommand{\ovJ}{\overline{J}}
\newcommand{\ovK}{\overline{K}}	\newcommand{\ovL}{\overline{L}}
\newcommand{\ovM}{\overline{M}}	\newcommand{\ovN}{\overline{N}}
\newcommand{\ovO}{\overline{O}}	\newcommand{\ovP}{\overline{P}}
\newcommand{\ovQ}{\overline{Q}}	\newcommand{\ovR}{\overline{R}}
\newcommand{\ovS}{\overline{S}}	\newcommand{\ovT}{\overline{T}}
\newcommand{\ovU}{\overline{U}}	\newcommand{\ovV}{\overline{V}}
\newcommand{\ovW}{\overline{W}}	\newcommand{\ovX}{\overline{X}}
\newcommand{\ovY}{\overline{Y}}	\newcommand{\ovZ}{\overline{Z}}
%Small Letters
\newcommand{\ova}{\overline{a}}	\newcommand{\ovb}{\overline{b}}
\newcommand{\ovc}{\overline{c}}	\newcommand{\ovd}{\overline{d}}
\newcommand{\ove}{\overline{e}}	\newcommand{\ovf}{\overline{f}}
\newcommand{\ovg}{\overline{g}}	\newcommand{\ovh}{\overline{h}}
\newcommand{\ovi}{\overline{i}}	\newcommand{\ovj}{\overline{j}}
\newcommand{\ovk}{\overline{k}}	\newcommand{\ovl}{\overline{l}}
\newcommand{\ovm}{\overline{m}}	\newcommand{\ovn}{\overline{n}}
\newcommand{\ovo}{\overline{o}}	\newcommand{\ovp}{\overline{p}}
\newcommand{\ovq}{\overline{q}}	\newcommand{\ovr}{\overline{r}}
\newcommand{\ovs}{\overline{s}}	\newcommand{\ovt}{\overline{t}}
\newcommand{\ovu}{\overline{u}}	\newcommand{\ovv}{\overline{v}}
\newcommand{\ovw}{\overline{w}}	\newcommand{\ovx}{\overline{x}}
\newcommand{\ovy}{\overline{y}}	\newcommand{\ovz}{\overline{z}}

%---------------------------------------
% Tilde
%---------------------------------------

%Captital Letters
\newcommand{\tdA}{\tilde{A}}	\newcommand{\tdB}{\tilde{B}}
\newcommand{\tdC}{\tilde{C}}	\newcommand{\tdD}{\tilde{D}}
\newcommand{\tdE}{\tilde{E}}	\newcommand{\tdF}{\tilde{F}}
\newcommand{\tdG}{\tilde{G}}	\newcommand{\tdH}{\tilde{H}}
\newcommand{\tdI}{\tilde{I}}	\newcommand{\tdJ}{\tilde{J}}
\newcommand{\tdK}{\tilde{K}}	\newcommand{\tdL}{\tilde{L}}
\newcommand{\tdM}{\tilde{M}}	\newcommand{\tdN}{\tilde{N}}
\newcommand{\tdO}{\tilde{O}}	\newcommand{\tdP}{\tilde{P}}
\newcommand{\tdQ}{\tilde{Q}}	\newcommand{\tdR}{\tilde{R}}
\newcommand{\tdS}{\tilde{S}}	\newcommand{\tdT}{\tilde{T}}
\newcommand{\tdU}{\tilde{U}}	\newcommand{\tdV}{\tilde{V}}
\newcommand{\tdW}{\tilde{W}}	\newcommand{\tdX}{\tilde{X}}
\newcommand{\tdY}{\tilde{Y}}	\newcommand{\tdZ}{\tilde{Z}}
%Small Letters
\newcommand{\tda}{\tilde{a}}	\newcommand{\tdb}{\tilde{b}}
\newcommand{\tdc}{\tilde{c}}	\newcommand{\tdd}{\tilde{d}}
\newcommand{\tde}{\tilde{e}}	\newcommand{\tdf}{\tilde{f}}
\newcommand{\tdg}{\tilde{g}}	\newcommand{\tdh}{\tilde{h}}
\newcommand{\tdi}{\tilde{i}}	\newcommand{\tdj}{\tilde{j}}
\newcommand{\tdk}{\tilde{k}}	\newcommand{\tdl}{\tilde{l}}
\newcommand{\tdm}{\tilde{m}}	\newcommand{\tdn}{\tilde{n}}
\newcommand{\tdo}{\tilde{o}}	\newcommand{\tdp}{\tilde{p}}
\newcommand{\tdq}{\tilde{q}}	\newcommand{\tdr}{\tilde{r}}
\newcommand{\tds}{\tilde{s}}	\newcommand{\tdt}{\tilde{t}}
\newcommand{\tdu}{\tilde{u}}	\newcommand{\tdv}{\tilde{v}}
\newcommand{\tdw}{\tilde{w}}	\newcommand{\tdx}{\tilde{x}}
\newcommand{\tdy}{\tilde{y}}	\newcommand{\tdz}{\tilde{z}}

%---------------------------------------
% Vec
%---------------------------------------

%Captital Letters
\newcommand{\vcA}{\vec{A}}	\newcommand{\vcB}{\vec{B}}
\newcommand{\vcC}{\vec{C}}	\newcommand{\vcD}{\vec{D}}
\newcommand{\vcE}{\vec{E}}	\newcommand{\vcF}{\vec{F}}
\newcommand{\vcG}{\vec{G}}	\newcommand{\vcH}{\vec{H}}
\newcommand{\vcI}{\vec{I}}	\newcommand{\vcJ}{\vec{J}}
\newcommand{\vcK}{\vec{K}}	\newcommand{\vcL}{\vec{L}}
\newcommand{\vcM}{\vec{M}}	\newcommand{\vcN}{\vec{N}}
\newcommand{\vcO}{\vec{O}}	\newcommand{\vcP}{\vec{P}}
\newcommand{\vcQ}{\vec{Q}}	\newcommand{\vcR}{\vec{R}}
\newcommand{\vcS}{\vec{S}}	\newcommand{\vcT}{\vec{T}}
\newcommand{\vcU}{\vec{U}}	\newcommand{\vcV}{\vec{V}}
\newcommand{\vcW}{\vec{W}}	\newcommand{\vcX}{\vec{X}}
\newcommand{\vcY}{\vec{Y}}	\newcommand{\vcZ}{\vec{Z}}
%Small Letters
\newcommand{\vca}{\vec{a}}	\newcommand{\vcb}{\vec{b}}
\newcommand{\vcc}{\vec{c}}	\newcommand{\vcd}{\vec{d}}
\newcommand{\vce}{\vec{e}}	\newcommand{\vcf}{\vec{f}}
\newcommand{\vcg}{\vec{g}}	\newcommand{\vch}{\vec{h}}
\newcommand{\vci}{\vec{i}}	\newcommand{\vcj}{\vec{j}}
\newcommand{\vck}{\vec{k}}	\newcommand{\vcl}{\vec{l}}
\newcommand{\vcm}{\vec{m}}	\newcommand{\vcn}{\vec{n}}
\newcommand{\vco}{\vec{o}}	\newcommand{\vcp}{\vec{p}}
\newcommand{\vcq}{\vec{q}}	\newcommand{\vcr}{\vec{r}}
\newcommand{\vcs}{\vec{s}}	\newcommand{\vct}{\vec{t}}
\newcommand{\vcu}{\vec{u}}	\newcommand{\vcv}{\vec{v}}
%\newcommand{\vcw}{\vec{w}}	\newcommand{\vcx}{\vec{x}}
\newcommand{\vcy}{\vec{y}}	\newcommand{\vcz}{\vec{z}}

%---------------------------------------
% Greek Letters:-
%---------------------------------------
\newcommand{\eps}{\epsilon}
\newcommand{\veps}{\varepsilon}
\newcommand{\lm}{\lambda}
\newcommand{\Lm}{\Lambda}
\newcommand{\gm}{\gamma}
\newcommand{\Gm}{\Gamma}
\newcommand{\vph}{\varphi}
\newcommand{\ph}{\phi}
\newcommand{\om}{\omega}
\newcommand{\Om}{\Omega}
\newcommand{\sg}{\sigma}
\newcommand{\Sg}{\Sigma}

\newcommand{\Qed}{\begin{flushright}\qed\end{flushright}}
\newcommand{\parinn}{\setlength{\parindent}{1cm}}
\newcommand{\parinf}{\setlength{\parindent}{0cm}}
\newcommand{\del}[2]{\frac{\partial #1}{\partial #2}}
\newcommand{\Del}[3]{\frac{\partial^{#1} #2}{\partial^{#1} #3}}
\newcommand{\deld}[2]{\dfrac{\partial #1}{\partial #2}}
\newcommand{\Deld}[3]{\dfrac{\partial^{#1} #2}{\partial^{#1} #3}}
\newcommand{\uin}{\mathbin{\rotatebox[origin=c]{90}{$\in$}}}
\newcommand{\usubset}{\mathbin{\rotatebox[origin=c]{90}{$\subset$}}}
\newcommand{\lt}{\left}
\newcommand{\rt}{\right}
\newcommand{\exs}{\exists}
\newcommand{\st}{\strut}
\newcommand{\dps}[1]{\displaystyle{#1}}
\newcommand{\la}{\langle}
\newcommand{\ra}{\rangle}
\newcommand{\cls}[1]{\textsc{#1}}
\newcommand{\prb}[1]{\textsc{#1}}
\newcommand{\comb}[2]{\left(\begin{matrix}
		#1\\ #2
\end{matrix}\right)}
%\newcommand[2]{\quotient}{\faktor{#1}{#2}}
\newcommand\quotient[2]{
	\mathchoice
	{% \displaystyle
		\text{\raise1ex\hbox{$#1$}\Big/\lower1ex\hbox{$#2$}}%
	}
	{% \textstyle
		#1\,/\,#2
	}
	{% \scriptstyle
		#1\,/\,#2
	}
	{% \scriptscriptstyle  
		#1\,/\,#2
	}
}

\newcommand{\tensor}{\otimes}
\newcommand{\xor}{\oplus}

\newcommand{\sol}[1]{\begin{solution}#1\end{solution}}
\newcommand{\solve}[1]{\setlength{\parindent}{0cm}\textbf{\textit{Solution: }}\setlength{\parindent}{1cm}#1 \hfill $\blacksquare$}
\newcommand{\mat}[1]{\left[\begin{matrix}#1\end{matrix}\right]}
\newcommand{\matr}[1]{\begin{matrix}#1\end{matrix}}
\newcommand{\matp}[1]{\lt(\begin{matrix}#1\end{matrix}\rt)}
\newcommand{\detmat}[1]{\lt|\begin{matrix}#1\end{matrix}\rt|}
\newcommand\numberthis{\addtocounter{equation}{1}\tag{\theequation}}
\newcommand{\handout}[3]{
	\noindent
	\begin{center}
		\framebox{
			\vbox{
				\hbox to 6.5in { {\bf Complexity Theory I } \hfill Jan -- May, 2023 }
				\vspace{4mm}
				\hbox to 6.5in { {\Large \hfill #1  \hfill} }
				\vspace{2mm}
				\hbox to 6.5in { {\em #2 \hfill #3} }
			}
		}
	\end{center}
	\vspace*{4mm}
}

\newcommand{\lecture}[3]{\handout{Lecture #1}{Lecturer: #2}{Scribe:	#3}}

\let\marvosymLightning\Lightning
\newcommand{\ctr}{\text{\marvosymLightning}\hspace{0.5ex}} % Requires marvosym package

\newcommand{\ov}[1]{\overline{#1}}
\newcommand{\thmref}[1]{\hyperref[th:#1]{Theorem \ref{th:#1}}}
\newcommand{\propref}[1]{\hyperref[th:#1]{Proposition \ref{th:#1}}}
\newcommand{\lmref}[1]{\hyperref[th:#1]{Lemma \ref{th:#1}}}
\newcommand{\corref}[1]{\hyperref[th:#1]{Corollary \ref{th:#1}}}

\newcommand{\thrmref}[1]{\hyperref[#1]{Theorem \ref{#1}}}
\newcommand{\propnref}[1]{\hyperref[#1]{Proposition \ref{#1}}}
\newcommand{\lemref}[1]{\hyperref[#1]{Lemma \ref{#1}}}
\newcommand{\corrref}[1]{\hyperref[#1]{Corollary \ref{#1}}}

\DeclareMathOperator{\enc}{Enc}
\DeclareMathOperator{\res}{Res}
\DeclareMathOperator{\spec}{Spec}
\DeclareMathOperator{\cov}{Cov}
\DeclareMathOperator{\Var}{Var}
\DeclareMathOperator{\Rank}{rank}
\newcommand{\Tfae}{The following are equivalent:}
\newcommand{\tfae}{the following are equivalent:}
\newcommand{\sparsity}{\textit{sparsity}}

\newcommand{\uddots}{\reflectbox{$\ddots$}} 

\newenvironment{claimwidth}{\begin{center}\begin{adjustwidth}{0.05\textwidth}{0.05\textwidth}}{\end{adjustwidth}\end{center}}

\setlength{\parindent}{0pt}

%%%%%%%%%%%%%%%%%%%%%%%%%%%%%%%%%%%%%%%%%%%%%%%%%%%%%%%%%%%%%%%%%%%%%%%%%%%%%%%%%%%%%%%%%%%%%%%%%%%%%%%%%%%%%%%%%%%%%%%%%%%%%%%%%%%%%%%%

\begin{document}
	
	%%%%%%%%%%%%%%%%%%%%%%%%%%%%%%%%%%%%%%%%%%%%%%%%%%%%%%%%%%%%%%%%%%%%%%%%%%%%%%%%%%%%%%%%%%%%%%%%%%%%%%%%%%%%%%%%%%%%%%%%%%%%%%%%%%%%%%%%
	
\textsf{\noindent \large\textbf{Soham Chatterjee} \& \textbf{Rishav Gupta} \hfill \textbf{Assignment - 1}\\
    Email: \href{sohamc@cmi.ac.in}{sohamc@cmi.ac.in} \& \href{rishavg@cmi.ac.in}{rishavg@cmi.ac.in}\hfill Roll: BMC202175 \& BMC202145\\
		\normalsize Course: Complexity Theory 1 \hfill Date: February 10, 2023}
	
	%%%%%%%%%%%%%%%%%%%%%%%%%%%%%%%%%%%%%%%%%%%%%%%%%%%%%%%%%%%%%%%%%%%%%%%%%%%%%%%%%%%%%%%%%%%%%%%%%%%%%%%%%%%%%%%%%%%%%%%%%%%%%%%%%%%%%%%%
	% Problem 1
	%%%%%%%%%%%%%%%%%%%%%%%%%%%%%%%%%%%%%%%%%%%%%%%%%%%%%%%%%%%%%%%%%%%%%%%%%%%%%%%%%%%%%%%%%%%%%%%%%%%%%%%%%%%%%%%%%%%%%%%%%%%%%%%%%%%%%%%%
	
	\begin{problem}{%problem statement
		}{p1% problem reference text
		}
		Prove that if a language L is in $DTIME(T(n))$ then it can be solved in time $T^{2}(n)$ by a one-tape TM.		
	\end{problem}
	
	\solve{
		%Solution
		Let $M$ be a turing machine with $k$ tapes which decides $L\in DTIME(T(n))$ in $T(n)$ steps. Now let $\mcM$ be a single tape turing machine. 
		\subsubsection*{Encoding $M$'s tapes on  $\mcM$'s tape:}\parinf
		The TM $\mcM$ encodes $k$ tapes of $M$ on a single tape by using locations $1, k+1,2 k+1, \dots$ to encode the first tape, l, locations $2, k+2,2k+2 \dots$ to encode the second tape, locations $3, k+3,2k+3 \dots$ to encode the third tape and so on.\parinn 
		
		For every symbol $a$ in $M$ 's alphabet, $\mcM$ will contain both the symbol $a$ and the symbol $\hat{a}$. In the encoding of each tape, exactly one symbol will be of the two types of symbols. $\hat{a}$  indicates that the corresponding head of $M$ is positioned in that location.
		
		 $\mcM$ will not touch the first $n+1$ locations of its tape (where the input is located) but rather start by taking $O\left(n^2\right)$ steps to copy the input bit by bit into the rest of the tape, while encoding it in the above way.
		 \subsubsection*{Simulation:}\parinf
		
		To simulate one step of $M$, the machine $\mcM$ makes two sweeps of its work tape: First it sweeps the tape in the left-to-right direction and records to its register the $k$ symbols that are marked by. Then $\mcM$ uses $M$'s transition function to determine the new state, symbols, and head movements and sweeps the tape back in the right-to-left direction to update the encoding accordingly. Clearly, $\mcM$ will have the same output as $M$.		
		\subsubsection*{Complexity Analysis:}
		Since on $n$-length inputs $M$ never reaches more than location $T(n)$ of any of its tapes, $\mcM$	 will never need to reach more than location $2n+k T(n) \leq (k+2) T(n)$ of its work tape, meaning that for each of the at most $T(n)$ steps of $M$, $\mcM$ performs at most $5kT(n)$ steps (sweeping back and forth requires about $4 \cdot k \cdot T(n)$ steps, and some additional steps may be needed for updating head movement and book keeping). SO total time it takes $5kT(n)\times T(n)=5kT^2(n)=O(T^(n))$ time.\parinn
		
		Hence $L$ can be solved in time $T^2(n)$ by a one-tape turing machine.
	}
	
	
	%%%%%%%%%%%%%%%%%%%%%%%%%%%%%%%%%%%%%%%%%%%%%%%%%%%%%%%%%%%%%%%%%%%%%%%%%
	% Problem 2

	%%%%%%%%%%%%%%%%%%%%%%%%%%%%%%%%%%%%%%%%%%%%%%%%%%%%%%%%%%%%%%%%%%%%%%%%%
	
	\begin{problem}{%problem statement
		}{p2% problem reference text
		}
		%Problem		
			Prove that if $L$ is accepted by a $k$ tape turing machine in $T (n)$ time then it
		can be also accepted by a two tape turing machine in $T (n) \log T (n)$ time.
	\end{problem}
	
	\solve{
		%Solution
		First we will prove a lemma.\parinf
		
		\textbf{\textit{Lemma: }}Define a bidirectional Turing machine to be a Turing Machine whose tapes are infinite in both directions. For every $f:\{0,1\}^*\to \{0,1\}$ and time-constructible $T:\mathbb{N}\to \mathbb{N}$ if $f$ is computable in time $T(n)$ by a bidirectional Turing Machine $M$ using $k$ tapes then it is computable in time $4T(n)$ by a standard Turing Machine $\hat{M}$.
		
		\textbf{\textit{Proof: }}The proof uses the idea of folding the bidirectional tape into an uni-directional tape . For implementing this idea we use the following construction . Let $T$ be our original bidirectional tape $TM$ with alphabets in $\Sigma\cup\{\_\}$ ,we construct $T'$ with alphabets in the set $\Sigma\cup\{\_\} \times \Sigma\cup\{\_\}$ .Now we enumerate the start cell of $T$ and $T'$ both with $0$ and the cell on right with $1,2\dots$ and the cells on left for $T$ with $-1,-2,\dots$ .Now the $ith$ cell of $T'$ will store two values one for the $ith$ cell of $T$ and other for the $-ith$ cell of $T$ that is the $ith $ cell content of $T'$ = ($ith$ cell content of $T$,$-ith$ cell content of $T$). Now if wee are at 0 th position and then the head moves left in $T$, then in $T'$ we will move right and make the change in the second coordinate and keep moving right and modifying the second coordinate  in $T'$ if we are moving in left in $T$ while being in a position with index $\leq$ 0, if we move right in $T$ while being at index$<0$ we will move left in $T'$ and modify the second coordinate, if we are at an index$>$0 in $T$ we will move accordingly in $T'$ modifying the first coordinate . Now by this process we can simulate a bidirectional $TM$ $T$ with an uni-directional $TM $ $T'$ without any blow up in time because for one shift in $T$ we are making exactly one shift in $T'$.       \Qed
		
		
		Let $L$ is accepted by the turing machine $M$ which is a $k$ tape turing machine in time $T(n)$ (we will write it as $T$). Therefore $M$'s one tape is input tape, one tape is output tape and rest are work tapes Let $\mathcal{M}$ be a  two tape turing machine. So one of the tape is input tape and the other one is work tape/output tape. Hence we will simulate $k-1$ tapes of $M$ in one tape of $\mcM$
		\subsubsection*{Encoding $M$'s tapes on  $\mcM$'s tape:}\parinf We add a special kind of blank symbol $\boxtimes$ to the alphabet of $M$’s parallel tapes with the properties that this symbol is ignored in the simulation. For example, if $M$'s tape contents are $010$ then, this can be encoded as $0\boxtimes 10$ or $0\boxtimes\boxtimes1\boxtimes 0$ and so on or just $010$\parinn
		
		For convenience, we think of $\mathcal{M}$'s parallel tapes as infinite in both the left and right directions by the Lemma. Thus we index their locations by $0,\pm 1,\pm 2,\dots$ Normally we keep $\mathcal{M}$'s head on location 0 of these parallel tapes. We will only move it temporarily to perform a shift when, following our general approach, we simulate a left hand movement by sifting the tape to the right and vice versa. At the end of the shift, we return the head to location 0.
		
		
		We split each of $\mathcal{M}$'s parallel tapes into zones that we denote by $R_0$, $L_0$, $R_1$, $L_1,\dots$ (We'll only need to go up to $R_{\log T}$, $L_{\log T}$ since the turing machine runs for time $T$, it can at most access $T$ cells of each tape.). The cell at location 0 is not at any zone. Zone $R_0$ contains two cells immediately right of location 0 (i.e. location $+1,+$). Zone $R_1$ contains the four cells $+3,+4,+5,+6$. Generally for every $i\geq 1$, Zone $R_1$ contains the $2\times 2^i$ cells that are right to the $R_{i-1}$ (i.e. locations $2^{i+1}-1,\dots, 2^{i+2}-2$). Similarly Zone $L_0$ contains two cells indexed by $-1$ and $-2$ and generally  Zone $L_i$ contains the cells $-2^{i+2}+2,\dots,- 2^{i+1}+1$
		
		\subsubsection*{Invariants:}\parinf We shall always maintain the following invariants:
		\begin{itemize}
			\item Each of the zones is either empty, full, or half-full with non-$\boxtimes$-symbols. That is, the number of symbols in zone $R_i$ that are not $\boxtimes$ is either $0$, $2^i$, or $2\times  2^i$ and the same holds for $L_i$ . (We treat the ordinary symbol $\square$ the same as any other symbol in $\Gamma$, and in particular a zone full of $\square$’s is considered full). We assume that initially all the zones are half-full. We can ensure this by filling half of each zone with $\boxtimes$ symbols in the first time we encounter it.
			\item The total number of non-$\boxtimes$-symbols in $R_i \cup L_i$ is $2 \times 2^i$. That is, either $R_i$ is empty and $L_i$ is full, or $R_i$ is full and $L_i$ is empty, or they are both half-full.
			\item Location 0 always contains a non-$\boxtimes$-symbol.
		\end{itemize}
		
		
		
		\subsubsection*{Performing a Shift:}\parinf
		Now when performing the shifts, we do not always have to move the entire tape, but we can restrict ourselves to only using some of the zones.\parinn
		
		We illustrate this by showing how $\mathcal{M}$ performs a left shift on the first of its parallel tapes
		
		\begin{enumerate}
			\item $\mathcal{M}$ finds the smallest $i_0$ such that $R_{i_0}$ is not empty (Hence $R_0,R_1,\dots,R_{i_0-1}$ are empty). Note that this is also the smallest $i_0$ such that $L_{i_0}$ is not full. We call this number $i_0$ the index of this particular shift.
			\item $\mathcal{M}$ puts the leftmost non-$\boxtimes$-symbol of $R_{i_0}$ in position 0. We now denote the new zones as $R_i, L_i$. Now there are $2\times 2^{i_0}-1$ non-$\boxtimes$-symbols remain in $R_{i_0}$ if $R_{i_0}$ is full and otherwise $2^{i_0}-1$ non-$\boxtimes$-symbols remain in $R_{i_0}$. \parinn
			
			Now if $R_{i_0}$ was full before we shift the $2\times 2^{i_0}-1$ to the $R_0,\dots,R_{i_0}$ each of the zones becomes half filled. In other words we shift the leftmost $ 2^{i_0}-1$ non-$\boxtimes$-symbols to the zones $R_0,\dots,R_{i_0-1}$ each of which becomes half filled because in total they take the space $\sum\limits_{i=0}^{i_0-1}2\times 2^i=2(2^{i_0}-1)$. And hence $R_{i_0}$ also becomes half filled because $2^{i_0}$ non-$\boxtimes$-symbols remain in $R_{i_0}$
			
			If $R_{i_0}$ was half-full before we shift the $2^{i_0}-1$ non-$\boxtimes$-symbols of $R_{i_0}$ into the zones $R_0,\dots, R_{i_0-1}$ filling up exactly half of the symbols of each $R_i$ zones where $i\in \{0,\dots, i_0-1\}$. Now the zone $R_{i_0}$ is empty.
			\item $\mathcal{M}$ performs symmetric operation to the left of position 0. That is for $j$ starting from $i_0-1$ down to 0, $\mathcal{M}$ literally moves the $2\times 2^j$ symbols of $L_j$ to fill the cells of $L_{j+1}$. Finally $\mathcal{M}$ moves the symbol originally in position 0 to $L_0$\parinn
			
			So if $R_{i_0}$ was full before $L_{i_0}$ was empty and all $L_0,\dots, L_{i_0-1}$ were full. After the shifting $L_{j}$ contains all the $2\times 2j$ symbols of $L_{j-1}$. So $L_{i_0}$ becomes half filled. And since at each step $L_{j-1}$ was emptied to move all its content to $L_{j}$ in the next step $L_{j-1}$ becomes half filled. Therefore all $L_0,\dots,L_{i_0-1}$ are half filled.
			
			If $R_{i_0}$ was half filled before $L_{i_0}$ was also half filled and all $L_0,\dots, L_{i_0-1}$ were full. After the shifting $L_{j}$ contains all the $2\times 2j$ symbols of $L_{j-1}$. So $L_{i_0}$ becomes full. And since at each step $L_{j-1}$ was emptied to move all its content to $L_{j}$ in the next step $L_{j-1}$ becomes half filled. Therefore all $L_0,\dots,L_{i_0-1}$ are half filled.
			\item At the end of the shift all of the zones $R_0,L_0,R_1,L_1,\dots,R_{i_0-1},L_{i_0-1}$ are half-full, $R_{i_0}$ has $2^{i_0}$ fewer non-$\boxtimes$-symbols and $L_{i_0}$ has $2^{i_0}$ additional non-$\boxtimes$-symbols. Thus our invariants are maintained
		\end{enumerate}
		
		\subsubsection*{Complexity Analysis:} \parinf
		The total cost of performing a shift is proportional to the total size of all zones involved $R_0,L_0,\dots,R_{i_0},L_{i_0}$. That is $$O\left(\sum_{j=0}^{i_0}2\times 2^j  \right)=O\left(2^{i_0}\right)\text{ operations}$$After performing a shift with index $i$ the zones $R_0,L_0,\dots, R_{i-1},L_{i-1}$ are half-full which means it will take at least $2^i-1$ left shifts before the zones $L_0,\dots,L_{i-1}$ becomes empty or at least $2^{i-1}-1$ right shifts before the zones $R_0,\dots,R_{i-1}$ becomes empty. Hence once we perform a shift with index $i$, the next $2^i-1$ shifts of that particular parallel tape will all have index less than $i$. This means that for every one of parallel tapes at most a $\frac{1}{2^i}$ fraction of the total number of shifts have index $i$. Since we perform $T$ shifts and the highest possible index is in the course of simulating $T$ steps of $M$ is $$O\left( (k-1)\sum_{i=1}^{\log T} \frac{T}{2^{i-1}}2^i \right)=O(T\log T)$$
	}
	\pagebreak
	
	%%%%%%%%%%%%%%%%%%%%%%%%%%%%%%%%%%%%%%%%%%%%%%%%%%%%%%%%%%%%%%%%%%%%%%%%%
	% Problem 3
	%%%%%%%%%%%%%%%%%%%%%%%%%%%%%%%%%%%%%%%%%%%%%%%%%%%%%%%%%%%%%%%%%%%%%%%%%
	
	\begin{problem}{%problem statement
		}{p3% problem reference text
		}
		Show that if $P = N P$ , then every $N P$ search problem can be solved in
		polynomial time.
				
	\end{problem}
	
	\solve{
		%Solution
		Let $L\in NP$ be any language. Let $N$ be the  turing machine where for input $x\in L$, $y$ is the certificate such that $|y|\leq g(|x|)$ where $g$ is a polynomial. which solves $L$ within $f(n)$ time where $f$ is a polynomial. Now since $SAT$ is a $NP-complete$ problem the language $L$ can be reduced to $SAT$ by a polynomial time Turing machine $M_1$ which converts an instance $x\in L$ to an instance of $\phi_x$ of $SAT$  in $p(|x|)$ time following the Cook-Levin Theorem where $p$ is a polynomial. 
		
		Now since $P=NP$ given, $SAT$ can be decided by a polynomial time turing machine $M_2$ which runs in $q(n)$ time where $q$ is a polynomial. Now we will prove that given an instance of $SAT$, $\phi$ we can find a satisfying assignment for $\phi$. Suppose $\phi$ is an $n-$variable boolean formula.
		
		\parinf
		
		\subsubsection*{Algorithm:}
		 
		\textbf{Step 0:} First we run $M_2$ on the given boolean formula $\phi$. If it accepts then $\phi $ is in $SAT$. Otherwise reject because it is not satisfiable, hence no satisfying assignment exists. Suppose it accepts. 
		
		\textbf{Step 1:} Then fix $x_1=1$. Now putting the value of $x_1$ in $\phi$ we have a new boolean formula $\phi'(x_2,\dots,x_n)=\phi(1,x_2,\dots,x_n)$. We run $M_2$ on $\phi'$, if it accepts then keep $x_1=1$. Otherwise keep $x_1=0$ since we know there is a satisfying assignment from step 0 and each variable can have only two values 0 or 1.
		
		\textbf{Step 2:} Now in $\phi'$ fix $x_2=1$ and repeat the same process as step 1. Then again keep the value for $x_2$ and fix $x_3$ Repeat then process for all variables $x_1,\dots,x_n$ till every variable is fixed.
		
		\textbf{Step 3:} At the end of the recursive process of step 2 we have a satisfying assignment $\overline{x}$ for the boolean formula $phi$.
	
		
		\subsubsection*{Complexity Analysis of Algorithm} Step 0 takes $q(|\phi|)$ time. Then in each $i-$th recursion of step 1 and step 2 takes $$q\Big(|\phi(1,\dots,1,x_{i+1},\dots,x_n)|\Big)\leq q(|\phi|)\text{ time}$$So in total it takes $(n+1)q(|\phi|)$ time to find a satisfying assignment for the given $SAT$ formula.
		
		\parinn
		
		Now if we input $x$ to $N$ then the satisfying assignment for $\phi_x$ gives us the full path description for a path which reaches the accept state on input $x$ ( if it reaches one) implies it contains the information of the accepting certificate for the input $x$ on $N$ which solves $L$ , due to the construction we used in Cook-Levin Theorem. So the satisfying assignment for the boolean formula $\phi_x$ gives us the description of certificate from which the certificate can be easily extracted in $O(l)$ where $l$ is the length of satisfying assignment which is $poly(|x|)$ . So we found the certificate for the accepting path which is essentially the solution for $L$ on input $x $ in polynomial time.
		
		Therefore every $NP$ search problem can be solved in polynomial time
	}
	
	
	%%%%%%%%%%%%%%%%%%%%%%%%%%%%%%%%%%%%%%%%%%%%%%%%%%%%%%%%%%%%%%%%%%%%%%%%%
	% Problem 4
	%%%%%%%%%%%%%%%%%%%%%%%%%%%%%%%%%%%%%%%%%%%%%%%%%%%%%%%%%%%%%%%%%%%%%%%%%
	
	\begin{problem}{%problem statement
		}{p4% problem reference text
		}
		Prove that $P\neq SPACE(O(n))$	.
	\end{problem}
	
	\solve{
		%Solution\
		Lets assume $P=SPACE(O(n))$ then we get  that if a problem can be solved in  $O(n)$ space then it can be also solved in time $n^{i}$ for some $i\in\bbN$. Now lets take a problem $L$ in $SPACE(O(n^{2}))$ then we make a $SPACE(O(n))$ problem from this problem by padding this problem .
		
		Suppose $L\in SPACE(O(n^{2}))$ then we define $L'$ to be 
		$$L'=\{x \# 0^{k}\  |\  x\in L\ \& \ k=|x|^{2} \}$$Now notice that $L'$ is in $SPACE(O(n))$ because checking that $|x|^{2}=k$ can be done in  $SPACE(k)$ easily and also since $L$ is in $SPACE(O(n^2))$ and $k=|x|^{2}$ checking if $x  \in L$ will take $SPACE(O(|x|^{2}))$ which is $SPACE(O(k))$ which implies this also takes $SPACE(O(n))$. Hence $L'$  is in $SPACE(O(n))$.
		
		 
		This implies that $L'$ can be solved in some $TIME(O(n^{i}))$ for some i.
		Now we claim using $L'$ we can claim that  $L $ can also be solved in 
		$TIME(O(n^{2i}))$ because  if we take an instance of $x\in L$ of  we can convert it into a instance of L' in $TIME(|x|^2)$  just by padding it by $|x|^{2} $ many 0's. Then we can solve this instance of $L'$ in $TIME(O(|x|^{2i}))$ which implies $L$ can be solved in $TIME(O(n^{2i}))$ hence we get that $L \in P$. Which implies that $L \in SPACE(O(n))$
		(because we assumed that $P=SPACE(O(n))$) $\implies SPACE(O(n^{2}))\subset SPACE(O(n))$ $\implies SPACE(O(n^{2}))= SPACE(O(n))$. Which is not true by Space Hierarchy Theorem. Hence contradiction. Therefore our assumption was wrong and $P\neq SPACE(O(n))$.

	}
	
	
	%%%%%%%%%%%%%%%%%%%%%%%%%%%%%%%%%%%%%%%%%%%%%%%%%%%%%%%%%%%%%%%%%%%%%%%%%
	% Problem 5
	%%%%%%%%%%%%%%%%%%%%%%%%%%%%%%%%%%%%%%%%%%%%%%%%%%%%%%%%%%%%%%%%%%%%%%%%%
	
%	\begin{problem}{%problem statement
%		}{p5% problem reference text
%		}
%		%Problem	
%		 A function $S(n)$ is said to be space constructible if there is a Turing machine
%		M which is $S(n)$ space bounded and for each $n$, there is some input of length $n$
%		on which $M$ actually uses $S(n)$ tape cells. If for all $n$, $M$ uses exactly $S(n)$ cells
%		for all inputs of length $n$, we say that $S(n)$ is fully time constructible. Prove
%		that any function that is bounded above by $\log_{2}\log_{2} n$ and bounded below by $c\log_{2}\log_{2} n$
%		, for some $c > 0$ infinitely often, is fully space constructible.	
%	\end{problem}
%	
%	\solve{
%		%Solution
%	}
	
	%%%%%%%%%%%%%%%%%%%%%%%%%%%%%%%%%%%%%%%%%%%%%%%%%%%%%%%%%%%%%%%%%%%%%%%%%
	% Problem 6
	%%%%%%%%%%%%%%%%%%%%%%%%%%%%%%%%%%%%%%%%%%%%%%%%%%%%%%%%%%%%%%%%%%%%%%%%%
	
	\begin{problem}{%problem statement
		}{p5% problem reference text
		}
		Show that if $P = N P$ then every language $A \in P$ except $A = \phi$ or $A = \Sigma^{*}$ is $N P$ complete.		
	\end{problem}
	
	\solve{
		We need to show if $P=NP$ then every problem in $P $ except $\phi$ and $\Sigma^*$ is $NP$-complete . For this let $A$ be a problem in $P$ where $A\neq \phi$ or $A\neq \Sigma^*$. There exists $w_i\in A$ and $w_o\notin A$. We need to show that  $A$  is $NP$-complete i.e for any problem $M$ in $NP$  $\exists$ a polynomial time reduction from $M$ to $A$ . The reduction is as follows given  a problem $M$ in $NP $ since we know $P=NP$ implies there exist a polynomial time Turing machine $T$ which solves $M$ in time $p(n)$ where $p$ is a polynomial. So for reducing an instance of $M$ to an instance of $A$ we will just naively run $T$ on the input $x$, and if the given input $x$ belongs to $M$ then we will naively search for a string which is in $A$ and return that. If it does not belongs to $M$ then we will search for a string which is not in $A$ and return it. 
		
		Now searching for a string which is in $A$ or not naively will take constant with respect to the input given to check in $M$ or not because on any size of input it will naively start searching and verifying from strings of length 1 till we get a satisfying or non-satisfying assignment of $A$ . We know this process of searching  will halt because $A$ is neither $\phi $ nor $\Sigma ^{*}$.Hence we reduced an instance of $M$ into an instance of $A$ using the above reduction such that a string $x\in M \iff f(x) \in A$ . And if $|x|=n$ then this reduction took at most $p(n)+ c$ many steps where $c$ is $max$(steps to find a satisfying assignment of $A$, steps to find a non-satisfying assignment of $A$)  since  the $TM$ $T$ will take at most $p(n)$  time and the assignment finder will take constant time with respect to $|x|$ . Hence this reduction is a polynomial time reduction. 
		
		This implies $A$ is $NP$-hard  and  we know $A \in NP$. Therefore $A$ is $NP-complete$.
	}
	%%%%%%%%%%%%%%%%%%%%%%%%%%%%%%%%%%%%%%%%%%%%%%%%%%%%%%%%%%%%%%%%%%%%%%%%%
	% Problem 7
	%%%%%%%%%%%%%%%%%%%%%%%%%%%%%%%%%%%%%%%%%%%%%%%%%%%%%%%%%%%%%%%%%%%%%%%%%
	
	\begin{problem}{%problem statement
		}{p5% problem reference text
		}
		%Problem	
		Define $ALL_{DFA} =\{ ⟨A⟩\ :\ A\text{ is a }DFA\text{ and }L(A) = \Sigma^{*}\}$. What is the complexity of this language? What do you think about $ALL_{N F A}$?	
	\end{problem}
	
	\solve{
		%Solutio
		\begin{itemize}
			\item We claim that $ALL_{DFA}$ is in $P$. Let $D=(Q,\Gamma, \delta, q_0,F)$ be any $DFA$ where $Q$ is the set of states, $\Gamma$ is the set of alphabets, $\delta$ is the transition function, $q_0$ is the starting state and $F$ is the set of final states. Now  let $R=L(D)$. Now we create a Turing Machine $M$ which converts $D$ to a $DFA$, $D'$ for which $L(D')=R^c$ which can be done by changing the final states from $F$ to $Q\setminus F$ which can be done in polynomial time. Now if $R=\Sigma^*$ then $L(D')=R^c=\phi$ which implies the initial state and the final states are not connected in the graph of the $DFA$, $D'$. So we run $DFS$ algorithm in the graph of $D'$ from the vertex $q_0$. If in the runtime of the $DFS$ algorithm at any point any vertex of $Q\setminus F$ is visited then we reject $D$ otherwise if at the end of the $DFS$ none of the vertices of $Q\setminus F$ are visited then we accept $D$. Thus $D$ is accepted $\iff$ $\forall\ q\in Q\setminus F$ $q_0$ and $q$ are not connected in $D'$ $\iff L(D')=\phi$. Therefore $L(M)=ALL_{DFA}$. \parinn
			
		The $DFS$ algorithm takes polynomial runtime on the graph of $D'$. And the conversion of $D$ to $D'$ takes also polynomial time. Hence $M$ takes polynomial time to decide $ALL_{DFA}$. Therefore $ALL_{DFA}\in P$
		\item Case for $ALL_{NFA} $ : We claim that $ALL_{NFA} $ is in $PSPACE$.\\
		\textbf{Proof-} Given an $NFA$ with $q$ states we claim that if the $NFA$ rejects any string it must reject a string of length at most $2^{q}$. Because if it rejects some word $w$ of length greater than $2^{q}$  then ,we know that a $NFA$ with $q$ states can be simulated by a $DFA$ $D$ with $2^{q}$ many states implies that on input $w$ ,$D$ will reach a rejecting state $r$ now the path from starting state $s$ to $r$ has length $>$ $2^{q}$ which is the number of nodes in the $DFA$ hence it must have got cycled somewhere if we keep removing these cycles we get a path from $s$ to $r$ which has a length smaller than $2^{q}$ implies that the $D$ must reject some string of length at most $2^{q}$ if it rejects some string.
		\\
		$\bullet$ Now using the above fact we will develop a $NPSPACE$ machine , $T$ for the complement of the given problem , which will show the above problem to be in $CONPSPACE$  and by Savitch's we know that $NPSPACE=PSAPACE\implies CONPSPACE=PSPACE$ which will imply that $ALL_{NFA}$ is in $PSPACE$. \\
		$\bullet$ \textbf{For proving this we will do the following on input $N$ on machine $T$ where $N$ is an $NFA$-}
		\\
		\textbf{Step 1-} Mark the start state $s$ of $N$.\\
		\textbf{Step 2-} Nondeterminstically select an input symbol and now unmark $N's$ start state then mark all states which can be reached by reading that letter to  simulate reading of that letter.\\
		\textbf{Step 3-} Now nondeterminstically  select another input symbol and then change the position of markers to all states which can be reached from marked states in the previous iteration by reading this guessed letter .\\
		\textbf{Step 4-} Now we will repeat \textbf{'Step 3'} for $2^{q}$ many steps recursively.\\
		\textbf{Step 5-}We will accept $N$ if at any stage none of the markers lie on an accept state of $N$ , otherwise we will reject $N$.\\
		$\bullet$ \textbf{Explanation-} Now we are accepting the machines which are rejecting a word of length at most $2^{q}$ since we know that if the language of $N$ is not $\Sigma^{*}$ it must reject a word of length at most $2^{q}$ ,this implies we are accepting the machines which are accepting the $NFA's$ which are rejecting some string i.e. , its language is not $\Sigma^{*}$. Hence we are accepting complement of $ALL_{NFA}$.\\
		$\bullet$ \textbf{Analysis-} Now at any iteration we just need to store marks of the previous iteration and then generate marks of the current iteration then forget marks of previous iteration . So if the number of states in the $NFA$ ,$N$ is $q$ then the space needed is $O(q)$.Implies this machine $T$ uses linear space on the size of graph implies complement of $ALL_{NFA}$ is in $NPSPACE=PSPACE$ which implies $ALL_{NFA}$ is in $PSPACE$.  		
		
		\end{itemize}
	}
	
	

	

	
\end{document}
