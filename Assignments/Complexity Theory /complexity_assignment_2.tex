\documentclass[a4paper, 11pt]{article}
\usepackage{comment} % enables the use of multi-line comments (\ifx \fi) 
\usepackage{fullpage} % changes the margin
\usepackage[a4paper, total={7in, 10in}]{geometry}
\usepackage{amsmath,mathtools}
\usepackage{amssymb,amsthm}  % assumes amsmath package installed
\usepackage{float}
\usepackage{xcolor}
\usepackage{mdframed}
\usepackage[shortlabels]{enumitem}
\usepackage{indentfirst}
\usepackage{hyperref}
\hypersetup{
	colorlinks=true,
	linkcolor=blue,
	filecolor=magenta,      
	urlcolor=blue!70!red,
	pdftitle={Complexity Assignment}, %%%%%%%%%%%%%%%%   WRITE ASSIGNMENT PDF NAME  %%%%%%%%%%%%%%%%%%%%
}
\usepackage[most,many,breakable]{tcolorbox}
\usepackage{algorithm}
\usepackage{algpseudocode}


\definecolor{mytheorembg}{HTML}{F2F2F9}
\definecolor{mytheoremfr}{HTML}{00007B}


\tcbuselibrary{theorems,skins,hooks}
\newtcbtheorem{problem}{Problem}
{%
	enhanced,
	breakable,
	colback = mytheorembg,
	frame hidden,
	boxrule = 0sp,
	borderline west = {2pt}{0pt}{mytheoremfr},
	sharp corners,
	detach title,
	before upper = \tcbtitle\par\smallskip,
	coltitle = mytheoremfr,
	fonttitle = \bfseries\sffamily,
	description font = \mdseries,
	separator sign none,
	segmentation style={solid, mytheoremfr},
}
{p}

% To give references for any problem use like this
% suppose the problem number is p3 then 2 options either 
% \hyperref[p:p3]{<text you want to use to hyperlink> \ref{p:p3}}
%                  or directly 
%                   \ref{p:p3}



%---------------------------------------
% BlackBoard Math Fonts :-
%---------------------------------------

%Captital Letters
\newcommand{\bbA}{\mathbb{A}}	\newcommand{\bbB}{\mathbb{B}}
\newcommand{\bbC}{\mathbb{C}}	\newcommand{\bbD}{\mathbb{D}}
\newcommand{\bbE}{\mathbb{E}}	\newcommand{\bbF}{\mathbb{F}}
\newcommand{\bbG}{\mathbb{G}}	\newcommand{\bbH}{\mathbb{H}}
\newcommand{\bbI}{\mathbb{I}}	\newcommand{\bbJ}{\mathbb{J}}
\newcommand{\bbK}{\mathbb{K}}	\newcommand{\bbL}{\mathbb{L}}
\newcommand{\bbM}{\mathbb{M}}	\newcommand{\bbN}{\mathbb{N}}
\newcommand{\bbO}{\mathbb{O}}	\newcommand{\bbP}{\mathbb{P}}
\newcommand{\bbQ}{\mathbb{Q}}	\newcommand{\bbR}{\mathbb{R}}
\newcommand{\bbS}{\mathbb{S}}	\newcommand{\bbT}{\mathbb{T}}
\newcommand{\bbU}{\mathbb{U}}	\newcommand{\bbV}{\mathbb{V}}
\newcommand{\bbW}{\mathbb{W}}	\newcommand{\bbX}{\mathbb{X}}
\newcommand{\bbY}{\mathbb{Y}}	\newcommand{\bbZ}{\mathbb{Z}}

%---------------------------------------
% MathCal Fonts :-
%---------------------------------------

%Captital Letters
\newcommand{\mcA}{\mathcal{A}}	\newcommand{\mcB}{\mathcal{B}}
\newcommand{\mcC}{\mathcal{C}}	\newcommand{\mcD}{\mathcal{D}}
\newcommand{\mcE}{\mathcal{E}}	\newcommand{\mcF}{\mathcal{F}}
\newcommand{\mcG}{\mathcal{G}}	\newcommand{\mcH}{\mathcal{H}}
\newcommand{\mcI}{\mathcal{I}}	\newcommand{\mcJ}{\mathcal{J}}
\newcommand{\mcK}{\mathcal{K}}	\newcommand{\mcL}{\mathcal{L}}
\newcommand{\mcM}{\mathcal{M}}	\newcommand{\mcN}{\mathcal{N}}
\newcommand{\mcO}{\mathcal{O}}	\newcommand{\mcP}{\mathcal{P}}
\newcommand{\mcQ}{\mathcal{Q}}	\newcommand{\mcR}{\mathcal{R}}
\newcommand{\mcS}{\mathcal{S}}	\newcommand{\mcT}{\mathcal{T}}
\newcommand{\mcU}{\mathcal{U}}	\newcommand{\mcV}{\mathcal{V}}
\newcommand{\mcW}{\mathcal{W}}	\newcommand{\mcX}{\mathcal{X}}
\newcommand{\mcY}{\mathcal{Y}}	\newcommand{\mcZ}{\mathcal{Z}}



%---------------------------------------
% Bold Math Fonts :-
%---------------------------------------

%Captital Letters
\newcommand{\bmA}{\boldsymbol{A}}	\newcommand{\bmB}{\boldsymbol{B}}
\newcommand{\bmC}{\boldsymbol{C}}	\newcommand{\bmD}{\boldsymbol{D}}
\newcommand{\bmE}{\boldsymbol{E}}	\newcommand{\bmF}{\boldsymbol{F}}
\newcommand{\bmG}{\boldsymbol{G}}	\newcommand{\bmH}{\boldsymbol{H}}
\newcommand{\bmI}{\boldsymbol{I}}	\newcommand{\bmJ}{\boldsymbol{J}}
\newcommand{\bmK}{\boldsymbol{K}}	\newcommand{\bmL}{\boldsymbol{L}}
\newcommand{\bmM}{\boldsymbol{M}}	\newcommand{\bmN}{\boldsymbol{N}}
\newcommand{\bmO}{\boldsymbol{O}}	\newcommand{\bmP}{\boldsymbol{P}}
\newcommand{\bmQ}{\boldsymbol{Q}}	\newcommand{\bmR}{\boldsymbol{R}}
\newcommand{\bmS}{\boldsymbol{S}}	\newcommand{\bmT}{\boldsymbol{T}}
\newcommand{\bmU}{\boldsymbol{U}}	\newcommand{\bmV}{\boldsymbol{V}}
\newcommand{\bmW}{\boldsymbol{W}}	\newcommand{\bmX}{\boldsymbol{X}}
\newcommand{\bmY}{\boldsymbol{Y}}	\newcommand{\bmZ}{\boldsymbol{Z}}
%Small Letters
\newcommand{\bma}{\boldsymbol{a}}	\newcommand{\bmb}{\boldsymbol{b}}
\newcommand{\bmc}{\boldsymbol{c}}	\newcommand{\bmd}{\boldsymbol{d}}
\newcommand{\bme}{\boldsymbol{e}}	\newcommand{\bmf}{\boldsymbol{f}}
\newcommand{\bmg}{\boldsymbol{g}}	\newcommand{\bmh}{\boldsymbol{h}}
\newcommand{\bmi}{\boldsymbol{i}}	\newcommand{\bmj}{\boldsymbol{j}}
\newcommand{\bmk}{\boldsymbol{k}}	\newcommand{\bml}{\boldsymbol{l}}
\newcommand{\bmm}{\boldsymbol{m}}	\newcommand{\bmn}{\boldsymbol{n}}
\newcommand{\bmo}{\boldsymbol{o}}	\newcommand{\bmp}{\boldsymbol{p}}
\newcommand{\bmq}{\boldsymbol{q}}	\newcommand{\bmr}{\boldsymbol{r}}
\newcommand{\bms}{\boldsymbol{s}}	\newcommand{\bmt}{\boldsymbol{t}}
\newcommand{\bmu}{\boldsymbol{u}}	\newcommand{\bmv}{\boldsymbol{v}}
\newcommand{\bmw}{\boldsymbol{w}}	\newcommand{\bmx}{\boldsymbol{x}}
\newcommand{\bmy}{\boldsymbol{y}}	\newcommand{\bmz}{\boldsymbol{z}}


%---------------------------------------
% Scr Math Fonts :-
%---------------------------------------

\newcommand{\sA}{{\mathscr{A}}}   \newcommand{\sB}{{\mathscr{B}}}
\newcommand{\sC}{{\mathscr{C}}}   \newcommand{\sD}{{\mathscr{D}}}
\newcommand{\sE}{{\mathscr{E}}}   \newcommand{\sF}{{\mathscr{F}}}
\newcommand{\sG}{{\mathscr{G}}}   \newcommand{\sH}{{\mathscr{H}}}
\newcommand{\sI}{{\mathscr{I}}}   \newcommand{\sJ}{{\mathscr{J}}}
\newcommand{\sK}{{\mathscr{K}}}   \newcommand{\sL}{{\mathscr{L}}}
\newcommand{\sM}{{\mathscr{M}}}   \newcommand{\sN}{{\mathscr{N}}}
\newcommand{\sO}{{\mathscr{O}}}   \newcommand{\sP}{{\mathscr{P}}}
\newcommand{\sQ}{{\mathscr{Q}}}   \newcommand{\sR}{{\mathscr{R}}}
\newcommand{\sS}{{\mathscr{S}}}   \newcommand{\sT}{{\mathscr{T}}}
\newcommand{\sU}{{\mathscr{U}}}   \newcommand{\sV}{{\mathscr{V}}}
\newcommand{\sW}{{\mathscr{W}}}   \newcommand{\sX}{{\mathscr{X}}}
\newcommand{\sY}{{\mathscr{Y}}}   \newcommand{\sZ}{{\mathscr{Z}}}


%---------------------------------------
% Math Fraktur Font
%---------------------------------------

%Captital Letters
\newcommand{\mfA}{\mathfrak{A}}	\newcommand{\mfB}{\mathfrak{B}}
\newcommand{\mfC}{\mathfrak{C}}	\newcommand{\mfD}{\mathfrak{D}}
\newcommand{\mfE}{\mathfrak{E}}	\newcommand{\mfF}{\mathfrak{F}}
\newcommand{\mfG}{\mathfrak{G}}	\newcommand{\mfH}{\mathfrak{H}}
\newcommand{\mfI}{\mathfrak{I}}	\newcommand{\mfJ}{\mathfrak{J}}
\newcommand{\mfK}{\mathfrak{K}}	\newcommand{\mfL}{\mathfrak{L}}
\newcommand{\mfM}{\mathfrak{M}}	\newcommand{\mfN}{\mathfrak{N}}
\newcommand{\mfO}{\mathfrak{O}}	\newcommand{\mfP}{\mathfrak{P}}
\newcommand{\mfQ}{\mathfrak{Q}}	\newcommand{\mfR}{\mathfrak{R}}
\newcommand{\mfS}{\mathfrak{S}}	\newcommand{\mfT}{\mathfrak{T}}
\newcommand{\mfU}{\mathfrak{U}}	\newcommand{\mfV}{\mathfrak{V}}
\newcommand{\mfW}{\mathfrak{W}}	\newcommand{\mfX}{\mathfrak{X}}
\newcommand{\mfY}{\mathfrak{Y}}	\newcommand{\mfZ}{\mathfrak{Z}}
%Small Letters
\newcommand{\mfa}{\mathfrak{a}}	\newcommand{\mfb}{\mathfrak{b}}
\newcommand{\mfc}{\mathfrak{c}}	\newcommand{\mfd}{\mathfrak{d}}
\newcommand{\mfe}{\mathfrak{e}}	\newcommand{\mff}{\mathfrak{f}}
\newcommand{\mfg}{\mathfrak{g}}	\newcommand{\mfh}{\mathfrak{h}}
\newcommand{\mfi}{\mathfrak{i}}	\newcommand{\mfj}{\mathfrak{j}}
\newcommand{\mfk}{\mathfrak{k}}	\newcommand{\mfl}{\mathfrak{l}}
\newcommand{\mfm}{\mathfrak{m}}	\newcommand{\mfn}{\mathfrak{n}}
\newcommand{\mfo}{\mathfrak{o}}	\newcommand{\mfp}{\mathfrak{p}}
\newcommand{\mfq}{\mathfrak{q}}	\newcommand{\mfr}{\mathfrak{r}}
\newcommand{\mfs}{\mathfrak{s}}	\newcommand{\mft}{\mathfrak{t}}
\newcommand{\mfu}{\mathfrak{u}}	\newcommand{\mfv}{\mathfrak{v}}
\newcommand{\mfw}{\mathfrak{w}}	\newcommand{\mfx}{\mathfrak{x}}
\newcommand{\mfy}{\mathfrak{y}}	\newcommand{\mfz}{\mathfrak{z}}

%---------------------------------------
% Bar
%---------------------------------------

%Captital Letters
\newcommand{\ovA}{\overline{A}}	\newcommand{\ovB}{\overline{B}}
\newcommand{\ovC}{\overline{C}}	\newcommand{\ovD}{\overline{D}}
\newcommand{\ovE}{\overline{E}}	\newcommand{\ovF}{\overline{F}}
\newcommand{\ovG}{\overline{G}}	\newcommand{\ovH}{\overline{H}}
\newcommand{\ovI}{\overline{I}}	\newcommand{\ovJ}{\overline{J}}
\newcommand{\ovK}{\overline{K}}	\newcommand{\ovL}{\overline{L}}
\newcommand{\ovM}{\overline{M}}	\newcommand{\ovN}{\overline{N}}
\newcommand{\ovO}{\overline{O}}	\newcommand{\ovP}{\overline{P}}
\newcommand{\ovQ}{\overline{Q}}	\newcommand{\ovR}{\overline{R}}
\newcommand{\ovS}{\overline{S}}	\newcommand{\ovT}{\overline{T}}
\newcommand{\ovU}{\overline{U}}	\newcommand{\ovV}{\overline{V}}
\newcommand{\ovW}{\overline{W}}	\newcommand{\ovX}{\overline{X}}
\newcommand{\ovY}{\overline{Y}}	\newcommand{\ovZ}{\overline{Z}}
%Small Letters
\newcommand{\ova}{\overline{a}}	\newcommand{\ovb}{\overline{b}}
\newcommand{\ovc}{\overline{c}}	\newcommand{\ovd}{\overline{d}}
\newcommand{\ove}{\overline{e}}	\newcommand{\ovf}{\overline{f}}
\newcommand{\ovg}{\overline{g}}	\newcommand{\ovh}{\overline{h}}
\newcommand{\ovi}{\overline{i}}	\newcommand{\ovj}{\overline{j}}
\newcommand{\ovk}{\overline{k}}	\newcommand{\ovl}{\overline{l}}
\newcommand{\ovm}{\overline{m}}	\newcommand{\ovn}{\overline{n}}
\newcommand{\ovo}{\overline{o}}	\newcommand{\ovp}{\overline{p}}
\newcommand{\ovq}{\overline{q}}	\newcommand{\ovr}{\overline{r}}
\newcommand{\ovs}{\overline{s}}	\newcommand{\ovt}{\overline{t}}
\newcommand{\ovu}{\overline{u}}	\newcommand{\ovv}{\overline{v}}
\newcommand{\ovw}{\overline{w}}	\newcommand{\ovx}{\overline{x}}
\newcommand{\ovy}{\overline{y}}	\newcommand{\ovz}{\overline{z}}

%---------------------------------------
% Tilde
%---------------------------------------

%Captital Letters
\newcommand{\tdA}{\tilde{A}}	\newcommand{\tdB}{\tilde{B}}
\newcommand{\tdC}{\tilde{C}}	\newcommand{\tdD}{\tilde{D}}
\newcommand{\tdE}{\tilde{E}}	\newcommand{\tdF}{\tilde{F}}
\newcommand{\tdG}{\tilde{G}}	\newcommand{\tdH}{\tilde{H}}
\newcommand{\tdI}{\tilde{I}}	\newcommand{\tdJ}{\tilde{J}}
\newcommand{\tdK}{\tilde{K}}	\newcommand{\tdL}{\tilde{L}}
\newcommand{\tdM}{\tilde{M}}	\newcommand{\tdN}{\tilde{N}}
\newcommand{\tdO}{\tilde{O}}	\newcommand{\tdP}{\tilde{P}}
\newcommand{\tdQ}{\tilde{Q}}	\newcommand{\tdR}{\tilde{R}}
\newcommand{\tdS}{\tilde{S}}	\newcommand{\tdT}{\tilde{T}}
\newcommand{\tdU}{\tilde{U}}	\newcommand{\tdV}{\tilde{V}}
\newcommand{\tdW}{\tilde{W}}	\newcommand{\tdX}{\tilde{X}}
\newcommand{\tdY}{\tilde{Y}}	\newcommand{\tdZ}{\tilde{Z}}
%Small Letters
\newcommand{\tda}{\tilde{a}}	\newcommand{\tdb}{\tilde{b}}
\newcommand{\tdc}{\tilde{c}}	\newcommand{\tdd}{\tilde{d}}
\newcommand{\tde}{\tilde{e}}	\newcommand{\tdf}{\tilde{f}}
\newcommand{\tdg}{\tilde{g}}	\newcommand{\tdh}{\tilde{h}}
\newcommand{\tdi}{\tilde{i}}	\newcommand{\tdj}{\tilde{j}}
\newcommand{\tdk}{\tilde{k}}	\newcommand{\tdl}{\tilde{l}}
\newcommand{\tdm}{\tilde{m}}	\newcommand{\tdn}{\tilde{n}}
\newcommand{\tdo}{\tilde{o}}	\newcommand{\tdp}{\tilde{p}}
\newcommand{\tdq}{\tilde{q}}	\newcommand{\tdr}{\tilde{r}}
\newcommand{\tds}{\tilde{s}}	\newcommand{\tdt}{\tilde{t}}
\newcommand{\tdu}{\tilde{u}}	\newcommand{\tdv}{\tilde{v}}
\newcommand{\tdw}{\tilde{w}}	\newcommand{\tdx}{\tilde{x}}
\newcommand{\tdy}{\tilde{y}}	\newcommand{\tdz}{\tilde{z}}

%---------------------------------------
% Vec
%---------------------------------------

%Captital Letters
\newcommand{\vcA}{\vec{A}}	\newcommand{\vcB}{\vec{B}}
\newcommand{\vcC}{\vec{C}}	\newcommand{\vcD}{\vec{D}}
\newcommand{\vcE}{\vec{E}}	\newcommand{\vcF}{\vec{F}}
\newcommand{\vcG}{\vec{G}}	\newcommand{\vcH}{\vec{H}}
\newcommand{\vcI}{\vec{I}}	\newcommand{\vcJ}{\vec{J}}
\newcommand{\vcK}{\vec{K}}	\newcommand{\vcL}{\vec{L}}
\newcommand{\vcM}{\vec{M}}	\newcommand{\vcN}{\vec{N}}
\newcommand{\vcO}{\vec{O}}	\newcommand{\vcP}{\vec{P}}
\newcommand{\vcQ}{\vec{Q}}	\newcommand{\vcR}{\vec{R}}
\newcommand{\vcS}{\vec{S}}	\newcommand{\vcT}{\vec{T}}
\newcommand{\vcU}{\vec{U}}	\newcommand{\vcV}{\vec{V}}
\newcommand{\vcW}{\vec{W}}	\newcommand{\vcX}{\vec{X}}
\newcommand{\vcY}{\vec{Y}}	\newcommand{\vcZ}{\vec{Z}}
%Small Letters
\newcommand{\vca}{\vec{a}}	\newcommand{\vcb}{\vec{b}}
\newcommand{\vcc}{\vec{c}}	\newcommand{\vcd}{\vec{d}}
\newcommand{\vce}{\vec{e}}	\newcommand{\vcf}{\vec{f}}
\newcommand{\vcg}{\vec{g}}	\newcommand{\vch}{\vec{h}}
\newcommand{\vci}{\vec{i}}	\newcommand{\vcj}{\vec{j}}
\newcommand{\vck}{\vec{k}}	\newcommand{\vcl}{\vec{l}}
\newcommand{\vcm}{\vec{m}}	\newcommand{\vcn}{\vec{n}}
\newcommand{\vco}{\vec{o}}	\newcommand{\vcp}{\vec{p}}
\newcommand{\vcq}{\vec{q}}	\newcommand{\vcr}{\vec{r}}
\newcommand{\vcs}{\vec{s}}	\newcommand{\vct}{\vec{t}}
\newcommand{\vcu}{\vec{u}}	\newcommand{\vcv}{\vec{v}}
%\newcommand{\vcw}{\vec{w}}	\newcommand{\vcx}{\vec{x}}
\newcommand{\vcy}{\vec{y}}	\newcommand{\vcz}{\vec{z}}

%---------------------------------------
% Greek Letters:-
%---------------------------------------
\newcommand{\eps}{\epsilon}
\newcommand{\veps}{\varepsilon}
\newcommand{\lm}{\lambda}
\newcommand{\Lm}{\Lambda}
\newcommand{\gm}{\gamma}
\newcommand{\Gm}{\Gamma}
\newcommand{\vph}{\varphi}
\newcommand{\ph}{\phi}
\newcommand{\om}{\omega}
\newcommand{\Om}{\Omega}
\newcommand{\sg}{\sigma}
\newcommand{\Sg}{\Sigma}

\newcommand{\Qed}{\begin{flushright}\qed\end{flushright}}
\newcommand{\parinn}{\setlength{\parindent}{1cm}}
\newcommand{\parinf}{\setlength{\parindent}{0cm}}
\newcommand{\del}[2]{\frac{\partial #1}{\partial #2}}
\newcommand{\Del}[3]{\frac{\partial^{#1} #2}{\partial^{#1} #3}}
\newcommand{\deld}[2]{\dfrac{\partial #1}{\partial #2}}
\newcommand{\Deld}[3]{\dfrac{\partial^{#1} #2}{\partial^{#1} #3}}
\newcommand{\uin}{\mathbin{\rotatebox[origin=c]{90}{$\in$}}}
\newcommand{\usubset}{\mathbin{\rotatebox[origin=c]{90}{$\subset$}}}
\newcommand{\lt}{\left}
\newcommand{\rt}{\right}
\newcommand{\exs}{\exists}
\newcommand{\st}{\strut}
\newcommand{\dps}[1]{\displaystyle{#1}}
\newcommand{\la}{\langle}
\newcommand{\ra}{\rangle}
\newcommand{\cls}[1]{\textsc{#1}}
\newcommand{\prb}[1]{\textsc{#1}}
\newcommand{\comb}[2]{\left(\begin{matrix}
		#1\\ #2
\end{matrix}\right)}
%\newcommand[2]{\quotient}{\faktor{#1}{#2}}
\newcommand\quotient[2]{
	\mathchoice
	{% \displaystyle
		\text{\raise1ex\hbox{$#1$}\Big/\lower1ex\hbox{$#2$}}%
	}
	{% \textstyle
		#1\,/\,#2
	}
	{% \scriptstyle
		#1\,/\,#2
	}
	{% \scriptscriptstyle  
		#1\,/\,#2
	}
}

\newcommand{\tensor}{\otimes}
\newcommand{\xor}{\oplus}

\newcommand{\sol}[1]{\begin{solution}#1\end{solution}}
\newcommand{\solve}[1]{\setlength{\parindent}{0cm}\textbf{\textit{Solution: }}\setlength{\parindent}{1cm}#1 \hfill $\blacksquare$}
\newcommand{\mat}[1]{\left[\begin{matrix}#1\end{matrix}\right]}
\newcommand{\matr}[1]{\begin{matrix}#1\end{matrix}}
\newcommand{\matp}[1]{\lt(\begin{matrix}#1\end{matrix}\rt)}
\newcommand{\detmat}[1]{\lt|\begin{matrix}#1\end{matrix}\rt|}
\newcommand\numberthis{\addtocounter{equation}{1}\tag{\theequation}}
\newcommand{\handout}[3]{
	\noindent
	\begin{center}
		\framebox{
			\vbox{
				\hbox to 6.5in { {\bf Complexity Theory I } \hfill Jan -- May, 2023 }
				\vspace{4mm}
				\hbox to 6.5in { {\Large \hfill #1  \hfill} }
				\vspace{2mm}
				\hbox to 6.5in { {\em #2 \hfill #3} }
			}
		}
	\end{center}
	\vspace*{4mm}
}

\newcommand{\lecture}[3]{\handout{Lecture #1}{Lecturer: #2}{Scribe:	#3}}

\let\marvosymLightning\Lightning
\newcommand{\ctr}{\text{\marvosymLightning}\hspace{0.5ex}} % Requires marvosym package

\newcommand{\ov}[1]{\overline{#1}}
\newcommand{\thmref}[1]{\hyperref[th:#1]{Theorem \ref{th:#1}}}
\newcommand{\propref}[1]{\hyperref[th:#1]{Proposition \ref{th:#1}}}
\newcommand{\lmref}[1]{\hyperref[th:#1]{Lemma \ref{th:#1}}}
\newcommand{\corref}[1]{\hyperref[th:#1]{Corollary \ref{th:#1}}}

\newcommand{\thrmref}[1]{\hyperref[#1]{Theorem \ref{#1}}}
\newcommand{\propnref}[1]{\hyperref[#1]{Proposition \ref{#1}}}
\newcommand{\lemref}[1]{\hyperref[#1]{Lemma \ref{#1}}}
\newcommand{\corrref}[1]{\hyperref[#1]{Corollary \ref{#1}}}

\DeclareMathOperator{\enc}{Enc}
\DeclareMathOperator{\res}{Res}
\DeclareMathOperator{\spec}{Spec}
\DeclareMathOperator{\cov}{Cov}
\DeclareMathOperator{\Var}{Var}
\DeclareMathOperator{\Rank}{rank}
\newcommand{\Tfae}{The following are equivalent:}
\newcommand{\tfae}{the following are equivalent:}
\newcommand{\sparsity}{\textit{sparsity}}

\newcommand{\uddots}{\reflectbox{$\ddots$}} 

\newenvironment{claimwidth}{\begin{center}\begin{adjustwidth}{0.05\textwidth}{0.05\textwidth}}{\end{adjustwidth}\end{center}}

\setlength{\parindent}{0pt}

%%%%%%%%%%%%%%%%%%%%%%%%%%%%%%%%%%%%%%%%%%%%%%%%%%%%%%%%%%%%%%%%%%%%%%%%%%%%%%%%%%%%%%%%%%%%%%%%%%%%%%%%%%%%%%%%%%%%%%%%%%%%%%%%%%%%%%%%

\begin{document}
	
	%%%%%%%%%%%%%%%%%%%%%%%%%%%%%%%%%%%%%%%%%%%%%%%%%%%%%%%%%%%%%%%%%%%%%%%%%%%%%%%%%%%%%%%%%%%%%%%%%%%%%%%%%%%%%%%%%%%%%%%%%%%%%%%%%%%%%%%%
	
\textsf{\noindent \large\textbf{Soham Chatterjee} \& \textbf{Rishav Gupta} \hfill \textbf{Assignment - 2}\\
    Email: \href{sohamc@cmi.ac.in}{sohamc@cmi.ac.in} \& \href{rishavg@cmi.ac.in}{rishavg@cmi.ac.in}\hfill Roll: BMC202175 \& BMC202145\\
		\normalsize Course: Complexity Theory \hfill Date: March 25, 2023}
	
	%%%%%%%%%%%%%%%%%%%%%%%%%%%%%%%%%%%%%%%%%%%%%%%%%%%%%%%%%%%%%%%%%%%%%%%%%%%%%%%%%%%%%%%%%%%%%%%%%%%%%%%%%%%%%%%%%%%%%%%%%%%%%%%%%%%%%%%%
	% Problem 1
	%%%%%%%%%%%%%%%%%%%%%%%%%%%%%%%%%%%%%%%%%%%%%%%%%%%%%%%%%%%%%%%%%%%%%%%%%%%%%%%%%%%%%%%%%%%%%%%%%%%%%%%%%%%%%%%%%%%%%%%%%%%%%%%%%%%%%%%%
	
	\begin{problem}{%problem statement
		}{p1% problem reference text
		}
		Give a full proof that $AM\subseteq \Pi_2$		
	\end{problem}
	
	\solve{
		%Solution\\
		We will use the following steps to prove -:
		\begin{itemize}
		\item Let $L  \in AM$.
		\item To show that $AM\subseteq \Pi_2$ we will use a similar idea as $BPP \subseteq \Pi_2 \cap \Sigma_2$.
		\item So if $L \in AM $	then we know that $\exists $ $p(n),q(n)$ polynomials in $n$ and a $DTIME $ machine $R$ such that $x \in L \iff$ For a random choice of $y_1\in \{0,1\}^{p(n)}$ $\exists $ $y_2\in \{0,1\}^{q(n)}$ such that $R(x,y_1,y_2)=1$ with probability $\geq \frac{3}{4}$ over random  $y_1$.  	
		\item	Now with  a similar approach as we used in the error reduction of $BPP$ we will reduce the error of our $AM$ protocol (by parallely giving multiple instances and accepting the majority) from $\frac{1}{4}$ to $\frac{1}{2^{n}}$.
		\item Let $S_{x}$ be the set of all random strings $y_1\in\{0,1\}^{p'(n)}$ such that  $\forall$ $y_2 \in \{0,1\}^{(q'(n))} $ for  which $R(x,y_1,y_2)=0$ .(We are doing this after the error reductions.)
		\item Now we will shift $S_{x}$ using some $k$ vectors $u_{1},u_{2} \cdots u_{k}$ then we will show for some $k$, $x\in L $ then  $\bigcup\limits_{i=1}^{k} (S_{x}+u_{i})$ covers the entire $\{0,1\}^{p'(n)}$  otherwise it wont cover it .
		\item Let $A_{x}=\bigcup\limits_{i=1}^{k} (S_{x}+u_{i})$
		\item We claim if $|S_{x}|\leq 2^{(p'(n)-n)}$ then there is no set of $k$ vectors such that $A_{x}=\{0,1\}^{p'(n)}$ if $k<2^{n}$
		\item Because $|\bigcup\limits_{i=1}^{k} (S_{x}+u_{i})|\leq$ $k \cdot |S|\leq$ $2^{(p'(n)-n)}\cdot k <$ $2^{m}$.
		\item $\therefore $ $A_{x} \neq \{0,1\}^{p'(n)}$ 
		\item If $|S|\geq(1-2^{-n})\cdot2^{(p'(n))}$ then $\exists$ vectors $u_{1},u_{2} \cdots u_{k}$ such that $A_{x} = \{0,1\}^{p'(n)}$ for $k> \frac{2^{(p'(n))}}{n}$.
		\item For $r \in {0,1}^{p'(n)}$ ,we need to find the probability $r \notin A_{x} $.
		\item $r \notin S_{x}+u_{i} \iff r+u_{i} \notin S_{x} $ , so let $r_{i}= r+u_{i}$ , so $r\notin A_{x} \iff$ $\forall i ,r_{i} \notin S_{x}$ .
		\item So now we have that $Pr(r\notin A_{x})=(1-\frac{|S_{x}|}{2^{(p'(n))}})^{k}\leq \frac{1}{2^{nk}}.$
		\item Now $Pr(A_{x} \neq \{0,1\}^{(p'(n))})\leq $ $\sum\limits_{j=1}^{2^{(p'(n))}} Pr(w_{j} \notin A_{x}) \leq \frac{2^{(p'(n))}}{2^{nk}}$ , where the words in $\{0,1\}^{(p'(n))}$ are indexed using the $w_{j}'s$ (follows from the union bound).
		\item Now if $k>\frac{2^{(p'(n))}}{n} $ then the above probability over random $\{u_{1},u_{2}\cdots u_{k}\}$ $<1$. So we get that for this $k, \exists $ such $k$ vectors for which $A_{x}$ cover the set .
		\item So choose a $k$ such that $\frac{2^{(p'(n))}}{n}<k<2^{n}$ .
		\item Now look at the following instance of $\Sigma_2$ .$x \in L'$ $\iff$ $\exists \{u_1,u_2\cdots u_k\}$ such that $\forall $ $r\in \{0,1\}^{(p'(n))}$, $\forall$ $y_2 \in \{0,1\}^{(q'(n))}$ 
		$$\bigvee_{i=1}^{k} R(x,r+u_i,y_2)=0$$
		Equivalently
			$$ \bigvee_{i=1}^{k} ((r+u_i) \in S_x)$$
		\item Now we claim $x\in L' \iff$ $x\notin L$ this is because if $x \notin L $ then  $|S_{x}| \geq (1-2^{-n})\cdot2^{(p'(n))}  $ and if $x \in L$ then $|S_x|\leq 2^{(p'(n)-n)}$ , so by above argument $L' = L^c$.
		\item Since we found a $\Sigma_2 $ formula for $L^c \implies $  $L \in \Pi_2$.
		\end{itemize}
		Hence we proved $AM \subseteq \Pi_2$.
	}
	
	
	%%%%%%%%%%%%%%%%%%%%%%%%%%%%%%%%%%%%%%%%%%%%%%%%%%%%%%%%%%%%%%%%%%%%%%%%%
	% Problem 2

	%%%%%%%%%%%%%%%%%%%%%%%%%%%%%%%%%%%%%%%%%%%%%%%%%%%%%%%%%%%%%%%%%%%%%%%%%
	
	\begin{problem}{%problem statement
		}{p2% problem reference text
		}
		%Problem		
		Prove that $MA\subseteq \Pi_2\cap \Sigma_2$
	\end{problem}
	
	\solve{
		%Solution
		We will use the following steps and motivation and results from \textbf{Problem 1} : We know that $MA \subseteq AM$ and from \textbf{Problem 1} we know $AM \subseteq \Pi_2 $ $\implies $ $MA \subseteq \Pi_2$.
		
		Now we only need to show $MA \subseteq \Sigma_2 $.
		\begin{itemize}
		\item For this let $L \in MA $ then we know that $\exists, p(n),q(n) $ polynomials in $n$ and $\exists$ a $DTIME $ machine $R$, such that $x\in L \iff$
		$\exists$ $y_1\in \{0,1\}^{(p(n))}$ such that for a random selection of $y_2\in \{0,1\}^{(q(n))}$   $R(x,y_1,y_2)=1$ with probability $\geq \frac{3}{4}$.
		\item Now we will derive an instance of $BPP $ from $L$.
		\item Now consider the language $\{L':<x,y_1>$ such that $R(x,y_1,y_2)=1$ with probability $\geq \frac{3}{4} $ over random $y_2$.$\}$.
		\item It is easy to see that $L' \in BPP$.
		\item By \textit{Sipser Gaccs Theorem} we know that $BPP \subseteq \Sigma_2$.
		 \item Hence we know that $L' \subseteq \Sigma_2$ .
		\item Now $x \in L$ , $n=|x|$  $\iff$ , $\exists , y_1 $ such that $<x,y_1> \in L'$ .
		\item Now let the $\Sigma_2$ form of  $L'$ be $<x,y_1>$ such that $\exists$ $u \in \{0,1\}^{(a(n))}$ such that $\forall$ $v \in \{0,1\}^{(b(n))} $ , $D(x,y_1,u,v)=1$.
		\item Now it is easy to see that $L \in \Sigma_2$ , as we will just the club the there exist quantifiers into one there exist quantifier.
		 \item Now the $\Sigma_2$ form of L is $x$ such that $\exists$ $(y_1,u) \in \{0,1\}^{(a(n)+p(n))}$ such that $\forall$ $v \in \{0,1\}^{(b(n))} $ , $D(x,y_1,u,v)=1$.
		\end{itemize}
		Hence we proved that $MA \subseteq \Sigma_2$ .	Hence we proved $MA \subseteq \Pi_2 \cap \Sigma_2$.

	}

	\pagebreak
	
	%%%%%%%%%%%%%%%%%%%%%%%%%%%%%%%%%%%%%%%%%%%%%%%%%%%%%%%%%%%%%%%%%%%%%%%%%
	% Problem 3
	%%%%%%%%%%%%%%%%%%%%%%%%%%%%%%%%%%%%%%%%%%%%%%%%%%%%%%%%%%%%%%%%%%%%%%%%%
	
	\begin{problem}{%problem statement
		}{p3% problem reference text
		}
		Let $k \leq n$. Prove that the following family $\mathcal{H}_{n, k}$ is a collection of pairwise independent function from $\{0,1\}^n \rightarrow\{0,1\}^k$ : For each $k \times n$ matrix $A$ with entries in $G F(2)$, and $b \in(G F(2))^k$, the family $\mathcal{H}_{n, k}$ contains functions $h_{A, b}(x)=A x+b$
				
	\end{problem}
	
	\solve{
		%Solution
		The proof is as follows -:
		 \begin{itemize}
			\item  A pair-wise disjoint hash family is a set of functions $H=\{h : P \to Q\}$ such that $\forall$ $u,v \in P$ and $\forall $ $x,y \in Q$ , we have $Pr_h[(h(u)=x) \wedge (h(v)=y)] = \frac{1}{|Q|^{2}}$ where the probability is taken uniformly over $h \in H$.
		 \item  $h_{(A,b)(x)}=Ax+b$. Now fix $u,v,x,y$ for the proof .
		 \item  We need to calculate $Pr_{(A,b)}[Au+b=x \wedge Av+b=y]$.
		  \item  Notice that for each row the calculation is independent as we take the dot product of the vector with $i th $  row of $A$ and then add the $ith $ element of the vector $b$ to get the $ith $ row element in the vector $Ax+b$.
		   \item  So we will prove for one row the rest will follow .
		  \item  Hence we will prove that $Pr_{(a_i \in \{0,1\}^n , b_i \in \{0,1\})}[(<a_i,u> + b_i=x_i) \wedge (<a_i,v>+b_i=y_i)]=\frac{1}{4}$ .
		   \item  Lets convert the probability into $Pr_{(a_i \in \{0,1\}^n , b_i \in \{0,1\})}[(<a_i,u+v>=x_i+y_i) \wedge (<a_i,v>+b_i=y_i)]$ .
		   \item  For proving this we will first prove $Pr_a[<x,a>=1]=\frac{1}{2}$  where $x \neq 0$ .
		   \item  Now for a given $x$ we have that if suppose $x$ has $k$ ones then number of strings for which it gives dot product as one is $\sum_{i=1}^{\lceil \frac{k}{2} \rceil}\binom{k}{2i-1}$ $ \cdot2^{(n-k)}=2^{(n-k)}\cdot 
		   \frac{(2^k)}{2}= 2^{(n-1)}$. (This sum arises because we can have anything in the $(n-k)$ places where the bit is $0$ and $1's$ in odd number of places from the remaining $k$ places).
		   \item  Now size of sample space will be  $2^n\cdot 2=2^{(n+1)} $ . 
		   \item Now for the event $E=[(<a_i,u+v>=x_i+y_i) \wedge (<a_i,v>+b_i=y_i)]$ where $(a_i \in \{0,1\}^n , b_i \in \{0,1\})$ to occur we will have to satisfy the first condition before $\wedge$ we get that $a_i$ should come from a certain  set of size $2^{(n-1)}$ as we proved in the last point because $(u,v,x_i,y_i)$ are fixed ,and for the second clause to satisfy for a choice of $a_i$ only one of $b _i$ $0$ or $1 $ will satisfy th second condition  hence the probability of event $E$ happening  is   $\frac{2^{(n-1)}}{2^{(n+1)}}=\frac{1}{2} \cdot \frac{1}{2} = \frac{1}{4}$.
		   \item  Since each row is independent the probability $Pr_{(A,b)}[Au+b=x \wedge Av+b=y]=\frac{1}{4^k}=\frac{1}{2^{2k}}=\frac{1}{|Q|^2}$.
		\end{itemize}
		    Hence proved.
	}
	
	
	%%%%%%%%%%%%%%%%%%%%%%%%%%%%%%%%%%%%%%%%%%%%%%%%%%%%%%%%%%%%%%%%%%%%%%%%%
	% Problem 4
	%%%%%%%%%%%%%%%%%%%%%%%%%%%%%%%%%%%%%%%%%place%%%%%%%%%%%%%%%%%%%%%%%%%%%%%%%%
	
	\begin{problem}{%problem statement
		}{p4% problem reference text
		}
		Prove that if $P=NP$ then $EXP=NEXP$
	\end{problem}
	
	\solve{
		%Solution
		We know $EXP\subseteq NEXP$. Let $P=NP$. Hence it is enough to show for any $L\in NEXP$, $L\in EXP$. Suppose $L\in NTIME\lt(2^{n^c}\rt)$ and $L$ is accepted by a nondeterministic turing machine $\mcN$. Then the language $$L_{pad}=\lt\{\la x,1^{2^{|x|^c}}\ra  \mid x\in L\rt\} $$is in $NP$ because we construct a nondeterministic turing machine $M$ which for any input string $y$  first checks if there is a string $z$ such that $y=\la z,1^{2^{|z|^c}}\ra$. If not then $M$ rejects $y$ otherwise $M$ simulates $\mcN$ on $z$ for $2^{|z|^c}$ steps non-deterministically  and accepts if $\mcN$ accepts otherwise reject. Hence $\mcL(M)=L_{pad}$. 
		
		Now $P=NP$ and $L_{pad}\in P$. Hence there exists a deterministic turing machine $\mcM$ for which $L_{pad}\in DTIME(n^c)$. Therefore $\mcM$ take $ |y|^c $ time for any input $y$. Since $y=\la x,1^{2^{|x|^c}}\ra $ $\mcM$ takes $O\lt(2^{|x|^c}  \rt) $ time to accepts or reject $y$ which means $\mcM$ accepts or rejects $x$ in $O\lt(2^{|x|^c}  \rt) $ time. Hence $L$ is in $EXP$. Therefore $NEXP=EXP$
	}
	
	
	%%%%%%%%%%%%%%%%%%%%%%%%%%%%%%%%%%%%%%%%%%%%%%%%%%%%%%%%%%%%%%%%%%%%%%%%%
	% Problem 5
	%%%%%%%%%%%%%%%%%%%%%%%%%%%%%%%%%%%%%%%%%%%%%%%%%%%%%%%%%%%%%%%%%%%%%%%%%
	
	\begin{problem}{%problem statement
		}{p5% problem reference text
		}
		Show that if $NP\subseteq BPP$ then $NP=RP$		
	\end{problem}
	
	\solve{
		%Solution
		Assume $NP \subseteq BPP$. Note that $RP \subseteq NP$ unconditionally, hence we just need to show that $NP \subseteq RP$. Since $SAT$ is $NP-complete$ it is enough to show that $SAT$ is in $RP$. By assumption $SAT$ is in $BPP$. Let $M$ be a probabilistic Turing machine running in polynomial time and accepting $SAT$ with error at most $1/2^n$. 
		
		Now we will give an algorithm $\mcA$ using $M$ for $SAT$
		
		\begin{algorithm}
		\textbf{Input:} $\varphi$ 
		\caption{$RP$ Algorithm for $SAT$}
		\begin{algorithmic}
		\State $\varphi' \gets \varphi$
		\For{$k=1,\dots,n$}
			\State $x_k\gets 0$
			\State $\varphi'\gets \varphi'$
			\If {$M(\varphi')==0$}
				\State $x_k\gets 1$
			\EndIf
		\EndFor
		\If {$M$ accepts $\varphi$ with the setting of $n$ variables}
			\State Accept 
		\Else 
			\State Reject  
		\EndIf
		\end{algorithmic}
	\end{algorithm}
If $\varphi$ is not satisfiable then no assignment of the variables satisfies $\varphi$. Hence the algorithm rejects $\vph$ with probability 1 i.e. $$\vph\notin SAT\implies Pr[\mcA(\vph)=1]=0$$. If $\vph$ is satisfiable. Let $\mcA$ rejects $\vph$. Then in the $n$ iterations $M$ at least gave one wrong answer at some point. Now probability of $M$ at least gives one wrong answer in $n$ iterations is $\frac{n}{2^{n}}<\frac12$. Hence $$\vph\in SAT\implies Pr[\mcA(\vph)=1]> 1-\frac12=\frac12$$Hence $SAT\in RP$. Therefore $NP=RP$
	}
	%%%%%%%%%%%%%%%%%%%%%%%%%%%%%%%%%%%%%%%%%%%%%%%%%%%%%%%%%%%%%%%%%%%%%%%%%
	% Problem 7
	%%%%%%%%%%%%%%%%%%%%%%%%%%%%%%%%%%%%%%%%%%%%%%%%%%%%%%%%%%%%%%%%%%%%%%%%%
	
	\begin{problem}{%problem statement
		}{p5% problem reference text
		}
		%Problem	
		If $PSPACE$ has polynomial size circuits then, show that $PSPACE=MA$
	\end{problem}
	
	\solve{
		%Solution
		We will follow the given steps to prove this -:
		\begin{itemize}
			\item First we will prove that $MA \subseteq PSPACE$.
		\item Let $L\in MA$.
		\item We need to describe a $PSPACE$ algorithm for $L$.
		\item For doing this we will use the fact that $L \in MA $ then  $\exists, p(n),q(n) $ polynomials in $n$ and $\exists$ a $DTIME $ machine $R$, such that $x\in L \iff$
		$\exists$ $y_1\in \{0,1\}^{(p(n))}$ such that for a random selection of $y_2\in \{0,1\}^{(q(n))}$   $R(x,y_1,y_2)=1$ with probability $\geq \frac{3}{4}$.
		\item Now what we will do is look over all $y_1\in\{0,1\}^{p(n)}$ and then calculate probability of success over $y_2$ by keeping a counter, if it is high enough we will accept $x$ otherwise we will keep searching for such an  $y_1$ until we exhaust the set$\{0,1\}^{(p(n))}$ if we are not able to find such $y_1$ then we will reject and for each branch described by $y_1 $ and $y_2$ we will reuse the space while doing so also we will clean the counter space after evaluating all possible $y_2$'s for a given $y_1$.
		 \item Now the calculation of the work space will take polynomial space and the counters will also take at most $q(n)$(since we clean the counter after every $y_1$'s all branches computation )  bits so the total space would b polynomial in $n$. 
		\item Hence we proved that $MA \subseteq PSPACE$.
		\item Now if $MA \in P/poly$ we need to show that $PSPACE \in MA$.
		\item For this we will use the fact that $PSPACE=IP$, we will in fact use a more stronger result that a problem in $PSPACE$  can be simulated using an $IP-protocol$ where whatever computation the prover does is in $PSPACE$ , that is the prover can be replaced by an $PSPACE $ machine. 
		\item The stronger result follows from $IP=PSPACE$ as the prover always outputs a polynomially bounded  output in every round and at every step the prover can go over all possibilities in $PSPACE$ and check if there is a string which will make the protocol accept with high probability.
		\item Now what we will do is in the first round $Merlin$ will send a polysized circuit family $\{C_i\}_{i=1}^{p(|x|)}$ upto some polynomially bounded length (snce in the IP-protocol the length of the messages will be bounded by some polynomial $p(|x|)$ ) for $Arthur $ to simulate this circuit family as the prover in the $IP-protocol$ whose prover is a $PSPACE$ machine and will accept iff at last the $IP-protocol$ simulation will accept it.
		\item Notice that the simulation will be polytime as the $IP-protocol $ has polynomially many rounds and we are substituting the prover by a poly sized circuit so  prover's computation can be done in polytime and the verifier is running in polytime by the definition of $IP-protocol$.(The overall computation is polytime as we are doing polytime computations for polynomially many rounds .)
		\item Now the circuit will be polysized in length of input , now if $x \in L$ then $\exists $ a  prover for which the $IP-protocol$ accepts with a high probability then $Merlin$ will send the circuit of this prover which is honest to $Arthur $ , now $Arthur$ will run the $IP-protocol $ with assuming this circuit as the prover , now $Arthur$ accepts iff this $IP-protocol$ accepts. Now by the completeness of the $IP$ protocol it will accept with a probability $\geq$ $\frac{3}{4}$).
		 \item If $x \notin L$ then we get that by the soundness of the $IP-protocol$ that for any prover the acceptance probability is $\leq \frac{1}{4}$ , hence it will accept with a probability $\leq \frac{1}{4 }$.
		\end{itemize}
	}
	
	

	

	
\end{document}
