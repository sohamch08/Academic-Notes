
\documentclass[twoside]{article}
\usepackage{url}

\usepackage[nottoc,numbib]{tocbibind}
\input{../../preamble-article}
\input{../../letterfonts}
\input{../../macros}

\definecolor{mycolor1}{HTML}{11998E}
\definecolor{mycolor2}{HTML}{38EF7D}
%\usepackage{anysize}
%\marginsize{1.5in}{1in}{0.5in}{0.5in}

\newcommand*{\titre}[2]{\begingroup
	\newlength{\drop}
	\setlength{\drop}{0.1\textheight}
	\centering
	\settowidth{\unitlength}{\Huge\scshape\bfseries #1 \hspace{10pt}}
	\vspace*{\baselineskip}
	\rule{\unitlength}{1.6pt}\vspace*{-\baselineskip}\vspace*{2pt}
	\rule{\unitlength}{0.4pt}\\[\baselineskip]
	{\Huge\scshape\color{white} #1}\\[\baselineskip]
	{\large\itshape Instructor: #2}\\[0.2\baselineskip]
	\rule{\unitlength}{0.4pt}\vspace*{-\baselineskip}\vspace{3.2pt}
	\rule{\unitlength}{1.6pt}\\[4\baselineskip]
	{\Large\scshape Scribe: Soham Chatterjee\\[10mm] sohamchatterjee999@gmail.com \\[2mm] Website: \href{https://sohamch08.github.io/}{\textcolor{black}{sohamch08.github.io}} }\par
	\vfill
	{\large\scshape 2024}\par
	\endgroup}
	






\begin{document}
%\begin{titlingpage}
%	\maketitle
%\end{titlingpage}
	
%----------------------------------------------------------------------------------------
%	TITLE PAGE
%----------------------------------------------------------------------------------------
\thispagestyle{empty}
%\titleBC{Algorithmic Coding Theory: 2023}

%{Soham Chatterjee}{\textbf{sohamchatterjee999@gmail.com \\ Website: \href{https://sohamch08.github.io/}{sohamch08.github.io}}
\begin{titlepage}
	
	\begin{tikzpicture}[remember picture,overlay]
		\node [xshift=\paperwidth/2,yshift=\paperheight/2] at (current page.south west)[minimum width=\paperwidth,minimum height=\paperheight,top color=mycolor1,bottom color=mycolor2]{};
	\end{tikzpicture}\\[3\baselineskip]
	
	\titre{Matroids: Combinatorial Optimization}{Prajakta Nimbhorkar}
\end{titlepage}
	
\pagebreak

\tableofcontents 
%\pagestyle{plain} 
\vfill
\pagebreak

\pagestyle{fancy}  

\section{Introduction}
\dfn{Matroid}{A matroid $M=(E,I)$ has a ground set $E$ and a collection $I$ of subsets of $E$ called the \textit{Independent Sets} st\begin{enumerate}
		\item Downward Closure: If $Y\in I$ then $\forall \ X\subseteq Y$, $X\in I$.
		\item Extension Property: If $x,Y\in I$, $|X|<|Y|$ then $\exs\ e\in Y-X$ such that $X\cup \{e\}$ also written as $X+e\in I$
\end{enumerate}}
\begin{observation*}
	A maximal independent set in a matroid is also a maximum independent set. All maximal independent sets have the same size.
\end{observation*}
\parinf

\textbf{Base:} Maximal Independent sets are called bases.

\textbf{Rank of $\boldsymbol{S\in I}$:} $\max\{|X|\colon X\subseteq S, X\in I\}$

\textbf{Rank of a Matroid:} Size of the base.

\textbf{Span of $\boldsymbol{S\in I}$:} $\{e\in E\colon rank(S)=rank(S+e)\}$

\section{Types of Matroids}
	\subsection{{Uniform Matroid:} }It is {denoted as $U_{k,n}$ where $E=[n]$ and $I=\{X\subseteq E\mid |X|\leq k\}$.}
	
	
	\textbf{Free Matroid: }When $k=n$ we take all possible subsets of $E$ into $I$. This matroid is called {Free Matroid} i.e. $U_{n,n}$\parinn
	
	\subsection{Partition Matroid:} Given $E=E_1\sqcup E_2\sqcup \cdots \sqcup E_l$ where $\{E_1,\dots, E_l\}$ is a partition of $E$ and $k_1,\dots,k_l\in \bbN\cup \{0\}$ $$I=\{X\subseteq E\colon |X\cap E_i|\leq k_i\ \forall \ i\in[l]\}$$then $M=(E,I)$ is a partition matroid.
	\begin{note}
		If the $E_i$'s are not a partition then suppose $E_1$, $E_2$ has nonempty partition then we will not have a matroid. 
		
		For example: $E_1=\{1,2\}$, $E_2=\{2,3\}$ and $k_1=k_2=1$ then $X=\{1,3\}$ is independent but $Y=\{2\}\subsetneq X$ is not a matroid. 
	\end{note}
	
	\subsection{Linear Matroid:} Given a $m\times n$ matrix denote its columns as $A_1,\dots, A_n$. Then $$I=\{ X\subseteq [n]\colon \text{Columns corresponding to $X$ are linearly independent}  \}$$\parinn
	
	Here if the underlying field is $\bbF_2$ then it is called \textit{Binary Matroid} and for $\bbF_3$ it is called \textit{Ternary Matroid}.
	
	\subsection{Representable Matroid:} A matroid with which we can associate a linear matroid is called a representable matroid.
	
	Eg: $U_{2,3}$. It can be represented by the matrix $A=\mat{1&1&0\\ 1& 0 & 1\\ 0& 1& 1}$, over $\bbF_2$. Over $\bbF_3$ it is same as $U_{3,3}$. 
	\nt{There are matroids which are not representable as linear matroids in some field. There are matroids which are not representable on any field as well.}
	\lem{}{$U_{2,4}$ is not representable over $\bbF_2$ but representable over $\bbF_3$}
	
	\subsection{Regular Matroid:} There are the matroids which are representable over all fields.
	\lem{}{Regular Matroids are precisely those which can be represented over $\bbR$ by a Totally Uni-modular matrix}
	
	\subsection{Graphic Matroid / Cyclic Matroid:} For a graph $G=(V,E)$ the graphic matroid $M_G=(E,I)$ where $$I=\{F\subseteq E\colon \text{$F$ is acyclic}\}$$
	
	Hence $I$ is the collection of forests of $G$. It follows the downward closure trivially. For extension property let $k=|F_1|<|F_2|=l$ and then there are $n-k$ and $n-l$ components. SO $n-k>n-l$. So $\exs$ an edge in $F_2$ which joins 2 components in $F_1$.
	\lem{}{A subset of columns is linearly independent iff the corresponding edges don't contain a cycle in the incidence matrix}
	\lem{}{Graphic Matroids are Regular Matroids}
	\begin{proof-idea}
		Use Incidence Matrix.
	\end{proof-idea}
	\subsection{Matching Matroids}
	We can try to define it like this but it will not work:
\pr{}{Is the following a matroid:
$E=$ Edges of a graph and $I=\{F\subseteq E\colon \text{$F$ is a matching}\}$
}\parinf

\solve{It is not a matroid since maximal matchings can not be extended to a maximum matching.}
Correct way will be: For a graph $G=(V,E)$ the ground set $=V$ and $$I=\{S\subseteq V\colon \exs \text{a matching that matches all vertices in $S$}\}$$\parinn

The downward closure property trivially holds. For extension property is $S|<|S'|$ then there exists another vertex in $S'$ which is not matched with $S$, so we can add that vertex to $S$. 

\section{Circuits}
Assume we have a matroid $M=(E,I)$. 
\dfn{Circuit}{A minimal dependent set $C$ such that $\forall\ e\in C$, $C-e$ is an independent set.}
\thm{}{Let $S\in I$. $S+e\notin I$. Then $\exs!$ $C\subseteq S+e$.}
\begin{proof}
	Given $S+e\notin I$. Take the set $\Sg$ where $T\in \Sg$ if $t\notin I$ and $T\subseteq S+e$. $\Sg$ is nonempty since $S+e\in \Sg$. Now under the ordering of inclusion $T$ has a minimal element. Hence this minimal element is the desired circuit $C$ which is minimal dependent set contained in $S+e$.
	
	Now suppose it is not unique. Let $C_1,C_2\subseteq S+e$ be circuits. Suppose $f\in C_1-C_2$. Then $S-e+f$ will still be dependent since $C_2\subseteq S-e+f$. Now by definition we get that $C_1-f$ is independent. Therefore we extend $C_1-f$ to an independent set by adding the elements of $S$ till we reach same size as $|S|$. Now $e\in C$ since $C_1$ was formed because of addition of $e$. Hence if we extend $C_1-f$ till same cardinality as $S$ we will add all the edges of $S$ not in $C_1-f$ except $f$ since adding $f$ will make $C$ be a dependent subset of an independent set which is not possible. Hence $C_1-f$ will be extended to $S-f+e$. Therefore $S+e-f$ is independent which contradicts our previous conclusion that $S+e-f$ is dependent. Hence contradiction.
\end{proof}

\section{Finding Max Weight Base}
The problem is given a matroid $M=(E,I)$ and a weight function $W:E\to \bbR$ find the maximum weight base of the matroid. We will solve this using basic greedy algorithm. 
\subsection{Algorithm}
\begin{algorithm}
	\KwIn{A matroid $M=(E,I)$ is given as an input as an oracle and a weight function $W:E\to \bbR$.}
	\KwOut{Find the maximum weight base of the matroid}
	\DontPrintSemicolon
	\Begin{
		Assume $w(1)\geq \cdots \geq w(n)$\;
		$S\leftarrow \emptyset$\;
		$I\leftarrow \{S\}$\;
		\For{$i=1$ to $n$}{\If{$S+i\in I$}{$S\leftarrow S+i$}}
		\Return{S}
	}
	\caption{Algorithm for Finding Max Weight Base}
\end{algorithm}
\subsection{Correctness Analysis}
\thm{}{The above algorithm outputs a maximum weight base iff $M$ is a matroid}
\begin{proof}
	$\Leftarrow:$\parinf
	
	Let $M$ be a matroid. We will prove that this greedy algorithm works by inducting on $i$. At any iteration $i$ we need to prove the following claim:
	
	\textbf{\textit{Claim:}} At any iteration $i$ there is a max weight base $B_i$ such that $S_i\subseteq B_i$ and $B_i\setminus S_i\subseteq \{i+1,\dots, n\}$.
	
	\textbf{\textit{Proof:}} Base case: $S=\emptyset$. So for base case the statement is true trivially. Assume that the statement is true up to $(i-1)$ iterations.\parinn
	
	Now $S_{i-1}\subseteq B_{i-1}$ where $B_{i-1}$ is a maximum weight base and $B_{i-1}-S_{i-1}\subseteq \{i,\dots, n\}$. Now three cases arise:
	\begin{enumerate}[label=\bfseries Case \arabic*:,leftmargin=1.5cm]
		\item If $i\in B_{i-1}$ then $S_{i-1}+i\subseteq B_{i-1}$. Therefore $S_{i-1}+i$ is independent. So now $B_i=B_{i-1}$ and $S_i=S_{i-1}+i$ and $B_i-S_i\subseteq \{i+1,\dots, n\}$.
		\item If $i\notin B_{i-1}$ and $S_{i-1}+i\notin I$. Then $S_i=S_{i-1}$ and $B_i=B_{i-1}$. And $B_i-S_i\subseteq \{i+1,\dots , n\}$.
		\item If $i\notin B_{i-1}$ but $S_{i-1}+i\in I$. Then $S_i=S_{i-1}+i$. Now $S_i$ can be extended to a $B'$ by adding all but one element of $B_{i-1}$. So $|B'|=|B_{i-1}|$. Let the element which is not added is $j\in B_{i-1}$. So $B'=B_{i-1}+i-j$. $$wt(B')=Wt(B_{i-1})-wt()+wt(i)$$But we have $wt(i)\geq wt(j)$. So $wt(B')\geq wt(B_{i-1})$. Now since $B_{i-1}$ has maximum weight we have $wt(B')=wt(B_{i-1})$. Then our $B_i=B'$. So $B_i-S_i\subseteq \{i+1,\dots, n\}$.
	\end{enumerate}
	Hence the claim is true for the $i$th stage as well. Therefore the claim is true.\Qed
	
	Therefore using the claim, after the algorithm finished we have no elements left to check, so the current set has the maximum weight which is also an independent set. So the algorithm successfully returns a maximum weight base.\parinf
	
	$\Rightarrow:$
	
	Assume $M$ is not a matroid.
	
\end{proof}
\section{Some Matroid Properties}
\subsection{Strong Base Exchange Property}
\subsection{Exchange Graph of a Matroid wrt ${S\in I}$}

\section{Matroid Intersection}
\subsection{Using Matroid Intersection to Solve Problems}
\subsubsection{Bipartite Matching}
\subsubsection{Colorful Spanning Tree}
\subsubsection{Min-Max Relation for Colorful Spanning Tree}
\subsubsection{Arborescence}
\subsection{Algorithm}





\pagebreak
%\printbibliography[heading=none]
%\addcontentsline{toc}{chapter}{Bibliography}
%\bibliographystyle{alpha}
%\bibliography{refs}
\end{document}
