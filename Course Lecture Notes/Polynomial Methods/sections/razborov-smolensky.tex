\section{Razborov Smolensky Lower Bound}
The result we will discuss the result that majority is strictly harder than the parity for $\prb{AC}^0$, since there is no polynomial-size $\cls{AC}^0$ circuit to compute majority even if we are given parity gates. The result is Razborov’s, and the proof technique uses ideas due to both Razborov and Smolensky. 

Consider the class $\cls{AC}^0$ of polynomial size circuits with constant depth  with unbounded fan-in. We consider the class $\cls{AC}^0(\xor)$ where we are give the parity gates $\xor$ which outputs $1$ if an odd number of its inputs are $1$. The main theorem which we will prove in this section is:
\begin{theorem}{Razborov-Smolensky}{razborovsmolensky}
	For any $d\in\bbN$ any any depth $d$ $\cls{AC}^0(\xor)$ circuit for \prb{Majority} has size $\geq 2^{\Om\lt(  n^{\frac{1}{2d}}\rt)}$
\end{theorem}

\subsection{Two Parts of Proving Lower Bound}
The proof of the above theorem requires two lemmas:

\begin{lemma}{}{boolean-func-poly}
$\forall\ \eps>0$ and $d\in\bbN$ the following is true:\parinn

If $f:\{0,1\}^n\to \{0,1\}$ can be computed by a size $s$ depth $d$ $\cls{AC}^0(\xor)$ circuit then $\exs$ a polynomial $g$ in $n$ variables and $\deg O\lt(\log \frac{s}{\eps}\rt)^d$ such that $$\underset{a\in\{0,1\}^n}{\bbP}[f(a)=g(a)]\geq 1-\eps$$
\end{lemma}
\begin{lemma}{}{poly-approx-maj}
	For all polynomials $p(x_1,\dots, x_n)$ with $\deg p=t$, $$\underset{a\in\{0,1\}^n}{\bbP}[g(a)=\prb{Maj}(a)]\leq \frac12+O\lt(\frac{t}{\sqrt{n}}\rt)$$
\end{lemma}


Now first we will show that with these two lemmas we can prove Razborov-Smolensky Lower Bound for \prb{Majority} function.\vspace*{2mm}

\begin{proof-of-theorem}[razborovsmolensky]
	Suppose \prb{Maj} has a $\cls{AC}^0(\xor)$ circuit of size $<2^{n^{\frac1{2d}-\dl}}$\parinf
	
	$\xRightarrow{\textbf{\lmref{boolean-func-poly}}}$ $\exs$ polynomial $g$ of degree $n^{\frac1{2d}-\dl}$ that approximates \prb{Maj} with error 0.1.
	
	
	$\xRightarrow{\textbf{\lmref{poly-approx-maj}}}$ $\forall$ polynomial $g$ of $\deg n^{\frac1{2d}-\dl}$ the error is $\geq 1-\lt[\frac12+O\lt( \frac{n^{\frac1{2d}-\dl}}{\sqrt{n}} \rt)\rt]\geq \frac12--\lt[\frac12+O\lt( \frac{n^{\frac1{2d}-\dl}}{\sqrt{n}} \rt)\rt]\geq  \frac12-o(1)$
	\parinn
	
	But $\frac12-o(1)<0.1$ is contradiction.
\end{proof-of-theorem}
\begin{alternate-proof}[razborovsmolensky]
	Suppose $C$ be an $\cls{AC}^0(\xor)$ circuit of size $s$ and depth $d$ computing $\prb{Majority}$\parinf
	
	$\xRightarrow{\textbf{\lmref{boolean-func-poly}}}$ $\exs$ polynomial $g$ of degree $O\lt(\log \frac{s}{\eps}\rt)^d$ with error probability $\leq \eps$.
	
	$\xRightarrow{\textbf{\lmref{poly-approx-maj}}}$ $\forall$ polynomial $g$ of $\deg O\lt(\log \frac{s}{\eps}\rt)^d$ the error is $\geq \frac12+O\lt(\frac{\lt(\log \frac{s}{\eps}\rt)^d}{\sqrt{n}}  \rt)$.\parinn
	
	Hence from these two results and setting $\eps=0.1$ we have $$\frac12+O\lt(\frac{\lt(\log \frac{s}{\eps}\rt)^d}{\sqrt{n}}  \rt)\geq 1-\eps\implies \lt(\log 10s\rt)^d\geq \sqrt{n}\implies s\geq 2^{\Om\lt(\frac{1}{2d}\rt)}$$
\end{alternate-proof}

Now that we proved our main objective theorem we will focus on proving the 2 lemmas in the following two sections.
\subsection{Approximating Boolean Function with Polynomials}
We first state and prove a lemma showing that every $\cls{AC}^0(\xor)$ circuit can be approximated by a low degree polynomial i.e. \lmref{boolean-func-poly}.  But to prove that we will show a more stronger lemma and then the lemma follows as a simple corollary of this stronger result.
\begin{lemma}{}{distribution-func-approx}
	For all $\cls{AC}^0(\xor)$ circuits $C$ of size $s$ of depth $d$ and $\forall\ \eps>0$ there exists a distribution $\sD$ of polynomials $p(x_1,\dots, x_n)\in\bbF_2[x_1,\dots, x_n]$  such that for all $a\in\{0,1\}^n$ $$\underset{p\in\sD}{\bbP}[p(a)=C(a)]\geq 1-\eps$$ where $\sD$ is supported on polynomials of degree $\leq \lt(\log \frac{s}{\eps}\rt)^d$
\end{lemma}

First we will show that this lemma implies \lmref{boolean-func-poly}. \vspace*{2mm}

\begin{proof-of-lemma}[boolean-func-poly]
	Consider the $|\{0,1\}^n|\times |\supp\sD|$ table for each $a\in\{0,1\}^n$, $a$ represents a row in the table. In the table at $(a,i)^{th}$ entry put $1$ if $i^{th}$ polynomial $p$ in $\sD$ satisfies $p(a)=C(a)$. For rest of the positions put $0$. \parinf\vspace*{2mm}
	
	$\xRightarrow{\textbf{\lmref{distribution-func-approx}}}$ $\forall\ \eps>0$ there exists a distribution $\sD$ such that for all $a\in\{0,1\}^n$ such that $\underset{p\in(\sD)}{\bbP}[p(a)=C(a)]\geq 1-\eps$. Hence in the table for each $a\in\{0,1\}^n$, at least $1-\eps$ many fraction of $|\supp(\sD)|$ entries in $a^{th}$ row have $1$. Therefore there are total at least $(1-\eps)\cdot |\{0,1\}^n|\cdot |\supp(\sD)|$ many $1$'s in total in the table. \parinn
	
	Hence by pigeon hole principle there is at least one column which has at least $(1-\eps)\cdot |\{0,1\}^n|$ many $1$'s. Therefore there is a polynomial $p\in\supp(\sD)$ which agrees with $C$ in at least $1-\eps$  fraction of total inputs. Hence $${\underset{a\in\{0,1\}^n}{\bbP}[p(a)=C(a)]\geq 1-\eps}$$
\end{proof-of-lemma}

Now we will prove the \lmref{distribution-func-approx}. Now before diving into the proof first let's see how can we approximate the gates in $\cls{AC}^0(\xor)$ circuits with low-degree polynomials. That way we can approximate any $\cls{AC}^0(\xor)$ circuit with low-degree polynomial.

So to for a $\neg x_i$ gate we can have the polynomial $1-x_i$. For a $\bigoplus\limits_{i=1}^k x_i$ we can use the polynomial $\sum\limits_{i=1}^k x_i$. So only $\wedge$ and $\vee$ gates are remaining. Now notice if we have a low degree polynomial for $\wedge $ we also have a low degree polynomial for $\vee$ since $$\bigvee_{i=1}^n x_i=\neg \lt( \bigwedge\limits_{i=1}^n (\neg x_i) \rt)$$So we will try to find a polynomial approximating  an $\wedge$ gate of degree $\leq \lt(\log\frac1{\eps}  \rt)^d$.  We can't approximate $\wedge$ by outputting $0$ every time since the desired correctness probability must hold for all inputs $x$. Multiplying a random constant-size subset of the bits will not work either, for the same reason. 

Naive way to have a polynomial for $\bigvee\limits_{i=1}^n x_i$ would be $1-\prod\limits_{i=1}^n(1-x_i)$. But with this the degree becomes very large. 
\begin{idea*}
	Check parity of random subset of $[n]$. So we take a random subset $S\subseteq [n]$ then we take the polynomial $p_S=\sum\limits_{i\in S}x_i$. 
\end{idea*}
\begin{lemma}{}{or-gate-approximate}
	If $S$ is a random subset of $[n]$ then $$\underset{S\subseteq [n]}{\bbP}\lt[p_S(x_1,\dots, x_n)=\bigvee\limits_{i=1}^n x_i\rt]\geq\frac12$$
\end{lemma}
\begin{proof}
	If $\ovx=(0,\dots, 0)$ then we have $p_S(x_1,\dots, x_n)=\bigvee\limits_{i=1}^n x_i$. Suppose $\ovx\neq (0,\dots, 0)$. Then  only way $p_S(x_1,\dots, x_n)\neq \bigvee\limits_{i=1}^n $ is when $S$ has an even number of $1$ bits. So let $T\subseteq [n]$ such that $i\in T\iff x_i=1$. Then $p_S(\ovx)=0\iff |S\cap T|\equiv 0\bmod 2$. Now $|S\cap T|\bmod 2$ can be either $1$ or $0$. Since $S$ is picked uniform at random the probability therefore the probability that $|S\cap T|\bmod 2=0$ is $\frac12$. Therefore $\underset{S\subseteq [n], S\neq \emptyset}{\bbP}\lt[p_S(x_1,\dots, x_n)\neq \bigvee\limits_{i=1}^n x_i\rt]\leq\frac12$. Hence we have $\underset{S\subseteq [n]}{\bbP}\lt[p_S(x_1,\dots, x_n)\neq \bigvee\limits_{i=1}^n x_i\rt]\geq\frac12$
\end{proof}

Hence we if we pick a subset $S\subseteq [n]$ uniformly at random then with probability $\geq \frac12$ we can approximate an $\vee$ gate or an $\wedge$ gate with a polynomial of degree $1$. To have error $\frac1{2^k}$ we can chose $k$ subsets of $[n]$ uniformly at random $S_1,\dots, S_k$. Then construct the polynomial $$p\st_{S_1,\dots, S_k}(x_1,\dots, x_n)=1-\prod_{i=1}^k\lt(1-p\st_{S_i}\rt)=1-\prod_{i=1}^k \lt(1-\sum_{j\in S_i} x_j\rt)$$This has error probability $\frac1{2^k}$. So we can approximate $\vee$ gate or $\wedge$ gate with $\frac1{2^k}$ error probability with a degree $k$ polynomial. \vspace*{2mm}

\begin{proof-of-lemma}[distribution-func-approx]
So like the above discussion we replace each gate with polynomials starting with leaf and then we proceed to the top:\begin{itemize}
	\item For $\neg x_i$ gate replace by $1-x_i$
	\item For $\bigoplus\limits_{i=1}^n x_i$ gate replace by $\sum\limits_{i=1}^n x_i$
	\item For  $\bigvee\limits_{i=1}^n x_i$ gate uniformly pick $k$ subsets $S_1,\dots, S_k$  of $[n]$ then construct the polynomial $$p\st_{\vee}(x_1,\dots, x_n)=1-\prod\limits_{i=1}^k \lt(1-\sum\limits_{j\in S_i} x_j\rt)$$ then the error probability becomes $\frac1{2^k}$ by  \lmref{or-gate-approximate}. For $\bigwedge\limits_{i=1}^n x_i$ use the formula $\bigwedge\limits_{i=1}^n x_i=\neg \lt( \bigvee\limits_{i=1}^n (\neg x_i) \rt)$ use the process for $\vee$ gates. So $$p\st_{\wedge}(x_1,\dots, x_n)= \prod\limits_{i=1}^n \lt(1-\sum\limits_{j\in S_i}(1-x_j)\rt)$$Here will choose $k$ later so that we have the necessary total error. 
\end{itemize}
The total polynomial for the circuit is constructed by composing of polynomials with each gate's $S_j$ for $j\in[k]$ sampled from the input wires.

Now degree increases by a factor of $k$ for each $\wedge$ gate or $\vee$ gate. Since the circuit has depth $d$, there can be $\vee$ gates or $\wedge$ gates in at most all depths. Hence degree of the final polynomial becomes $O(k^d)$. 

For the error let $\eps_l$ denote the errors for each gate at depth $l$. Then for each gate $g$ at depth $l-1$ we have error for $g$ is $\leq \frac1{2^k}+|fanin(g)|\eps_l$.\parinf\vspace*{2mm}
\begin{center}
	
\begin{minipage}{0.9\textwidth}
	\textbf{\textit{Claim:}} $\eps_d\leq \frac{s}{2^k}$

\begin{proof}
	We will prove this by induction. For base case $d=1$ this is trivial. Let this is true for $d-1$. For $d$ consider all the children of the root gate $v$. Then $$\eps_d\leq \frac{1}{2^k}+\sum\limits_{u\in \text{Child}(v)}\frac{|C_u|}{2^k}=\frac{1+\sum\limits_{u\in \text{Child}(v)}|C_u|}{2^k}=\frac{|C_v|}{2^k}$$ Hence by mathematical induction we have $\eps_d\leq \frac{s}{d}$
\end{proof}
\end{minipage}
\end{center}

Hence the total error is $\frac{s}{2^k}$. We want the error to be at most $\eps$. Therefore $$\frac{s}{2^k}\leq \eps\implies k=\log \frac{s}{\eps}$$Hence the degree of the final polynomial approximating the circuit is $\lt(\log \frac{s}{\eps}\rt)^d$. Therefore the support of $\sD$ has the polynomials of degree $\leq \lt(\log \frac{s}{\eps}\rt)^d$
\end{proof-of-lemma}


\subsection{Degree-Error Trade of to Approximate \prb{Majority}}

Now we will prove the \lmref{poly-approx-maj}.  But before that we first make some observations.\begin{note}
	The polynomial which approximates \prb{Majority} can be made multilinear without changing its evaluation in $\{0,1\}^n$ just by replacing $x_i^k$ by $x_i$ for each variable and for each power.  
\end{note}

Now we will show that if $\prb{Maj}$ has an approximating polynomial of low-degree then every $n-$variable boolean function $f:\{0,1\}^n \to \{0,1\}$ has an approximating polynomial of low degree. 

\begin{Theorem}{Versatility of \prb{Majority}}{maj-versatility}
	$\forall\ f:\{0,1\}^n\to \{0,1\}$, $\exs\ g,h\in \bbF_2[x_1,\dots, x_n]$ such that $$\forall \ x, f(x) =g(x)\cdot \prb{Maj}(x)+ h(x),\text{ where }\deg g, \deg h\leq \frac{n}{2}$$
\end{Theorem}
 Before proving this theorem first let's see what results we get from this theorem.
 
 \begin{lemma}{}{}
 	Let $f\in\bbF_2[x_1,\dots , x_n]$ such that for all $x\in \{0,1\}^n$, $f(x)=\prb{Maj}(x)$. Then $\deg f\geq \frac{n}2$.
 \end{lemma}
\begin{proof}
	Suppose $\exs\ p\in \bbF_2[x_1,\dots, x_n]$ such that $\deg p< \frac{n}2$ and for all $x\in\{0,1\}^n$ we have $p(x)=\prb{Maj}(x)$.\parinf\vspace*{2mm}
	
	$\xRightarrow{\textbf{\lmref{maj-versatility}}}$ $\forall\ f:\{0,1\}^n\to \{0,1\}$ such that $f(x)=g(x)\cdot \prb{Maj}(x)+g(x)$ for all $x\in\{0,1\}^n$. Then the polynomial $f(x)=g(x)\cdot p(x)+h(x)$ for all $x\in\{0,1\}^n$. Then $\deg f\leq n-1$. Hence all boolean function of $n-$variables can be computed by a polynomial of degree $\leq n-1$. \parinn
	
	But number of boolean functions over $n-$variables are $2^{2^n}$. Number of polynomials of $n-$variables of degree $<n$ is $\leq 2^{2^{n}}-1$. Hence there exists a boolean function which can not be computed by polynomial of degree $< n$. Contradiction. 
\end{proof}

Therefore $\deg(\prb{Maj})\geq \frac{n}{2}$. Now we will prove \lmref{poly-approx-maj} using the above theorem. \vspace*{2mm}

\begin{proof-of-lemma}[poly-approx-maj]
Let $p\in \bbF_2[x_1,\dots, x_n]$ be a polynomial of degree $t$. Let $S\subseteq \{0,1\}^n$ be the set of inputs where $p$ and $\prb{Maj}$ agree. \parinf

$\xRightarrow{\textbf{\lmref{maj-versatility}}}$ $\forall\ f:\{0,1\}^n\to \{0,1\}$ there exists $g,h\in\bbF_2[x_1,\dots, x_n]$ with $\deg g, \deg g\leq \frac{n}2$ such that $\forall \  z\in\{0,1\}^n$ $$f(a)=g(a)\prb{Maj}(a)+h(a)$$Hence every function $f|_S:S\to \{0,1\}^n$ can be computed by the polynomial $g(x)\cdot p(x)+h(x)\in \bbF_2[x_1,\dots, x_n]$ which has degree $\leq \frac{n}{2}+t$. \parinn

Let $\mcF$ be the vector space  of all functions $f|_S:S\to \{0,1\}$ for all $f:\{0,1\}^n\to \{0,1\}$ and let $\mcP$ be the vector space of all polynomials in $\bbF_2[x_1,\dots, x_n]$ of degree at most $\frac{n}{2}+t$. By the above argument we get that $\forall f|_S\in \mcF$, $\exs \ p\st_f\in \mcP$ such that $f|_S$ is computed by $\mcP$. Hence $\dim\mcF\leq \dim\mcP$. Now $$\dim \mcP=\sum_{i=0}^{\frac{n}2+t}\binom{n}{i}=\sum_{i=0}^{\frac{n}2}\binom{n}{i}+\sum_{i=\frac{n}2+1}^{\frac{n}2+t}\binom{n}{i}=\frac12\; 2^n+\sum_{i=\frac{n}2+1}^{\frac{n}2+t}\binom{n}{i}\leq 2^{n-1}+t\frac{2^n}{\sqrt{n}}=2^n\lt(\frac12+\frac{t}{\sqrt{n}}\rt)$$Now $\dim \mcF=|S|$. Hence $$|S|\leq 2^n\lt(\frac12+\frac{t}{\sqrt{n}}\rt)\implies \frac{|S|}{2^n}\leq \frac12+\frac{t}{\sqrt{n}}$$Therefore for any polynomial $p\in \bbF_2[x_1,\dots, x_n]$ with degree $t$ we have $\underset{a\in\{0,1\}^n}{\bbP}[p(a)=\prb{Maj}(a)]\leq \frac12+O\lt(\frac{t}{\sqrt{n}}\rt)$.
\end{proof-of-lemma}


\begin{observation*}
	Now let for any $f:\{0,1\}^n\to \{0,1\}$, $S_0=\prb{Maj}^{-1}(0)$ and $S_1=\prb{Maj}^{-1}(1)$.  Suppose we can compute the polynomials $u,v\in\bbF_2[x_1,\dots, x_n]$ with $\deg u,\deg v\leq \frac{n}{2}$ such that $u$,$f$ agree on $S_0$ and $v$, $f$ agree on $S_1$ i.e. $f|\st_{S_0}$ can be computed by $u$ and $f|\st_{S_1}$ can be computed by $v$. Then $\forall \ x\in\{0,1\}^n$ we have $$f(x)=u(x)(1-\prb{Maj}(x))+v(x)\prb{Maj}(x)$$
\end{observation*}
Hence by the observation we can conclude that computing the polynomial for $f$ on $S_0$ or $S_1$ is enough. Now we will prove the \hyperref[th:maj-versatility]{Versatility of \prb{Majority} Theorem}. \vspace*{2mm}

\begin{proof-of-theorem}[maj-versatility]
	So assume $S_0=\prb{Maj}^{-1}(0)$ and $S_1=\prb{Maj}^{-1}(1)$. We want to show that these are interpolating sets for polynomials of degree $\leq \frac{n}{2}$ i.e. $\deg f|\st_{S_0},\deg f|\st_{S_1}\leq \frac{n}{2}$. 
	
	Now we will show the process to find $u\in\bbF_2[x_1,\dots, x_n]$ with $\deg u\leq \frac{n}{2}$ where $u$ agrees with $f$ in $S_0$. We will follow the same process to find the polynomial $v\in\bbF_2[x_1,\dots, x_n]$ with $\deg v\leq \frac{n}{2}$ where $v$ agrees with $f$ in $S_1$. Now for any $S\subseteq [n]$ we denote $x^S\coloneqq \prod\limits_{i\in S}x_i$. Then $$u(\ovx)=\sum_{S\subseteq[n], |S|\leq \frac{n}{2}}c_Sx^S\quad \forall \ S\subseteq[n],|S|\leq\frac{n}{2},\  c_S\in\{0,1\}$$We have to compute the coefficients of $u$. Now $u(a)=f(a)$ for all $a\in S_0$. Therefore we have a system of linear equations. 
	
	We we take a matrix $M$ with rows and columns indexed by by subsets $S\subseteq [n]$ where $|S|\leq\frac{n}{2}$ and they are ordered so that the size is non-decreasing and lexicographically and use this same ordering for both the rows and columns. Now for $S\subseteq [n]$, $|S|\leq \frac{n}{2}$ the $S^{th}$ column indicates the monomial $x^S$ and the $S^{th}$ row indicates the binary number $a\in\{0,1\}^n$ where $a_i=1\iff i\in S$. Naturally for any $S,T\subseteq [n]$ with $|S|,|T|\leq\frac{n}2$ we have $M(S,T) =1\iff S\subseteq T$. We have the coefficient vector $C$ indexed same as rows of $M$ as the variable vector and we have the column vector $F_0$ of values of $f$ at every point of $S_0$. Then we have the equation $MC=F_0$. 
	
	To have a solution to exists we need $\det M\neq0$. We will show $M$ is an lower triangular matrix with all diagonal entries is $1$. This is true because $M(S,T)=1\iff S\subseteq T$ therefore in the diagonal all entries are $1$ and up the triangle it has $0$'s.. So we have $\det M\neq 0$. And therefore there is an unique solution for $u$. 
	
	Similarly we find $v$. And then we have $f(a)=u(a)(1-\cdot\prb{Maj}(a))+v(a)\prb{Maj}(a)=(v-u)(a)\cdot \prb{Maj}(a)+u(a)$.  This proves the theorem.
\end{proof-of-theorem}