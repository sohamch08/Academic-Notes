\section{Kakeya and Nikodym Problem}
\dfn[kakeya-set]{Kakeya Sets}{In a finite field $\bbF_q$, $K\subseteq \bbF^n$ is a Kakeya Set if $\forall\ a\in\bbF^n$, $\exs \ b\in\bbF^n$ such that $$L_{a,b}=\{b+at\colon t\in \bbF_q\}\subseteq K$$i.e. informally it has a line in every direction}

Now notice that we can take the whole $\bbF^n_q$ as the Kakeya Set. We can also remove a point from $\bbF_q^n$ and it will still be a Kakeya Set.  Having defined the Kakeya sets the biggest question which is studied is:

\begin{question}{}{}
	How small can a Kakeya Set be?
\end{question}
\subsection{Lower Bound on Nikodym Sets}
\subsection{Lower Bound on Kakeya Sets}
\subsubsection{Hasse Derivative}