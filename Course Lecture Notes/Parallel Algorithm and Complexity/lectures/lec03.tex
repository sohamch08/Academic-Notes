\begin{theorem}[Formula Depth Reduction]
	If $f$ has a formula $F$ of size $s$ then there exists an equivalent formula $F'$ of depth $d= O(\log s)$
\end{theorem}
\begin{proof}
	We will prove this by induction on $s$. When $s=1 $ or 2 this is trivially true. 
	
	For the induction step, assume that the inequality holds for all formulas of size strictly 	less than $s$. Given a formula $F$ of size $s$, we find a node $g$ in the formula such that the 	number of leaves in the subformula below $g$ is at least $\frac{s}{2}$ but individually the left and right subformula below $g$ have strictly less than $\frac{s}2$ leaves. Such a node can easily be found by 	starting at the root and checking if the current node satisfies the properties required. If so 	we have found $g$, else there exists a child of $g$ such that number of leaves below it is at least $\frac{s}2$. We move to this child and check if this node satisfies the properties needed. Continuing 	in this way we find the node $g$. We use the following equivalence $$F\equiv (g\wedge F|_{g=1})\vee (g\wedge F|_{g=0})$$
	
	Let $g=g_LBg_R$ where $B\in \{\wedge,\vee\}$. By the choice of $g$, we have that the size of each subformula $g_L$, $g_R$, $F|_{g=0}$, and $F|+{g=1}$ is at most $\frac{s}{2}$. Inductively we balance these subformulas to obtain the formula $F'$. Hence the depth of the formula thus constructed is given by the following recurrence $$d(s)=d\lt(\frac{s}{2}\rt)+O(1)\implies d(s)=O(\log s)$$
	%For size now we have two gates $g_L$ and $g_R$ both with sizes at most $\frac{s}{2}$. Hence $$size(s)=2size\lt(\frac{s}{2}+O(1)\implies size(s)=2^{O(\log s)}=s^{O(1)}  \rt)$$
	Therefore $depth(F')=O(\log s)$.% and $size(F')=O(s^{O(1)})$.
\end{proof}
\section{Iterated Multiplication}
\parinf
\textbf{Problem:} $\textsc{IterMULT}_{m,n}$

\textbf{Input:} $m$ many $n$-bit numbers are given

\textbf{Output:} The product of the given numbers

\parinn
\begin{theorem}[Chinese Remainder Theorem]\label{crt}
	Let $k\in \bbN$, $P=p_1p_2\cdots p_k$. Let $a_1,\dots, a_k$ be given. Then there is a unique number $a\in \{0,\dots, P-1\}$ such that $a\equiv a_i\mod{p_i}$ for $1\leq i\leq k$. More specifically, we have $$a=\lt[ \sum_{i=1}^k (a_i\pmod{p_i}) \rt]\pmod{P}$$where $r_i=\frac{P}{p_i}$ and $s_i=\lt(\frac{P}{p_i}\rt)^{-1} \pmod{p_i}$ (i.e. $s_i$ is the multiplicative inverse of $r_i\pmod{p_i}$).
\end{theorem}
\begin{theorem}[Prime Number Theorem]
	Let $\Pi(n)$ denote the number of prime numbers not greater than $n$. Then $$\Pi(n)=\Theta\lt( \frac{n}{\log n}\rt)$$
\end{theorem}
\begin{corollary}\label{approxpn}
	Let $p_k$ denote the $k$th prime. $p_n=O(n\log n)$
\end{corollary}
\begin{theorem}
	Let $p_i$ denote the $i$th prime. Then $$\prod_{i=1}^n p_i\leq 4^{n\log n}$$
\end{theorem}


\begin{theorem}
	$\textsc{IterMULT}_{n,n}\in \textsc{TC}^0$
\end{theorem}
\begin{proof}
	Let $a_i=a_{i,n-1}\cdots a_{i,1}a_{i,0}$ for $1\leq i\leq n$ be the given numbers which we want to multiply up. Hence $$a=a_1\cdot a_2\cdots a_n<  (2^n)^n=2^{n^2}<p_1\cdot p_2\cdots p_{n^2}$$Then take $p=p_1\cdot p_2\cdots p_{n^2}$. Then our goal is to compute $a\pmod{p}$. Our first goal is to compute $a\pmod{p_i}$ for all $1\leq i\leq n^2$. 
	
	Now take $r_i=\frac{p}{p_i}$ and $s_i=\lt(\frac{p}{p_i}\rt)^{-1}\pmod{p_i}$. Then by \hyperref[crt]{Chinese Remainder Theorem} we have $$a=\sum_{i=1}^{n^2}(a\pmod{p_j}) \cdot r_i\cdot s_i$$ Now we have $$a\pmod{p_j}=\prod_{i=1}^na_i\pmod{p_j}=\prod_{i=1}^n (a_i\pmod{p_j})$$Take $$a'_{i,j}\coloneqq a_i\pmod{p_j}=\lt(\sum_{k=0}^{n-1}a_{i,k}\cdot  2^k\rt)\pmod{p_j}= \sum_{k=0}^{n-1}a_{i,k}\cdot  (2^k\pmod{p_j})\quad\text{for }1\leq i\leq n,\ 1\leq j\leq n^2$$This gives $n^3$ many numbers all of which can be computed in constant depth. The values $2^k\pmod{p_j}$ can be hardwired into the circuit since they do not depend on the input but only the input length. So we can compute $a'_{i,j}$ using $\textsc{IterADD}$ and $\textsc{MULT}$ gates (since we know they are already in $\textsc{TC}^0$) in constant depth. Using the \hyperref[approxpn]{Corollary \ref{approxpn}} we have $p_j=O(n^2\log n)$ for all $1\leq j\leq n^2$. Therefore all the $a'_{i,j}$ consists of $O(\log n)$ bits only.
	
	For $1\leq j\leq n^2$ let $g_j$ be the generator of the multiplicative group $\bbZ^*_{p_j}$ group of the finite field $\bbZ_{p_j}$. Let for $k\in \{1,\dots, p_j-1\}$ take $g^{m_j(k)}=k$ where $m_j(k)\in \{0,\dots, p_j-2\}$. And we set $m_j(0)\coloneqq p_j-1$. So now we will compute $m_j\lt(a'_{i,j}\rt)$. Since the numbers $a'_{i,j}$ contains $O(\log n)$ many bits, these computations can be done in constant depth. Therefore $$m_j\lt( \prod_{i=1}^{n}a_i \pmod{p_j} \rt)=m_j\lt( \prod_{i=1}^{n}(a_i \pmod{p_j}) \rt)=\lt( \sum_{i=1}^n m_j\lt(a'_{i,j}\rt) \rt)\pmod{p_j-1}$$ for $1\leq j\leq n^2$. This can be computed in constant depth using $\textsc{IterADD}$ gates. The case when $a'_{i,j}=0$ can be taken care of. These $n^2$ numbers have logarithmically many bits. Thus we can compute the value $\prod\limits_{i=1}^n a_i\pmod{p_j}$ for $1\leq j\leq n^2$ in constant depth. Finally we compute $$a'\coloneqq \sum_{i=1}^{n^2}\lt( \prod\limits_{i=1}^n a_i\pmod{p_j} \rt)\cdot r_j\cdot s_j$$ in constant depth with $\textsc{IterADD}$ and $\textsc{MULT}$ gates. The values $r_j$ and $s_j$ can be hardwired into the circuit since they don't depend on the input instead the length of the input.
	
	By \hyperref[crt]{Chinese Remainder Theorem} we have $a=a'\pmod{a'}$ that is $q\coloneqq \lt\lfloor \frac{a'}{p}\rt\rfloor$ we have $a=a'-q\cdot p$. Since $r_j=\frac{p}{p_j}$ and $s_j\leq p_j$ we have $r_j\cdot s_j\leq p$. Therefore $$a'= \sum_{i=1}^{n^2}\lt( \prod\limits_{i=1}^n a_i\pmod{p_j} \rt)\cdot r_j\cdot s_j\leq  \sum_{i=1}^{n^2}\lt( \prod\limits_{i=1}^n a_i\pmod{p_j} \rt)p\leq  \sum_{i=1}^{n^2}p_{n^2}p=n^2\cdot p_{n^2}\cdot p\implies q\leq n^2\cdot p_{n^2}$$ This the value of $q$ is  polynomial in $n$ and can be determined by parallely testing for all $i\leq n^2\cdot p_{n^2}$ if $i\cdot p\leq a'\leq (i+1)p$ which can be done in constant depth and $q$ is the one value of $i$ for which the answer of this is yes. Then the result of the iterated multiplication is $a=a'-q\cdot p$.
\end{proof}
\begin{theorem}
	$\textsc{IterMULT}\in \textsc{FO}[\textsc{Maj}]$
\end{theorem}
\section{Some Special Problems}
\begin{definition}[\textsc{Approximate-Majority} or $\textsc{AM}$]
	For $x\in \{0,1\}^n$ let $wt(x)$ be the number of ones in $x$.  Then $$\textsc{AM}(x)=\begin{cases}
		0                 & \text{if } wt(x)\leq \frac{n}{4} \\
		1                 & \text{if } wt(x)\geq \frac{3n}4  \\
		\text{don't care} & \text{otherwise}
	\end{cases}$$
\end{definition}
\begin{theorem}[{\cite[Section 4]{AJTAIapproximatemaj}}]
	$\textsc{Approximate-Majority}\in \textsc{AC}^0$
\end{theorem}

\begin{theorem}[{\cite[Theorem 2.1, Corollary 2.2]{barringtoncorebettdyckk}}]
	$\textsc{Dyck}_k\in \textsc{DLogTime-Uniform TC}^0$
\end{theorem}

\begin{theorem}[{\cite{formulavalprob}}]
	Boolean Formula Value Problem $\in \textsc{ALogTime}$
\end{theorem}
\begin{corollary}
	$\textsc{NC}^1=\textsc{ALogTime}$
\end{corollary}
\begin{theorem}[{\cite{ganradilohrey}}]
	Tree Balancing $\in \textsc{DLogTime-Uniform TC}^0$
\end{theorem}