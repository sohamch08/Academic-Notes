\chapter{Combinatorics}
\section{Twelve Problems: $n$ Balls in $m$ Bins}
\begin{Theorem}{}{}
	\begin{center}
		\begin{tabularx}{0.9\textwidth}{>{\raggedright}p{0.2\linewidth}|>{\centering\arraybackslash}m{0.2\linewidth} |>{\centering\arraybackslash}m{0.2\linewidth}|>{\centering\arraybackslash}m{0.2\linewidth}}
			& $\leq 1$ balls/bin $(m\geq n)$ & $\geq 1$ balls/bin $(m\leq n)$ & Unrestricted\\\hline
			Identical Balls, Identical Bins& 1 & $P(n,m)$ & $\sum\limits_{i=1}^mP(n,i)$\\[5mm]
			Identical Balls, Distinguishable Bins & $\displaystyle{\binom{m}{n}}$ & $\displaystyle{\binom{m-1}{n-1}}$ & $\displaystyle{\binom{n+m-1}{m-1}}$\\[5mm]
			Distinguishable Balls, Identical Bins & $1$ & $S_2(n,m)$ & $\sum\limits_{i=1}^mS_2(ni)$\\[5mm]
			Distinguishable Balls, Distinguishable Bins & $\displaystyle{\binom{m}{n}n!}$ & $S_2(n,m)m! $ & $m^n$
		\end{tabularx}
	\end{center}
\end{Theorem}
\begin{proof}
	
\end{proof}
\section{Stirling Numbers}
\subsection{Stirling Number of Second Kind}
\begin{Definition}{Stirling Number of The Second Kind}{}
	It is the number of ways to partition the set $[n]$ into $m$ nonempty parts. 
\end{Definition}

Clearly if we take the $n$ balls to be the set $[n]$ the balls become distinguishable and each partition is bin and the order order of the partition doesn't matter the bins are identical. So the it becomes the number of ways $n$ distinguishable balls divided into $m$ identical bins.
\begin{lemma}{}{s2recrel1}
	$S_2(n,m)=S_2(n-1,m-1)+mS_2(n-1,m)$
\end{lemma}
\begin{combi-proof}
	We have the balls $[n]$. Then there are two cases. The bin containing ball `1' can has only 1 ball or it can have $\geq 2$ balls. 
	
	For the first case the bin containing ball `1' has only one balls. So the rest of the $n-1$ balls are divided into the rest of the $m-1$ bins. The number of ways this is done is $S_2(n-1,m-1)$.
	
	For the second case the bin containing ball `1' has at least $2$ balls. In that case apart from the ball `1' all the other balls are filled into $m$ identical bins where each bin has at least $1$ ball. So we can think this scenario in other way that is first we fill bins with all the balls except `1' and then we choose where to put the ball `1'. So the number of ways the balls, $\{2,3,\dots, n\}$ i.e. $n-1$ distinguishable balls can be divided into $m$ bins is $S_2(n-1,m)$. Now there are $m$ choices for the ball `1' to be partnered up. Hence for this case there are $mS_2(n-1,m)$ many ways.
	
	Therefore the total number of ways the $n$ distinguishable balls can be divided into $m$ bins so that each bin has at least $1$ ball is $S_(n-1,m-1)+mS_2(n-1,m)$. Therefore we get $S_2(n,m)=S_2(n-1,m-1)+mS_2(n-1,m)$.
\end{combi-proof}

\begin{lemma}{}{s2recrel2}
	$S_2(n+1,m+1)= \displaystyle\sum\limits_{i=m}^n  \binom{n}{i}  S_2(i,m)$
\end{lemma}
\begin{combi-proof}
On the $LHS$ we are counting the number of ways to partition $[n+1]$ into $m+1$ parts. 

For the $RHS$ let's focus on the element $n+1$. So we drop the element from $[n+1]$ in the $(m+1)^{th}$ part. The $(m+1)^{th}$ block can have $k$ elements from $[n]$ which are partnered by $n+1$ where $0\leq k\leq n-m$. We have $k\leq n-m$ since all the other $m$ parts have at least 1 element that leaves us $n-m$ elements to choose. So there are $\binom{n}{k}$ ways to choose the $k$ elements. The remaining $n-k$ elements are divided into $m$ parts which can be done in $S_2(n-k,m)$ many choices. So in total we have $\sum\limits+{k=0}^{n-m}S_2(n-k,m)$ ways to divide $[n+1]$ into $m+1$ parts. Therefore we have $$S_2(n+1,m+1=\sum\limits_{i=0}^{n-m}\binom{n}{i}S_2(n-i,m)=\sum\limits_{i=0}^{n-m}\binom{n}{n-i}S_2(n-i,m)=\sum\limits_{i=m}^{n}\binom{n}{i}S_2(i,m)$$
\end{combi-proof}
\begin{alg-proof}We will prove by Induction. 
	\begin{align*}
		S_2(n+1,m+1) & = \mathcolor{black}{S_2(n,m)}+\mathcolor{blue}{(m+1)S_2(n,m)} & [\text{Using \lmref{s2recrel1}}]\\
		             & = \mathcolor{black}{\sum_{i=m-1}^{n-1}\binom{n-1}{i}S_2(i,m-1)} +\mathcolor{blue}{ (m+1)\sum_{j=m}^{n-1}\binom{n-1}{j}S_2(j,m)}& [\text{Induction Hypothesis}]\\
		             & = \mathcolor{black}{\sum_{i=m-1}^{n-1}\binom{n-1}{i}S_2(i,m-1)}+\mathcolor{red}{m\sum_{j=m}^{n-1}\binom{n-1}{j}S_2(j,m)}+\mathcolor{blue}{\sum_{j=m}^{n-1}\binom{n-1}{j}S_2(j,m)}& [\text{Using \lmref{s2recrel1}}]\\
		             & = \sum_{i=m}^{n-1}\binom{n-1}{i-1}S_2(i-1,m-1)+\mathcolor{red}{m\sum_{j=m}^{n-1}\binom{n-1}{j}S_2(j,m)}+\mathcolor{blue}{\sum_{j=m}^{n-1}\binom{n-1}{j}S_2(j,m)}\\
	\end{align*}
\end{alg-proof}