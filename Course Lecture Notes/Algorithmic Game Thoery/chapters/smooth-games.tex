\chapter{Price of Anarchy Bounds in Smooth Games}
\section{Facility Location Game}
\begin{Definition}{Facility Location Game}{}
	A Facility Location Game consists of \begin{enumerate}[label=(\roman*)]
		\item There is a set $L$, of $n$ locations.
		\item $SP$ is the set of $k$ service providers or players.
		\item Player $i\in SP$ has its strategy set some $S_i\subseteq L$. For player $i$, $S_i$ represents the places where player $i$ might build a facility. For some player $i\in SP$, it may also be the case that $S_i=\emptyset$, i.e. player $i$ prefers nowhere.
		\item A set $C$ of $m$ clients. 
		\item Each client $j\in C$ has some value $\pi_j\geq 0$. Think of this as how much the client is willing to pay for the service that the facilities provide.
		\item For all $l\in L$ and $j\in C$ let $c(l,j)$ is the transportation cost.
		\item For all $i\in SP$ and $j\in C$ let $p(i,j)$ is the price of $i$ for serving $j$.
	\end{enumerate}
\end{Definition}


\begin{assumption}We will have the following assumptions for the game:
	\begin{itemize}
		\item $c(l,j)\neq c(l',j)$ for all $l,l'\in L$, $l\neq l'$ and $j\in C$.
		\item $\pi_j\geq c(l,j)$ for all $l\in L$ and $j\in C$.
	\end{itemize}
\end{assumption}
\subsection{Utilities: Definition}
We will define the utilities of client and service providers. Let $s\in S$ be any strategy profile. If a client $j\in C$ chooses the service provider $i\in SP$  then we denote $SP(j)=i$. 

So for any client $j\in C$ the utility for strategy profile $s$ is $$u_j(s)=\pi_j-p(SP(j),j)$$  and for  any service provider $i\in SP$, the utility of $i$ is $$u_i(s)=\sum\limits_{j\colon SP(j)=i}p(i,j)-c(s_i,j)$$ Now we define the utilitarian social welfare for a strategy profile $s$ to be $$V(s)=\sum\limits_{i\in SP}u_i(s)+\sum\limits_{j\in C}u_j(s)=\sum\limits_{j\in C}\pi_j-c\lt(s\st_{SP(j)},j\rt)$$
To make the utilities of every service provider to be non-negative we have another assumption:
\begin{assumption}
	For any strategy profile $s$, $\forall\ i\in SP$ and $\forall\ j\in C$, $p(i,j)\geq c(s_i,j)$
\end{assumption}
\begin{remark}
	This social welfare considers the clients as players as well but with simple strategies choosing the least price service provider
\end{remark}

\subsection{Choosing Prices}
Note that technically prices too are chosen by the service providers. However we will show that given a strategy profile the prices are fixed at equilibrium.

Now the service provider $i$ at location $s_i$ can get profit from client $j\in C$ only if it's the closest i.e. transportation cost satisfies $c(s_i,j)\leq c(l,j)$ for all $j\in L$.  Any client $j\in C$ chooses the service provider $i\in SP$ if the price charged by the $i$ to $j$ is minimum i.e. $p(i,j)=\min\limits_{i'\in SP}p(i',j)$.
\begin{observation*}
	For any client $j\in C$ and any service provider $i\in SP$ in a strategy profile $s$,  $SP(j)=i$ if \begin{enumerate}[label=(\roman*)]
		\item $i\in \arg\min\limits_{i'\in SP}p(i,j)$
		\item $c(s_i,j)=\min\limits_{i'\in SP}c(s_{i'},j)$
	\end{enumerate}
\end{observation*}

\begin{lemma}{}{price-in-flg}
	For any client $j\in C$ and any service provider $i\in SP$ in a strategy profile $s$,  $SP(j)=i$ if\begin{enumerate}[label=(\roman*)]
		\item $c(s_i,j)=\min\limits_{i'\in SP}c(s_{i'},j)$
		\item $p(i,j)=\max\lt\{c(s_i,j),\min\limits_{\substack{i'\in SP\\ i'\neq i'}}c(s_{i'},j)\rt\}$
			\end{enumerate}
\end{lemma}
\begin{proof}
	The first condition directly follows from the observation.  We will prove the second one.   Let $\hat{i}=\arg\min\limits_{i'\neq i}c(s_{i'},j)$. Suppose $c(s_i,j)<c(s_{\hat{i}},j)$. Then we have to show $p(i,j)=c(s_{\hat{i}},j)$.  
	
	Since the service providers want to maximize their utility they want to maximize the price charged to client. Now if $p(i,j)\neq c(s_{\hat{i}},j)$ then we can assume $p(i,j)>c(s_{\hat{i}},j)$. If $p(i,j)>c(s_{\hat{i}},j)$ then if $p(\hat{i},j)=\frac12\lt(p(i,j)+c(s_{\hat{i}},j)\rt)$ then $p(\hat{i},j)<p(i,j)$. Hence $SP(j)=\hat{i}$ but we are given $SP(j)=i$. Hence contradiction \ctr Hence we have the lemma.
\end{proof}
\subsection{Potential Game}
We will now show that Facility Location Game is a potential game. We will take the function $V$ which is the total utilitarian social welfare as the potential function of the game. And we will show now that this follows the condition of potential games.
\begin{Theorem}{}{}
	For any strategy profile $s$, for all $i\in SP$, $\forall\ s_i'\in S_i$ $$V(s_i',s_{-i})-V(s)=u_i(s_i',s_{-i})-u_i(s)$$
\end{Theorem}
\begin{proof}
	We will show that $u_i(s_i'=\emptyset, s_{-i})-u_i(s)=V(s'_i=\emptyset, s_{-i})-V(s)$. Now we have $$V(s)=\sum\limits_{j\in C} \pi_j-c\lt(s\st_{SP(j)},j\rt)=\sum\limits_{j\in C}\pi_j-\min\limits_{i'\in SP}c(s_{i'},j)\qquad V(s_i',s_{-i})=\sum\limits_{j\in C}\pi_j-\min\limits_{\substack{i\in SP\\ i'\neq i}}c(s_{i'},j) $$Therefore $$V(s_i'=\emptyset, s_{-i})-V(s)=\sum\limits_{j\in C}\min\limits_{i'\in SP}c(s_{i'},j)-\min\limits_{\substack{i\in SP\\ i'\neq i}}c(s_{i'},j)$$Now for $j\in C$ for which $SP(j)\neq i$ we have $\min\limits_{i'\in SP}c(s_{i'},j)=\min\limits_{\substack{i\in SP\\ i'\neq i}}c(s_{i'},j$. By \lmref{price-in-flg} we have $\min\limits_{\substack{i\in SP\\ i'\neq i}}c(s_{i'},j=p(i,j)$ for $SP(j)=i$. Then we have $$V(s_i'=\emptyset, s_{-i})-V(s)=\sum\limits_{j:SP(j)=i}p(i,j)-c(s_i,j)=u_i(s_i',s_{-i})-u_i(s)$$Therefore $V$ is a potential function of satisfying the condition of potential game. Hence Facility Location Game is a potential game.
\end{proof}
\begin{corolary}{}{}
	The $\pos$ of Facility Location Games is $1$
\end{corolary}
\begin{proof}
	Since the potential function and the utility functions are same the strategy which maximizes the utility and the $\pne$ which has maximum utility are the same strategy profiles. Therefore $\pos$ is $1$.
\end{proof}
\section{Valid Utility Games}

\section{Smooth Games}
\begin{Definition}{$(\lm,\mu)$-smooth Cost Minimization Game}{}
	A cost minimization game is $(\lm,\mu)$ smooth if for all strategy profiles $s,s'\in S$ $$\sum\limits_{i=1}^n c_i(s'_i,s_{-i})\leq \lm \textit{cost}(s')+\mu\textit{cost}(s)$$
\end{Definition}

Therefore (Atomic) Network Congestion Games is $\lt(\frac53,\frac13\rt)$-smooth. If a game is $(\lm,\mu)$ smooth then we get an upper bound on the $\poa$ directly. 
\begin{Theorem}{}{}
	For any $(\lm,\mu)$-smooth cost-minimization game the $\poa\leq \frac{\lm}{1-\mu}$. 
\end{Theorem}
\begin{proof}
	Let $s^*$ be an equilibrium and $s$ be minimum cost strategy profile. Since the game is $(\lm,\mu)$ smooth we have $$\textit{cost}(s^*)=\sum\limits_{i=1}^n c_i(s^*)\leq \sum\limits_{i=1}^n c_i(s_i,s_{-i}^*)\leq \lm \textit{cost}(s)+\mu\textit{cost}(s^*)$$Therefore $\frac{\textit{cost}(s^*)}{\textit{cost}(s)}\leq \frac{\lm}{1-\mu}$.
\end{proof}
\subsection{Bound of \textsf{PoA} for \textsf{CCE}}
We can also extend this bound of $\poa$ for coarse correlated equilibriums too. So we have the following theorem:
\begin{Theorem}{}{}
	For a $(\lm,\mu)$-smooth cost-minimization game, the $\poa$ for $\cce$'s is at most $\frac{\lm}{1-\mu}$.
\end{Theorem}
\begin{proof}
	We know the distribution $\sD\in\Delta_{|S|}$ is a $\cce$ if $\forall \ i\in[n]$, $\forall\ s_i'\in S_i$, $\underset{s\sim \sD}{\bbE}[c_i(s)]\leq \underset{s\sim \sD}{\bbE}[c_i(s_i',s_{-i})]$. Let $s^*$ be the minimum cost pure strategy profile. Then we have 
	\[	\underset{s\sim \sD}{\bbE}[\textit{cost}(s)] =\sum_{i=1}^n \underset{s\sim \sD}{\bbE}[c_i(s)]\leq \sum_{i=1}^n \underset{s\sim \sD}{\bbE}[c_i(s_i^*,s_{-i})]\leq \sum_{i=1}^n \lm \underset{s\sim \sD}{\bbE}[c_i(s^*)]+\mu\underset{s\sim \sD}{\bbE}[c_i(s)]=\lm \textit{cost}(s^*)+\mu \underset{s\sim \sD}{\bbE}[\textit{cost}(s)]\]Therefore we have $\frac{\underset{s\sim \sD}{\bbE}[\textit{cost}(s)]}{\textit{cost}(s^*)}\leq \frac{\lm}{1-\mu}$. Hence we have the upper bound $\poa$. 

\end{proof}
\subsection{Bound of \textsf{PoA} for \texorpdfstring{$\varepsilon$}{ε}-\textsf{PNE}}
\begin{Definition}{$\veps-\pne$}{}
	For a cost-minimization game $s\in S$ is an $\veps-\pne$ if $\forall \ i\in[n]$, $\forall\ s_i'\in S_i$ if $$c_i(s_i',s_{-i})\geq \frac{1}{1+\veps}c_i(s)$$\end{Definition}

We can extend the bound of $\poa$ for $\pne$'s  to even approximate $\pne$'s too but we don't get the exact $\frac{\lm}{1-\mu}$ bound. Instead there is a 

\subsection{Utility Maximization}
\begin{Definition}{$(\lm,\mu)$-smooth Utility Maximization Game}{}
	A utility maximization game is $(\lm,\mu)$ smooth if for all strategy profiles $s,s'\in S$ $$\sum\limits_{i=1}^n u_i(s'_i,s_{-i})\geq \lm U(s')-\mu U(s)$$
\end{Definition}

Therefore any Valid Utility Game is $(1,1,)$-smooth game. And like in the case of cost minimization game we get an upper bound on the $\poa$ for utility maximization game.
\begin{Theorem}{}{}
	For any $(\lm,\mu)$-smooth utility maximization game the $\poa\leq \frac{\lm}{1+\mu}$. 
\end{Theorem}
\section{Load Balancing Game}