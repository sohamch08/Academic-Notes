\chapter{Kruskal Algorithm with Data Structure}

\section{Kruskal Algorithm}
\begin{center}
\hspace*{4mm}\begin{tikzpicture}[vertex/.style={circle, draw, fill=white, inner sep=2pt}]
\begin{scope}[shift={(-5,0)}]
		\node[vertex] (1) at (0, 0) {};
	\node[vertex] (2) at (1, 1) {};
	\node[vertex] (3) at (1, -1) {};
	\node[vertex] (4) at (2, 0) {};
	\node[vertex] (5) at (4, 0.5) {};
	\node[vertex] (6) at (6, 2) {};
	\node[vertex] (7) at (5, -0.5) {};
	\node[vertex] (8) at (3, -1) {};
	\node[vertex] (9) at (0, -2) {};
	
	%    % Edges with weights (some edges have bends to avoid crossing)
	\draw[shorten >=1pt, shorten <=1pt] (1) -- (2) node[midway, xshift=-1mm,yshift=2mm] {6};
	\draw[shorten >=1pt, shorten <=1pt] (1) -- (3) node[midway, xshift=1mm,yshift=2mm] {3};
	\draw[shorten >=1pt, shorten <=1pt] (2) -- (3) node[midway, right] {14};
	\draw[shorten >=1pt, shorten <=1pt] (2) -- (4) node[midway, above right] {12};
	\draw[shorten >=1pt, shorten <=1pt] (3) -- (4) node[midway, xshift=1mm,yshift=-2mm] {2};
	\draw[shorten >=1pt, shorten <=1pt] (4) -- (5) node[midway, below right] {7};
	\draw[shorten >=1pt, shorten <=1pt] (2) to [bend left=60] (5) node[xshift=-1.5cm, yshift=1.2cm] {15};
	\draw[shorten >=1pt, shorten <=1pt] (5) -- (6) node[midway, above left] {18};
	\draw[shorten >=1pt, shorten <=1pt] (6) -- (7) node[midway, xshift=-3mm, yshift=-1mm] {24};
	\draw[shorten >=1pt, shorten <=1pt] (7) -- (8) node[midway, above left]{16};
	\draw[shorten >=1pt, shorten <=1pt] (3) -- (8) node[midway, below] {5};
	\draw[shorten >=1pt, shorten <=1pt] (3) to [bend right=80, looseness=1.5] (6) node[xshift=-1.5cm, yshift=-4.1cm] {10};
	\draw[shorten >=1pt, shorten <=1pt] (3) -- (9) node[midway, xshift=-1mm,yshift=2mm] {9};
\end{scope}
\draw[ultra thick, -Stealth] (2,0)  -- (4.2,0) node[midway, above] {Kruskal} node[midway, below] {Algorithm};
\begin{scope}[shift={(5,0)}]
	\node[vertex] (1) at (0, 0) {};
	\node[vertex] (2) at (1, 1) {};
	\node[vertex] (3) at (1, -1) {};
	\node[vertex] (4) at (2, 0) {};
	\node[vertex] (5) at (4, 0.5) {};
	\node[vertex] (6) at (6, 2) {};
	\node[vertex] (7) at (5, -0.5) {};
	\node[vertex] (8) at (3, -1) {};
	\node[vertex] (9) at (0, -2) {};
	
	%    % Edges with weights (some edges have bends to avoid crossing)
	\draw[shorten >=1pt, shorten <=1pt, red, thick] (1) -- (2) node[midway, xshift=-1mm,yshift=2mm] {6};
	\draw[shorten >=1pt, shorten <=1pt, red, thick] (1) -- (3) node[midway, xshift=1mm,yshift=2mm] {3};
	\draw[shorten >=1pt, shorten <=1pt] (2) -- (3) node[midway, right] {14};
	\draw[shorten >=1pt, shorten <=1pt] (2) -- (4) node[midway, above right] {12};
	\draw[shorten >=1pt, shorten <=1pt, red, thick] (3) -- (4) node[midway, xshift=1mm,yshift=-2mm] {2};
	\draw[shorten >=1pt, shorten <=1pt, red, thick] (4) -- (5) node[midway, below right] {7};
	\draw[shorten >=1pt, shorten <=1pt] (2) to [bend left=60] (5) node[xshift=-1.5cm, yshift=1.2cm] {15};
	\draw[shorten >=1pt, shorten <=1pt] (5) -- (6) node[midway, above left] {18};
	\draw[shorten >=1pt, shorten <=1pt] (6) -- (7) node[midway, xshift=-3mm, yshift=-1mm] {24};
	\draw[shorten >=1pt, shorten <=1pt, red, thick] (7) -- (8) node[midway, above left]{16};
	\draw[shorten >=1pt, shorten <=1pt, red, thick] (3) -- (8) node[midway, below] {5};
	\draw[shorten >=1pt, shorten <=1pt, red, thick] (3) to [bend right=80, looseness=1.5] (6) node[xshift=-1.5cm, yshift=-4.1cm] {10};
	\draw[shorten >=1pt, shorten <=1pt, red, thick] (3) -- (9) node[midway, xshift=-1mm,yshift=2mm] {9};
\end{scope}
\end{tikzpicture}
\end{center}

\section{Data Structure 1: Array}

\section{Data Structure 2: Left Child Right Siblings Tree}

\section{Data Structure 3: Union Find}
\subsection{Analyzing the Union-Find Data-Structure}
We call a node in the union-find data-structure a \textit{leader} if it is the root of the (reversed) tree.
\begin{lemma}{}{}
	Once a node stop being a {leader} (i.e. the node in top of a tree). it can never become a leader again.
\end{lemma}
\begin{proof}
	A node $x$ stops being a {leader} only because of the \prb{Union} operation which made $x$ child of a node $y$ which is a {leader} of a tree. From this point on, the only operation that might change the parent pointer of $x$ is the \prb{Find} operation which traverses through $x$. Since path-compression only change the parent pointer of $x$ to point to some other node $y$. Therefore the parent pointer of $x$ will never become equal to itself i.e. $x$ can never be a {leader} again. Hence once $x$ stops being a {leader} it can never be a {leader} again.
\end{proof}
\begin{lemma}{}{}
	Once a node stop being a leader then its rank is fixed.
\end{lemma}
\begin{proof}
	The rank of a node changes only by an \prb{Union} operation. But the \prb{Union} operation only changes the rank of nodes that are {leader} after the operation is done. Therefore once a node stops being a {leader} it's rank will not being changed by an \prb{Union} operation. Hence once a node stop being a leader then its rank is fixed.
\end{proof}
\begin{lemma}{}{}
	Ranks are monotonically increasing in the reversed trees, as we travel from a node to
	the root of the tree.
\end{lemma}
\begin{proof}
	To show that the ranks are monotonically increasing  it suffices to prove that for all edge $u\to v$ in the data structure we have $\Rank(u)<\Rank(v)$. 
\end{proof}

\begin{lemma}{}{}
	When a node gets rank $k$ than there are at least $\geq 2^k$ elements in its subtree.
\end{lemma}
\begin{corolary}{}{}
	For all vertices $v$, $v.rank\leq \lt\lfloor \log n\rt\rfloor$
\end{corolary}
\begin{corolary}{}{}
	Height of any tree $\leq \lt\lfloor\log _2n\rt\rfloor$
\end{corolary}
\begin{lemma}{}{}
	The number of nodes that get assigned rank $k$ throughout the execution of the Union-Find data-structure is at most $\frac{n}{2^k}$. 
\end{lemma}
Define $N(r)=\#$vertices with rank at least $k$. Then by the above lemma we have $N(r)\leq \frac{n}{2^k}$.
\begin{lemma}{}{}
	 The time to perform a single find operation when we perform union by rank and path
	compression is $O(\log n)$ time.
\end{lemma}
We will show that we can do much better. In fact we will show that for $m$ operations over $n$ elements the overall running time is $O((n+m)\log ^*n)$

\begin{lemma}{}{}
	During a single $\prb{Find}(x)$ operation, the number of jumps between blocks along the search path is $O(\log^*n)$.
\end{lemma}
\begin{lemma}{}{}
	At most $|\textit{Block}(i)|\leq \textit{Tower}(i)$ many \prb{Find} operations can pass through an element $x$ which is in the $i^{th}$ block (i.e. $\prb{index}_B(x)=i$) before $x.\textit{parent}$ is no longer in the $i^{th}$ block. That is $\prb{index}_B(x.\textit{parent})>i$.
\end{lemma}

\begin{lemma}{}{}
	There are at most $\frac{n}{\textit{Tower}(i)}$ nodes that have ranks in the $i^{th}$ block throughout the algorithm execution.
\end{lemma}

\begin{lemma}{}{}
	The number of internal jumps performed, inside the $i^{th}$ block, during the lifetime of \prb{Union-Find} data structure is $O(n)$.
\end{lemma}

\begin{Theorem}{}{}
	The number of internal jumps performed by the \prb{Union-Find} data structure overall $O(n\log^*n)$.
\end{Theorem}
\begin{Theorem}{}{}
	The overall time spent on $m$ \prb{Find} operations, throughout the lifetime of a Union-Find data structure defined over $n$ elements is $O((n+m)\log^*n)$.
\end{Theorem}