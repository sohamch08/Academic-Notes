\chapter{Max Flow Min Cut}
\section{Maximum Flow}
Suppose we are given a directed graph $G=(V,E)$ with a source vertex $s$ and a target vertex $t$. And additionally for every edge $e\in E$ we are given a number $c_e\in \bbZ_+$ which is called the capacity of the edge.
\begin{Definition}{Flow}{}
	An $s-t$ flow is a function $f:E\to\bbR_+$ which satisfies the following:\begin{enumerate}[label=\protect\circled{\small\arabic*}]
		\item $\forall\ e\in E$, $f(e)\leq c_e$
		\item $\forall\ v\in V\setminus\{s,t\}$, $\sum\limits_{e\in \textit{in}(v)}f(e)=\sum\limits_{e\in \textit{out}(v)}f(e)$
	\end{enumerate}
Also the value of a flow $f$ is denoted by $|f|\coloneqq \sum\limits_{e\in \textit{out}(s)}f(e)$.
\end{Definition}Before proceeding into the setup and the problem first we will assume some things
\begin{assumption*} 
	\begin{itemize}
		\item $\textit{in}(s)=\emptyset$ i.e. there is no edge into $s$.
		\item $\textit{out}(t)=\emptyset$ i.e. there is no edge out of $t$.
		\item There are no parallel edges
	\end{itemize}
\end{assumption*}
\begin{lemma}{}{}
	For any flow $f$, $|f|=\sum\limits_{e\in \textit{in}(t)}f(e)$
\end{lemma}
\begin{proof}
	We have for every edge $e\in E$, $\exs\ v\in V$ such that $e\in \textit{in}(v)$ and $\exs\ u\in V$ such that $e\in \textit{out}(u)$. Hence we get $$\sum_{e\in E}f(e)=\sum_{v\in V}\sum_{e\in\textit{in}(v)}f(e)=\sum_{v\in V}\sum){e\in\textit{out}(v)}f(e)\implies \sum_{v\in V}\lt[\sum_{e\in\textit{in}(v)}f(e)-\sum_{e\in\textit{out}(v)}f(e)\rt]=0$$
	Now we know $\forall\ v\in V\setminus\{s,t\}$. $\sum\limits_{e\in\textit{in}(v)}f(e)=\sum\limits_{e\in\textit{out}(v)}f(e)$. Therefore we get $$ \sum_{v\in V}\lt[\sum_{e\in\textit{in}(v)}f(e)-\sum_{e\in\textit{out}(v)}f(e)\rt]=0\implies \sum_{v\in \{s,t\}}\lt[\sum_{e\in\textit{in}(v)}f(e)-\sum_{e\in\textit{out}(v)}f(e)\rt]=0\implies \sum\limits_{e\in \textit{out}(s)}f(e)-\sum\limits_{e\in \textit{in}(t)}f(e)$$Hence we have $|f|=\sum\limits_{e\in \textit{in}(t)}f(e)$.
\end{proof}

%Now the problem we will study is to given such a graph find a flow which maximizes the value.
\begin{algoprob}
	\problemtitle{Max Flow}
	\probleminput{A directed graph $G=(V,E)$ with source vertex $s$ and target vertex $t$ and for all edge $e\in E$ capacity of the edge $c_e\in\bbZ_+$}
	\problemquestion{Given such a graph and its capacities find an $s-t$ flow which has the maximum value}
\end{algoprob}

\begin{Example}{}{}
	Consider the following directed graph with capacities: $V=\{s,t,u,v\},\quad c_{s,u}=2,c_{s,v}=c_{u,t}=c_{v,t}=c_{u,v}=1$. Firstly the following function: $f':f'(s,u)=2=f(u,t)$. It is not a flow since $f(u,t)=2>1=c_{u,t}$. Now we define three different flow functions:
	\begin{center}
	\begin{tikzpicture}[scale=1.5]
	% Draw vertices with circles around them
\begin{scope}[shift={(-5,0)}]
		\node[draw, circle] (A) at (0, 0) {$s$};
	\node[draw, circle] (B) at (2, 1) {$u$};
	\node[draw, circle] (C) at (2, -1) {$v$};
	\node[draw, circle] (D) at (4, 0) {$t$};
	
	% Draw directed edges with values and reduced bends
	\draw[,-latex, bend left=15] (A) to node[below] {2} (B);
	\draw[,-latex, bend left=15] (B) to node[below] {1} (D);
	\draw[,-latex] (B) -- node[left] {1} (C);
	\draw[,-latex, bend right=15] (A) to node[above] {1} (C);
	\draw[,-latex, bend right=15] (C) to node[above] {1} (D);
	
	
    % Draw a single continuous red path from A to B to C to D
\draw[red!80!black, thick, ,-latex, rounded corners=5pt] 
($(A) + (0.1, 0.4)$) to [bend left=15] node[above] {1} ($(B) + (-0.4, 0.2)$) 
-- node[left] {1} ($(C) + (-0.4, -0.35)$) to [bend right=18] node[below] {1}($(D) + (-0.2, -0.5)$);
\draw[blue, thick, ,-latex, rounded corners=12pt, bend left=70] ($(A) + (0, 0.6)$) to node[above, xshift=-2.5cm, yshift=-0.85cm]{1} node[above, xshift=2.5cm, yshift=-0.85cm] {1}($(D) + (-0, 0.6)$);
\draw[blue, thick, ,-latex, rounded corners=12pt, bend right=70] ($(A) + (0, -0.6)$) to node[below, xshift=-2.5cm, yshift=0.85cm]{1} node[below, xshift=2.5cm, yshift=0.85cm]{1} ($(D) + (0, -0.6)$);
\draw[green!60!black, thick, rounded corners=12pt, bend left=23] ($(A) + (0.5, 0)$) to node[below]{2} ($(B) + (0.2,-0.39)$);
\draw[green!60!black, thick,,-latex, rounded corners=12pt, bend left=17]  ($(B) + (0.2,-0.39)$)  to node[below]{1} ($(D) + (-0.5,0)$)  ;
\draw[green!60!black, thick, rounded corners=5pt,-latex]  ($(B) + (0.2,-0.39)$)  to  node[right]{1}($(C) + (0.2,0.35)$)  to [bend right=23] node[above]{1} ($(D) + (-0.5,0)$);

\end{scope}
\begin{scope}[shift={(1.5,0)}]
	\draw (1,0) node[text width=8cm]{\begin{itemize}
			\item \textcolor{red!80!black}{$f\colon f(s,u)=f(u,v)=f(v,t)=1$ and otherwise $0$. Therefore $|f|=1$}
			\item \textcolor{blue}{$g\colon g(s,u)=g(u,t)=1$, $g(s,v)=g(v,t)=1$ and otherwise $0$. Therefore $|g|=2$}
			\item \textcolor{green!50!black}{$h\colon h(s,u)=2$, $h(u,t)=h(u,v)=h(v,t)=1$ and otherwise $0$. Therefore $|h|=2$}
	\end{itemize}\vspace*{5mm}

Notice here $g$ and $h$ has the maximum flow value.};
\end{scope}
\end{tikzpicture}
\end{center}
\end{Example}


\section{Ford Fulkerson Algorithm}
\begin{Definition}{Residual Graph}{}
	Given a directed graph $G=(V,E)$ and capacities $C_e$ for all $e\in E$ and an $s-t$ flow $f$ the residual graph $G_f=(V,E_f)$ has the edges with the following properties:
	\begin{enumerate}[label=\protect\circled{\arabic*}]
		\item If $(u,v)\in E$ and $f(u,v)>0$  then $(v,u)\in E_f$ and $c_{v,u}^f=f(u,v)$. Such an edge is called a ``backward" edge.
		\item If $(u,v)\in E$ and $f(u,v)<c_{u,v}$ then $(u,v)\in E_f$ and $c_{u,v}^f=c_{u,v}-f(u,v)$. It is called ``forward" edge.
	\end{enumerate}
\end{Definition}

\begin{algorithm}
\DontPrintSemicolon
\SetKwComment{Comment}{//}{}
\KwIn{Directed graph $G=(V,E)$, source $s$, target $t$ and edge capacities $C_e$ for all $e\in E$}
\KwOut{Flow $f$ with maximum value}
\Begin{
	\For{$e\in E$}{$f(e)=0$}
	\While{$\exs\ s\rightsquigarrow t$ path $P$ in $G_f$}{
	$\dl\longleftarrow \min\limits_{e\in P}\{c_e^f\}$
	\For{$e=(u,v)\in P$}{
	\If{$e$ is Forward Edge}{$f(u,v)\longleftarrow f(u,v)+\dl$}
	\Else{$f(u,v)\longleftarrow f(v,u)-\dl$}
}
}	
}
\caption{\prb{Max-Flow}}
\end{algorithm}