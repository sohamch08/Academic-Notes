\chapter{Dijkstra Algorithm with Data Structures}
\begin{algoprob}
	\problemtitle{Minimum Weight Path}
	\probleminput{Directed Graph $G=(V,E)$, $s\in V$ is source and $W=\{w_e\in \bbZ_+\colon e\in E\}$}
	\problemquestion{$\forall\ v\in V-\{s\}$ find minimum weight path $s\rightsquigarrow v$.}
\end{algoprob}

This is the problem we will discuss in this chapter. In this chapter we will often use the term `shortest distance' to denote the minimum weight path distance. One of the most famous algorithm for finding out minimum weight paths to all vertices from a given source vertex is Dijkstra's Algorithm
\section{Dijkstra Algorithm}
We will assume that the graph is given as adjacency list. Dijkstra Algorithm is basically dynamic programming. Suppose $\delta(v)$ is the shortest path distance from $s\rightsquigarrow v$.  Then we have the following relation: 
$$\delta(v)=\min\limits_{u:(u,v)\in E}\{\delta(u)+e(u,v)\}$$And suppose for any vertex $v\in V-\{s\}$, $dist(v)$ be the distance from $s$ estimated by the algorithm at any point. This is why  Dijkstra's algorithm maintains a set $S$  of vertices   whose final shortest-path weights from the source $s$ have already been determined. The algorithm repeatedly selects the vertex $u\in V-S$ with minimum shortest-path estimate and estimates the distances of neighbors of $u$. So here is the algorithm:

\begin{algorithm}
	\SetKwComment{Comment}{// }{ }
	\DontPrintSemicolon
	\KwIn{Adjacency Matrix of digraph $G=(V,E)$, source vertex $s\in V$ and weight function $W=\{w_e\in \bbZ_+\colon e\in E\}$}
	\KwOut{$\forall\ v\in V-\{s\}$ minimum weight path from $s\rightsquigarrow v$}
	\Begin{
		$S\longleftarrow \emptyset$, $U\longleftarrow V$\;
		$dist(s)\longleftarrow 0$, $\forall\ v\in V-\{s\}$, $dist(v)\longleftarrow\infty$\;
		\While{$U\neq \emptyset$}{
			$u\longleftarrow \min\limits_{u\in U} dist(u)$ and remove $u$ from $U$\;
			$S\longleftarrow S\cup \{u\}$\;
			\For{$e=(u,v)\in E$}{$dist(v)\longleftarrow \min\{dist(v), dist(u)+w(u,v)\}$}
		}
	}
\caption{\prb{Dijkstra}$(G,s,W)$}
\end{algorithm}

Here below we give an example of how the Dijkstra algorithm works:
%\begin{center}
	
\begin{figure}[h]
	\centering
\begin{tikzpicture}[
		node distance = 12mm and 14mm,
		every state/.append style = {inner sep=0pt, fill=gray!10,
			minimum size=7mm},
		every edge/.style = {draw, -Stealth, bend angle=15},
		auto=right,
		]
		
		%
		%	\draw[blue!20, line width=5pt] 	(s1) to                     (s2);
		%
		\begin{scope}[shift={(-6,0)}]
			\node (s1) [state,fill=gray!30,label={left:$s$}]         {$\boldsymbol{0}$};
			\node (s2) [state, above right=of s1]   {$\boldsymbol{\infty}$};
			\node (s3) [state, right=of s2]         {$\boldsymbol{\infty}$};
			\node (s4) [state, below right =of s1]   {$\boldsymbol{\infty}$};
			\node (s5) [state, right=of s4]          {$\boldsymbol{\infty}$};
			\draw   (s1) edge [ "$10$"]             (s2)
			(s1) edge ["$5$"]						(s4)
			(s2) edge ["$1$"]             			(s3)
			(s2) edge [bend right, "$2$"]			(s4)
			(s3) edge [bend right, "$4$"] 			(s5)
			(s5) edge [bend right,"$6$"]  			(s3)
			(s5) edge [bend left=80,looseness=1.2,  "$7$", swap]	(s1)
			(s4) edge ["$2$"] 						(s5)
			(s4) edge ["$9$"]						(s3)
			(s4) edge [bend right, "$3$"]			(s2);
		\end{scope}
		
		\begin{scope}[shift={(0,0)}]
			\node (s1) [state,fill=gray!30,label={left:$s$},fill=black,text=white]         {$\boldsymbol{0}$};
			\node (s2) [state, above right=of s1]   {$\boldsymbol{10}$};
			\node (s3) [state, right=of s2]         {$\boldsymbol{\infty}$};
			\node (s4) [state,fill=gray!30, below right =of s1]   {$\boldsymbol{5}$};
			\node (s5) [state, right=of s4]          {$\boldsymbol{10}$};
			\draw[blue!20, line width=5pt] (s1) -- (s2);
			\draw[blue!20, line width=5pt] (s1) -- (s4);	
			\draw   (s1) edge [ "$10$"]             			(s2)
			(s1) edge ["$5$"]									(s4)
			(s2) edge ["$1$"]             						(s3)
			(s2) edge [bend right, "$2$"]						(s4)
			(s3) edge [bend right, "$4$"] 						(s5)
			(s5) edge [bend right,"$6$"]  						(s3)
			(s5) edge [bend left=80,looseness=1.2,"$7$", swap]	(s1)
			(s4) edge ["$2$"] 									(s5)
			(s4) edge ["$9$"]									(s3)			     			
			(s4) edge [bend right, "$3$"]						(s2);
			
		\end{scope}
		
		\begin{scope}[shift={(6,0)}]
			\node (s1) [state,label={left:$s$},fill=black,text=white]         {$\boldsymbol{0}$};
			\node (s2) [state, above right=of s1]   {$\boldsymbol{8}$};
			\node (s3) [state, right=of s2]         {$\boldsymbol{13}$};
			\node (s4) [state, below right =of s1,fill=black,text=white]   {$\boldsymbol{5}$};
			\node (s5) [state, right=of s4,fill=gray!30]          {$\boldsymbol{7}$};
			\draw[red!20, line width=5pt] (s1) -- (s4);
			\draw[blue!20, line width=5pt] (s4) to [bend right=15] (s2);
			\draw[blue!20, line width=5pt] (s4) -- (s5);	
			\draw[blue!20, line width=5pt] (s4) -- (s3);
			\draw   (s1) edge [ "$10$"]             			(s2)
			(s1) edge ["$5$"]									(s4)
			(s2) edge ["$1$"]             						(s3)
			(s2) edge [bend right, "$2$"]						(s4)
			(s3) edge [bend right, "$4$"] 						(s5)
			(s5) edge [bend right,"$6$"]  						(s3)
			(s5) edge [bend left=80,looseness=1.2,"$7$", swap]	(s1)
			(s4) edge ["$2$"] 									(s5)
			(s4) edge ["$9$"]									(s3)
			(s4) edge [bend right, "$3$"]						(s2);
		\end{scope}
		
		\begin{scope}[shift={(-6,-6)}]
			\node (s1) [state,fill=gray!30,label={left:$s$},fill=black,text=white]      {$\boldsymbol{0}$};
			\node (s2) [state,fill=gray!30, above right=of s1]   						{$\boldsymbol{8}$};
			\node (s3) [state, right=of s2]        		 								{$\boldsymbol{13}$};
			\node (s4) [state, below right =of s1,fill=black,text=white]   				{$\boldsymbol{5}$};
			\node (s5) [state, right=of s4,fill=black,text=white]          {$\boldsymbol{7}$};
			\draw[red!20, line width=5pt] (s1) -- (s4);
			\draw[red!20, line width=5pt] (s4) to [bend right=15] (s2);
			\draw[red!20, line width=5pt] (s4) -- (s5);	
			\draw[blue!20, line width=5pt] (s5) to [bend right=15] (s3);
			\draw[blue!20, line width=5pt] (s5) to [bend left=80,looseness=1.3]	(s1);	
			\draw   (s1) edge [ "$10$"]             			(s2)
			(s1) edge ["$5$"]									(s4)
			(s2) edge ["$1$"]             						(s3)
			(s2) edge [bend right, "$2$"]						(s4)
			(s3) edge [bend right, "$4$"] 						(s5)
			(s5) edge [bend right,"$6$"]  						(s3)
			(s5) edge [bend left=80,looseness=1.3,"$7$", swap]	(s1)
			(s4) edge ["$2$"] 									(s5)
			(s4) edge ["$9$"]									(s3)
			(s4) edge [bend right, "$3$"]						(s2);
		\end{scope}
		
		\begin{scope}[shift={(0,-6)}]
			\node (s1) [state,fill=gray!30,label={left:$s$},fill=black,text=white]         {$\boldsymbol{0}$};
			\node (s2) [state, above right=of s1,fill=black,text=white]   {$\boldsymbol{8}$};
			\node (s3) [state, right=of s2,fill=gray!30]         {$\boldsymbol{9}$};
			\node (s4) [state, below right =of s1,fill=black,text=white]   {$\boldsymbol{5}$};
			\node (s5) [state, right=of s4,fill=black,text=white]          {$\boldsymbol{7}$};
			\draw[red!20, line width=5pt] (s1) -- (s4);
			\draw[red!20, line width=5pt] (s4) to [bend right=15] (s2);
			\draw[red!20, line width=5pt] (s4) -- (s5);	
			\draw[red!20, line width=5pt] (s5) to [bend right=15] (s3);	
			\draw[blue!20, line width=5pt] (s2) -- (s3);	
			\draw   (s1) edge [ "$10$"]             			(s2)
			(s1) edge ["$5$"]									(s4)
			(s2) edge ["$1$"]             						(s3)
			(s2) edge [bend right, "$2$"]						(s4)
			(s3) edge [bend right, "$4$"] 						(s5)
			(s5) edge [bend right,"$6$"]  						(s3)
			(s5) edge [bend left=80,looseness=1.3,"$7$", swap]	(s1)
			(s4) edge ["$2$"] 									(s5)
			(s4) edge ["$9$"]									(s3)
			(s4) edge [bend right, "$3$"]						(s2);
		\end{scope}
		
		\begin{scope}[shift={(6,-6)}]
			\node (s1) [state,fill=gray!30,label={left:$s$},fill=black,text=white]         {$\boldsymbol{0}$};
			\node (s2) [state, above right=of s1,fill=black,text=white]   {$\boldsymbol{8}$};
			\node (s3) [state, right=of s2,fill=black,text=white]         {$\boldsymbol{9}$};
			\node (s4) [state, below right =of s1,fill=black,text=white]   {$\boldsymbol{5}$};
			\node (s5) [state,fill=gray!30, right=of s4,fill=black,text=white]          {$\boldsymbol{7}$};
			\draw[red!20, line width=5pt] (s1) -- (s4);
			\draw[red!20, line width=5pt] (s4) to [bend right=15] (s2);
			\draw[red!20, line width=5pt] (s4) -- (s5);	
			%	\draw[red!20, line width=5pt] (s5) to [bend right=15] (s3);	
			\draw[red!20, line width=5pt] (s2) -- (s3);	
			\draw   (s1) edge [ "$10$"]             			(s2)
			(s1) edge ["$5$"]									(s4)
			(s2) edge ["$1$"]             						(s3)
			(s2) edge [bend right, "$2$"]						(s4)
			(s3) edge [bend right, "$4$"] 						(s5)
			(s5) edge [bend right,"$6$"]  						(s3)
			(s5) edge [bend left=80,looseness=1.3,"$7$", swap]	(s1)
			(s4) edge ["$2$"] 									(s5)
			(s4) edge ["$9$"]									(s3)
			(s4) edge [bend right, "$3$"]						(s2);
		\end{scope}
	\end{tikzpicture} 
\caption{The execution of Dijkstra’s algorithm. The source s is the leftmost vertex. The 	shortest-path estimates appear within the vertices, and shaded edges indicate predecessor values. Black vertices are in the set $S$ and at any iteration of while loop the shaded vertex has the minimum value. At any iteration the red edges are the edges considered in minimum weight path from $s$ using only vertices in $S$.}
\end{figure}
%\end{center}      
%\begin{tikzpicture}[
%	every state/.append style = {inner sep=0pt, fill=gray!10,
%		minimum size=7mm},
%	every edge/.style = {draw}, auto=right,
%	shortcut/.code={\def\pv##1{\pgfkeysvalueof{/tikz/#1/##1}}},
%	third corner of triangle/.style={shortcut=triangle pars,
%		triangle pars/.cd,#1,
%		/tikz/insert path={
%			let \p1=($(\pv{A})-(\pv{B})$),\n1={sqrt(pow(\x1/1cm,2)+pow(\y1/1cm,2))},
%			\n2={atan2(\y1,\x1)} in
%			(intersection cs:first line={(\pv{A})--($(\pv{A})+({\n2-cosinelaw(\n1,\pv{b},\pv{a})}:1)$)},
%			second line={(\pv{B})--($(\pv{B})+({\n2+cosinelaw(\n1,\pv{a},\pv{b})}:1)$)})
%	}},
%	declare function={cosinelaw(\a,\b,\c)=acos((\a*\a+\b*\b-\c*\c)/(2*\a*\b));},
%	triangle pars/.cd,
%	A/.initial=A,B/.initial=B,a/.initial=2,b/.initial=2]
%	
%	\begin{scope}[shift={(-5,0)}]
%		\path[scale=0.3] 
%		node[state,label={left:$s$}, label={[blue]below:$0$}](1){1}
%		++ (-15:7)  node[state,label={[red]below:$\infty$}](2){2} 
%		edge["$7$"] (1)
%		[third corner of triangle={A=2,B=1,a=9,b=10}]
%		node[state,label={[red]above:$\infty$}] (3){3}
%		edge["$9$"] (1)
%		edge["$10$"] (2)
%		[third corner of triangle={A=2,B=3,a=11,b=15}] 
%		node[state, label={[red]above:$\infty$}] (4){4}
%		edge["$15$"] (2)
%		edge["$11$"] (3)
%		[third corner of triangle={A=3,B=1,a=14,b=7}] 
%		node[state,label={[red]above:$\infty$}] (6){6}
%		edge["$14$"] (1)
%		edge["$7$"] (3)
%		[third corner of triangle={A=4,B=6,a=9,b=6}]
%		node[state,label={[red]above:$\infty$}] (5){5}
%		edge["$6$"]  (4)
%		edge["$9$"] (6); 
%	\end{scope}
%	
%	\begin{scope}[shift={((5,0))}]
%		\path[scale=0.3] 
%		node[state,label={left:$s$}, label={[blue]below:$0$}](1){1}
%		++ (-15:7)  node[state,label={[red]below:$\infty$}](2){2} 
%		edge["$7$"] (1)
%		[third corner of triangle={A=2,B=1,a=9,b=10}]
%		node[state,label={[red]above:$\infty$}] (3){3}
%		edge["$9$"] (1)
%		edge["$10$"] (2)
%		[third corner of triangle={A=2,B=3,a=11,b=15}] 
%		node[state, label={[red]above:$\infty$}] (4){4}
%		edge["$15$"] (2)
%		edge["$11$"] (3)
%		[third corner of triangle={A=3,B=1,a=14,b=7}] 
%		node[state,label={[red]above:$\infty$}] (6){6}
%		edge["$14$"] (1)
%		edge["$7$"] (3)
%		[third corner of triangle={A=4,B=6,a=9,b=6}]
%		node[state,label={[red]above:$\infty$}] (5){5}
%		edge["$6$"]  (4)
%		edge["$9$"] (6); 
%		\draw[green!70!black, -latex, shorten <=1pt, shorten >=1pt] (1) to [bend right=30] (2);
%	\end{scope}
%\end{tikzpicture}  


Suppose at any iteration $t$, let $dist_t(v)$ denotes the distance $v$ from $s$ calculated by algorithm for any $v\in V$ and $S^{(t)}$ denote the content of $S$ at $t^{th}$ iteration. In order to show that the algorithm correctly computes the distances we prove the following lemma:
\begin{Theorem}{}{}
	For each $v\in S^{(t)}$, $\dl(v)=dist_t(v)$ for any iteration $t$. 
\end{Theorem}
\begin{proof}
	We will prove this induction. Base case is $|S^{(1)}|=1$. $S$ grows in size. Then only time $|S^{(1)}|=1$ is when $S^{(1)}=\{s\}$ and $d(s)=0=\dl(s)$. Hence for base case this is correct.
	
	Suppose this is also true for $t-1$. Let at $t^{th}$ iteration the vertex $u\in V-S$ is picked. By induction hypothesis for all $v\in S^{(t)}-\{u\}$, $dist_t(v)=dist_{t-1}(v)=\dl(v)$. So we have to show that $dist_{t}(u)=\dl(u)$. 
	
	Suppose for contradiction the shortest path from $s\rightsquigarrow u$ is $P$ and has total weight $=\dl(u)=w(P)<dist_t(u)$. Now $P$ starts with vertices from $S^{(t)}$ by eventually leaves $S$. Let $(x,y)$ be the first edge in $P$ which leaves $S$ i.e. $x\in S$ but $y\notin S$. By inductive hypothesis $dist_t(x)=\dl(x)$. Let $P_y$ denote the path $s\rightsquigarrow y$ following $P$. Since $y$ appears before $u$ we have $$w(P_y)=\dl(y)\leq \dl(u)=w(P)$$Now $$dist_t(y)\leq dist_t(x)+w(x,y)$$ since $y$ is adjacent to $x$. Therefore $$dist_t(y)\leq dist_t(x)+w(x,y)=\dl(y)\leq dist_t(y)\implies dist_t(y)=\dl(y)$$Now since both $u,y\notin S^{(t)}$ and the algorithm picked up $u$ we have $\dl(u)<dist_t(u)\leq dist_t(y)=\dl(y)$. But we can not have both $\dl(y)\leq \dl(u)$ and $\dl(u)<\dl(y)$. Hence contradiction. Therefore $\dl(u)=dist_t(u)$. Hence by mathematical induction for any iteration $t$, for all $v\in S^{(t)}$, $\dl(v)=dist_t(v)$. 
\end{proof}

Therefore by the lemma  after all iterations $S$ has all the vertices with their shortest distances from $s$ and henceforth the algorithm runs correctly.\parinf


\section{Data Structure 1: Linear Array}

\section{Data Structure 2: Min Heap}

\section{Amortized Analysis}

\section{Data Structure 3: Fibonacci Heap}