\chapter{Perfect Matching Polytope}
\section{Matching Polytope}
\section{Perfect Matching Polytope}
\dfn[pm-polytope]{Perfect Matching Polytope}{
	Let $G=(V,E)$ be a graph. For any perfect matching $M$ of $G$, consider the incidence vector $x^M=(x_e)_{e\in E}\in \bbR^E$ given by $$c_e^M=\begin{cases}
		1& \text{if $e\in M$}\\  0 & \text{o/w}
	\end{cases}$$ For any perfect matching $M$ of $G$ this vector $x^M$ is called as a \textit{Perfect Matching Point}. The bipartite perfect matching polytope of the graph $G$ is defined to the convex hull of all its perfect matching points,  $$PM(G)=\conv\{x^M\mid \text{$M$ is a perfect matching in $G$}\}$$
}
% https://courses.engr.illinois.edu/cs598csc/sp2010/Lectures/Lecture9.pdf
\section{Bipartite Perfect Matching Polytope}
It also defined like the perfect matching polytope where we just take the graph to be a bipartite graph. The following lemma form \cite{LovaszPlummer_1986_BOOK} gives a simple description of the perfect matching polytope of a bipartite graph $G$ 
\begin{Theorem}{\cite{LovaszPlummer_1986_BOOK}}{bipartite-polytope-equations}
	Let $G=(V,E)$ be a bipartite graph and $x=(x_e)_{e\in E}\in \bbR^E$. Then $x\in PM(G)$ if and only if \begin{align*}
		\sum_{e\in\dl(v)}x_e=1 &\quad v\in V,\\
		x_e\geq 0&\quad e\in E
	\end{align*}where for any $v\in V$, $\delta(v)$ denotes the set of edges incident on the vertex $v$.
\end{Theorem}
\begin{proof}

\end{proof}

