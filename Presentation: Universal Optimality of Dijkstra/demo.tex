\documentclass[10pt]{beamer}

\usetheme{metropolis}
\usepackage{appendixnumberbeamer}

\usepackage{booktabs}
\usepackage[scale=2]{ccicons}

\usepackage{pgfplots}
\usepgfplotslibrary{dateplot}

\usepackage{xspace}
\newcommand{\themename}{\textbf{\textsc{metropolis}}\xspace}
\DeclareMathOperator{\order}{Order}

\title{Universal Optimality of Dijkstra ALgorithm}
\subtitle{Using Fibonacci-Like Priority Queue with Working Sets}
\date{\today}
\author{Soham Chatterjee}
\institute{Oral Qualifier, STCS}
% \titlegraphic{\hfill\includegraphics[height=1.5cm]{logo.pdf}}
\metroset{block=fill}
\begin{document}

\maketitle

\begin{frame}{Table of contents}
  \setbeamertemplate{section in toc}[sections numbered]
  \tableofcontents[hideallsubsections]
\end{frame}

\section{Introduction}

\begin{frame}[fragile]{Metropolis}

  The \themename theme is a Beamer theme with minimal visual noise
  inspired by the \href{https://github.com/hsrmbeamertheme/hsrmbeamertheme}{\textsc{hsrm} Beamer
  Theme} by Benjamin Weiss.

  Enable the theme by loading

  \begin{verbatim}    \documentclass{beamer}
    \usetheme{metropolis}\end{verbatim}

  Note, that you have to have Mozilla's \emph{Fira Sans} font and XeTeX
  installed to enjoy this wonderful typography.
\end{frame}
\begin{frame}[fragile]{Sections}
  Sections group slides of the same topic

\begin{verbatim}    \section{Elements}\end{verbatim}

  for which \themename provides a nice progress indicator \ldots
\end{frame}

\section{Lower Bounding Query Complexity}

\begin{frame}{$OPT_Q(G)=\Omega(\log(\order(G)))$}
	\begin{theorem}
      For any directed or undirected graph $G$, any algorithm for the DO problem needs $\Omega(\log(\order(G)))$ comparison queries in expectation.
  \end{theorem}%\pause

  \begin{itemize}
      \item Let $A$ is any correct algorithm and  $L\in \order(G)$. 
      \item Given $L$ we have a weight assignment $w_L$ such that $L$ is unique order obtained from $w_L$ upon running Dijkstra. For each $L$ fix $w_L$. Let $\mathcal{W}$ be the collection of all such $w_L$. %\pause
      \item Let $C_L\in\{-1,0,1\}^*$ be the sequence of answers of comparisons made by $A$ on $(G,w_L)$. Then $\mathcal{C}:\mathcal{W}\to \{-1,0,1\}^*$, $\mathcal{C}(w_L)=C_L$ is a ternary prefix free code.%\pause
      \item By Shannon's source coding theorem for symbol codes any such code has expected length $\Omega(\log(|\mathcal{W}|))=\Omega(\log(\order(G)))$
  \end{itemize} 
  
\end{frame}

\begin{frame}{Barrier Sequence}
    Let $T$ be any tree. A \emph{Barrier}, $B\subseteq V(T)$ is a set of nodes where for any two vertices $u,v\in B$, $u$ is not ancestor of $v$ in $T$. \setlength{\parindent}{1cm}

    For two disjoint barriers, $B_1\prec B_2$ if no node of $B_2$ is predecessor of a node in $B_1$.

    $(B_1,\dots, B_k)$ is a \emph{barrier sequence} if whenever $i<j$, $B_i\prec B_j$. 
\vfill

    \begin{theorem}
        A sequence $(B_1,\dots, B_k)$ of pairwise disjoint vertex sets is barrier sequence if and only if for all $1\leq i\leq j\leq k$, $v\in B_j$ is not ancestor of any $u\in B_i$ in $T$.
    \end{theorem}
\end{frame}

\begin{frame}{Barriers Give Lower Bounds}
    \begin{theorem}
        Let $T$ be any spanning tree and $(B_1,\dots, B_k)$ be a barrier sequence of $T$. Then $\log(\order(G))=\Omega\left(\sum\limits_{i=1}^k|B_i|\log|B_i|\right)$
    \end{theorem}

    \begin{itemize}
        \item We have $\log(\order(G))\geq \log(\order(T))$. We'll show $\log(\order(T))\geq |B_1|!|B_2|!\cdots|B_k|!$. 
        \item Delete vertices of $B_k$ to get $T'$. By induction for the barrier sequence $(B_1,\dots, B_{k-1})$ for $T'$, $\log(\order(T'))\geq |B_1|!|B_2|!\cdots|B_{k-1}|!$. 
    \end{itemize}
\end{frame}

\begin{frame}{Barriers Give Lower Bounds}

    \begin{itemize}
        \item We can order vertices of $B_k$ in any order we want. There are $|B_k|!$ many orders. 
        \item For each order of $B_k$ and any order of $\order(T')$ we can just concatenate them to get an order of $T$.
    \end{itemize}
\vfill

    So finally we got the result: 
    \begin{alertblock}{Result}
		If $T$ is a spanning tree of $G$ and $(B_1,\dots, B_k)$ is a barrier sequence for $T$ then $$OPT_Q(G)=\Omega\left(\sum\limits_{i=1}^k|B_i|\log|B_i|\right)$$
	\end{alertblock}
\end{frame}

\begin{frame}{Construction of Barrier Sequence}
    Consider running Dijkstra algorithm until some time. Let $S$ is the set of nodes that are in the priority queue. 
    \begin{itemize}
        \item Notice that $S$ are the leaves of the partial exploration tree built so far which is a subgraph of final exploration tree.
        \item Therefore, $S$ is an incomparable set of the final exploration tree.
        \item $S$ forms a barrier. 
    \end{itemize}\vfill

    \begin{alertblock}{Result}
		At any time of the algorithm the set of elements in the priority queue forms a barrier
	\end{alertblock}
\end{frame}
\begin{frame}{Intersecting Coloring}
    \begin{definition}[Intersecting Coloring]
      An intersecting coloring of $\mathcal{I}$ with $k$ colors is a function $C:\mathcal{I}\to [k]$ that assigns a color to every interval and additionally for every color $i\in[k]$, $\bigcap\limits_{I\in\mathcal{I}, C(I)=i}I\neq \emptyset$.
    \end{definition}\vfill

    Every intersecting coloring induces a barrier sequence in the exploration tree in following way: For any color $c$, \begin{itemize}
        \item  $B_c=\{v\in V(G)\mid C(I(v))=c\}$
        \item $t_c=\min\{t\mid \forall\ v\in B_c, t\in I(v)\}$
        \item Order $\{B_c\}$ by increasing order of $\{t_c\}$. WLOG $t_1<\cdots <t_k$. 
        \item $(B_1,\dots, B_k)$ is a barrier sequence.
    \end{itemize}
\end{frame}

\begin{frame}{Intersecting Coloring Gives Lower Bounds}
    Let $C$ be an intersecting coloring of $\mathcal{I}$ with $k$ colors. 
    Let $(B_1,\dots, B_k)$ is the barrier sequence induced by $C$. Then let the energy of $C$ is defined to be $$E(C)=\sum\limits_{i=1}^k|B_i|\log |B_i|$$
\vfill

    \begin{alertblock}{Result}
		If $\mathcal{I}$ is the interval set induced by Dijkstra and $C$ be any arbitrary intersecting coloring of $\mathcal{I}$ then $$OPT_Q(G)=\Omega(E(C))$$
	\end{alertblock}
\end{frame}

\begin{frame}{Good Intersecting Coloring gives Optimality}
    \textbf{Goal:} Find an intersecting coloring of $\mathcal{I}$, $C$ such that $E(C)\geq Cost(\mathcal{I})$ 

    \begin{itemize}
        \item Then time complexity of all \textsc{ExtractMin} operations is $O(n+Cost(\mathcal{I}))=O(n+E(C))$.
        \item We have  $OPT_Q(G)=\Omega (E(C))$. 
        \item So overall Cost of \textsc{ExtractMin} in Dijkstra is upper bounded by $O(n+OPT_Q(G))$. 
        \item Dijkstra achieves universal optimality for time complexity. 
    \end{itemize}
\vfill

    We will find such a good intersecting coloring recursively.
\end{frame}
\begin{frame}{Deleting Intervals from $\mathcal{I}$}
    \begin{theorem}
        Let $\mathcal{I}$ an interval set and $x\in\mathcal{I}$. $k=\max\limits_t|\{I\in\mathcal{I}\mid t\in I\}|$. Then $$Cost(\mathcal{I})\leq Cost(\mathcal{I}\setminus \{x\})+\log |W_x|+\log k$$
    \end{theorem}\vfill

    \begin{itemize}
      \item Let $I_1,\dots, I_l\in\mathcal{I}$ are the only intervals which had nonempty intersection with $x$. So $l\leq k-1$.\vfill 
      \item Let $t_i$ is starting point of $I_i$. WLOG assume $t_l>\cdots>t_1$.\vfill 
      \item Let $W_i, W_i'$ are working sets of $I_i$ before and after removing $x$. 
    \end{itemize}
\end{frame}
\begin{frame}{Deleting Intervals from $\mathcal{I}$}
  \begin{itemize}
    \item Let $t$ is starting point of $x$. Then $W_{i,t}$ contains $x, I_1,\dots, I_i$. So $|W_i|\geq i+1$.
    \item $|W_i|\in\{|W_i'|,|W_i'|+1\}$ for all $i\in[l]$.
  \end{itemize}
     \begin{align*}
         &Cost(\mathcal{I})-Cost(\mathcal{I}\setminus \{x\})-\log |W_x|  \\ 
         =& \sum\limits_{i=1}^l \log|W_i|-\log|W_i'| \\ 
         \leq &  \sum\limits_{i=1}^l \log(i+1)-\log i =\log (l+1) \leq \log k
     \end{align*}

      \begin{alertblock}{Fact}
		For any working set $|W_x|=k$ we have $$Cost(\mathcal{I})\leq Cost(\mathcal{I}\setminus W_x)+2|W_x|\log |W_x|$$
	\end{alertblock}
\end{frame}
\begin{frame}{Finding Good Intersecting Coloring}
    We will construct $C$ by induction on $|\mathcal{I}|$. 
\end{frame}
\begin{frame}[standout]
  Questions?
\end{frame}

\appendix

\begin{frame}[fragile]{Backup slides}
  Sometimes, it is useful to add slides at the end of your presentation to
  refer to during audience questions.

  The best way to do this is to include the \verb|appendixnumberbeamer|
  package in your preamble and call \verb|\appendix| before your backup slides.

  \themename will automatically turn off slide numbering and progress bars for
  slides in the appendix.
\end{frame}

\begin{frame}[allowframebreaks]{References}

  \bibliography{demo}
  \bibliographystyle{abbrv}

\end{frame}

\end{document}
