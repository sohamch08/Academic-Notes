\documentclass[10pt]{beamer}

\usetheme{metropolis}
\usepackage{appendixnumberbeamer,amsmath}
\usepackage[ruled,vlined]{algorithm2e}
\usepackage{booktabs}
\usepackage[scale=2]{ccicons}

\usepackage{pgfplots}
\usepgfplotslibrary{dateplot}

\usepackage{xspace}
\newcommand{\themename}{\textbf{\textsc{metropolis}}\xspace}
\DeclareMathOperator{\order}{Order}

\usepackage{libertine}
\usepackage[libertine]{newtxmath}
\usepackage{mathrsfs}
\title{Universal Optimality of Dijkstra Algorithm}
\subtitle{Using Fibonacci-Like Priority Queue with Working Sets}
\date{\today}
\author{Soham Chatterjee}
\institute{Oral Qualifier, STCS}
\renewcommand{\thealgocf}{}
% \titlegraphic{\hfill\includegraphics[height=1.5cm]{logo.pdf}}
\metroset{block=fill}
\pagestyle{empty}
\begin{document}

\maketitle

% \begin{frame}{Table of contents}
% 	\setbeamertemplate{section in toc}[sections numbered]
% 	\tableofcontents[hideallsubsections]
% \end{frame}

\begin{frame}{Introduction}
	\begin{itemize}
		\item Dijkstra algorithm is a foundation algorithm solving Single Source Shortest Path problem (SSSP) both for directed and undirected graphs.\vfill
		\item Using Fibonacci Heaps we have the worst-case time complexity $\mathcolor{blue}{O(m+n\log n)}$. \vfill \pause
		\item Recently Duan, Mao, Shu and Yin in 2023 solved SSSP for undirected graphs with expected time $\mathcolor{blue}{O(m\sqrt{\log n\log\log n})}$\vfill\pause
		\item This year in STOC Duan, Mao, Mao, Shu, Yin solved SSSP for directed graphs in $\mathcolor{blue}{O(m\log^{\frac23}n)}$ time.
	\end{itemize}
\end{frame}
\begin{frame}{Assumptions}\begin{itemize}
		\item Input graph is always connected. \vfill
		\item All trees are rooted at $s$. \vfill
		\item For any vertex $v$, $T(v)$ denote the subtree of $T$ rooted at $v$.\vfill
		\item The weights of the graph are positive real numbers. \vfill
		\item We allow the $\infty$ in the weights.
	\end{itemize}

\end{frame}
\begin{frame}{Dijkstra Algorithm}

	\begin{algorithm}[H]
		\DontPrintSemicolon
		$F\longleftarrow \emptyset$, $\textsc{Insert}(F,s)$, 	$dist(s)\longleftarrow 0$\;
		\While{$F\neq \emptyset$}{
		$u\longleftarrow \textsc{ExtractMin}(F)$\;
		\For{$e=(u,v)\in E$}{
			If $v$ is unseen, $\textsc{Insert}(F,v)$ \;
			$\textsc{DecreaseKey}( F, v,\min\{dist(v), dist(u)+w(u,v)\})$}
		}
		\caption{\textsc{Dijkstra}$(G,s,w)$}
	\end{algorithm}\pause
	Dijkstra solves three problems:
	\begin{itemize}
		\item  Computes Shortest Distances\pause
		\item  Build Shortest Path Tree\pause
		\item Sorts vertices by Shortest Distance (DO)
	\end{itemize}
\end{frame}
\begin{frame}{Comparison-Addition Model}
	Notice the Dijkstra algorithm does the following operations: \begin{itemize}
		\item Adds two values
		\item Compares two values.
		\item Stores Values.
	\end{itemize}So we will work on a model where all possible operations are  addition, compare and storage. \vfill\pause

	For a given graph:
	\begin{itemize}
		\item $\mathcolor{blue}{OPT_Q(G)}$ is the number of comparison queries of an optimal algorithm for this graph.\vfill
		\item $\mathcolor{blue}{OPT(G)}$ be the number of total steps taken by an optimal correct algorithm for the graph.
	\end{itemize}

\end{frame}
\begin{frame}{Universal Optimality}
	\begin{itemize}
		\item Let $\mathcolor{blue}{\mathcal{A}}$ is the set of all correct algorithms.
		\item $\mathcolor{blue}{\mathcal{G}_{n,m}}$ is the set of all graphs with $n$ vertices and $m$ edges.
		\item $\mathcolor{blue}{\mathcal{W}_G}$ is the set of all possible weights for a graph $\mathcolor{blue}{G\in\mathcal{G}_{n,m}}$.
	\end{itemize} \pause\vfill
	A correct algorithm $\mathcolor{blue}{A^*}$ is \emph{existentially optimal} if $$\mathcolor{blue}{\forall\ n,m: \sup\limits_{\substack{G\in\mathcal{G}_{n,m}\\ w\in \mathcal{W}_G} }A^*(G,w)\leq \alpha\inf\limits_{A\in \mathcal{A}}\sup\limits_{\substack{G\in\mathcal{G}_{n,m}\\ w\in \mathcal{W}_G} } A(G,w)}$$ where $\mathcolor{blue}{\alpha=\tilde{O}(1)}$. This corresponds to being optimal wrt worst-case complexity.\pause  \vfill

	But this is not good. It is just saying $\mathcolor{blue}{A^*}$ may take as much time as it takes in a star-graph or more complicated one.
\end{frame}
\begin{frame}{Universal Optimality}
	We want a notion of optimality which says your algorithm is optimal compared to any other algorithm if you \textcolor{red}{\textbf{fix the graph}}.\pause  \vfill

	A correct algorithm $\mathcolor{blue}{}A^*$ is \emph{universally optimal} if$$\mathcolor{blue}{\forall\ n,m,\ \forall\ G\in \mathcal{G}_{n,m}: \sup\limits_{w\in \mathcal{W}_G} A^*(G,w)\leq \alpha\inf\limits_{A\in \mathcal{A}}\sup\limits_{w\in \mathcal{W}_G} A(G,w)}$$ where $\mathcolor{blue}{\alpha=\tilde{O}(1)}$.  \pause \vfill

	In this discussion we will focus solely on  $\mathcolor{blue}{\alpha=O(1)}$.
\end{frame}
\begin{frame}{Exploration Tree and DO}
	Consider a run of Dijkstra. Whenever a vertex is extracted add the unexplored neighbors of that vertex as children of that vertex. The tree built this way is called the exploration tree.\pause \vfill

	\begin{itemize}
		\item Let $\mathcolor{blue}{T}$ be the exploration tree. Let $\mathcolor{blue}{\prec}$ be the final distance ordering of the vertices.\pause \vfill
		\item Then for every edge $\mathcolor{blue}{(u,v)\in T$, $u\prec v}$.
	\end{itemize}
\end{frame}
\begin{frame}{Order of Vertices by a Tree}
	\begin{definition}[Order of $T$]
		Let $\mathcolor{blue}{T}$ be any tree in $\mathcolor{blue}{G}$. An order of $\mathcolor{blue}{T}$ is a total order of $\mathcolor{blue}{V(T)}$ such that for every edge $\mathcolor{blue}{(u,v)\in E(T)}$ we have $\mathcolor{blue}{u\prec v}$ in the order.
	\end{definition}\pause\vfill
	The DO after Dijkstra is an order of exploration tree.
	\begin{itemize}
		\item $\mathcolor{blue}{L}$ is an order of $\mathcolor{blue}{G}$ if there exists a spanning tree $\mathcolor{blue}{T}$ of $\mathcolor{blue}{G}$ such that $\mathcolor{blue}{L}$ is an order of $\mathcolor{blue}{T}$.\pause
		\item $\mathcolor{blue}{\order(G)}$ is the number of all possible orders of $\mathcolor{blue}{G}$.
	\end{itemize}\pause \vfill

	\begin{lemma}
		For any graph $\mathcolor{blue}{G}$, $\mathcolor{blue}{L}$ is an order of $\mathcolor{blue}{G}$ iff there exists non-negative weights $\mathcolor{blue}{w}$ such that \begin{enumerate}
			\item For every two nodes $\mathcolor{blue}{u\neq v}$, $\mathcolor{blue}{d_w(s,u)\neq d_w(s,v)}$.
			\item $\mathcolor{blue}{u\prec_L v}$ if and only if $\mathcolor{blue}{d_w(s,u)<d_w(s, v)}$.
		\end{enumerate}
	\end{lemma}
\end{frame}
\begin{frame}{Dijkstra Induced Interval Set}
	For any vertex $\mathcolor{blue}{v\in V(G)}$\begin{itemize}
		\item $\mathcolor{blue}{l_v}$: When $\mathcolor{blue}{v}$ was first discovered and added to the heap.
		\item $\mathcolor{blue}{r_v}$: When $\mathcolor{blue}{v}$ was removed from heap.
		\item $\mathcolor{blue}{[l_v,r_v]}$: Interval set of $\mathcolor{blue}{v}$
	\end{itemize} \pause \vfill

	A run of Dijkstra induces intervals for each vertex $v\in V$ with the operations \textcolor{red}{\textsc{Insert}} and \textcolor{red}{\textsc{ExtractMin}}.\pause  \vfill

	An interval set $\mathcolor{blue}{\mathcal{I}}$ is collection of intervals for each vertex. It is called Dijkstra Induced when all the intervals for each vertex in $\mathcolor{blue}{\mathcal{I}}$ is induced by a run of Dijkstra on some $\mathcolor{blue}{(G,w)}$.
\end{frame}
\begin{frame}{Working set of an Interval Set}
	Let $\mathcolor{blue}{\mathcal{I}}$ any interval set.\pause

	\begin{itemize}
		\item For any vertex $\mathcolor{blue}{v\in V(G)}$ at any time $\mathcolor{blue}{t\in I(v)}$ the working set $\mathcolor{blue}{W_{v,t}}$ is the set of vertices inserted after $\mathcolor{blue}{v}$ and still present at time $\mathcolor{blue}{t}$. So $$\mathcolor{blue}{W_{v,t}=\{[l_u,r_u]\in\mathcal{I}\colon l_v\leq l_u\leq t\leq r_u\}}$$\pause \vfill
		\item Working set of $\mathcolor{blue}{v}$, $\mathcolor{blue}{W_v=W_{v,t^*}}$ such that $\mathcolor{blue}{t^*=\arg\max\limits_t|W_{v,t}|}$.\pause  \vfill
		\item The cost of a vertex $\mathcolor{blue}{v\in V(G)}$ is $\mathcolor{blue}{Cost(v)=\log|W_v|}$. And so $\mathcolor{blue}{Cost(\mathcal{I})=\sum\limits_{v\in V(G)} \log |W_v|}$.
	\end{itemize}
\end{frame}
\begin{frame}{Fibonacci-Like Priority Queue with Working Set Property}
	FPQWSP  is a type of Fibonacci Heap which satisfies the amortized time complexity for any sequence of operations as follows:\vfill \pause

	\begin{center}
		\begin{tabular}{|c|c|c|}
			\hline
			                     & FPQWSP           & Fibonacci Heap \\ \hline
			\textsc{Insert}      & $O(1)$           & $O(1)$         \\ \hline
			\textsc{DecreaseKey} & $O(1)$           & $O(1)$         \\ \hline
			\textsc{ExtractMin}  & $O(1+\log|W_x|)$ & $O(\log n)$    \\ \hline
		\end{tabular}
	\end{center}\pause
	\vfill


	\begin{alertblock}{Fact}
		There is a FPQWSP for Dijkstra. We will use this data structure in every argument from now on by default.
	\end{alertblock}
\end{frame}
\begin{frame}{Time Complexity of Dijkstra}
	In Dijkstra Algorithm it runs $\mathcolor{blue}{n}$ times \textcolor{red}{\textsc{ExtractMin}} calls for each vertex and $\mathcolor{blue}{m}$ times \textcolor{red}{\textsc{DecreaseKey}} calls.\pause
	\begin{itemize}
		\item Hence total time taken by all \textcolor{red}{\textsc{DecreaseKey}} calls is $\mathcolor{blue}{O(m)}$.\pause
		\item Total time taken by all \textcolor{red}{\textsc{ExtractMin}} calls is
	\end{itemize}
	$$
		\mathcolor{blue}{\sum\limits_{v\in V(G)}O(1+\log|W_v|)= O\left(n+\sum\limits_{v\in V(G)} \log |W_v| \right)= O(n+Cost(\mathcal{I}))}
	$$
	\begin{itemize}
		\item Total time taken by Dijkstra is $\mathcolor{blue}{O(m+n+Cost(\mathcal{I}))}$
	\end{itemize}
\end{frame}
\begin{frame}{Main Theorem}
	\begin{theorem}
		Dijkstra implemented by FPQWSP in Comparison-Addition model has time complexity $\mathcolor{blue}{O(OPT_Q(G)+m+n)}$.
	\end{theorem}

	\textbf{Goal}: We'll show $\mathcolor{blue}{OPT_Q(G)=\Omega(Cost(\mathcal{I}))}$.\vfill

	\begin{itemize}
		\item $\mathcolor{blue}{OPT_Q(G)\leq OPT(G)}$\pause
		\item $\mathcolor{blue}{OPT(G)=\Omega(n)}$\pause
		\item $\mathcolor{blue}{OPT(G)=\Omega(m)}$
	\end{itemize}\vfill

	So $\mathcolor{blue}{OPT_Q(G)+n+m=O(OPT(G))}$.

\end{frame}
% \section{Lower Bounding Query Complexity}
\begin{frame}{Proof Flow}
	\begin{alertblock}{Fact}
		$\mathcolor{blue}{OPT_Q(G)=\Omega(\log(\order(G)))}$
	\end{alertblock}\vfill\pause
	\begin{itemize}
		\item Partition the exploration tree into non-comparable sets $\mathcolor{blue}{(B_1,\dots, B_k)}$ with $\mathcolor{blue}{i<j}$ then no node of $\mathcolor{blue}{B_j}$ is ancestor of any node of $\mathcolor{blue}{B_i}$. \vfill\pause
		\item For any such partition $\mathcolor{blue}{\log(\order(G))=\Omega\left(\sum\limits_{i=1}^k|B_i|\log|B_i|\right)}$\vfill\pause
		\item There is a partition such that $\mathcolor{blue}{2\sum\limits_{i=1}^k|B_i|\log|B_i|\geq Cost(\mathcal{I})}$
	\end{itemize}

\end{frame}


\begin{frame}{Barrier Sequence}
	\begin{definition}[Barrier]
		Let $\mathcolor{blue}{T}$ be any tree. A \emph{Barrier}, $\mathcolor{blue}{B\subseteq V(T)}$ is a set of nodes where for any two vertices $\mathcolor{blue}{u,v\in B}$, $\mathcolor{blue}{u}$ is not ancestor of $\mathcolor{blue}{v}$ in $\mathcolor{blue}{T}$.
	\end{definition}
	\pause
	\begin{itemize}
		\item For two disjoint barriers, $\mathcolor{blue}{B_1\prec B_2}$ if no node of $\mathcolor{blue}{B_2}$ is predecessor of a node in $\mathcolor{blue}{B_1}$.\pause
		\item $\mathcolor{blue}{(B_1,\dots, B_k)}$ is a \emph{barrier sequence} if  $\mathcolor{blue}{i<j\implies B_i\prec B_j}$.\pause
	\end{itemize}
	\vfill

	\begin{lemma}
		A sequence $\mathcolor{blue}{(B_1,\dots, B_k)}$ of pairwise disjoint vertex sets is barrier sequence if and only if for all $\mathcolor{blue}{1\leq i\leq j\leq k}$, $\mathcolor{blue}{v\in B_j}$ is not ancestor of any $\mathcolor{blue}{u\in B_i}$ in $\mathcolor{blue}{T}$.
	\end{lemma}
\end{frame}

\begin{frame}{Barriers Give Lower Bounds}
	\begin{lemma}
		Let $T$ be any spanning tree and $\mathcolor{blue}{(B_1,\dots, B_k)}$ be a barrier sequence of $\mathcolor{blue}{T}$. Then $\mathcolor{blue}{\log(\order(G))=\Omega\left(\sum\limits_{i=1}^k|B_i|\log|B_i|\right)}$
	\end{lemma}\vfill \pause

	\begin{itemize}
		\item We have $\mathcolor{blue}{\order(G))\geq \order(T)}$. We'll show $\mathcolor{blue}{\order(T)\geq |B_1|!|B_2|!\cdots|B_k|!}$.\pause \vfill
		\item Delete vertices of $\mathcolor{blue}{B_k}$ to get $\mathcolor{blue}{T'}$. By induction for the barrier sequence $\mathcolor{blue}{(B_1,\dots, B_{k-1})}$ for $\mathcolor{blue}{T'}$, $\mathcolor{blue}{\order(T')\geq |B_1|!|B_2|!\cdots|B_{k-1}|!}$.
	\end{itemize}
\end{frame}

\begin{frame}{Barriers Give Lower Bounds}

	\begin{itemize}
		\item We can order vertices of $\mathcolor{blue}{B_k}$ in any order we want. There are $\mathcolor{blue}{|B_k|!}$ many orders. \pause
		\item For each order of $\mathcolor{blue}{B_k}$ and any order of $\mathcolor{blue}{\order(T')}$ we can just concatenate them to get an order of $\mathcolor{blue}{T}$.\pause
	\end{itemize}
	\vfill

	So finally we got the result:
	\begin{alertblock}{Result}
		If $T$ is a spanning tree of $\mathcolor{blue}{G}$ and $\mathcolor{blue}{(B_1,\dots, B_k)}$ is a barrier sequence for $\mathcolor{blue}{T}$ then $$\mathcolor{blue}{OPT_Q(G)=\Omega\left(\sum\limits_{i=1}^k|B_i|\log|B_i|\right)}$$
	\end{alertblock}
\end{frame}

\begin{frame}{Barriers in the Heap}
	Consider running Dijkstra algorithm until some time. Let $\mathcolor{blue}{S}$ is the set of nodes that are in the priority queue.\pause
	\begin{itemize}
		\item Notice that $\mathcolor{blue}{S}$ is the leaves of the partial exploration tree built so far which is a subgraph of final exploration tree.\pause
		\item Therefore, $\mathcolor{blue}{S}$ is an incomparable set of the final exploration tree.
		\item $\mathcolor{blue}{S}$ forms a barrier.
	\end{itemize}\pause \vfill

	\begin{alertblock}{Result}
		At any time of the algorithm the set of elements in the priority queue forms a barrier
	\end{alertblock}
\end{frame}
\begin{frame}{Intersecting Coloring}
	A barrier sequence is basically coloring vertices in a certain way where vertices in a barrier have same color.\pause
	\begin{definition}[Intersecting Coloring]
		An intersecting coloring of $\mathcolor{blue}{\mathcal{I}}$ with $\mathcolor{blue}{k}$ colors is a function $\mathcolor{blue}{C:\mathcal{I}\to [k]}$ that assigns a color to every interval and additionally for every color $\mathcolor{blue}{i\in[k]}$, $\mathcolor{blue}{\bigcap\limits_{I\in\mathcal{I}, C(I)=i}I\neq \emptyset}$.
	\end{definition}\pause \vfill

	Every intersecting coloring induces a barrier sequence in the exploration tree in following way: For any color $\mathcolor{blue}{c}$,\pause
	\begin{itemize}
		\item  $\mathcolor{blue}{B_c=\{v\in V(G)\mid C(I(v))=c\}}$\pause
		\item $\mathcolor{blue}{t_c=\min\{t\mid \forall\ v\in B_c, t\in I(v)\}}$\pause
		\item Order $\mathcolor{blue}{\{B_c\}}$ by increasing order of $\mathcolor{blue}{\{t_c\}}$. WLOG $\mathcolor{blue}{t_1<\cdots <t_k}$.\pause
		\item $\mathcolor{blue}{(B_1,\dots, B_k)}$ is a barrier sequence for exploration tree.
	\end{itemize}
\end{frame}

\begin{frame}{Intersecting Coloring Gives Lower Bounds}
	Let $\mathcolor{blue}{C}$ be an intersecting coloring of $\mathcolor{blue}{\mathcal{I}}$ with $\mathcolor{blue}{k}$ colors.
	Let $\mathcolor{blue}{(B_1,\dots, B_k)}$ is the barrier sequence induced by $\mathcolor{blue}{C}$. Then let the energy of $\mathcolor{blue}{C}$ is defined to be $$\mathcolor{blue}{E(C)=2\sum\limits_{i=1}^k|B_i|\log |B_i|}$$\pause
	\vfill

	\begin{alertblock}{Result}
		If $\mathcolor{blue}{\mathcal{I}}$ is the interval set induced by Dijkstra and $\mathcolor{blue}{C}$ be any arbitrary intersecting coloring of $\mathcolor{blue}{\mathcal{I}}$ then $$\mathcolor{blue}{OPT_Q(G)=\Omega(E(C))}$$
	\end{alertblock}
\end{frame}

\begin{frame}{Good Intersecting Coloring gives Optimality}
	\textbf{Goal:} Find an intersecting coloring of $\mathcolor{blue}{\mathcal{I}}$, $\mathcolor{blue}{C}$ such that $\mathcolor{blue}{E(C)\geq Cost(\mathcal{I})}$\pause

	\begin{itemize}
		\item Then time complexity of all \textcolor{red}{\textsc{ExtractMin}} operations is $\mathcolor{blue}{O(n+Cost(\mathcal{I}))=O(n+E(C))}$.\pause
		\item We have  $\mathcolor{blue}{OPT_Q(G)=\Omega (E(C))}$.\pause
		\item So overall Cost of \textcolor{red}{\textsc{ExtractMin}} in Dijkstra is upper bounded by $\mathcolor{blue}{O(n+OPT_Q(G))}$.\pause
		\item Dijkstra achieves universal optimality for time complexity.\pause
	\end{itemize}
	\vfill

	We will find such a good intersecting coloring recursively.
\end{frame}
\begin{frame}{Finding Good Intersecting Coloring}
	\begin{itemize}
		\item We will construct $\mathcolor{blue}{C}$ by induction on $\mathcolor{blue}{|\mathcal{I}|}$.\pause
		\item Find the interval $\mathcolor{blue}{x\in\mathcal{I}}$ with the largest $\mathcolor{blue}{W_x}$. Use induction on $\mathcolor{blue}{\mathcal{I}'=\mathcal{I}\setminus W_x}$\pause
		\item Let $\mathcolor{blue}{C'}$ is the coloring for $\mathcolor{blue}{\mathcal{I}'}$ such that $\mathcolor{blue}{E(C')\geq Cost(\mathcal{I}')}$. Add a new color for all the elements in $\mathcolor{blue}{W_x}$ to get new coloring $\mathcolor{blue}{C}$.\pause
		\item $\mathcolor{blue}{E(C)=E(C')+2|W_x|\log|W_x|}$ by definition. \pause
	\end{itemize}







	\begin{alertblock}{Fact}
		For working set $\mathcolor{blue}{W_x}$ with the largest size $$\mathcolor{blue}{Cost(\mathcal{I})\leq Cost(\mathcal{I}\setminus W_x)+2|W_x|\log |W_x|}$$
	\end{alertblock}\pause
	\begin{itemize}
		\item $\mathcolor{blue}{Cost(\mathcal{I})\leq Cost(\mathcal{I}')+2|W_x|\log|W_x|}$. Hence, $\mathcolor{blue}{E(C)\geq Cost(\mathcal{I})}$.
	\end{itemize}

\end{frame}
\begin{frame}[standout]
	Thank You
\end{frame}
\begin{frame}{$OPT_Q(G)=\Omega(\log(\order(G)))$}
	\begin{lemma}
		For any directed or undirected graph $\mathcolor{blue}{G}$, any algorithm for the DO problem needs $\mathcolor{blue}{\Omega(\log(\order(G)))}$ comparison queries in expectation.
	\end{lemma}

	\begin{itemize}
		\item Let $\mathcolor{blue}{A}$ is any correct algorithm and  $\mathcolor{blue}{L\in \order(G)}$.
		\item Given $\mathcolor{blue}{L}$ we have a weight assignment $\mathcolor{blue}{w_L}$ such that $\mathcolor{blue}{L}$ is unique order obtained from $\mathcolor{blue}{w_L}$ upon running Dijkstra. For each $\mathcolor{blue}{L}$ fix $\mathcolor{blue}{w_L}$. Let $\mathcolor{blue}{\mathcal{W} }$ be the collection of all such $\mathcolor{blue}{w_L}$.
		\item Let $\mathcolor{blue}{C_L\in\{-1,0,1\}^*}$ be the sequence of answers of comparisons made by $\mathcolor{blue}{A}$ on $\mathcolor{blue}{(G,w_L)}$. Then $\mathcolor{blue}{\mathcal{C}:\mathcal{W}\to \{-1,0,1\}^*}$, $\mathcolor{blue}{\mathcal{C}(w_L)=C_L}$ is a ternary prefix free code.
		\item By Shannon's source coding lemma for symbol codes any such code has expected length $\mathcolor{blue}{\Omega(\log(|\mathcal{W}|))=\Omega(\log(\order(G)))}$
	\end{itemize}

\end{frame}
\begin{frame}{Deleting Intervals from $\mathcal{I}$}
	\begin{lemma}
		Let $\mathcolor{blue}{\mathcal{I}}$ an interval set and $\mathcolor{blue}{x\in\mathcal{I}}$. $\mathcolor{blue}{k=\max\limits_t|\{I\in\mathcal{I}\mid t\in I\}|}$. Then $$\mathcolor{blue}{Cost(\mathcal{I})\leq Cost(\mathcal{I}\setminus \{x\})+\log |W_x|+\log k}$$
	\end{lemma}\vfill

	\begin{itemize}
		\item Let $\mathcolor{blue}{I_1,\dots, I_l\in\mathcal{I}}$ are the only intervals which had nonempty intersection with $\mathcolor{blue}{x}$. So $\mathcolor{blue}{l\leq k-1}$.\vfill
		\item Let $\mathcolor{blue}{t_i}$ is starting point of $\mathcolor{blue}{I_i}$. WLOG assume $\mathcolor{blue}{t_l>\cdots>t_1}$.\vfill
		\item Let $\mathcolor{blue}{W_i, W_i'}$ are working sets of $\mathcolor{blue}{I_i}$ before and after removing $\mathcolor{blue}{x}$.
	\end{itemize}
\end{frame}
\begin{frame}{Deleting Intervals from $\mathcal{I}$}
	\begin{itemize}
		\item Let $\mathcolor{blue}{t}$ is starting point of $\mathcolor{blue}{x}$. Then $\mathcolor{blue}{W_{i,t}}$ contains $\mathcolor{blue}{x, I_1,\dots, I_i}$. So $\mathcolor{blue}{|W_i|\geq i+1}$.
		\item $\mathcolor{blue}{|W_i|\in\{|W_i'|,|W_i'|+1\}}$ for all $\mathcolor{blue}{i\in[l]}$.
	\end{itemize}
	\begin{align*}
		                       & \mathcolor{blue}{Cost(\mathcal{I})-Cost(\mathcal{I}\setminus \{x\})-\log |W_x|} \\
		\mathcolor{blue}{=}    & \mathcolor{blue}{\sum\limits_{i=1}^l \log|W_i|-\log|W_i'|}                      \\
		\mathcolor{blue}{\leq} & \mathcolor{blue}{\sum\limits_{i=1}^l \log(i+1)-\log i =\log (l+1) \leq \log k}
	\end{align*}

	\begin{alertblock}{Fact}
		For any working set $\mathcolor{blue}{|W_x|=k}$ we have $$\mathcolor{blue}{Cost(\mathcal{I})\leq Cost(\mathcal{I}\setminus W_x)+2|W_x|\log |W_x|}$$
	\end{alertblock}
\end{frame}

\end{document}
