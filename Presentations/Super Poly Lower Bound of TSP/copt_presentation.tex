% !TEX program = xelatex
\documentclass[aspectratio=1610, handout]{beamer}
\usepackage[T1]{fontenc}
\usetheme{wildcat}
\usepackage{xcolor, mathtools, optidef}
\usepackage{tikz}
\usetikzlibrary{decorations.pathreplacing, arrows.meta, shapes, calc, positioning}
\DeclareMathOperator{\poa}{\mathsf{PoA}}
\input{../../letterfonts}
\input{macros}
\title{Super-Polynomial Lower Bound of TSP Extended Formula}
\date{May 2025}
\author{Soham Chatterjee}

% You change the titlegraphic to whatever you want, or comment it out to remove it.
% \titlegraphic{\includegraphics[scale=0.25]{logo-northwestern.pdf}}

% % You can directly change the colors using the following macros.
% % You must redefine colors AFTER the theme is loaded.
% % For example, these provide shades of Yale Blue (#00356b)
% \definecolor{wcprimary}{RGB}{0,53,107}      % Main color
% \definecolor{wcprimary140}{RGB}{0, 34, 70}
% \definecolor{wcprimary130}{RGB}{0, 40, 80}
% \definecolor{wcprimary120}{RGB}{0, 45, 91}
% \definecolor{wcprimary110}{RGB}{0, 50, 102}
% \definecolor{wcprimary40}{RGB}{153, 174, 196}
% \definecolor{wcprimary30}{RGB}{179, 194, 211}
% \definecolor{wcprimary20}{RGB}{204, 215, 225}
% \definecolor{wcprimary10}{RGB}{230, 235, 240}

% % Now for the alerted orange (#bd5319) and example green (#5f712d)
% \definecolor{wcalerted}{RGB}{189,83,25}
% \definecolor{wcexample}{RGB}{95,113,45}

% % If you want to change the slide background color, 
% % you can use the following command:
%\setbeamercolor{background canvas}{bg=nupurple10!30}

% % Turn off section slides
% \AtBeginSection{}

% Change the font theme
%\usefonttheme{wildcat-overleaf}

% Change the bg pattern manually: Simple Single Color
% \renewcommand{\bgpattern}{
%     \draw[color=wcprimary,fill=wcprimary] (0,0) rectangle (\paperwidth,\paperheight);
% }

\definecolor{doc}{HTML}{DCBCD0}
\definecolor{myg}{RGB}{56, 140, 70}
\definecolor{myb}{RGB}{45, 111, 177}
\definecolor{myr}{RGB}{199, 68, 64}
\definecolor{mybg}{HTML}{F2F2F9}
\definecolor{mytheorembg}{HTML}{F2F2F9}
\definecolor{mytheoremfr}{HTML}{00007B}
\definecolor{myexamplebg}{HTML}{F2FBF8}
\definecolor{myexamplefr}{HTML}{88D6D1}
\definecolor{myexampleti}{HTML}{2A7F7F}
\definecolor{mydefinitbg}{HTML}{E5E5FF}
\definecolor{mydefinitfr}{HTML}{3F3FA3}
\definecolor{notesgreen}{RGB}{0,162,0}
\definecolor{myp}{RGB}{197, 92, 212}
\definecolor{mygr}{HTML}{2C3338}
\definecolor{myred}{RGB}{127,0,0}
\definecolor{myyellow}{RGB}{169,121,69}
\definecolor{OrangeRed}{HTML}{ED135A}
\definecolor{Dandelion}{HTML}{FDBC42}
\definecolor{light-gray}{gray}{0.95}
\definecolor{Emerald}{HTML}{00A99D}
\definecolor{RoyalBlue}{HTML}{0071BC}
\definecolor{mytoccolor}{HTML}{886830}


\renewcommand{\P}{\ensuremath{\mathsf{P}}}
\newcommand{\PSPACE}{\ensuremath{\mathsf{PSPACE}}}
\newcommand{\TQBF}{\ensuremath{\mathsf{TQBF}}}
\newcommand{\CONP}{\ensuremath{\mathsf{coNP}}}
\newcommand{\NP}{\ensuremath{\mathsf{NP}}}
\newcommand{\ONP}{\ensuremath{\mathsf{ONP}}}
\newcommand{\EXP}{\ensuremath{\mathsf{EXP}}}
\newcommand{\RP}{\ensuremath{\mathsf{RP}}}
\newcommand{\RE}{\ensuremath{\mathsf{RE}}}
\newcommand{\RL}{\ensuremath{\mathsf{RL}}}
\newcommand{\R}[1]{\ensuremath{\mathsf{R#1}}}
\newcommand{\BPP}{\ensuremath{\mathsf{BPP}}}
\newcommand{\SIZE}{\ensuremath{\mathsf{SIZE}}}
\newcommand{\DTIME}{\ensuremath{\mathsf{DTIME}}}
\newcommand{\DSPACE}{\ensuremath{\mathsf{DSPACE}}}
\newcommand{\NL}{\ensuremath{\mathsf{NL}}}
\newcommand{\NSIZE}{\ensuremath{\mathsf{NSIZE}}}
\newcommand{\poly}{\ensuremath{\mathsf{poly}}}
\newcommand{\AC}{\ensuremath{\mathsf{AC}}}
\newcommand{\NC}{\ensuremath{\mathsf{NC}}}
\newcommand{\PH}{\ensuremath{\mathsf{PH}}}
\newcommand{\SAT}{\ensuremath{\mathsf{SAT}}}
\newcommand{\CONE}{\ensuremath{\mathsf{coNEXP}}}
\newcommand{\NE}{\ensuremath{\mathsf{NEXP}}}

% ibliographystyle{alpha}
% \bibliography{refs}\b

\begin{document}

\begin{frame}
	\titlepage
\end{frame}



\begin{frame}{Introduction}
	\begin{definition}[Travelling Salesman]
		Given a graph $G=(V,E)$, $S\subseteq V$ and weights $w:E\to \bbR$ find minimum weight cycle which visits every vertex of $S$ exactly once.
	\end{definition}\pause

	We will focus on $S=V$.
	\begin{itemize}
		\item We know Traveling Salesman Problem is $\NP$-complete.
		\item In [Yannkakis, 1988, STOC] he proved every symmetric LP for the TSP has expnential size.
		\item Here we will show TSP admits no polynomial-size LP.
		\item This proof also shows unconditional super-polynomial lower bound on the number of inequalities.
		\item Therefore it is impossible to prove $\P=\NP$ by means of a polynomial size LP.
	\end{itemize}
\end{frame}
\begin{frame}{Table of Contents}
    \tableofcontents
\end{frame}
\section{Preliminaries}
\begin{frame}{Definitions}
	Let $P=\{x\in\bbR^n\mid Ax\leq b\}=conv(V)$ is a polytope  with $A\in\bbR^{m\times d}, b\in \bbR^m$ and $V\subseteq \bbR^d$. We will consider $V$ as the characteristic vector for all hamiltonian paths.\pause

	\begin{definition}[Extension Polytope]
		An extension of $P$ is a polytope $Q\subseteq \bbR^{d+e}$ such that there is a linear map $\pi:\bbR^{d+e}\to\bbR^{d}$ such that $\pi(Q)=P$.
	\end{definition}\pause

	\begin{definition}[Extended Formula]
		An EF $Q$ is an extension of $P$ is a linear system in variable s $(x,y)$ such that $$x\in P\iff \exs\ y\ (x,y)\in Q$$\pause

	\end{definition}
\end{frame}
\begin{frame}{Extension Complexity}
	Extension complexity of $P$ is the minimum size  EF of $P$ where size of a polytope is the number inequalities. We denote by $xc(P)$.\pause\vspace*{1cm}

	\begin{lemma}
		Let $P,Q$ and $F$ be polytopes. Then the following holds:
		\begin{enumerate}[label=(\roman*)]
			\item If $F$ is an extension of $P$ then $xc(F)\geq xc(P)$.
			\item If $F$ is a face of $Q$ then $xc(Q)\geq xc(F)$.
		\end{enumerate}
	\end{lemma}
\end{frame}
\begin{frame}{Slack Matrix}
	\begin{definition}
		Let $P=\{x\in\bbR^n\mid Ax\leq b\}=conv(V)$ is a polytope  with $A\in\bbR^{m\times d}, b\in \bbR^m$ and $V\subseteq \bbR^d$. Let $V=\{v_1,\dots, v_n\}$. Then $S\in\bbR^{m\times n}_{0}$ is called the slack matrix of $P$ wrt $Ax\leq b$ and $V$ where $$S(i,j)=b_i-A_iv_j$$
	\end{definition}

	Some times we may refer to the submatrix of slack matrix induced by rows corresponding to facets as the slack matrix of $P$ denoted by $S(P)$.

	% \begin{theorem}
	% 	The following are equivalent:\begin{enumerate}[label=(\roman*)]
	% 		\item $S$ has non-negative rank at most $r$.
	% 		\item $P$ has an extension of size at most $r$.
	% 		\item $P$ has an EF of size at most $r$.
	% 	\end{enumerate}
	% \end{theorem}
\end{frame}
\begin{frame}{Some Polytopes}

	\begin{itemize}
		\item $TSP(n)$ is the traveling salesman polytope for $K_n=(V_n,E_n)$. Let $C\subseteq E_n$ denotes a tour of $K_n$. Then $\chi^C$ denotes the characteristic vector of $C$. Then \pause

		      $$TSP(n)\coloneqq conv\{\chi^C\mid C\subseteq E_n\text{ is a tour of $K_n$}\}$$\pause

		\item Given $G=(V,E)$, for any $S\subseteq V$, $\chi^S$ denote characteristic vector of $S$. Then the independent set polytope\pause

		      $$IND(G)\coloneqq conv\{\chi^S\mid S\text{ is independent set of $G$}\}$$
		\item The correlation polytope $COR(n)$ is $$COR(n)\coloneqq conv\{bb^T\mid b\in\{0,1\}^n\}$$
	\end{itemize}
\end{frame}
\begin{frame}{Proof Flow}
	\begin{theorem}
		$xc(TSP(n))=2^{\Om\left(n^{\frac14}\right)}$
	\end{theorem}\pause

	\begin{enumerate}[label=Step \arabic*:]
		\item First we will prove $xc(COR(n))=2^{\Om(n)}$\pause \vspace*{5mm}

		\item For all $n$, $\exs\ $ graph $G_n$ with $n$ vertices such that $xc(IND(G_n))\geq xc(COR(n'))$ where $n'=n^{\frac1d}$ for some $d>1$.\pause\vspace*{5mm}

		\item For any $n$-vertex graph $G$, $IND(G)$ is linear projection of a face of $TSP(k)$ where $k=O(n^2)$.
	\end{enumerate}
\end{frame}

\section{Covering Bound of Matrix and Non-negative Factorization}


\begin{frame}{Covering Bound of Matrix with Rectangles}
	\begin{itemize}
		\item Let $M\in\{0,1\}^{n\times n}$ matrix. \pause\vspace*{5mm}
		
		\item A monochromatic rectangle $R$ in $M$ means a submatrix $N$ of $M$ whose all entries are $1$.\pause\vspace*{5mm}
		
		\item A collection of rectangles $\mcC$ covers $M$ if their union covers all the nonzero entries of $M$.\pause\vspace*{5mm}
		
		\item $|\mcC|$ is called a covering bound of $M$. $Cov(X)=\min\{|\mcC|\colon \mcC\text{ covers $M$}\}$
	\end{itemize}
\end{frame}

\begin{frame}{Covering Bound of Simple Matrix}
	Consider A matrix $X$ of dimension $2^n\times 2^n$ where the rows and columns are indexed by strings from $\{0,1\}^n$. Let $X(a,b)=(1-a^Tb)^2$ where $a,b\in\{0,1\}^n$.\pause\vspace*{5mm}
	
	\begin{theorem}[Yannkakis, 1988, STOC]
		Every monochromatic rectangle cover of $suppmat(X)$ has size $2^{\Om(n)}$ i.e. $$Cov(suppmat(X))\geq 2^{\Om(n)}$$
	\end{theorem}
\end{frame}

\begin{frame}{Non-negative Factorization}
	\begin{itemize}
		\item A rank $r$ non-negative factorization of a matrix $M$ is a factorization $M=TU$ where $T,U$ are non-negative matrix with $r$ columns and $r$ rows respectively.\pause
		
		\item Non-negative rank of $M$ is the minimum rank of a non-negative factorization of $M$. Denote it by $\Rank_+(M)$.
	\end{itemize}\pause
		\begin{theorem}[Factorization Theorem]
		For a polytope $P=\{x\mid Ax\leq b\}$ where $S$ is the slack matrix of $P$ the following are equivalent:\begin{enumerate}[label=(\roman*)]
			\item $S$ has non-negative rank at most $r$.
			\item $P$ has an extension of size at most $r$.
			\item $P$ has an EF of size at most $r$.
		\end{enumerate}
	\end{theorem}
	
	We get $xc(P)=\Rank_+(S)$.
\end{frame}


\begin{frame}{Factorization and Covering Bound Relation}
	For any matrix $M\in\bbR^{m\times n}$ let $suppmat(M)\in\{0,1\}^{m\times n}$ is a matrix where the $(i,j)^{th}$ element is $1$ if $M(i,j)\neq 0$ and otherwise $0$.\vspace*{1cm}

	\begin{theorem}[Yannkakis, 1988, STOC]
		Let $M$ be any matrix with non-negative real entries. Then $$\Rank_+(M)\geq Cov(suppmat(M))$$
	\end{theorem}
\end{frame}
\section{Correlation Polytope Lower Bound}
\begin{frame}{Polytope equations}
	Consider the inner product of two matrices $A,B\in\bbR^{m\times n}$ be $\la A,B\ra=Tr(A^TB)$\pause

	For all $a\in \{0,1\}^n$ there are some $b\in\{0,1\}^n$ such that $$\la 2\text{diag}(a)-aa^T,bb^T\ra=1$$ \begin{align*}
		1-\la \text{diag}(a)-aa^T,bb^T\ra & = 1-2\la \text{diag}(a),bb^T\ra+\la aa^T,bb^T\ra\\
		& = 1-2a^Tb+(a^Tb)^2=(1-a^Tb)^2
	\end{align*}\pause%Hence take $b$ such that $\text{supp}(b)\subseteq [n]\setminus\text{supp}(a)$.\pause

	\begin{tblock}{Remark}
		Because of above prove for all $b\in COR(n)$, for all $a\in\{0,1\}^n $, $\la 2\text{diag}(a)-aa^T,bb^T\ra\leq 1$. 
	\end{tblock}
	Hence let $A,b$ be such that $COR(n)=\{x\mid Ax\leq b\}$ where $(A,b)$ includes these inequities. So the slack matrix $S$ of $COR(n)$ contains $X$.
\end{frame}
\begin{frame}{Lower Bound}
	Let $S$ is the slack matrix of $COR(n)$. Then $S$ contains the matrix $X$. \vspace*{5mm}
	
	\begin{itemize}
		\item By Factorization Theorem $xc(COR(n))=\Rank_+(S)$.\pause \vspace*{5mm}
		
		\item Since $X$ is submatrix of $S$ we have $\Rank_+(S)\geq \Rank_+(X)$.\pause\vspace*{5mm}
		
		
		\item By Covering-Factorization Relation $\Rank_+(X)\geq Cov(suppmat(X))\geq 2^{\Om(n)}$.
	\end{itemize}\vspace*{5mm}
		
	\begin{theorem}
		$xc(COR(n))=2^{\Om(n)}$.
	\end{theorem}
\end{frame}
\section{Independent Set Polytope Lower Bound}
\begin{frame}{New Graph Construction}
	Let fix an $n$. Now consider the complete graph $K_n$. Now we will construct a graph $H_n=(V_n,E_n)$ with $O(n^2)$ vertices.\pause
	
	\begin{itemize}
		\item Each vertex $i\in K_n$ there is a 2-clique on $i,\hat{i}$ in $H_n$.\pause
		
		\item Each edge $(i,j)\in K_n$ \begin{itemize}
			\item There is a 4-clique on the vertices $\{ij,\hat{i}j,i\hat{j},\hat{i}\hat{j}\}$.
			\item The additional edges\begin{align*}
				(i,\hat{i}j)  && (\hat{i},ij) && (j,i\hat{j}) && (\hat{j}, ij)\\
				(i,\hat{i}\hat{j}) && (\hat{i},i\hat{j}) && (j,\hat{i}\hat{j}) && (\hat{j},\hat{i}j)
			\end{align*}
		\end{itemize}
	\end{itemize}\pause

	Let $F$ is the face of $IND(H_n)$ containing independent sets which have exactly one vertex from each vertex-clique and one vertex from each edge-clique
\end{frame}
\begin{frame}{$COR(n)$ Inside  Independent Set Polytope}
	Take the linear map $\pi:\bbR^{V_n}\to \bbR^{n\times n}$. Let $\pi(x)=y$. Then $$y_{ij}=y_{ji}=x_{ij}$$\pause
	\begin{itemize}
		\item $S$ is independent set of $H_n$. $\chi^S$ is the characteristic vector.\pause
		
		\item Define $b\in \{0,1\}^n$ where $b_i=1$ iff $ii\in S$ otherwise $0$
	\end{itemize}\pause

	Observe: For edge $(i,j)\in K_n$, $ij\in S\iff ii,jj\in S$.

	Then $\pi(\chi^S)=bb^T$. So $\pi(F)\subseteq COR(n)$\pause

	\begin{itemize}
		\item $b\in\{0,1\}^n$. Consider $bb^T$. \pause
		
		\item $S$ contains a vertex $ii$ if $b_i=1$ and $S$ contains $\hat{i}$ if $b_i=0$.
	\end{itemize} \pause

	$\chi^S\in F$. So $\pi(\chi^S)=bb^T$. So $$\pi(F)=COR(n)$$ So $COR(n)$ is a face of $IND(H_n)$.
\end{frame}
\begin{frame}{Lower Bound}
	Above $H_n$ has $2n+\binom{n}2$ vertices.\pause\vspace*{5mm}
	
	\begin{itemize}
		\item For any $n$ consider $p$ to be the maximum such that $2p+\binom{p}2\leq n$. \vspace*{5mm}\pause
		
		\item Take the graph $H_p$ and add $n-2p-\binom{p}2$ isolated vertices to construct $G_n$. \vspace*{5mm}\pause
		
		\item $IND(H_p)$ isomorphic to $IND(G_n)$
	\end{itemize}\pause

	$$xc(IND(G_n))=xc(IND(H_p))\geq xc(COR(p))\geq 2^{\Om(p)}=2^{\Om\lt(n^{\frac12}\rt)}$$
\begin{theorem}
	For all $n\in\bbN$ there exists graph $G_n$, $xc(IND(G_n))=2^{\Om\lt(n^{\frac12}\rt)}$
\end{theorem}
\end{frame}
\section{TSP Polytope Lower Bound}
\begin{frame}
	\begin{theorem}[Yannkakis, 1988, STOC]
		Every $p$-vertex graph $G$, $IND(G)$ is the linear projection of a face of $TSP(n)$ with $n=O(p^2)$.
	\end{theorem}
\vspace*{5mm}\pause

	Therefore 
	
	$$xc(TSP(n))\geq xc(IND(G_p))=2^{\Om\lt(p^{\frac12}\rt)}=2^{\Om\lt(n^{\frac14}\rt)}$$
\end{frame}

\standout{Thank You}
\end{document}