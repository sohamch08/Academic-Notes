\chapter{Hensel Lifting}
The hensel method described here will lift an approximate factorization of a polynomial over a Hensel Ring $R$ with valuation $v$ where the factors are relatively prime. We will show a linear convergence and a quadratic convergence behavior for the liftings.
\section{Hensel Ring}
\dfn{Hensel Ring}{
A ring with valuation $v:R\to \bbR_{\geq 0}$ is called a Hensel Ring if:\begin{enumerate}[label=(\roman*)]
	\item $\forall\ a\in R$, $v(a)\leq 1$
	\item $\forall\ a,b\in R$, $\forall\ \eps>0$, $\exs\ c\in R$ such that $(v(a)\leq v(b)\implies v(a-bc)\leq \eps)$
\end{enumerate}
}
In other words $R$ is Hensel iff it is contained and dense in the valuation ring of its quotient field (with respect to the unique extension of $v$). We sometimes call such $v$ a Hensel Valuation. 

In condition (ii) we assume we can compute the $c$ efficiently.
\begin{Theorem}{}{}
	Condition (i) of Hensel Ring $\implies v$ is Non-Archimedean.
\end{Theorem}
\begin{proof}
	Let $a,b\in R$. Now \begin{align*}
		v(a+b)^k & = v((a+b)^k)\\
		& = v\lt(\sum_{i=0}^k{{k}\choose{i}}a^{n-i}b^i  \rt)\leq \sum_{i=0}^kv\lt(\binom{k}{i}\rt)v(a)^{n-i}v(b)^i\\
		&\leq \sum_{i=0}^kv(a)^{n-i}v(b)^i\leq \sum_{i=0}^kM^{n-i}M^i &[m=\max{v(a),v(b)}]\\
		&=\sum_{i=0}^kM^k=M^k(k+1)
	\end{align*}Hence $$\lt(\frac{v(a+b)}{M}\rt)^k\leq (k+1)\iff \frac{v(a+b)}{M}\leq (1+k)^{\frac1k}$$As $k\to \infty$ the RHS approaches 1 so $v(a+b)\leq M$. 
\end{proof}

\begin{example}{$p-$adic Valuations}{}
	\begin{itemize}
		\item $\bbZ$ with $p-$adic valution $v_p$ where $p\in\bbN$ is prime is a Hensel ring. Here $v_p(a)=p^{-n}$ where $n=\max\{k\geq 0\mid p^k\mid a\}$
		\item $\bbF[y]$ with $p-$adic valutaion $v_p$ where $p\in \bbF[y]$ is an irreducible polynomial is a Hensel Ring. Here $v_p(f)=2^{-n\deg p}$ where $n=\max\{k\geq 0\mid p^k\mid f\}$
	\end{itemize}
\end{example}

 \nt{From the valuation $v$ over $R$ we naturally get a valuation $v$ over the polynomial ring $R[x]$ by defining $$\forall\ f\in R[x],\ \text{let }f=\sum_{i=0}^nf_ix^i,\text{ then } v\lt(\sum_{i=0}^nf_ix^i\rt)=\max_{i}\{v(f_i)\}$$}
\section{Conditions related to Hensel's Lemma}
We will define 5 conditinos. First suppose we have:
\begin{flushleft}
	\begin{tabular}{rll}
		(1) & $f\in R[x]$ & \\
		(2) & $f_0,\dots, f_m\in R[x]$ & $\mcF=\{f_i\colon 0\leq i\leq m\}$\\
		(3) & $f^*_0,\dots, f^*_m\in R[x]$ & $\mcF^*=\{f^*_i\colon 0\leq i\leq m\}$\\
		(4) & $s_0,\dots, s_m\in R[x]$ & $\mcS=\{s_i\colon 0\leq i\leq m\}$\\
		(5) & $s^*_0,\dots, s^*_m\in R[x]$ & $\mcS^*=\{s^*_i\colon 0\leq i\leq m\}$\\
		(6) & $z\in R$ & \\
		(7) & $\alpha,\delta,\eps\in \bbR$ & \\
		(8) & $\delta ^*\in \bbR$ & \\
		(9) & $\gm=\max\{\delta,\alpha \eps\}$		
	\end{tabular}
\end{flushleft}
As you can see the set $\mcF^*$ basically represents the lift of $\mcF$ but here since we are saying the conditions in more generality we are not assuming any relations among them and we define some conditions involving them.
\begin{itemize}
	\item $\boldsymbol{H_1(m,f,\mcF,\mcS,\eps)}\coloneqq v\lt(f-\prod\limits_{i=0}^mf_i\rt)\leq\eps<1$ 
	\item $\boldsymbol{H_2(m,f,\mcF,\mcS,z,\delta)}\coloneqq v\lt(\sum\limits_{i=0}^m s_i\prod\limits_{j\neq i}f_i -z   \rt)<leq \delta<1$
	\item $\boldsymbol{H_3(m,f,\mathcal{F},S,z,\alpha,\delta,\epsilon)}\coloneqq$ 
	$\begin{array}[t]{@{}rl@{}}
		(1) &f_1,\dots,f_m \text{ are monic}\\
		(2) &\deg\left(\prod\limits_{i=0}^mf_i\right)\leq \deg f\\
		(3) &\deg s_i\leq \deg f_i \ \forall\  i\in[m]\\
		(4) &\alpha\delta\leq 1, \alpha\epsilon^2\leq 1\\
		(5) &1\leq \alpha v(z)
	\end{array}$
	\item $\boldsymbol{H_4(m,\mcF,\mcF^*,\mcS,\mcS^*,\alpha,\delta,\eps)}\coloneqq$ $\begin{array}[t]{@{}rll@{}}
		(1) & v(f^*_i-f_i)\leq \alpha\eps\ &\forall\ 0\leq i\leq m\\
		(2) & v(s^*_i-s_i)\leq \alpha\eps\ &\forall\ 0\leq i\leq m\\
		(3) &\deg f_i^*= \deg f_i\ &\forall \ i\in[m]\\
		(4) &\deg s_i<\deg f_i\implies \deg s_i^*<\deg f_i^*\ &\forall\ i\in [m]
	\end{array}$
	\item $\boldsymbol{H_5(m,f,\mcF,\mcF^*,\mcS,\mcS^*, \alpha,\delta,\eps,\delta^*)}\coloneqq $ Let $p\in[m]$. Then suppose
	\begin{itemize}
		\item $\mcI_p=\{I_0,I_1,\dots,I_p\}$ be a partition of $\{0,\dots,m\}$ with $o\in I_0$. 
		\item $\overline{\mcF}_p^m=\{\ovf_i\colon i\in[p]\}\subseteq R[x]$ be a set of monic polynomials
	\end{itemize}
	Then define: $$F_i=\prod_{j\in I_i} f_j,\qquad F_i^*=\prod_{j\in I_i}f_j^*,\qquad \mfs_i^*=\sum_{j\in I_i} s_j\frac{F_i^*}{f_i^*}$$So now we denote:$$\sF=\{F_i\colon 0\leq i\leq p\},\qquad \sF^*=\{F_i\colon 0\leq^* i\leq p\}, \qquad \sS=\{\mfs_i^*\colon 0\leq i\leq p\} $$
	Assume:\begin{enumerate}
		\item $v(\ovf_i-F_i)\leq \alpha \eps $ $\forall \ i\in[p]$
		\item $\alpha v(s_i)\leq 1$ $\forall\ 0\leq i\leq m$
		\item $\alpha\delta <1$, $\alpha^2\delta \leq 1$
		\item $\alpha^2\eps<1$, $\alpha^3\eps\leq 1$
	\end{enumerate}Then the following are equivalent:
	\begin{enumerate}[label=(\roman*)]
		\item $\exs$ $\ovf_0,$ $\ovs_0,\dots,\ovs_p\in R[x]$ denote $$\overline{\mcF}=\{\ovf_i\colon 0\leq i\leq p\},\qquad \overline{\mcS}=\{\ovs_i\colon 0\leq i\leq p\}$$then the following conditions are true:\begin{enumerate}
			\item $H_1(p,f,\overline{\mcF},\overline{\mcS},\eps^*)$
			\item $H_2(p,f,\overline{\mcF},\overline{\mcS},z,\delta^*)$
			\item $H_3(p,f,\overline{\mcF},\overline{\mcS},z,\alpha^*, \delta^*,\eps^*)$
			\item $H_4(p,f,\sF,\overline{\mcF},\sS, \overline{\mcS},z,\alpha^*,\delta^*,\eps^*)$
		\end{enumerate}where $\alpha^*=\alpha$, $\eps^*=\alpha\eps\gm$
		\item $\exs$ $\ovf_0\in R[x]$ such that $H_1(p,f,\overline{\mcF},\overline{\mcS},\eps^*)$ is true
		\item $\forall\ i\in[p]$ we have $v(\ovf_i-F_i^*)\leq \eps^*$.
	\end{enumerate}
\end{itemize}
The first 3 conditions here togather imply that: From $H_1$ we get that $f_0\cdots f_m$ is a good approximation of factorization of $f$ with $\eps$-precision, $H_2\implies z$ plays a similar role to the gcd of $f_0,\dots,f_m$ and it shows the generalized bezout's identity for gcd  for multiple elements. In the usual treatment of Hensel's Lemma $f_0,\dots, f_m$ are relatively prime (more precisely their images in the residue class field or $R$ modulo the maximal ideal $\la a\in R\mid v(a)<1\ra$ satisfy the assumption then one can find $s_0,\dots,s_m,\delta$ satisfying $H_2$ with $z=1$. One can set $\alpha=1$ or in general one can choose $\alpha=\frac1{v(z)}$. Thus $H_2$ staes that $f_0,\dots, f_m$ are approximately pairwise relatively prime.

$H_4$ shows the connection between the lifts $f_i^*,s_i^*$ and $f_i,s_i$.

$H_5$ basically states that the lifts are unique in the sense that one can group some of the $f_i's$ to form $F_0,\dots, F_p$ and change $F_i$ to $\ovf_i$ with precision $\eps^*$ and still one will have the factorization of $f$ with precision $\eps^*$. $H_5$ is very important for the factorization algorithm in chapter 6. 

Now we will state the Hensel's Lemma and will later give the algorithm to obtain the lifts.
\section{Hensel's Lemma}
First we will prove a helping lemma which will be very much usefull in the proof of Hensel's Lemma  then we will state the actual theorem.
\thm[hlm]{}{
\begin{enumerate}[label=(\roman*)]
	\item Let $a,f,p,s\in R[x]$ such that $f$ is monic and $s=pf+a$ with $\deg a<\deg f$. Then we have $v(p)\leq v(s) \text{ and } v(a)\leq v(s)$
	\item Let $h_0,\dots, h_m\in R[x]$ and $h_o^*,\dots,h_m^*\in R[x]$ such that we have $v(h_i^*-h_i)\leq \eps$ for all $0\leq i\leq m$. Then we have $v\lt( \prod\limits_{i=0}^m h_i-\prod\limits_{i=0}^mh_i^*\rt)\leq \eps$
\end{enumerate}
}
\thm[hslem]{Hensel's Lemma}{
Assume that we have $f\in R[x]$, $\mcF=\{f_0,\dots,f_m\}\subseteq R[x]$, $\mcS=\{s_0,\dots, s_m\}\subseteq R[x]$, $z\in R$ and $\alpha,\delta,\eps\in \bbR$ which satisfy:\begin{enumerate}
	\item $H_1(m,f,\mcF,\mcS,\eps)$
	\item $H_2(m,f,\mcF,\mcS,z,\delta)$
	\item $H_3(m,f,\mathcal{F},S,z,\alpha,\delta,\epsilon)$
\end{enumerate}Then we can compute efficiently $$\mcF^*=\{f^*_i\colon 0\leq i\leq m\}\qquad \text{and}\qquad T=\{t_0,\dots, t_m\}$$ such that \begin{enumerate}[label=(\roman*)]
\item \textbf{Linear Case}: $\mcS^*=\mcS$ and $\delta^*=\gm$, $\eps^*=\alpha\gm\eps$. Then we have the following conditions hold:\begin{enumerate}[label=(\alph*)]
	\item $H_1(m,f,\mcF^*,\mcS^*,\eps^*)$
	\item $H_2(m,f,\mcF^*,\mcS^*,z,\delta^*)$
	\item $H_3(m,f,\mcF^*,\mcS^*,z,\alpha,\delta^*,\eps^*)$
	\item $H_4(m,f,\mcF,\mcF^*,\mcS,\mcS^*,\alpha,\delta,\eps)$
	\item $H_5(m,f,\mcF,\mcF^*,\mcS,\mcS^*,\alpha,\delta,\eps,\delta^*)$
\end{enumerate}
\item \textbf{Quadratic Case}: $\mcS^*=T$ and $\delta^*=\alpha\gm^2$, $\eps^*=\alpha\gm\eps$. Assume that $\deg s_i>\deg f_i$ for $0\leq i\leq m$. Then we have the following conditions hold:\begin{enumerate}[label=(\alph*)]
	\item $H_1(m,f,\mcF^*,\mcS^*,\eps^*)$
	\item $H_2(m,f,\mcF^*,\mcS^*,z,\delta^*)$
	\item $H_3(m,f,\mcF^*,\mcS^*,z,\alpha,\delta^*,\eps^*)$
	\item $H_4(m,f,\mcF,\mcF^*,\mcS,\mcS^*,\alpha,\delta,\eps)$
	\item $H_5(m,f,\mcF,\mcF^*,\mcS,\mcS^*,\alpha,\delta,\eps,\delta^*)$
\end{enumerate}
\end{enumerate}
}
\section{Hensel's Algorithm}
\section{Proof of Hensel's Lemma}