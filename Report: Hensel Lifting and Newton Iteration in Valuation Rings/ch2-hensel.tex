\chapter{Hensel Lifting}
The hensel method described here will lift an approximate factorization of a polynomial over a Hensel Ring $R$ with valuation $v$ where the factors are relatively prime. We will show a linear convergence and a quadratic convergence behavior for the liftings.
\section{Hensel Ring}
\dfn{Hensel Ring}{
A ring with valuation $v:R\to \bbR_{\geq 0}$ is called a Hensel Ring if:\begin{enumerate}[label=(\roman*)]
	\item $\forall\ a\in R$, $v(a)\leq 1$
	\item $\forall\ a,b\in R$, $\forall\ \eps>0$, $\exs\ c\in R$ such that $(v(a)\leq v(b)\implies v(a-bc)\leq \eps)$
\end{enumerate}
}
In other words $R$ is Hensel iff it is contained and dense in the valuation ring of its quotient field (with respect to the unique extension of $v$). We sometimes call such $v$ a Hensel Valuation. 

In condition (ii) we assume we can compute the $c$ efficiently.
\begin{Theorem}{}{}
	Condition (i) of Hensel Ring $\implies v$ is Non-Archimedean.
\end{Theorem}
\begin{proof}
	Let $a,b\in R$. Now \begin{align*}
		v(a+b)^k & = v((a+b)^k)\\
		& = v\lt(\sum_{i=0}^k{{k}\choose{i}}a^{n-i}b^i  \rt)\leq \sum_{i=0}^kv\lt(\binom{k}{i}\rt)v(a)^{n-i}v(b)^i\\
		&\leq \sum_{i=0}^kv(a)^{n-i}v(b)^i\leq \sum_{i=0}^kM^{n-i}M^i &[m=\max{v(a),v(b)}]\\
		&=\sum_{i=0}^kM^k=M^k(k+1)
	\end{align*}Hence $$\lt(\frac{v(a+b)}{M}\rt)^k\leq (k+1)\iff \frac{v(a+b)}{M}\leq (1+k)^{\frac1k}$$As $k\to \infty$ the RHS approaches 1 so $v(a+b)\leq M$. 
\end{proof}

\begin{example}{$p-$adic Valuations}{}
	\begin{itemize}
		\item $\bbZ$ with $p-$adic valution $v_p$ where $p\in\bbN$ is prime is a Hensel ring. Here $v_p(a)=p^{-n}$ where $n=\max\{k\geq 0\mid p^k\mid a\}$
		\item $\bbF[y]$ with $p-$adic valutaion $v_p$ where $p\in \bbF[y]$ is an irreducible polynomial is a Hensel Ring. Here $v_p(f)=2^{-n\deg p}$ where $n=\max\{k\geq 0\mid p^k\mid f\}$
	\end{itemize}
\end{example}

 \nt{From the valuation $v$ over $R$ we naturally get a valuation $v$ over the polynomial ring $R[x]$ by defining $$\forall\ f\in R[x],\ \text{let }f=\sum_{i=0}^nf_ix^i,\text{ then } v\lt(\sum_{i=0}^nf_ix^i\rt)=\max_{i}\{v(f_i)\}$$}
\section{Conditions related to Hensel's Lemma}
We will define 5 conditinos. First suppose we have:
\begin{flushleft}
	\begin{tabular}{rll}
		(1) & $f\in R[x]$ & \\
		(2) & $f_0,\dots, f_m\in R[x]$ & $\mcF=\{f_i\colon 0\leq i\leq m\}$\\
		(3) & $f^*_0,\dots, f^*_m\in R[x]$ & $\mcF^*=\{f^*_i\colon 0\leq i\leq m\}$\\
		(4) & $s_0,\dots, s_m\in R[x]$ & $\mcS=\{s_i\colon 0\leq i\leq m\}$\\
		(5) & $s^*_0,\dots, s^*_m\in R[x]$ & $\mcS^*=\{s^*_i\colon 0\leq i\leq m\}$\\
		(6) & $z\in R$ & \\
		(7) & $\alpha,\delta,\eps\in \bbR$ & \\
		(8) & $\delta ^*\in \bbR$ & \\
		(9) & $\gm=\max\{\delta,\alpha \eps\}$		
	\end{tabular}
\end{flushleft}
As you can see the set $\mcF^*$ basically represents the lift of $\mcF$ but here since we are saying the conditions in more generality we are not assuming any relations among them and we define some conditions involving them.
\begin{itemize}
	\item $\boldsymbol{H_1(m,f,\mcF,\mcS,\eps)}\coloneqq v\lt(f-\prod\limits_{i=0}^mf_i\rt)\leq\eps<1$ 
	\item $\boldsymbol{H_2(m,f,\mcF,\mcS,z,\delta)}\coloneqq v\lt(\sum\limits_{i=0}^m s_i\prod\limits_{j\neq i}f_i -z   \rt)\leq \delta<1$
	\item $\boldsymbol{H_3(m,f,\mathcal{F},\mcS,z,\alpha,\delta,\epsilon)}\coloneqq$ 
	$\begin{array}[t]{@{}rl@{}}
		(1) &f_1,\dots,f_m \text{ are monic}\\
		(2) &\deg\left(\prod\limits_{i=0}^mf_i\right)\leq \deg f\\
		(3) &\deg s_i\leq \deg f_i \ \forall\  i\in[m]\\
		(4) &\alpha\delta\leq 1, \alpha\epsilon^2\leq 1\\
		(5) &1\leq \alpha v(z)
	\end{array}$
	\item $\boldsymbol{H_4(m,\mcF,\mcF^*,\mcS,\mcS^*,\alpha,\delta,\eps)}\coloneqq$ $\begin{array}[t]{@{}rll@{}}
		(1) & v(f^*_i-f_i)\leq \alpha\eps\ &\forall\ 0\leq i\leq m\\
		(2) & v(s^*_i-s_i)\leq \alpha\eps\ &\forall\ 0\leq i\leq m\\
		(3) &\deg f_i^*= \deg f_i\ &\forall \ i\in[m]\\
		(4) &\deg s_i<\deg f_i\implies \deg s_i^*<\deg f_i^*\ &\forall\ i\in [m]
	\end{array}$
	\item $\boldsymbol{H_5(m,f,\mcF,\mcF^*,\mcS,\mcS^*, \alpha,\delta,\eps,\delta^*)}\coloneqq $ Let $p\in[m]$. Then suppose
	\begin{itemize}
		\item $\mcI_p=\{I_0,I_1,\dots,I_p\}$ be a partition of $\{0,\dots,m\}$ with $o\in I_0$. 
		\item $\overline{\mcF}_p^m=\{\ovf_i\colon i\in[p]\}\subseteq R[x]$ be a set of monic polynomials
	\end{itemize}
	Then define: $$F_i=\prod_{j\in I_i} f_j,\qquad F_i^*=\prod_{j\in I_i}f_j^*,\qquad \mfs_i^*=\sum_{j\in I_i} s_j\frac{F_i^*}{f_i^*}$$So now we denote:$$\sF=\{F_i\colon 0\leq i\leq p\},\qquad \sF^*=\{F_i\colon 0\leq^* i\leq p\}, \qquad \sS=\{\mfs_i^*\colon 0\leq i\leq p\} $$
	Assume:\begin{enumerate}
		\item $v(\ovf_i-F_i)\leq \alpha \eps $ $\forall \ i\in[p]$
		\item $\alpha v(s_i)\leq 1$ $\forall\ 0\leq i\leq m$
		\item $\alpha\delta <1$, $\alpha^2\delta \leq 1$
		\item $\alpha^2\eps<1$, $\alpha^3\eps\leq 1$
	\end{enumerate}Then the following are equivalent:
	\begin{enumerate}[label=(\roman*)]
		\item $\exs$ $\ovf_0,$ $\ovs_0,\dots,\ovs_p\in R[x]$ denote $$\overline{\mcF}=\{\ovf_i\colon 0\leq i\leq p\},\qquad \overline{\mcS}=\{\ovs_i\colon 0\leq i\leq p\}$$then the following conditions are true:\begin{enumerate}
			\item $H_1(p,f,\overline{\mcF},\overline{\mcS},\eps^*)$
			\item $H_2(p,f,\overline{\mcF},\overline{\mcS},z,\delta^*)$
			\item $H_3(p,f,\overline{\mcF},\overline{\mcS},z,\alpha^*, \delta^*,\eps^*)$
			\item $H_4(p,f,\sF,\overline{\mcF},\sS, \overline{\mcS},z,\alpha^*,\delta^*,\eps^*)$
		\end{enumerate}where $\alpha^*=\alpha$, $\eps^*=\alpha\eps\gm$
		\item $\exs$ $\ovf_0\in R[x]$ such that $H_1(p,f,\overline{\mcF},\overline{\mcS},\eps^*)$ is true
		\item $\forall\ i\in[p]$ we have $v(\ovf_i-F_i^*)\leq \eps^*$.
	\end{enumerate}
\end{itemize}
The first 3 conditions here togather imply that: From $H_1$ we get that $f_0\cdots f_m$ is a good approximation of factorization of $f$ with $\eps$-precision, $H_2\implies z$ plays a similar role to the gcd of $f_0,\dots,f_m$ and it shows the generalized bezout's identity for gcd  for multiple elements. In the usual treatment of Hensel's Lemma $f_0,\dots, f_m$ are relatively prime (more precisely their images in the residue class field or $R$ modulo the maximal ideal $\la a\in R\mid v(a)<1\ra$ satisfy the assumption then one can find $s_0,\dots,s_m,\delta$ satisfying $H_2$ with $z=1$. One can set $\alpha=1$ or in general one can choose $\alpha=\frac1{v(z)}$. Thus $H_2$ staes that $f_0,\dots, f_m$ are approximately pairwise relatively prime.

$H_4$ shows the connection between the lifts $f_i^*,s_i^*$ and $f_i,s_i$.

$H_5$ basically states that the lifts are unique in the sense that one can group some of the $f_i's$ to form $F_0,\dots, F_p$ and change $F_i$ to $\ovf_i$ with precision $\eps^*$ and still one will have the factorization of $f$ with precision $\eps^*$. $H_5$ is very important for the factorization algorithm in chapter 6. 

Now we will state the Hensel's Lemma and will later give the algorithm to obtain the lifts.
\section{Hensel's Lemma}
First we will prove a helping lemma which will be very much usefull in the proof of Hensel's Lemma  then we will state the actual theorem.
\thm[hlm]{}{
\begin{enumerate}[label=(\roman*)]
	\item Let $a,f,p,s\in R[x]$ such that $f$ is monic and $s=pf+a$ with $\deg a<\deg f$. Then we have $v(p)\leq v(s) \text{ and } v(a)\leq v(s)$
	\item Let $h_0,\dots, h_m\in R[x]$ and $h_o^*,\dots,h_m^*\in R[x]$ such that we have $v(h_i^*-h_i)\leq \eps$ for all $0\leq i\leq m$. Then we have $v\lt( \prod\limits_{i=0}^m h_i-\prod\limits_{i=0}^mh_i^*\rt)\leq \eps$
\end{enumerate}
}

\begin{proof}
	\begin{enumerate}[label=(\roman*)]
		\item \parinn Suppose $\deg s \geq\deg f \geq 0$ otherwise $\deg s< \deg f$ Then $s= 0\times f + a=a$ so we get $v(a)=v(s)$ and $v(p)=v(0)=0\leq v(s)$. Hence assume $l=\deg s-\deg f\geq 0$. Then $p=\sum\limits_{i=0}^l p_ix^i$ where $p_i\in R$ for all $0\leq i\leq l$.
		
		Now we will induct on $l-i$. Let $\deg f=n$ and $\deg s=m$. Hence assume $f=\sum\limits_{i=0}^n f_ix^i$ and $s=\sum\limits_{i=0}^m s_ix^i$ where for all $0\leq i\leq n$ and $0\leq j\leq m$, $f_i,s_j\in R$. Since $f$ is monic $f_n=1$
		
		\underline{Base Case ($i=l$)} : $$v(p_l)= v(p_l)v(f_n)= v(p_lf_n)=v(s_m)\leq v(s)$$ 
		
		\underline{Inductive Step} : $v(p_i)\leq v(s)$ for all $l-k\leq i\leq l$. Now coefficient of $x^{(l-k-1)+n}$ in $S$ is $$s_{l-k-1+n}=\sum\limits_{i=l-k-1}^{l}p_if_{l-k-1+n-i}$$Therefore $p_{l-k-1}f_n=s_{l-k-1+n}-\sum\limits_{i=l-k}^l p_if_{l-k-1+n-i}$. Therefore 
		\begin{align*}
			v(p_{l-k-1})=v(p_{l-k-1}f_n) & \leq \max\{  v(s_{l-k-1+n}), v(p_if_{l-k-1+n-i})\mid l-k\leq i\leq l  \} \\
			                             & \leq \max\{v(s_{l-k-1+n}), v(p_i)\mid l-k\leq i\leq l\}                  \\
			                             & \leq v(s)
		\end{align*}
		Therefore by induction we have $v(p_i)\leq v(s)$ for all $0\leq i\leq l$. Then $v(p)\leq v(s)$.
		
		Hence $$v(a)=v(s-pf)\leq \max\{v(s),v(pf)\}\leq \max\{v(s),v(p)\}\leq v(s)$$Therefore we have $v(a)\leq v(s)$.
		\item \parinn We will induct on $0\leq i\leq m$. 
		
		\underline{Base Case} : $v(h_0-g_0^*)\leq \eps$ Given
		
		\underline{Inductive Step} : Let this is true for $i=k$ i.e.$$v\lt(\prod_{j=0}^kh_j-\prod_{j=0}^k h_j^*  \rt)\leq \eps$$Now\begin{align*}
			\prod_{j=0}^kh_j(h_{k+1}-h_{k+1}^*)+\prod_{j=0}^kh_j^*(h_{k+1}-h_{k+1}^*) &=\lt[\prod_{j=0}^{k+1}h_j-\prod_{j=0}^{k+1}h_j^*\rt] -\lt[ h_{k+1}^*\prod_{j=0}^kh_j-h_{k+1}\prod_{j=0}^{k}h_j^*  \rt]
		\end{align*}Therefore we have $$\prod_{j=0}^{k+1}h_j-\prod_{j=0}^{k+1}h_j^*=	\prod_{j=0}^kh_j(h_{k+1}-h_{k+1}^*)+\prod_{j=0}^kh_j^*(h_{k+1}-h_{k+1}^*)+ \lt[ h_{k+1}^*\prod_{j=0}^kh_j-h_{k+1}\prod_{j=0}^{k}h_j^*  \rt]  $$
	\end{enumerate}Hence we have \begin{align*}
	v\lt(  \prod_{j=0}^{k+1}h_j-\prod_{j=0}^{k+1}h_j^* \rt)& \leq \max\lt\{ v\lt(\prod_{j=0}^kh_j\rt)v(h_{k+1}-h_{k+1}^*), v\lt(\prod_{j=0}^kh_j^*\rt)v(h_{k+1}-h_{k+1}^*),v\lt( h_{k+1}^*\prod_{j=0}^kh_j-h_{k+1}\prod_{j=0}^{k}h_j^*  \rt)  \rt\}\\
	& \leq \max \lt\{v(h_{k+1}-h_{k+1}^*), v\lt( h_{k+1}^*\prod_{j=0}^kh_j-h_{k+1}\prod_{j=0}^{k}h_j^*  \rt)  \rt\}\\
	& \leq \max\lt\{ \eps, v\lt( h_{k+1}^*\prod_{j=0}^kh_j-h_{k+1}\prod_{j=0}^{k}h_j^*  \rt)   \rt\}
	\end{align*}Now we need to show that $$ v\lt( h_{k+1}^*\prod\limits_{j=0}^kh_j-h_{k+1}\prod\limits_{j=0}^{k}h_j^*  \rt)  \leq \eps$$Now
	\begin{align*}
		 h_{k+1}^*\prod_{j=0}^kh_j-h_{k+1}\prod_{j=0}^{k}h_j^*   & = \lt( h_{k+1}^*\prod_{j=0}^k h_j-h_{k+1}^*h_k^*\prod_{j=0}^{k-1}h_j \rt)         +         \lt( \lt[\prod_{j=k}^{k+1}h_j^*\rt]\lt[\prod_{j=0}^{k-1}h_j  \rt]-\lt[\prod_{j=k-1}^{k+1}h_j^*\rt]\lt[\prod_{j=0}^{k-2}h_j  \rt]\rt)\\
		 & + \lt( \lt[\prod_{j=k-1}^{k+1}h_j^*\rt]\lt[\prod_{j=0}^{k-2}h_j  \rt]-\lt[\prod_{j=k-2}^{k+1}h_j^*\rt]\lt[\prod_{j=0}^{k-3}h_j  \rt]\rt)\\
		 & \qquad\qquad \qquad\vdots\qquad\qquad \qquad\qquad\qquad\vdots\\
		 &+ \lt( \lt[\prod_{j=t+1}^{k+1}h_j^*\rt]\lt[\prod_{j=0}^{t}h_j  \rt]-\lt[\prod_{j=t}^{k+1}h_j^*\rt]\lt[\prod_{j=0}^{t-1}h_j  \rt]\rt)\\
		 & \qquad\qquad \qquad\vdots\qquad\qquad \qquad\qquad\quad\vdots\\
		 & + \lt( \lt[\prod_{j=1}^{k+1}h_j^*\rt]h_0  -\prod_{j=0}^{k+1}h_j^*\rt)+\lt(\prod_{j=0}^{k+1}h_j^*  -\lt[\prod_{j=0}^{k}h_j^*\rt]h_{k+1}  \rt)\\
		 & = h_{k+1}^*\prod_{j=0}^{k-1}h_j^*(h_{k}-h_k^*)+\lt[\prod_{j=k}^{k+1}h_j^*\rt]\lt[\prod_{j=0}^{k-2}h_j  \rt](h_{k-1}-h_{k-1}^*) \\
		 & +\lt[\prod_{j=k-1}^{k+1}h_j^*\rt]\lt[\prod_{j=0}^{k-3}h_j  \rt](h_{k-2}-h_{k-2}^*)\\
		 & \qquad\qquad\qquad\vdots\\
		 & +\lt[\prod_{j=t+1}^{k+1}h_j^*\rt]\lt[\prod_{j=0}^{t-1}h_j  \rt](h_{t}-h_{t}^*)\\
		  & \qquad\qquad\qquad\vdots\\
		  & + \lt[\prod_{j=1}^{k+1}h_j^*\rt](h_0-h_0^*)+  \lt[\prod_{j=1}^{k}h_j^*\rt](h_{k+1}^*-h_{k+1})
	\end{align*}
	Now for each $0\leq t\leq k$ we have $$v\lt(  \lt[\prod_{j=t+1}^{k+1}h_j^*\rt]\lt[\prod_{j=0}^{t-1}h_j  \rt](h_{t}-h_{t}^*)\rt)\leq v(h_t-h_t^*)\leq\eps, \qquad v\lt(  \lt[\prod_{j=1}^{k}h_j^*\rt](h_{k+1}^*-h_{k+1})  \rt)\leq v(h_{k+1}^*-h_{k+1})\leq \eps$$Hence $$v\lt( h_{k+1}^*\prod\limits_{j=0}^kh_j-h_{k+1}\prod\limits_{j=0}^{k}h_j^*  \rt)\leq \max_{0\leq t\leq k}\lt\{  v\lt(  \lt[\prod_{j=t+1}^{k+1}h_j^*\rt]\lt[\prod_{j=0}^{t-1}h_j  \rt](h_{t}-h_{t}^*)\rt),  v\lt(  \lt[\prod_{j=1}^{k}h_j^*\rt](h_{k+1}^*-h_{k+1})  \rt)\rt\}\leq \eps$$Therefore we have  $$ v\lt( h_{k+1}^*\prod\limits_{j=0}^kh_j-h_{k+1}\prod\limits_{j=0}^{k}h_j^*  \rt)  \leq \eps\implies 	v\lt(  \prod_{j=0}^{k+1}h_j-\prod_{j=0}^{k+1}h_j^* \rt)\leq \eps$$Hence by induction we have $$	v\lt(  \prod_{j=0}^{m}h_j-\prod_{j=0}^{m}h_j^* \rt)\leq \eps$$
\end{proof}


\thm[hslem]{Hensel's Lemma}{
Assume that we have $f\in R[x]$, $\mcF=\{f_0,\dots,f_m\}\subseteq R[x]$, $\mcS=\{s_0,\dots, s_m\}\subseteq R[x]$, $z\in R$ and $\alpha,\delta,\eps\in \bbR$ which satisfy:\begin{enumerate}
	\item $H_1(m,f,\mcF,\mcS,\eps)$
	\item $H_2(m,f,\mcF,\mcS,z,\delta)$
	\item $H_3(m,f,\mathcal{F},S,z,\alpha,\delta,\epsilon)$
\end{enumerate}Then we can compute efficiently $$\mcF^*=\{f^*_i\colon 0\leq i\leq m\}\qquad \text{and}\qquad T=\{t_0,\dots, t_m\}$$ such that \begin{enumerate}[label=(\roman*)]
\item \textbf{Linear Case}: $\mcS^*=\mcS$ and $\delta^*=\gm$, $\eps^*=\alpha\gm\eps$. Then we have the following conditions hold:\begin{enumerate}[label=(\alph*)]
	\item $H_1(m,f,\mcF^*,\mcS^*,\eps^*)$
	\item $H_2(m,f,\mcF^*,\mcS^*,z,\delta^*)$
	\item $H_3(m,f,\mcF^*,\mcS^*,z,\alpha,\delta^*,\eps^*)$
	\item $H_4(m,f,\mcF,\mcF^*,\mcS,\mcS^*,\alpha,\delta,\eps)$
	\item $H_5(m,f,\mcF,\mcF^*,\mcS,\mcS^*,\alpha,\delta,\eps,\delta^*)$
\end{enumerate}
\item \textbf{Quadratic Case}: $\mcS^*=T$ and $\delta^*=\alpha\gm^2$, $\eps^*=\alpha\gm\eps$. Assume that $\deg s_i>\deg f_i$ for $0\leq i\leq m$. Then we have the following conditions hold:\begin{enumerate}[label=(\alph*)]
	\item $H_1(m,f,\mcF^*,\mcS^*,\eps^*)$
	\item $H_2(m,f,\mcF^*,\mcS^*,z,\delta^*)$
	\item $H_3(m,f,\mcF^*,\mcS^*,z,\alpha,\delta^*,\eps^*)$
	\item $H_4(m,f,\mcF,\mcF^*,\mcS,\mcS^*,\alpha,\delta,\eps)$
	\item $H_5(m,f,\mcF,\mcF^*,\mcS,\mcS^*,\alpha,\delta,\eps,\delta^*)$
\end{enumerate}
\end{enumerate}
}
\pagebreak 
\section{Hensel's Computation}
\begin{algorithm}
	\KwIn{\begin{enumerate}
			\item $f\in R[x]$, $\mcF=\{f_0,\dots, f_m\}\subseteq R[x]$, $\mcS=\{s_0,\dots, s_m\}\subseteq R[x]$
			\item  $z\in R$ 
			\item  $\alpha,\delta,\eps\in \bbR$\end{enumerate}
	}
	\KwOut{$\mcF^*=\{f_0^*,\dots, f_m^*\}$, $T=\{t_0,\dots, t_m\}$}
	\DontPrintSemicolon
	\Begin{
		Set $\gm=\max\{\delta,\alpha\eps\}$, $\alpha^*=\alpha$, $\eps^*=\alpha\gm\eps$ and $e=f-\prod\limits_{i=0}^mf_i$\;
		\For{$1\leq i\leq m$}{
		Compute $a_i,b_i,p_i\in R[x]$ such that $$s_ie=p_if_i+a_i, \qquad v(zb_i-a_i)\leq \eps\gm,\qquad \deg b_i\leq \deg a_i< \deg f_i$$
		}
		Compute $a_0,b_0\in R[x]$ such that $$a_0=s_0e+f_0\sum\limits_{i=1}^m p_i,\qquad v(zb_0-a_0)\leq \eps\gm,\qquad  \deg b_0\leq \deg f-\def \prod\limits_{i=1}^m f_i$$\;
		\For{$0\leq i\leq m$}{
		$f_i^*=f_i+b_i$
		}
		\For{$1\leq i\leq m$}{
		Compute $c_i,d_i,g_i^*q_i\in R[x]$ such that $$g_i^*=\prod_{j\neq i}f_j^*,\qquad s_i(s_ig_i^*-z)=q_if_i^*+c_i,\qquad v(zd_i-c_i)\leq \gm^2,\qquad \deg d_i\leq \deg c_i<\deg f_i^*$$}
		Compute $g_0^*=\prod\limits_{i=1}^mf_i^*$ and $c_0,d_0\in R[x]$ such that $$c_0=s_o\lt(\sum_{i=0}^m s_ig_i^*-z  \rt)+f_0^*\sum_{i=1}^m\lt[ q_i+s_i\lt(\sum_{j\neq i} s_j\frac{g_j^*}{f_j^*} \rt)\rt],\quad v(zd_0-c_0)\leq \gm^2,\quad \deg d_0\leq \deg f-\deg g_0$$\;
		\For{$0\leq i\leq m$}{$t_i=s_i-d_i$}
		\Return{$\mcF^*=\{f_i^*\colon 0\leq i\leq m\}$, $T=\{t_i\colon 0\leq i\leq m\}$}
		
	}
	\caption{Hensel's Computation}
\end{algorithm}
\section{Proof of Hensel's Lemma}