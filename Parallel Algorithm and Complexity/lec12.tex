\section{Counting Perfect Matchings in Planar Graph}

\begin{definition}[Tutte Matrix]
	The Tutte Matrix of a digraph $G=V(E)$ is the matrix $T_G[X]$ where $$T_G[X]=\begin{cases}
		x_{ij}&\text{when $(i,j)\in E$}\\
		-x_{ij}&\text{when $(i,j)\in E$}\\
		0 & \text{when $i=j$}\\
		0&\text{otherwise}
	\end{cases}$$
\end{definition}
Hence we can see that $T_G[X]$ is an skew symmetric matrix.


\begin{definition}[Nice Cycle]
	The cycle $C$ which is of even length and $G-V(C)$ has a perfect matching is called a nice cycle.
\end{definition}
\begin{definition}[Oddly Oriented]
	An even cycle $C$ is called oddly oriented if for  either choice of direction of traversal around $C$ the number of edges of $C$ directed in the direction of the traversal is odd.
	
	In other words the number of edges of $C$ directed in the opposite direction of the traversal is odd.
\end{definition}
\begin{definition}[Pfaffian Orientation]
	Pfaffian Orientation of a graph $G$ is an orientation where every nice cycle of $G$ is oddly oriented.
\end{definition}
\begin{definition}[Pfaffian Graph]
	A graph $G$ is Pfaffian if it has a Pfaffian orientation.
\end{definition}
\begin{lemma}\label{lmoddcyclecancel}
	Let $\sigma\in S_n$ where $\sigma=(C_1)(C_2)\cdots (C_k)$. If $\exs\ i\in [k]$ such that $C_i$ is an odd cycle then the monomial $\prod\limits_{i=1}^n T_{i,\sigma(i)}$ gets canceled out in $\det(T_G[X])$
\end{lemma}
\begin{proof}
	$\sg=(C_1)(C_2)\cdots (C_k)$. Therefore $\sg$ represents a cycle cover $\mcC$ of $G$. Given that $\mcC$ has an odd cycle. Now for any cycle $C$ define the head of the cycle to be the vertex with the smallest index. . Let the odd cycle in $\mcC$ with smallest head is $C_j$. Now $C)j$ corresponds to a permutation $\tau\in S_n$. Therefore $C_j$ reversed i.e $C_j^R$ corresponds to $\tau^{-1}$. Then take the cycle cover $\mcC'=\{C_1,\dots,C_{j-1},C_j^R,C_{j+1},\dots,C_k\}$. $\mcC'$ corresponds to the permutation $\sg'=(C_1)(C_2)\cdots(C_{j-1})(C_j^R)(C_{j+1})\cdots (C_k)$. 
	
	After reversing $C_{j}$ for any edge $u\to v\in C_j$ the edge $v\to u\in C_j^R$. In $T_G[X]$ we have $T_{v,u}=-T_{u,v}$. Hence for all edge in $C_j$ the sign is changed after reversing. Since the length of $C_j$ is odd we have $sgn(C_j)=-sgn(C_j^R)$. Since all the other cycles in $\mcC'$ and $\mcC$ are same there sign is not changed. Hence $sgn(\sg')=-sgn(\sg)$. Since the monomial generated by $\sg$ and $\sg'$ are same only their sign is reversed they will cancel each other in the determinant. Hence $\prod\limits_{i=1}^n T_{i,\sigma(i)}$ gets canceled out in $\det(T_G[C])$.	 
\end{proof}
\begin{theorem}[Tutte]
	$\det(T_G[X])\neq 0\iff G$ has a perfect matching
\end{theorem}
\begin{proof}
	$(\Rightarrow):$ Suppose $\det(T_G[X])\neq 0$. Then there exists a $\sg\in S_n$ such that the monomial $\prod\limits_{i=1}^nT_{i,\sg(i)}$ in not canceled out. Now by \lmref{lmoddcyclecancel} the cycle cover $\mcC$ corresponding to $\sg$ has only even cycles. So for each cycle $C\in \mcC$ we get a matching for the vertices in $C$. [Just take the odd (even) positioned edges.] Since the cycles in $\mcC$ are edge disjoint union of all these edges gives a perfect matching in $G$. Hence $G$ has a perfect matching.
	
	$(\Leftarrow):$ We will prove the contrapositive. Let $G$
	 has no perfect matching. Then suppose $\det(T_G[X])\neq 0$. Then by previous arguments there exists a $\sg\in S_n$ such that the monomial $\prod\limits_{i=1}^nT_{i,\sg(i)}$ in not canceled out. Now again by \lmref{lmoddcyclecancel} the cycle cover $\mcC$ corresponding to $\sg$ has only even cycles. Each of these cycles will give out a matching for the vertices in that cycle. Hence we obtain a perfect matching for $G$. Contradiction \ctr Hence we have $\det(T_G[X])=0$. 
\end{proof}

If $G$ has a perfect matching then you can take the cycle cover of all the transpositions for each edge in the perfect matching. Then the monomial for this permutation will not get canceled in the determinant. 

\begin{remark}
	From now on by matching we mean to say perfect matching. Also we define the set $$S=\{\sg\in S_n\mid \text{every cycle in the cycle cover corresponding to  $\sg$ is even cycle}\}$$Also we define the set $$\mcM_G=\{\text{set of all perfect matching of $G$}\}$$We will use heavily in this section. 
\end{remark}
\subsection{Pfaffian, Matchings, Determinant}
\begin{lemma}\label{lmstomatchpair}
	There is a bijection $\vph$ between $\vph: S\to \mcM_G\times \mcM_G$
\end{lemma}
\begin{proof}
	content...
\end{proof}
Let $\pi$ be any partition of the vertices of $G$. Let $$\pi=\{\{i_1,j_1\},\{i_2,j_2\},\dots,\{i_k,j_k\}\}$$Now we will define some quantities:\begin{itemize}
	\item By $wt(\pi)$ we define the quantity $$ wt(\pi)=sgn\underbrace{\matp{1&2&3&4&\cdots & 2k-1&2k\\ i_1& j_1 & i_2 & j_2 & \cdots & i_k & j_k  }}_{\coloneqq \sg_{\pi}} \prod_{l=1}^k T_{i_l,j_l}$$
	\item $Pf(T_G)\coloneqq \sum\limits_{pi}wt(\pi)=\sum\limits_{M\in \mcM_G}wt(M)$
	\item Let $D$ be any orientation of $G$. Then we define $$sgn_D(M)=sgn\underbrace{\matp{1&2&3&4&\cdots & 2k-1&2k\\ i_1& j_1 & i_2 & j_2 & \cdots & i_k & j_k  }}_{\coloneqq \sg_{M}}$$ where the edge direction of for $\{i_l,j_l\}$ is $i_l\to j_l$ $\forall\ l\in [k]$ and we start from the smallest index $i_1$.
\end{itemize}
Clearly $\pi$  is a matching of $G$. So we can write this as $wt(M)$ where $M\in \mcM_G$. Since $T_G$ is a skew symmetric matrix $wt(\pi)$ or $wt(M)$ does not depend on the order of the blocks in $\pi$ (Because if any 2 blocks' positions get swapped then in the permutation we have to multiply 2 transpositions to $\sg_{\pi}$ or $\sg_M$ which doesn't change the sign) or the order of the elements in a block in $\pi$ (Because if their order is changed then we have to multiply a single transposition to $\sg_{\pi} $ or $\sg_M$ multiplies the sign with $(-1)$ and from the variables let $m$ is the index of the block where the positions of $i_m$ and $j_m$ is changed then $T_{j_m,i_m}=-T_{i_m,j_m}$ hence we get another $(-1)$ which is multiplied hence the sign remains same.)

\begin{lemma}\label{lmpfmatchingsign}
	$D$ is Pfaffian orientation $\iff\ \forall \ M,_1M_2\in\mcM_G$, we have  $sgn_D(M_1)=sgn_D(M_2)$
\end{lemma}
\begin{proof}
	$(\Rightarrow):$ Let $M_1,M_2\in \mcM_G$. By \lmref{lmstomatchpair}  $\exs\ \sg\in S$ such that $\vph(\tau)=(M_1,M_2)$. Let $\tau=(C_1)(C_2)\cdots (C_k)$. Now for each $C_i$ is even cycle. And by the bijection $\vph$ $G-V(C_i)$ has a perfect matching. Therefore each $C_i$ is nice cycle. We denote $\sg_{M_i}=\sg_i$ for $i=1,2$ for simplicity. Since $D$ is Pfaffian orientation for each $i\in [k]$ there are odd number of edges in $C_i$ which are oddly oriented. Now because of orientation we can say that if $u\to v$ is an edge of $M_1$ then any one  of $\tau(u)\to \tau(v)$ and $\tau(v)\to \tau(u)$ is an edge of $M_2$. Or we can say if for any $l\in [\frac{n}2]$ $\sg_1(2l-1)\to sg_1(2l)$ is an edge of $M_1$ then any one of $\tau\circ \sg_1(2l-1)\to \tau\circ\sg(2l)$ and $\sg_1(2l-1)\to sg_1(2l)$ is an edge of $M_2$. Now from $\tau\circ\sg_1$ in order to obtain $\sg_2$ we need to multiply some transpositions since some of the edges are in opposite direction. So 3 cases arise:
	
	\begin{enumerate}[label=\bfseries Case \arabic*:,itemindent=2cm,leftmargin=0cm]
		\item If an opposite direction edge is in $M_1$ and the next edge is in correct direction which is in $M_2$. Let the opposite direction edge is $\sg_1(2l)\to sg_1(2l-1)$ then the next edge which is in $M_2$ is $\tau\circ\sg_1(2l-1)\to\tau\circ \sg_1(2l)$. Hence we have to multiply the transposition $\matp{2l-1& 2l}$ to $\tau\circ\sg_1$. Hence a negative sign gets multiplied.
		\item If a edge of $M_1$ is in correct direction but the next edge is in opposite direction which is in $M_2$. Let the  edge is $\sg_1(2l-1)\to sg_1(2l)$ then the next edge which is in $M_2$ is $\tau\circ \sg_1(2l)\to \tau\circ\sg(2l-1)$. Hence we have to multiply the transposition $\matp{2l-1& 2l}$ to $\tau\circ\sg_1$. Hence a negative sign gets multiplied.
		\item 2 consecutive same direction edges either in opposite to the cycle direction or in following the cycle direction the first one is in $M_1$ and the next one is in $M_2$. Let the edge in $M_1$ $\sg_1(2l-1)\to sg_1(2l)$ then the next edge which is in $M_2$ is $\tau\circ \sg_1(2l-1)\to \tau\circ\sg(2l)$. Hence no need to multiply transposition. So no need to change anything.
	\end{enumerate}
Now since there are odd number of edges which are in opposite direction for Case 3 the opposite edges appear in pairs apart from Case 3 there are still odd number of edges which are in opposite direction. Hence with Case 1 and 2 after multiplying the transpositions the multiply odd number of transpositions which changes the sign. Hence over all a $(-1)$ gets multiplied. Since we do this for all cycles $C_i$ we multiply $(-1)$ for each cycle. Therefore we can write $sgn(\sg_2)=(-1)^ksgn(\tau\circ \sg_1)$. Now since $\tau$ has $k$ many even cycles we have $sgn(\tau)=(-1)^k$. Therefore we have $$sgn(\sg_2)=(-1)^ksgn(\tau\circ \sg_1)=(-1)^ksgn(\tau)sgn(\sg_1)=(-1)^k\times (-1)^k sgn(\sg_1)=sgn(\sg_1)$$Therefore for all $M_1,M_2\in\mcM_G$ we have $sgn_D(M_1)=sgn_D(M_2)$

$(\Leftarrow):$ We have $\forall \ M_1,M_2$ $sgn_D(M_1)=sgn_D(M_2)$. Now take a nice cycle $C$. Then $G-V(C)$ has a matching, $M'$. Now since $C$ is even we get two matching $M_1$ and $M_2$ by taking the odd edges in $M_{C_1}$ and even edges in $M_{C_2}$ of $V(C)$ from $C$. Now we create two matching of $G$, $M_1=M'\cup M_{C_1}$ and $M_2=M'\cup M_{C_2}$. So by \lmref{lmstomatchpair} $\exs\ \tau\in S$ such that $\vph(\tau)=(M_1,M_2)$. So in the cycle cover $\mcC$ obtained from $\tau$ for all the vertex pairs not in $C$ forms a 2-cycle but because of orientation we have both the edges in same direction so one in the direction of the 2-cycle and the other is in the opposite direction of the direction of the 2-cycle. 

Like in the forward direction proof we have 3 cases of occurrences of opposite direction edges. In order to obtain $\sg_2$ form $\tau\circ\sg_1$ we have a transposition multiplied to $\tau\circ\sg_1$ for each 2-cycle. Let there are $m$ many 2-cycles. So $m$ many transpositions are multiplied to $\tau\circ \sg_1$ and hence $(-1)^m$ is multiplied to $sgn(\tau\circ \sg_1)$ because of the transpositions due to the 2-cycles. Let $(-1)^{m'}$ is multiplied because of the transpositions of $C$. Now since there are $m+1$ many cycles in $\tau$ $sgn(\tau)=(-1)^{m+1}$. So therefore from $sgn(\sg_1)=sgn(\sg_2)$ we have $(-1)^m\times (-1)^{m'}=(-1)^{m+1}\implies (-1)^{m'}=(-1)$ Hence there are odd number of transpositions multiplied to $\tau\circ \sg_1$ because of $C$. Hence there are odd number of edges in $C$ which are in opposite directions following the 3 cases described in the forward direction proof. Hence for every nice cycle there are odd number of edges which are oddly oriented. Hence $D$ is a Pfaffian orientation.
\end{proof}
\begin{theorem}[\cite{Cayley+1849+93+96}]\label{thdetpfsq}
	$\det(T_G[X])=Pf(T_G)^2$
\end{theorem}
\begin{proof}
	We have $$\det(T_G)=\sum_{\sg\in S_n}sgn(\sg)\prod_{i=1}^nT_{i,\sg(i)}$$By \lmref{lmoddcyclecancel} and since the diagonal entries are 0 the cycle cover corresponding to $\sg\in S_n$ containing at least one odd cycle cancels out and the permutations having $\sg(i)=i$ for any $i\in[n]$ the monomial becomes zero. So we have $$\det(T_G)=\sum_{\sg\in S}sgn(\sg)\prod_{i=1}^nT_{i,\sg(i)}$$ We have $$Pf(T_G)^2=\sum_{M_1,M_2\in \mcM_G\times \mcM_G}wt(M_1)wt(M_2)$$By \lmref{lmstomatchpair} there exists $\sg\in S$ such that $\vph(\sg)=(M_1,M_2)$. Let $\sg=(C_1)(C_2)\cdots (C_m)$. Now $$wt(M_1)=sgn_D(M_1)\prod_{l=1}^kT_{\sg_1({2l-1}),\sg_1({2l})}\qquad wt(M_2)=sgn_D(M_2)\prod_{l=1}^kT_{\sg_2({2l-1}),\sg_2({2l})}$$Where $D$ is some orientation on $G$.
	
	Now to obtain $\sg_2$ from $\tau\circ\sg_1$ we have to multiply some transposition in case of any opposite direction edges in the cycles. Let the product of all these transpositions are denoted by $\psi$. Then we have $$\sg_2=\psi\circ\tau\circ \sg_1\implies sgn(\sg_1)sgn(\sg_2)=sgn(\psi)sgn(\tau)$$Now for each transposition $\psi'=\matp{i,& \tau(i)}$  in $\psi$ for some $i$ it means the edge $\tau(i)\to i$ present in a cycle of the cycle cover of $\tau$ which  is opposite  to the direction of cycle following the orientation $D$. Since we have $T_{\tau(i),i}=-T_{i,\tau(i)}$ we can replace the monomial $T_{\sg(i), i}$ in $wt(M_1)wt(M_2)$ by $-T_{i,\sg(i)}$. So we do this for all the transpositions in $\psi$. Hence after replacing all these variables we can say $$\lt[ \prod_{l=1}^kT_{\sg_1({2l-1}),\sg_1({2l})}\rt] \lt[  \prod_{l=1}^kT_{\sg_2({2l-1}),\sg_2({2l})}\rt] =sgn(\psi)\prod_{i=1}^nT_{i,\tau(i)}$$So we get \begin{align*}
		wt(M_1)wt(M_2) & =sgn(\sg_1)sgn(\sg_2)\lt[ \prod_{l=1}^kT_{\sg_1({2l-1}),\sg_1({2l})}\rt] \lt[  \prod_{l=1}^kT_{\sg_2({2l-1}),\sg_2({2l})}\rt]\\
		& = sgn(\sg_1)sgn(\sg_2)\times \lt[sgn(\psi)\prod_{i=1}^nT_{i,\tau(i)}\rt]\\
		& = sgn(\psi)sgn(\tau)\times \lt[sgn(\psi)\prod_{i=1}^nT_{i,\tau(i)}\rt]\\
		& =sgn(\tau)\prod_{i=1}^nT_{i,\tau(i)}
	\end{align*}
Therefore we get $Pf(T_G)^2=\det(T_G)$
\end{proof}

\subsection{Pfaffian Orientation of Planar Graph}
\begin{lemma}\label{lmfaceoddpf}
	$G$ is a planar graph. Suppose $G$ admits an orientation $D$ such that every internal face $F$ is oddly oriented in clockwise traversal. Then $D$ is Pfaffian
\end{lemma}
\begin{proof}
	We will prove a stronger claim here.
	\begin{claim}
		For any simple cycle $C$ addition of the number $f$ of edges in forward direction and the number $k$ of vertices strictly inside $C$ is odd.
	\end{claim}
\begin{proof-of-claim}{1}
	We will induct on $k$. Let $k=0$. Then basically the cycle is the boundary cycle of an internal face. Then this is true by the hypothesis of the lemma. 
	
	Now suppose $C$ encloses more than one face. We can view $C$ is the symmetric difference of two smaller simple cycles $C_1$ and $C_2$.  Let $f_1,k_1$ and $f_2,k_2$ be	the number of forward direction edges and internal vertices for $C_1$ and $C_2$ respectively. So $f_1+k_1\equiv 1\pmod{2}$ and $f_2+k_2\equiv 1\pmod2$. Now let the path $P$ shared by $C_1$ and $C_2$ contains $m$ vertices. So the number of vertices of $C$ is $k=k_1+k_2+b$. Since all the edges of $P$ is either in forward direction with respect to $C_1$ or $C_2$ if we take $f_1+f_2$ then this number has all the edges of $P$. So the number of edges in forward direction of $C$ is $f=f_1+f_2-(b+1)$. Hence $$k+f=(k_1+k_2+b)+(f_1+f_2-(b+1))=(k_1+f_1)+(k_2+f_2)-1\equiv 1\pmod2$$So by induction this is true for any simple cycle.
\end{proof-of-claim}

Now if we take any nice cycle $C$ in $G$. Then $G-V(C)$ has a matching. Since $C$ is even cycle the vertices in $C$ also has a matching. So we obtain a matching of $G$. Hence if there are odd number of vertices strictly inside $C$ there exists at least on edge of in the matching which connects a vertex inside $C$ and a vertex outside $C$. But this edge would be cutting the cycle $C$ at some point which is not possible since $G$ is planar graph. Therefore the number of vertices strictly inside $C$ is even. By the claim we can say that the number of edges in forward direction in $C$ are odd. Since this is true for any nice cycle in $G$. $D$ is Pfaffian.
\end{proof}
\begin{theorem}[\cite{planegraphpfaffian}]
	Every Planar graph is Pfaffian 
\end{theorem}
\begin{proof}
	content...
\end{proof}
\begin{theorem}
	Let $D$ be the pfaffian orientation of $G$. Then let $B$ is the adjacency matrix of $G$ following the orientation $D$. Then we have $$\det(B)=|\mcM_G|^2$$
\end{theorem}
\begin{proof}
	Since $G$ is planar such a pfaffian orientation $D$ by \lmref{lmpfmatchingsign} we have for all $M_1,M_2\in \mcM_G$ $sgn_D(M_1)=sgn_D(M_2)$. Therefore $sgn(wt(M_1)wt(M_2))=1$ and each $T_{\sg_j(2l-1),\sg_j(2l)}=1$ for all $l\in [k]$ in \thmref{thdetpfsq}. Hence for all $M_1,M_2\in \mcM_G$ we have the value of $wt(M_1)wt(M_2)=1$. So we have by  $\det(B)=|\mcM_G|^2$
\end{proof}


\section{Perfect Matching in Bipartite Planar Graph}

