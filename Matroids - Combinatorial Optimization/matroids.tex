\documentclass{article}
\usepackage{url}

\usepackage[nottoc,numbib]{tocbibind}
\input{../preamble-article}
\input{../letterfonts}
\input{../macros}

\definecolor{mycolor1}{HTML}{11998E}
\definecolor{mycolor2}{HTML}{38EF7D}
%\usepackage{anysize}
%\marginsize{1.5in}{1in}{0.5in}{0.5in}

\newcommand*{\titre}[2]{\begingroup
	\newlength{\drop}
	\setlength{\drop}{0.1\textheight}
	\centering
	\settowidth{\unitlength}{\Huge\scshape #1 \hspace{10pt}}
	\vspace*{\baselineskip}
	\rule{\unitlength}{1.6pt}\vspace*{-\baselineskip}\vspace*{2pt}
	\rule{\unitlength}{0.4pt}\\[\baselineskip]
	{\Huge\scshape\color{white} #1}\\[\baselineskip]
	{\large\itshape Instructor: #2}\\[0.2\baselineskip]
	\rule{\unitlength}{0.4pt}\vspace*{-\baselineskip}\vspace{3.2pt}
	\rule{\unitlength}{1.6pt}\\[4\baselineskip]
	{\Large\scshape Scribe: Soham Chatterjee\\[10mm] sohamchatterjee999@gmail.com \\[2mm] Website: \href{https://sohamch08.github.io/}{\textcolor{black}{sohamch08.github.io}} }\par
	\vfill
	{\large\scshape 2024}\par
	\endgroup}
	

%\title{\huge Algorithmic Coding Theory - Amit Kumar Sinhababu}
%\author{ \vspace*{5mm} \LARGE Scribed: Soham Chatterjee\\\large sohamchatterjee999@gmail.com\\ \large Website: \href{https://sohamch08.github.io/}{sohamch08.github.io}}
%\date{\LARGE 2023}

%\input{../titlepageC}
%\renewcommand{\thefootnote}{\fnsymbol{footnote}}
%\addbibresource{refs.bib}
\begin{document}
%\begin{titlingpage}
%	\maketitle
%\end{titlingpage}
	
%----------------------------------------------------------------------------------------
%	TITLE PAGE
%----------------------------------------------------------------------------------------
\thispagestyle{empty}
%\titleBC{Algorithmic Coding Theory: 2023}

%{Soham Chatterjee}{\textbf{sohamchatterjee999@gmail.com \\ Website: \href{https://sohamch08.github.io/}{sohamch08.github.io}}
\begin{titlepage}
	
	\begin{tikzpicture}[remember picture,overlay]
		\node [xshift=\paperwidth/2,yshift=\paperheight/2] at (current page.south west)[minimum width=\paperwidth,minimum height=\paperheight,top color=mycolor1,bottom color=mycolor2]{};
	\end{tikzpicture}\\[3\baselineskip]
	
	\titre{Matroids: Combinatorial Optimization}{Prajakta Nimbhorkar}
\end{titlepage}
	
\pagebreak
\tableofcontents 
\pagebreak

%sfasdfffffffffffffffffffffffffff\footnotemark[2]{}
%fffffffffffff
%
%fffffffffffffffffff\footnotemark[2]{}
%
%\footnotetext[2]{A footnote with two references.}
\section{Introduction}
\dfn{Matroid}{A matroid $M=(E,I)$ has a ground set $E$ and a collection $I$ of subsets of $E$ called the \textit{Independent Sets} st\begin{enumerate}
		\item Downward Closure: If $Y\in I$ then $\forall \ X\subseteq Y$, $X\in I$.
		\item Extension Property: If $x,Y\in I$, $|X|<|Y|$ then $\exs\ e\in Y-X$ such that $X\cup \{e\}$ also written as $X+e\in I$
\end{enumerate}}
\begin{observation*}
	A maximal independent set in a matroid is also a maximum independent set. All maximal independent sets have the same size.
\end{observation*}
\parinf

\textbf{Base:} Maximal Independent sets are called bases.

\textbf{Rank of $\boldsymbol{S\in I}$:} $\max\{|X|\colon X\subseteq S, X\in I\}$

\textbf{Rank of a Matroid:} Size of the base.

\textbf{Span of $\boldsymbol{S\in I}$:} $\{e\in E\colon rank(S)=rank(S+e)\}$

\section{Types of Matroids}
	\subsection{{Uniform Matroid:} }It is {denoted as $U_{k,n}$ where $E=[n]$ and $I=\{X\subseteq E\mid |X|\leq k\}$.}
	
	
	\textbf{Free Matroid: }When $k=n$ we take all possible subsets of $E$ into $I$. This matroid is called {Free Matroid} i.e. $U_{n,n}$\parinn
	
	\subsection{Partition Matroid:} Given $E=E_1\sqcup E_2\sqcup \cdots \sqcup E_l$ where $\{E_1,\dots, E_l\}$ is a partition of $E$ and $k_1,\dots,k_l\in \bbN\cup \{0\}$ $$I=\{X\subseteq E\colon |X\cap E_i|\leq k_i\ \forall \ i\in[l]\}$$then $M=(E,I)$ is a partition matroid.
	\begin{note}
		If the $E_i$'s are not a partition then suppose $E_1$, $E_2$ has nonempty partition then we will not have a matroid. 
		
		For example: $E_1=\{1,2\}$, $E_2=\{2,3\}$ and $k_1=k_2=1$ then $X=\{1,3\}$ is independent but $Y=\{2\}\subsetneq X$ is not a matroid. 
	\end{note}
	
	\subsection{Linear Matroid:} Given a $m\times n$ matrix denote its columns as $A_1,\dots, A_n$. Then $$I=\{ X\subseteq [n]\colon \text{Columns corresponding to $X$ are linearly independent}  \}$$\parinn
	
	Here if the underlying field is $\bbF_2$ then it is called \textit{Binary Matroid} and for $\bbF_3$ it is called \textit{Ternary Matroid}.
	
	\subsection{Representable Matroid:} A matroid with which we can associate a linear matroid is called a representable matroid.
	
	Eg: $U_{2,3}$. It can be represented by the matrix $A=\mat{1&1&0\\ 1& 0 & 1\\ 0& 1& 1}$, over $\bbF_2$. Over $\bbF_3$ it is same as $U_{3,3}$. 
	\nt{There are matroids which are not representable as linear matroids in some field. There are matroids which are not representable on any field as well.}
	\lem{}{$U_{2,4}$ is not representable over $\bbF_2$ but representable over $\bbF_3$}
	
	\subsection{Regular Matroid:} There are the matroids which are representable over all fields.
	\lem{}{Regular Matroids are precisely those which can be represented over $\bbR$ by a Totally Uni-modular matrix}
	
	\subsection{Graphic Matroid / Cyclic Matroid:} For a graph $G=(V,E)$ the graphic matroid $M_G=(E,I)$ where $$I=\{F\subseteq E\colon \text{$F$ is acyclic}\}$$
	
	Hence $I$ is the collection of forests of $G$. It follows the downward closure trivially. For extension property let $k=|F_1|<|F_2|=l$ and then there are $n-k$ and $n-l$ components. SO $n-k>n-l$. So $\exs$ an edge in $F_2$ which joins 2 components in $F_1$.
	\lem{}{A subset of columns is linearly independent iff the corresponding edges don't contain a cycle in the incidence matrix}
	\lem{}{Graphic Matroids are Regular Matroids}
	\begin{proof-idea}
		Use Incidence Matrix.
	\end{proof-idea}
	\subsection{Matching Matroids}
	We can try to define it like this but it will not work:
\pr{}{Is the following a matroid:
$E=$ Edges of a graph and $I=\{F\subseteq E\colon \text{$F$ is a matching}\}$
}\parinf

\solve{It is not a matroid since maximal matchings can not be extended to a maximum matching.}
Correct way will be: For a graph $G=(V,E)$ the ground set $=V$ and $$I=\{S\subseteq V\colon \exs \text{a matching that matches all vertices in $S$}\}$$\parinn

The downward closure property trivially holds. For extension property is $S|<|S'|$ then there exists another vertex in $S'$ which is not matched with $S$, so we can add that vertex to $S$. 

\section{Circuits}
Assume we have a matroid $M=(E,I)$. 
\dfn{Circuit}{A minimal dependent set $C$ such that $\forall\ e\in C$, $C-e$ is an independent set.}
\thm{}{Let $S\in I$. $S+e\notin I$. Then $\exs!$ $C\subseteq S+e$.}


\section{Finding Max Weight Base}
\subsection{Algorithm}
\subsection{Correctness Analysis}
\section{Some Matroid Properties}
\subsection{Strong Base Exchange Property}
\subsection{Exchange Graph of a Matroid wrt $\boldsymbol{S\in I}$}

\section{Using Matroid Intersection to Solve Problems}
\subsection{Bipartite Matching}
\subsection{Colorful Spanning Tree}
\subsection{Min-Max Relation for Colorful Spanning Tree}
\subsection{Arborescence}
\section{Solving Matroid Intersection Problem}





\pagebreak
%\printbibliography[heading=none]
%\addcontentsline{toc}{chapter}{Bibliography}
%\bibliographystyle{alpha}
%\bibliography{refs}
\end{document}
