\chapter{Lecture 3: Introduction}

\section{Recap/Questions}
In the last lecture, we covered the linearity tests using the following two equivalent definitions:
\begin{itemize}
    \item $\exists \alpha \in \F_n^2$ such that $\Pr(f(x) = \inn{\alpha}{x}) \geq 1 - \epsilon$
    \item $\Pr(f(x)+f(y) = f(x+y) \geq 1 - \epsilon$
\end{itemize}

Recall that the characters $\chi_\alpha$ are the online linear maps in $\sF$, with $\chi_{\alpha+\beta} = \chi_\alpha \cdot \chi_\beta$. Note that $f(x) = \chi_\alpha(x) \cdot -1$ are affine maps, not linear. Recall that BLR test only fails when the map is not linear, i.e. has one sided error. Hence, it is amenable to boosting.

Recall the problem discussed in the last class:

\textbf{Problem 1: } Suppose $f \in \sF$ is close to some unknown parity function $\chi_\alpha$. Can we find out $\chi_\alpha$ where its close in term of $l_\infty$ norm?

\textbf{Solution:} Report $\widehat{\chi_\alpha}(x) = \E_{y \in \F_2^n}[f(y+x)f(y)]$. 

Observe that by the definition of $\epsilon$-close, we have:
\begin{align}
% \Pr_{x,y \sim \F_2^n}[f(x)\cdot f(y)  = f(x+y)] \geq 1 - \epsilon
\Pr_{y \in \F_2^n}[f(y) \neq \chi_\alpha(y)] \leq \epsilon\\
\Pr_{y \in \F_2^n}[f(x+y) \neq \chi_\alpha(x+y)] \leq \epsilon
\end{align}

Combining through union bound, we get:

\[
\Pr{y \in \F_2^n}[f(y)f(y+x) = \chi_\alpha(x)] \geq 1 - \epsilon
\]

\textbf{Problem 2: } Can we estimate the distance of a function $f$ from the nearest linear function? 

\textbf{Idea: } If $dist(f,l_\alpha) = \epsilon$, then BLR 'fails' with probability $\geq \epsilon$. Can we do a coin-toss like argument on the BLR test to obtain an estimate of $\epsilon$?

\begin{theorem}
    Let $P_i$ be the failure probability of BLR test where $\epsilon = dist(f,l_\alpha)$. Then, $P_1 \in [\epsilon,5\epsilon]$.

    \begin{proof}
        
    \end{proof}
\end{theorem}

\begin{idea}[Unique Decoding Theorem]
    To be added.
\end{idea}

\textbf{Problem 3:} Let $\func{f}{\F_2^n}{\rbool}$. Show that the number of characters $\chi_\alpha$ such that $\norm{\fhat(\chi_\alpha)} > \frac{1}{2}$ is at most one.


\section{Affine functions and Beyond}

Up till now, we have concerned ourselves with checking and approximating linear functions. Can we extend these results to the case of affine functions?

\begin{definition}[Affine Functions]
There can be multiple alternative definitions for affine functions:
\begin{itemize}
    \item A function $f \in \sF$ is said to be an affine function if it is of the form $f(x) = a + \inn{\alpha}{x}$.
    \item A function $f \in \sF$ is said to be an affine function if $\forall x,y,z \in \F_2^n$, we have $f(x+y+z) = f(x)+f(y)+f(z)$
\end{itemize}
\end{definition}

To argue something along the lines of the BLR test, we will require some robust definitions for affine functions as well.

\begin{definition}[Robust($\epsilon$-far) affine functions]
    Again, we introduce multiple robust definitions for affine functions:
\end{definition}

\begin{algorithm}
    \caption{Atomic Tester for Affine|Modified BLR Test}\label{Alg: Atomic Affine Test}
    \begin{algorithmic}
        \Require Black-box access to $\func{f}{\F_2^n}{\rbool}$
        \State Pick $x,y,z \in \F_2^n$ uniformly and independently at random
        \State If $f(x+y+z) = f(x)f(y)f(z)$, return $1$; otherwise return $0$
    \end{algorithmic}
\end{algorithm}

\begin{theorem}
    The modified BLR test (Algorithm \ref{Alg: Atomic Affine Test}) returns 0 with probability $q_\epsilon \in [\epsilon,6\epsilon]$ if $dist(f,l_\alpha) = \epsilon$

    \begin{proof}[Proof Sketch]
        Let us again consider the functions to be $\func{f}{\F_2^n}{\rbool}$. Let us consider the event $E = \set{f(x)f(y)f(z) = f(x+y+z)}$. We also have:
        \[\indicator_E(x,y,z) = \frac{1}{2} + \frac{1}{2} f(x)f(y)f(z)f(x+y+z) \]
        Then, we have:
        \begin{align*}
            \Pr[E] &= \E[\indicator_E]\\
                    &=\E[\frac{1}{2} + \frac{1}{2} f(x)f(y)f(z)f(x+y+z)]\\
                    &= \frac{1}{2} + \frac{1}{2} \E[f(x)f(y)f(z)f(x+y+z)]\\
                    &= ...\\
                    &= \frac{1}{2} +\frac{1}{2}\sum_{\alpha\in\F_2^n} |\fhat^4(\chi_\alpha)|
        \end{align*}    
        \textbf{Observation: } The only possible affine functions are $\chi_\alpha$ and $-1 \cdot \chi_\alpha$.
    \end{proof}
\end{theorem}

\begin{idea}[Roth's Theorem]
    To be added. Has similar analysis. Doesn't work for four term-arithmetic progressions, for similar reasons why we can't go to qudratic.
\end{idea}

However, can we go one step ahead and go to higher degrees, starting with quadratic functions? Turns out, no. Let us consider a particular function $\func{g}{\F_n^2}{\F_2}$ defined as:
\[g(x_1,...,x_n) = x_1x_2 + x_3x_4+...+x_{n-1}x_n\]
And correspondingly the function $\func{G}{\F_2^n}{\rbool}$ defined as:
\[G(x) = (-1)^g(x)\]
We can show that $\forall \alpha\in\F_2^n$,
\[|\hat{G}(\chi_\alpha) = 2^{-\frac{n}{2}}\]

\begin{idea}[Higher Order Fourier]
    For the cases of higher degree, similarly for four term extension to the Roth's Theorem, the framework of Higher Order Fourier expansion was introduced by Timothy Gowers.
\end{idea}

\section{Convolutions}

\begin{definition}
    Let us consider two functions $f,g \in \sF$ with range $\rbool$, we define their convolution to be:
    \[f*g(x) = \E_{y \in \F_2^n} f(y)g(x-y)\]
\end{definition}

Let us list down some important properties of the convolution operation:

\begin{itemize}
    \item $f * g = g * f$
    \item $f * g = \inn{f}{g}$
    \item $\widehat{f*g}(\chi_\alpha) = \fhat(\chi_\alpha)\cdot\hat{g}(\chi_\alpha)$
\end{itemize}

\begin{definition}[Probability Density Function]
    $\func{\phi}{\F_2^n}{\R_{\geq 0}}$ is a probability density function if $\E_{x \sim \F_2^n} [\phi(x)] = 1$. 

    Probability mass at $x$ w.r.t. $\phi$ is $\frac{\phi}{2^n}$.
\end{definition}

\begin{itemize}
    \item If $x,y$ are random vectors in $\F_2^n$ drawn according to  probability density functions $\phi,\psi$, then $X+Y \sim \phi*\psi$.
\end{itemize}






