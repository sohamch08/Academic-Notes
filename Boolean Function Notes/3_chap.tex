\chapter{Lecture 4}

Recall the two formulations $\func{f}{\rbool^n}{\rbool}$ and $\func{f}{\fbool^n}{\rbool}$. 

We focus on understanding the functions that captures different voting rules. Let us start with a few examples:

\begin{itemize}
    \item $\func{Maj_n}{\rbool^n}{\rbool}$, defined as $Maj_n(x_1,x_2,...,x_n) = sign(\sum_{i \in [n]} x_i$
    \item $\func{AND}{\rbool^n}{\rbool}$, defined as:
    \[AND(x_1,x_2,...,x_n) = \begin{dcases}-1 & if x = (-1,...,-1)\\ 1 & otherwise\end{dcases}\]
    \item $\func{OR}{\rbool^n}{\rbool}$, defined as:
    \[OR(x_1,x_2,...,x_n) = \begin{dcases}+1 & if x = (+1,...,+1)\\ -1 & otherwise\end{dcases}\]
    \item $\func{Dic_i}{\rbool^n}{\rbool}$, defined as $Dic_i(x) = x_i$. These are essentially the characteristic functions of singleton sets. Henceforward referred to as $\chi_i$
    \item $\func{k-junta}{\rbool^n}{\rbool}$, is a k-junta function if $\exists S \subseteq [n]$ with $|S| = k$, and a corresponding $\func{g}{\rbool^k}{\rbool}$ such that $\forall x \in \rbool^n$, we have $f(x) = g(x_S)$.
    \item $\func{Threshold_a}{\rbool^n}{\rbool}$, given an $a \in \R^n$, is defined as $Threshold_a(x) = sign(\inn{a}{x})$.

    Pictorially, we can think about this function as a hyperplane separating the boolean hypercube into two sets. Any such labelling that divides the hypercube into two disjoint compact convex sets can be represented through this function, using \emph{Hyperplane Separation Theorem}.

    \textbf{Note: } Sometimes called LTF(Linear Threshold Function), with general threshold functions defined w.r.t. function based threshold.

    \item $\func{Tribes}{\rbool^sw}{\rbool}$ is defined to be a $(s,w)$-tribe function such that  it can be represented as the $OR$ of $AND$ of $s$ groups of size $w$. The function outputs true when any of the group has all $1$s. 

    Observe that the probability of success, i.e.  getting $-1$ to be equal to getting a probability of failure, i.e. getting $1$, we must have $w = \log{s}$.
\end{itemize}

\begin{note}[Testability]
    Let us recall the property testing where given a property $\sP$ and query access to function $f$, we check whether $f \in \sP$ or $dist(f,\sP) > \epsilon$. A property is said to be testable if this can be done with query complexity independent of the size of the truth table, i.e. n.
\end{note}

\begin{idea}[Conjecture: Characterization of Testability of Boolean Function]
    Recall that testability in the case of Graph Property Testing is completely characterized by Regularity Lemma. Two separate groups [Eric Blaise and Sourav-Fischer] conjectured that the testability of properties of Boolean Functions can be completely characterized by partial symmetry.
\end{idea}

\begin{definition}[Symmetric Function]
    A function $\func{f}{\rbool^n}{\rbool}$ is said to be a symmetric function if $f(x) = f(\Pi(x))$, for all permutations $\Pi$ and $x \in \rbool^n$.
\end{definition}

\begin{note}[Symmetric Function is Testable]
    Symmetric function is testable. The idea is to choose a random string and check whether the string and a random permutation of itself gives the same output.
    
    Observe that a randomly chosen boolean string of length $n$ has level $n/2$ on average and in $[n/2-\sqrt{n},n/2+\sqrt{n}]$, with high probability. Also note that any permutation functions $\Pi$ preserves the level of a string.

    \textbf{Exercise: } Formalize the algorithm and the proof.
\end{note}

Let us define the obvious partial order on $\rbool^n$. For any $x,y \in \rbool$, $x \preceq y$ if $\forall i$, $x_i \leq y_i$

\begin{definition}[Monotone Function]
    We define a function $\func{f}{\rbool^n}{\rbool}$ to be monotone if $\forall x \preceq y$, $f(x) \leq f(y)$.
\end{definition}

\begin{definition}[Transitive Symmetric Function]
    Given a function $\func{f}{\rbool^n}{\rbool}$, define $Aut(f) = \{\Pi \in S_n| \forall x, f(x) = f(\Pi(x))\}$. We then define $f$ to be a transitive symmetric function if $\forall i \neq j$, $\exists \Pi \in Aut(f)$ such that $\Pi(i) = j$.
\end{definition}

\begin{definition}[Unanimous function]
    A function $\func{f}{\R^n}{\R}$ is said to be unanimous if $\forall r \in \R$, we have $f(r,r,...,r) = r$.
\end{definition}

\begin{definition}[Good Voting Rule]
    A voting rule is said to be \emph{"good"} must necessarily satisfy:
    \begin{itemize}
        \item The function should be \emph{monotone}. If more votes go towards one value, the vote should not conclude in the opposite direction.
        \item The function should be \emph{symmetric}; alternatively a weaker condition can be for the function to be \emph{Transitive symmetric}.
        \item The function should be \emph{Odd}.
        \item The function should be \emph{unanimous}.
    \end{itemize}
\end{definition}

\textbf{Exercise: Check whether functions defined at the beginning of this lecture are \textit{good voting rules}.}

