\chapter{Pushdown Automata and Context-Free Languages}
\begin{enumerate}
	\addtocounter{enumi}{68}
	\item In the grammar a '$b'$ is generated with the non-terminal $B$ and $B$ is always replaced by $bA$. Hence for every $b$ in any word of the grammar there will be a $a$ immediately after $b$. Hence. 
	Now $a$ is generated by the non-terminal $A$. And the production rule $A\to aA\mid a$ ensures that  there can be any number of $a$'s consecutively.\begin{enumerate}
		\item Hence $aabaab$ is not in $L(G)$ because there is no $a$ after the last $b$
		\item $aaaaba$ is in $L(G)$ and $$S\to AB\to aAB\to aaAB\to aaaAB\to aaaaB\to aaaabA\to aaaaba$$
		\item $aabbaa$ is not in $L(G)$ because there is no $a$ just after the first $b$.
		\item $abaaba$ is in $L(G)$ and $$S\to ABS\to aBS\to abAS\to abaS\to abaAB\to abaaB\to abaabA\to abaaba$$ 
	\end{enumerate}
\item We can change the grammar by removing $\eps$-productions
\begin{align*}
	S & \to   aAB \mid aBA \mid bAA \\
	A & \to aS \mid bAAA\mid a\\
	B & \to aABB \mid  aBAB \mid aBBA \mid bS \mid b
	\end{align*}
Now at any stage if previously $\# a+\# A=2(\# b+\# B)$ then is we use any production rule replacing $S$ the number of $\# a+\#A$ is increased by 2 and number of $\#b+\#B$ is increased by 1 so still the relation $\# a+\# A=2(\# b+\# B)$ is maintained. If any production rule replacing $A$ is used then either no $a,A,b$ or $B$ is added or number of $\# a+\#A$ is increased by 2 and number of $\#b+\#B$ is increased by 1. Hence the relation $\# a+\# A=2(\# b+\# B)$ is maintained. And if $B$ is replaced then either number of $\# a+\#A$ is increased by 2 and number of $\#b+\#B$ is increased by 1 or no $a,A,b$ or $B$ is added. Hence the relation $\# a+\# A=2(\# b+\# B)$ is maintained. is satisfied in every level of the parse tree. Therefore $L(G)$ contains the set $L$.
\parinn

Now consider the function $f(w)=\#_a(w)-2\#_b(w)$. Now we know for all word $w\in L$, $f(w)=0$. Now for any word if we plot the graph of $f$ as it gradually reads the whole word we may consider upward diagonal movement by $\frac12$ unit if it reads $a$ and downward diagonal movement by one unit if it reads $b$. WLOG suppose the first letter is $a$. Then if the last letter is $a$ then the function $f$ must have reached the $x$-axis at some point after the first letter and the before the last letter. Hence $w=w_1w_2$ where both $w_1$, $w_2$ has $\#a=2\# b$ . By induction $w\in L(G)$.

  If the last letter is $b$. Then if the second letter is $b$ then we have touched the $x$-axis. So $w=ab$
  
  
  \item \begin{align*}
  	S &\to a S b b \mid T \mid  abb \\
  	T &\to b T a a\mid S\mid  baa
  \end{align*}
\end{enumerate}