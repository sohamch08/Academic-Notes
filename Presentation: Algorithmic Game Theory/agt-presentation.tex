% !TEX program = xelatex
\documentclass[aspectratio=1610]{beamer}
\usepackage[T1]{fontenc}
\usetheme{wildcat}
\usepackage{xcolor, mathtools, optidef}
\usepackage{tikz}
\usetikzlibrary{decorations.pathreplacing, arrows.meta, shapes, calc, positioning}
\DeclareMathOperator{\poa}{\mathsf{PoA}}
%---------------------------------------
% BlackBoard Math Fonts :-
%---------------------------------------

%Captital Letters
\newcommand{\bbA}{\mathbb{A}}	\newcommand{\bbB}{\mathbb{B}}
\newcommand{\bbC}{\mathbb{C}}	\newcommand{\bbD}{\mathbb{D}}
\newcommand{\bbE}{\mathbb{E}}	\newcommand{\bbF}{\mathbb{F}}
\newcommand{\bbG}{\mathbb{G}}	\newcommand{\bbH}{\mathbb{H}}
\newcommand{\bbI}{\mathbb{I}}	\newcommand{\bbJ}{\mathbb{J}}
\newcommand{\bbK}{\mathbb{K}}	\newcommand{\bbL}{\mathbb{L}}
\newcommand{\bbM}{\mathbb{M}}	\newcommand{\bbN}{\mathbb{N}}
\newcommand{\bbO}{\mathbb{O}}	\newcommand{\bbP}{\mathbb{P}}
\newcommand{\bbQ}{\mathbb{Q}}	\newcommand{\bbR}{\mathbb{R}}
\newcommand{\bbS}{\mathbb{S}}	\newcommand{\bbT}{\mathbb{T}}
\newcommand{\bbU}{\mathbb{U}}	\newcommand{\bbV}{\mathbb{V}}
\newcommand{\bbW}{\mathbb{W}}	\newcommand{\bbX}{\mathbb{X}}
\newcommand{\bbY}{\mathbb{Y}}	\newcommand{\bbZ}{\mathbb{Z}}

%---------------------------------------
% MathCal Fonts :-
%---------------------------------------

%Captital Letters
\newcommand{\mcA}{\mathcal{A}}	\newcommand{\mcB}{\mathcal{B}}
\newcommand{\mcC}{\mathcal{C}}	\newcommand{\mcD}{\mathcal{D}}
\newcommand{\mcE}{\mathcal{E}}	\newcommand{\mcF}{\mathcal{F}}
\newcommand{\mcG}{\mathcal{G}}	\newcommand{\mcH}{\mathcal{H}}
\newcommand{\mcI}{\mathcal{I}}	\newcommand{\mcJ}{\mathcal{J}}
\newcommand{\mcK}{\mathcal{K}}	\newcommand{\mcL}{\mathcal{L}}
\newcommand{\mcM}{\mathcal{M}}	\newcommand{\mcN}{\mathcal{N}}
\newcommand{\mcO}{\mathcal{O}}	\newcommand{\mcP}{\mathcal{P}}
\newcommand{\mcQ}{\mathcal{Q}}	\newcommand{\mcR}{\mathcal{R}}
\newcommand{\mcS}{\mathcal{S}}	\newcommand{\mcT}{\mathcal{T}}
\newcommand{\mcU}{\mathcal{U}}	\newcommand{\mcV}{\mathcal{V}}
\newcommand{\mcW}{\mathcal{W}}	\newcommand{\mcX}{\mathcal{X}}
\newcommand{\mcY}{\mathcal{Y}}	\newcommand{\mcZ}{\mathcal{Z}}



%---------------------------------------
% Bold Math Fonts :-
%---------------------------------------

%Captital Letters
\newcommand{\bmA}{\boldsymbol{A}}	\newcommand{\bmB}{\boldsymbol{B}}
\newcommand{\bmC}{\boldsymbol{C}}	\newcommand{\bmD}{\boldsymbol{D}}
\newcommand{\bmE}{\boldsymbol{E}}	\newcommand{\bmF}{\boldsymbol{F}}
\newcommand{\bmG}{\boldsymbol{G}}	\newcommand{\bmH}{\boldsymbol{H}}
\newcommand{\bmI}{\boldsymbol{I}}	\newcommand{\bmJ}{\boldsymbol{J}}
\newcommand{\bmK}{\boldsymbol{K}}	\newcommand{\bmL}{\boldsymbol{L}}
\newcommand{\bmM}{\boldsymbol{M}}	\newcommand{\bmN}{\boldsymbol{N}}
\newcommand{\bmO}{\boldsymbol{O}}	\newcommand{\bmP}{\boldsymbol{P}}
\newcommand{\bmQ}{\boldsymbol{Q}}	\newcommand{\bmR}{\boldsymbol{R}}
\newcommand{\bmS}{\boldsymbol{S}}	\newcommand{\bmT}{\boldsymbol{T}}
\newcommand{\bmU}{\boldsymbol{U}}	\newcommand{\bmV}{\boldsymbol{V}}
\newcommand{\bmW}{\boldsymbol{W}}	\newcommand{\bmX}{\boldsymbol{X}}
\newcommand{\bmY}{\boldsymbol{Y}}	\newcommand{\bmZ}{\boldsymbol{Z}}
%Small Letters
\newcommand{\bma}{\boldsymbol{a}}	\newcommand{\bmb}{\boldsymbol{b}}
\newcommand{\bmc}{\boldsymbol{c}}	\newcommand{\bmd}{\boldsymbol{d}}
\newcommand{\bme}{\boldsymbol{e}}	\newcommand{\bmf}{\boldsymbol{f}}
\newcommand{\bmg}{\boldsymbol{g}}	\newcommand{\bmh}{\boldsymbol{h}}
\newcommand{\bmi}{\boldsymbol{i}}	\newcommand{\bmj}{\boldsymbol{j}}
\newcommand{\bmk}{\boldsymbol{k}}	\newcommand{\bml}{\boldsymbol{l}}
\newcommand{\bmm}{\boldsymbol{m}}	\newcommand{\bmn}{\boldsymbol{n}}
\newcommand{\bmo}{\boldsymbol{o}}	\newcommand{\bmp}{\boldsymbol{p}}
\newcommand{\bmq}{\boldsymbol{q}}	\newcommand{\bmr}{\boldsymbol{r}}
\newcommand{\bms}{\boldsymbol{s}}	\newcommand{\bmt}{\boldsymbol{t}}
\newcommand{\bmu}{\boldsymbol{u}}	\newcommand{\bmv}{\boldsymbol{v}}
\newcommand{\bmw}{\boldsymbol{w}}	\newcommand{\bmx}{\boldsymbol{x}}
\newcommand{\bmy}{\boldsymbol{y}}	\newcommand{\bmz}{\boldsymbol{z}}


%---------------------------------------
% Scr Math Fonts :-
%---------------------------------------

\newcommand{\sA}{{\mathscr{A}}}   \newcommand{\sB}{{\mathscr{B}}}
\newcommand{\sC}{{\mathscr{C}}}   \newcommand{\sD}{{\mathscr{D}}}
\newcommand{\sE}{{\mathscr{E}}}   \newcommand{\sF}{{\mathscr{F}}}
\newcommand{\sG}{{\mathscr{G}}}   \newcommand{\sH}{{\mathscr{H}}}
\newcommand{\sI}{{\mathscr{I}}}   \newcommand{\sJ}{{\mathscr{J}}}
\newcommand{\sK}{{\mathscr{K}}}   \newcommand{\sL}{{\mathscr{L}}}
\newcommand{\sM}{{\mathscr{M}}}   \newcommand{\sN}{{\mathscr{N}}}
\newcommand{\sO}{{\mathscr{O}}}   \newcommand{\sP}{{\mathscr{P}}}
\newcommand{\sQ}{{\mathscr{Q}}}   \newcommand{\sR}{{\mathscr{R}}}
\newcommand{\sS}{{\mathscr{S}}}   \newcommand{\sT}{{\mathscr{T}}}
\newcommand{\sU}{{\mathscr{U}}}   \newcommand{\sV}{{\mathscr{V}}}
\newcommand{\sW}{{\mathscr{W}}}   \newcommand{\sX}{{\mathscr{X}}}
\newcommand{\sY}{{\mathscr{Y}}}   \newcommand{\sZ}{{\mathscr{Z}}}


%---------------------------------------
% Math Fraktur Font
%---------------------------------------

%Captital Letters
\newcommand{\mfA}{\mathfrak{A}}	\newcommand{\mfB}{\mathfrak{B}}
\newcommand{\mfC}{\mathfrak{C}}	\newcommand{\mfD}{\mathfrak{D}}
\newcommand{\mfE}{\mathfrak{E}}	\newcommand{\mfF}{\mathfrak{F}}
\newcommand{\mfG}{\mathfrak{G}}	\newcommand{\mfH}{\mathfrak{H}}
\newcommand{\mfI}{\mathfrak{I}}	\newcommand{\mfJ}{\mathfrak{J}}
\newcommand{\mfK}{\mathfrak{K}}	\newcommand{\mfL}{\mathfrak{L}}
\newcommand{\mfM}{\mathfrak{M}}	\newcommand{\mfN}{\mathfrak{N}}
\newcommand{\mfO}{\mathfrak{O}}	\newcommand{\mfP}{\mathfrak{P}}
\newcommand{\mfQ}{\mathfrak{Q}}	\newcommand{\mfR}{\mathfrak{R}}
\newcommand{\mfS}{\mathfrak{S}}	\newcommand{\mfT}{\mathfrak{T}}
\newcommand{\mfU}{\mathfrak{U}}	\newcommand{\mfV}{\mathfrak{V}}
\newcommand{\mfW}{\mathfrak{W}}	\newcommand{\mfX}{\mathfrak{X}}
\newcommand{\mfY}{\mathfrak{Y}}	\newcommand{\mfZ}{\mathfrak{Z}}
%Small Letters
\newcommand{\mfa}{\mathfrak{a}}	\newcommand{\mfb}{\mathfrak{b}}
\newcommand{\mfc}{\mathfrak{c}}	\newcommand{\mfd}{\mathfrak{d}}
\newcommand{\mfe}{\mathfrak{e}}	\newcommand{\mff}{\mathfrak{f}}
\newcommand{\mfg}{\mathfrak{g}}	\newcommand{\mfh}{\mathfrak{h}}
\newcommand{\mfi}{\mathfrak{i}}	\newcommand{\mfj}{\mathfrak{j}}
\newcommand{\mfk}{\mathfrak{k}}	\newcommand{\mfl}{\mathfrak{l}}
\newcommand{\mfm}{\mathfrak{m}}	\newcommand{\mfn}{\mathfrak{n}}
\newcommand{\mfo}{\mathfrak{o}}	\newcommand{\mfp}{\mathfrak{p}}
\newcommand{\mfq}{\mathfrak{q}}	\newcommand{\mfr}{\mathfrak{r}}
\newcommand{\mfs}{\mathfrak{s}}	\newcommand{\mft}{\mathfrak{t}}
\newcommand{\mfu}{\mathfrak{u}}	\newcommand{\mfv}{\mathfrak{v}}
\newcommand{\mfw}{\mathfrak{w}}	\newcommand{\mfx}{\mathfrak{x}}
\newcommand{\mfy}{\mathfrak{y}}	\newcommand{\mfz}{\mathfrak{z}}

%---------------------------------------
% Bar
%---------------------------------------

%Captital Letters
\newcommand{\ovA}{\overline{A}}	\newcommand{\ovB}{\overline{B}}
\newcommand{\ovC}{\overline{C}}	\newcommand{\ovD}{\overline{D}}
\newcommand{\ovE}{\overline{E}}	\newcommand{\ovF}{\overline{F}}
\newcommand{\ovG}{\overline{G}}	\newcommand{\ovH}{\overline{H}}
\newcommand{\ovI}{\overline{I}}	\newcommand{\ovJ}{\overline{J}}
\newcommand{\ovK}{\overline{K}}	\newcommand{\ovL}{\overline{L}}
\newcommand{\ovM}{\overline{M}}	\newcommand{\ovN}{\overline{N}}
\newcommand{\ovO}{\overline{O}}	\newcommand{\ovP}{\overline{P}}
\newcommand{\ovQ}{\overline{Q}}	\newcommand{\ovR}{\overline{R}}
\newcommand{\ovS}{\overline{S}}	\newcommand{\ovT}{\overline{T}}
\newcommand{\ovU}{\overline{U}}	\newcommand{\ovV}{\overline{V}}
\newcommand{\ovW}{\overline{W}}	\newcommand{\ovX}{\overline{X}}
\newcommand{\ovY}{\overline{Y}}	\newcommand{\ovZ}{\overline{Z}}
%Small Letters
\newcommand{\ova}{\overline{a}}	\newcommand{\ovb}{\overline{b}}
\newcommand{\ovc}{\overline{c}}	\newcommand{\ovd}{\overline{d}}
\newcommand{\ove}{\overline{e}}	\newcommand{\ovf}{\overline{f}}
\newcommand{\ovg}{\overline{g}}	\newcommand{\ovh}{\overline{h}}
\newcommand{\ovi}{\overline{i}}	\newcommand{\ovj}{\overline{j}}
\newcommand{\ovk}{\overline{k}}	\newcommand{\ovl}{\overline{l}}
\newcommand{\ovm}{\overline{m}}	\newcommand{\ovn}{\overline{n}}
\newcommand{\ovo}{\overline{o}}	\newcommand{\ovp}{\overline{p}}
\newcommand{\ovq}{\overline{q}}	\newcommand{\ovr}{\overline{r}}
\newcommand{\ovs}{\overline{s}}	\newcommand{\ovt}{\overline{t}}
\newcommand{\ovu}{\overline{u}}	\newcommand{\ovv}{\overline{v}}
\newcommand{\ovw}{\overline{w}}	\newcommand{\ovx}{\overline{x}}
\newcommand{\ovy}{\overline{y}}	\newcommand{\ovz}{\overline{z}}

%---------------------------------------
% Tilde
%---------------------------------------

%Captital Letters
\newcommand{\tdA}{\tilde{A}}	\newcommand{\tdB}{\tilde{B}}
\newcommand{\tdC}{\tilde{C}}	\newcommand{\tdD}{\tilde{D}}
\newcommand{\tdE}{\tilde{E}}	\newcommand{\tdF}{\tilde{F}}
\newcommand{\tdG}{\tilde{G}}	\newcommand{\tdH}{\tilde{H}}
\newcommand{\tdI}{\tilde{I}}	\newcommand{\tdJ}{\tilde{J}}
\newcommand{\tdK}{\tilde{K}}	\newcommand{\tdL}{\tilde{L}}
\newcommand{\tdM}{\tilde{M}}	\newcommand{\tdN}{\tilde{N}}
\newcommand{\tdO}{\tilde{O}}	\newcommand{\tdP}{\tilde{P}}
\newcommand{\tdQ}{\tilde{Q}}	\newcommand{\tdR}{\tilde{R}}
\newcommand{\tdS}{\tilde{S}}	\newcommand{\tdT}{\tilde{T}}
\newcommand{\tdU}{\tilde{U}}	\newcommand{\tdV}{\tilde{V}}
\newcommand{\tdW}{\tilde{W}}	\newcommand{\tdX}{\tilde{X}}
\newcommand{\tdY}{\tilde{Y}}	\newcommand{\tdZ}{\tilde{Z}}
%Small Letters
\newcommand{\tda}{\tilde{a}}	\newcommand{\tdb}{\tilde{b}}
\newcommand{\tdc}{\tilde{c}}	\newcommand{\tdd}{\tilde{d}}
\newcommand{\tde}{\tilde{e}}	\newcommand{\tdf}{\tilde{f}}
\newcommand{\tdg}{\tilde{g}}	\newcommand{\tdh}{\tilde{h}}
\newcommand{\tdi}{\tilde{i}}	\newcommand{\tdj}{\tilde{j}}
\newcommand{\tdk}{\tilde{k}}	\newcommand{\tdl}{\tilde{l}}
\newcommand{\tdm}{\tilde{m}}	\newcommand{\tdn}{\tilde{n}}
\newcommand{\tdo}{\tilde{o}}	\newcommand{\tdp}{\tilde{p}}
\newcommand{\tdq}{\tilde{q}}	\newcommand{\tdr}{\tilde{r}}
\newcommand{\tds}{\tilde{s}}	\newcommand{\tdt}{\tilde{t}}
\newcommand{\tdu}{\tilde{u}}	\newcommand{\tdv}{\tilde{v}}
\newcommand{\tdw}{\tilde{w}}	\newcommand{\tdx}{\tilde{x}}
\newcommand{\tdy}{\tilde{y}}	\newcommand{\tdz}{\tilde{z}}

%---------------------------------------
% Vec
%---------------------------------------

%Captital Letters
\newcommand{\vcA}{\vec{A}}	\newcommand{\vcB}{\vec{B}}
\newcommand{\vcC}{\vec{C}}	\newcommand{\vcD}{\vec{D}}
\newcommand{\vcE}{\vec{E}}	\newcommand{\vcF}{\vec{F}}
\newcommand{\vcG}{\vec{G}}	\newcommand{\vcH}{\vec{H}}
\newcommand{\vcI}{\vec{I}}	\newcommand{\vcJ}{\vec{J}}
\newcommand{\vcK}{\vec{K}}	\newcommand{\vcL}{\vec{L}}
\newcommand{\vcM}{\vec{M}}	\newcommand{\vcN}{\vec{N}}
\newcommand{\vcO}{\vec{O}}	\newcommand{\vcP}{\vec{P}}
\newcommand{\vcQ}{\vec{Q}}	\newcommand{\vcR}{\vec{R}}
\newcommand{\vcS}{\vec{S}}	\newcommand{\vcT}{\vec{T}}
\newcommand{\vcU}{\vec{U}}	\newcommand{\vcV}{\vec{V}}
\newcommand{\vcW}{\vec{W}}	\newcommand{\vcX}{\vec{X}}
\newcommand{\vcY}{\vec{Y}}	\newcommand{\vcZ}{\vec{Z}}
%Small Letters
\newcommand{\vca}{\vec{a}}	\newcommand{\vcb}{\vec{b}}
\newcommand{\vcc}{\vec{c}}	\newcommand{\vcd}{\vec{d}}
\newcommand{\vce}{\vec{e}}	\newcommand{\vcf}{\vec{f}}
\newcommand{\vcg}{\vec{g}}	\newcommand{\vch}{\vec{h}}
\newcommand{\vci}{\vec{i}}	\newcommand{\vcj}{\vec{j}}
\newcommand{\vck}{\vec{k}}	\newcommand{\vcl}{\vec{l}}
\newcommand{\vcm}{\vec{m}}	\newcommand{\vcn}{\vec{n}}
\newcommand{\vco}{\vec{o}}	\newcommand{\vcp}{\vec{p}}
\newcommand{\vcq}{\vec{q}}	\newcommand{\vcr}{\vec{r}}
\newcommand{\vcs}{\vec{s}}	\newcommand{\vct}{\vec{t}}
\newcommand{\vcu}{\vec{u}}	\newcommand{\vcv}{\vec{v}}
%\newcommand{\vcw}{\vec{w}}	\newcommand{\vcx}{\vec{x}}
\newcommand{\vcy}{\vec{y}}	\newcommand{\vcz}{\vec{z}}

%---------------------------------------
% Greek Letters:-
%---------------------------------------
\newcommand{\eps}{\epsilon}
\newcommand{\veps}{\varepsilon}
\newcommand{\lm}{\lambda}
\newcommand{\Lm}{\Lambda}
\newcommand{\gm}{\gamma}
\newcommand{\Gm}{\Gamma}
\newcommand{\vph}{\varphi}
\newcommand{\ph}{\phi}
\newcommand{\om}{\omega}
\newcommand{\Om}{\Omega}
\newcommand{\sg}{\sigma}
\newcommand{\Sg}{\Sigma}
\newcommand{\Qed}{\begin{flushright}\qed\end{flushright}}
\newcommand{\parinn}{\setlength{\parindent}{1cm}}
\newcommand{\parinf}{\setlength{\parindent}{0cm}}
\newcommand{\del}[2]{\frac{\partial #1}{\partial #2}}
\newcommand{\Del}[3]{\frac{\partial^{#1} #2}{\partial^{#1} #3}}
\newcommand{\deld}[2]{\dfrac{\partial #1}{\partial #2}}
\newcommand{\Deld}[3]{\dfrac{\partial^{#1} #2}{\partial^{#1} #3}}
\newcommand{\uin}{\mathbin{\rotatebox[origin=c]{90}{$\in$}}}
\newcommand{\usubset}{\mathbin{\rotatebox[origin=c]{90}{$\subset$}}}
\newcommand{\lt}{\left}
\newcommand{\rt}{\right}
\newcommand{\exs}{\exists}
\newcommand{\st}{\strut}
\newcommand{\dps}[1]{\displaystyle{#1}}
\newcommand{\la}{\langle}
\newcommand{\ra}{\rangle}
\newcommand{\cls}[1]{\textsc{#1}}
\newcommand{\prb}[1]{\textsc{#1}}
\newcommand{\comb}[2]{\left(\begin{matrix}
		#1\\ #2
\end{matrix}\right)}
%\newcommand[2]{\quotient}{\faktor{#1}{#2}}
\newcommand\quotient[2]{
	\mathchoice
	{% \displaystyle
		\text{\raise1ex\hbox{$#1$}\Big/\lower1ex\hbox{$#2$}}%
	}
	{% \textstyle
		#1\,/\,#2
	}
	{% \scriptstyle
		#1\,/\,#2
	}
	{% \scriptscriptstyle  
		#1\,/\,#2
	}
}

\newcommand{\tensor}{\otimes}
\newcommand{\xor}{\oplus}

\newcommand{\sol}[1]{\begin{solution}#1\end{solution}}
\newcommand{\solve}[1]{\setlength{\parindent}{0cm}\textbf{\textit{Solution: }}\setlength{\parindent}{1cm}#1 \hfill $\blacksquare$}
\newcommand{\mat}[1]{\left[\begin{matrix}#1\end{matrix}\right]}
\newcommand{\matr}[1]{\begin{matrix}#1\end{matrix}}
\newcommand{\matp}[1]{\lt(\begin{matrix}#1\end{matrix}\rt)}
\newcommand{\detmat}[1]{\lt|\begin{matrix}#1\end{matrix}\rt|}
\newcommand\numberthis{\addtocounter{equation}{1}\tag{\theequation}}
\newcommand{\handout}[3]{
	\noindent
	\begin{center}
		\framebox{
			\vbox{
				\hbox to 6.5in { {\bf Complexity Theory I } \hfill Jan -- May, 2023 }
				\vspace{4mm}
				\hbox to 6.5in { {\Large \hfill #1  \hfill} }
				\vspace{2mm}
				\hbox to 6.5in { {\em #2 \hfill #3} }
			}
		}
	\end{center}
	\vspace*{4mm}
}

\newcommand{\lecture}[3]{\handout{Lecture #1}{Lecturer: #2}{Scribe:	#3}}

\let\marvosymLightning\Lightning
\newcommand{\ctr}{\text{\marvosymLightning}\hspace{0.5ex}} % Requires marvosym package

\newcommand{\ov}[1]{\overline{#1}}
\newcommand{\thmref}[1]{\hyperref[th:#1]{Theorem \ref{th:#1}}}
\newcommand{\propref}[1]{\hyperref[th:#1]{Proposition \ref{th:#1}}}
\newcommand{\lmref}[1]{\hyperref[th:#1]{Lemma \ref{th:#1}}}
\newcommand{\corref}[1]{\hyperref[th:#1]{Corollary \ref{th:#1}}}

\newcommand{\thrmref}[1]{\hyperref[#1]{Theorem \ref{#1}}}
\newcommand{\propnref}[1]{\hyperref[#1]{Proposition \ref{#1}}}
\newcommand{\lemref}[1]{\hyperref[#1]{Lemma \ref{#1}}}
\newcommand{\corrref}[1]{\hyperref[#1]{Corollary \ref{#1}}}

\DeclareMathOperator{\enc}{Enc}
\DeclareMathOperator{\res}{Res}
\DeclareMathOperator{\spec}{Spec}
\DeclareMathOperator{\cov}{Cov}
\DeclareMathOperator{\Var}{Var}
\DeclareMathOperator{\Rank}{rank}
\newcommand{\Tfae}{The following are equivalent:}
\newcommand{\tfae}{the following are equivalent:}
\newcommand{\sparsity}{\textit{sparsity}}

\newcommand{\uddots}{\reflectbox{$\ddots$}} 

\newenvironment{claimwidth}{\begin{center}\begin{adjustwidth}{0.05\textwidth}{0.05\textwidth}}{\end{adjustwidth}\end{center}}
\title{Bounding $\poa$ using LP, QP and Fenchel Duality}
\date{April 2025}
\author{Soham Chatterjee}

% You change the titlegraphic to whatever you want, or comment it out to remove it.
% \titlegraphic{\includegraphics[scale=0.25]{logo-northwestern.pdf}}

% % You can directly change the colors using the following macros.
% % You must redefine colors AFTER the theme is loaded.
% % For example, these provide shades of Yale Blue (#00356b)
% \definecolor{wcprimary}{RGB}{0,53,107}      % Main color
% \definecolor{wcprimary140}{RGB}{0, 34, 70}
% \definecolor{wcprimary130}{RGB}{0, 40, 80}
% \definecolor{wcprimary120}{RGB}{0, 45, 91}
% \definecolor{wcprimary110}{RGB}{0, 50, 102}
% \definecolor{wcprimary40}{RGB}{153, 174, 196}
% \definecolor{wcprimary30}{RGB}{179, 194, 211}
% \definecolor{wcprimary20}{RGB}{204, 215, 225}
% \definecolor{wcprimary10}{RGB}{230, 235, 240}

% % Now for the alerted orange (#bd5319) and example green (#5f712d)
% \definecolor{wcalerted}{RGB}{189,83,25}
% \definecolor{wcexample}{RGB}{95,113,45}

% % If you want to change the slide background color, 
% % you can use the following command:
%\setbeamercolor{background canvas}{bg=nupurple10!30}

% % Turn off section slides
% \AtBeginSection{}

% Change the font theme
%\usefonttheme{wildcat-overleaf}

% Change the bg pattern manually: Simple Single Color
% \renewcommand{\bgpattern}{
%     \draw[color=wcprimary,fill=wcprimary] (0,0) rectangle (\paperwidth,\paperheight);
% }

\definecolor{doc}{HTML}{DCBCD0}
\definecolor{myg}{RGB}{56, 140, 70}
\definecolor{myb}{RGB}{45, 111, 177}
\definecolor{myr}{RGB}{199, 68, 64}
\definecolor{mybg}{HTML}{F2F2F9}
\definecolor{mytheorembg}{HTML}{F2F2F9}
\definecolor{mytheoremfr}{HTML}{00007B}
\definecolor{myexamplebg}{HTML}{F2FBF8}
\definecolor{myexamplefr}{HTML}{88D6D1}
\definecolor{myexampleti}{HTML}{2A7F7F}
\definecolor{mydefinitbg}{HTML}{E5E5FF}
\definecolor{mydefinitfr}{HTML}{3F3FA3}
\definecolor{notesgreen}{RGB}{0,162,0}
\definecolor{myp}{RGB}{197, 92, 212}
\definecolor{mygr}{HTML}{2C3338}
\definecolor{myred}{RGB}{127,0,0}
\definecolor{myyellow}{RGB}{169,121,69}
\definecolor{OrangeRed}{HTML}{ED135A}
\definecolor{Dandelion}{HTML}{FDBC42}
\definecolor{light-gray}{gray}{0.95}
\definecolor{Emerald}{HTML}{00A99D}
\definecolor{RoyalBlue}{HTML}{0071BC}
\definecolor{mytoccolor}{HTML}{886830}

\begin{document}

\begin{frame}
\titlepage
\end{frame}

% \begin{frame}{Table of Contents}
%     \tableofcontents
% \end{frame}

\begin{frame}{Introduction}
    The Wildcat theme is a Beamer theme for Northwestern University, but which can be modified easily with different colors, fonts, and even background patterns. 
    \\ ~ \\
    The theme is inspired by the \href{https://github.com/matze/mtheme}{Metropolis theme} by Matthias Vogelgesang. It incorporates the Northwestern University facet design pattern, but otherwise has a clean, simple look, and relatively few bells and whistles. It is licensed under the GNU GENERAL PUBLIC LICENSE.
\end{frame}

\section{Weighted Congestion Games}
\begin{frame}
    \frametitle{Definitions}
    \begin{itemize}
        \item $\mcN$: Set of players
        \item $\mcE$: The ground set of resources
        \item For each player $j \in \mcN$, let $S_j\subseteq 2^{\mcE}$ be the set of strategies available to player $j$. Let $S=\bigtimes\limits_{j\in\mcN}S_i$.
        \item For each $j\in \mcN$ and each $e\in \mcE$ there is a weight of the resource $w_{ej}\in\bbR^+$.
        \item For each $e\in \mcE$ the cost of resource $e$ is an affine function $C_e:\bbR\to\bbR$ where $c_e(x)=a_e\cdot x+b_e$
        \item For any strategy profile $f\in S$, the cost of player $j$ is $\mathcolor{mytheoremfr}{\textbf{Cost}(f)_j=\sum\limits_{e\in f_j}w_{ej}\cdot c_e(l_e(f))}$ where $\mathcolor{mytheoremfr}{l_e(f)=\sum\limits_{j':e\in f_{j'}}w_{ej'}}$ is the load on resource $e$. Do $$\text{Cost}(f)=\sum_{j\in\mcN}\sum_{e\in f_j}w_{ej}\cdot c_e(l_e(f))=\sum_{e\in\mcE}a_e\cdot l_e(f)+b_e\cdot l_e(f)$$
    \end{itemize}
\end{frame}
\begin{frame}
\frametitle{Convex program of WCG}

\framesubtitle{Setting up the variables}
For any player $j\in\mcN$ and $f_j\in S_j$ let          \only<1>{
    \(
    \color{myr}{L\st_{j,f_j} = \sum\limits_{e \in f_j} w_{ej} \cdot c_e(w_{ej})}
    \)
}
% From slide 2 onwards: Normal weight and black
\only<2->{
    \(
    L_{j,f_j} = \sum\limits_{e \in f_j} w_{ej} \cdot c_e(w_{ej})
    \)
}i.e. the cost incurred by player $j$ when it plays strategy $f_j$.\pause

\begin{itemize}[itemsep=2em, topsep=2em]
    \item  \only<2>{$\mathcolor{myr}{\boldsymbol{x\st_{j,f_j}}}$} \only<3->{$x\st_{j,f_j}$} $\coloneqq$ Variable for player $j$ playing strategy $f_j$ for all $j\in\mcN$ and $f_j\in S_j$\pause

    \item {$\mathcolor{myr}{\boldsymbol{y_e}}$} $\coloneqq$ Variable for the load on resource $e$ for all $e\in\mcE$
\end{itemize}
\end{frame}

% \begin{frame}{Convex program of WCG}
%     \framesubtitle{Quadratic Program}  
% \begin{mini*}{}{\sum_{j\in\mcN}\sum_{f_j\in S_j}x\st_{j,f_j}\cdot L\st_{j,f_j}+\sum_{e\in\mcE}a_e\cdot y_e^2}{}{}
%     \addConstraint{\sum_{f_j\in S_j}x\st_{j,f_j}}{\leq 1}{\quad\forall\ j\in\mcN}
%     \addConstraint{\sum_{j\in\mcN}\sum_{f_j\in S_j}\sum_{e\in f_j}w_{ej}\cdot x\st_{j,f_j}}{\leq y_e}{\quad\forall \ e\in\mcE}
%     \addConstraint{x\st_{j,f_j}}{\geq 0}{\quad \forall\ j\in\mcN,\ f_j\in S_j}
% \end{mini*}
    
% \end{frame}

\begin{frame}{Convex program of WCG}
    \framesubtitle{Quadratic Program}  
    
    \visible<1->{\begin{mini*}{}{\sum_{j\in\mcN}\sum_{f_j\in S_j}x\st_{j,f_j}\cdot L\st_{j,f_j}+\sum_{e\in\mcE}a_e\cdot y_e^2}{}{}
        \addConstraint{\sum_{f_j\in S_j}x\st_{j,f_j}}{\leq 1}{\quad\forall\ j\in\mcN}
        \addConstraint{\sum_{j\in\mcN}\sum_{f_j\in S_j}\sum_{e\in f_j}w_{ej}\cdot x\st_{j,f_j}}{\leq y_e}{\quad\forall \ e\in\mcE}
        \addConstraint{x\st_{j,f_j}}{\geq 0}{\quad \forall\ j\in\mcN,\ f_j\in S_j}
    \end{mini*}}
    
    \only<2>{  \begin{tikzpicture}[remember picture, overlay]
        % Coordinates for the first constraint
        \coordinate (constraint) at ($(current page.center)+(2,0.2)$);
        
        % UPWARD-FACING curly brace under the first constraint
        \draw 
            ($(constraint)+(-2.6,-0.65)$) rectangle ($(constraint)+(1.5,+0.45)$);
            
        % Arrow start point (below the brace)
        \coordinate (arrowstart) at ($(constraint)+(0,-0.65)$);
        
       
            % Explanation box
            \node[draw, fill=doc, 
                  text width=10cm, align=left,line width=1pt, 
                  below left=1cm and -1.5cm of arrowstart] (explanation) {
                This constraint makes sure only one strategy is played by each player.
            };
            
            % Straight tapered arrow
            \fill[black, draw=black] 
                (arrowstart) -- 
                ($(explanation.north)-(0.2,0)$) -- 
                ($(explanation.north)+(0.2,0)$) -- 
                cycle;                
    \end{tikzpicture}}  
    \only<3>{  \begin{tikzpicture}[remember picture, overlay]
        % Coordinates for the first constraint
        \coordinate (constraint) at ($(current page.center)+(2,0.2)$);
        
        % UPWARD-FACING curly brace under the first constraint
        \draw 
            ($(constraint)+(-4.65,-1.85)$) rectangle ($(constraint)+(1.5,-0.65)$);
            
        % Arrow start point (below the brace)
        \coordinate (arrowstart) at ($(constraint)+(-2.5,-1.85)$);
        
       
            % Explanation box
            \node[draw, fill=doc, 
                  text width=11cm, align=left,line width=1pt, 
                  below left=1.2cm and -5cm of arrowstart] (explanation) {
                This constraint makes sure that the load on each resource is at least sum of the weights of the players using that resource.
            };
            
            % Straight tapered arrow
            \fill[black, draw=black] 
                (arrowstart) -- 
                ($(explanation.north)+(0.2,0)$) -- 
                ($(explanation.north)+(0.4,0)$) -- 
                cycle;                
    \end{tikzpicture}}
\end{frame}

\begin{frame}{Dual Program}
We denote the dual variables by $\{\mu_j\}_{j\in\mcN}$, $\{\Phi_e\}_{e\in\mcE}$ and $\{\Psi_e\}_{e\in\mcE}$. Then we use the Fenchel Duality to obtain the dual  of the convex program.
\only<1>{\begin{maxi*}{}{\sum_{j\in\mcN}\mu_j-\sum_{e\in\mcE} \frac1{4a_e}\cdot \Phi_e^2}{}{}
    \addConstraint{\mu_j-\sum\limits_{e\in f_j}w_{e,j}\cdot \Psi_e}{\leq L\st_{j,f_j}}{\quad\forall\ j\in\mcN,f_j\in S_j}
    \addConstraint{\Psi_e}{\leq\Phi_e}{\quad\forall\ e\in\mcE}
    \addConstraint{\mu_j}{\geq 0}{\quad\forall\ j\in\mcN}
    \addConstraint{\Phi_e}{\geq 0}{\quad\forall\ e\in\mcE}
\end{maxi*}}

\only<2>{\begin{maxi*}{}{\sum_{j\in\mcN}\mu_j-\sum_{e\in\mcE} \frac1{4a_e}\cdot \Phi_e^2}{}{}
    \addConstraint{\mu_j-\sum\limits_{e\in f_j}w_{e,j}\cdot \Phi_e}{\leq L\st_{j,f_j}}{\quad\forall\ j\in\mcN,f_j\in S_j}
    \addConstraint{\mu_j}{\geq 0}{\quad\forall\ j\in\mcN}
    \addConstraint{\Phi_e}{\geq 0}{\quad\forall\ e\in\mcE}
\end{maxi*}}\pause

\begin{tblock}{Remark}
    We can take $\Phi_e=\Psi_e$ for all $e\in \mcE$ as from every \textsf{CCE} we will assign $\Phi_e$ and $\Psi_e$ to be the same value
\end{tblock}


\end{frame}

\begin{frame}{$\left(1+\frac1\dl\right)$-Approximate Solution from Primal}

    Consider the following changed primal program:
    \begin{mini*}{}{\mathcolor{myr}{\boldsymbol{\frac1{\dl}}}\sum_{j\in\mcN}\sum_{f_j\in S_j}x\st_{j,f_j}\cdot L\st_{j,f_j}+\sum_{e\in\mcE}a_e\cdot y_e^2}{}{}
    \addConstraint{\sum_{f_j\in S_j}x\st_{j,f_j}}{\leq 1}{\quad\forall\ j\in\mcN}
    \addConstraint{\sum_{j\in\mcN}\sum_{f_j\in S_j}\sum_{e\in f_j}w_{ej}\cdot x\st_{j,f_j}}{\leq y_e}{\quad\forall \ e\in\mcE}
    \addConstraint{x\st_{j,f_j}}{\geq 0}{\quad \forall\ j\in\mcN,\ f_j\in S_j}
\end{mini*}

If $\dl=1$ we get our original program. For any $\dl>0$ we get a $\lt(1+\frac1{\dl}\rt)$-approximate solution.
\end{frame}

\begin{frame}{Dual don't need to change}
    Taking the dual of the new program we get the following:
    \begin{maxi*}{}{\sum_{j\in\mcN}\mu_j-\sum_{e\in\mcE} \frac1{4a_e}\cdot \Phi_e^2}{}{}
    \addConstraint{\mu_j-\sum\limits_{e\in f_j}w_{e,j}\cdot \mathcolor{myr}{\boldsymbol{\Phi_e}}}{\leq \mathcolor{myr}{\boldsymbol{\frac{L\st_{j,f_j}}{\dl}}}}{\quad\forall\ j\in\mcN,f_j\in S_j}
    \addConstraint{\mu_j}{\geq 0}{\quad\forall\ j\in\mcN}
    \addConstraint{\Phi_e}{\geq 0}{\quad\forall\ e\in\mcE}
    \end{maxi*}

So instead if we work with the old dual program and scale our variables $\mu_j$, $\Phi_e$ and $\Psi_e$ by $\frac1{\dl}$ we still get a feasible solution to the new dual program.
\end{frame}
\begin{frame}{Setting the Dual Variables}
    Let $\sg$ is any \textsf{CCE} of the  game. Set
    \begin{itemize}
        \item $\mu_j=\dfrac1{\dl}\cdot\underset{f\sim \sg}{\bbE}[\text{Cost}_j(f)]$
        \item $\Phi_e=\dfrac1{\dl}\cdot a_e\cdot \underset{f\sim \sg}{\bbE}[l_e(f)]$
    \end{itemize}\pause

    \begin{align*}
        \text{Cost}_j(f_j,\theta_{-j}) & \leq \sum_{e\in f_j}w_{e,j}\cdot (a_e(l_e(\theta)+w_{e,j})+b_e)\\
        & = \sum_{e\in f_j}w_{e,j}(a_e\cdot w_{e,j}+b_e)+\sum_{e\in f_j}w_{e,j}\cdot a_e\cdot l_e(\theta)\\
        & = L\st_{j,f_j}+\sum_{e\in f_j}w_{e,j}\cdot a_e\cdot l_e(\theta)
    \end{align*}\pause

    \begin{tblock}{Remark}
        It is a feasible solution to the dual program. 
    \end{tblock}
\end{frame}


\begin{frame}{Bound on \textsf{PoA} : I}
    \vspace*{-8mm}
    \begin{align*}
        \sum_{e\in \mcE} \frac1{ a_e}\cdot a_e^2\cdot \underset{f\sim \sg}{\bbE}[l_e(f)]^2 & = \sum_{e\in \mcE}  a_e\cdot\underset{f\sim \sg}{\bbE}[l_e(f)]^2 \\
        & \leq \underset{f\sim \sg}{\bbE}\lt[ \sum_{e\in\mcN} a_e\cdot l_e^2(f)  \rt]& [\text{Jensen}]\\
        & \leq \underset{f\sim \sg}{\bbE}\lt[ \sum_{e\in\mcN} \text{Cost}_j(f)\rt] =\sum_{j\in\mcN}\underset{f\sim \sg}{\bbE}[\text{Cost}_j(f)]
    \end{align*}
\end{frame}
\begin{frame}{Bound on \textsf{PoA} : II}
    \begin{align*}
        \text{Primal-Sol} & \geq \sum\limits_{j\in \mcN}\frac1{\dl}\cdot \underset{f\sim \sg}{\bbE}[\text{Cost}_j(f)]-\sum_{e\in \mcE}\frac1{\dl^2}\cdot \frac1{4}a_e\cdot \underset{f\sim \sg}{\bbE}[l_e(f)]^2\\
        & \geq  \frac{1}{\dl}\sum_{j\in\mcN}\underset{f\sim \sg}{\bbE}[\text{Cost}_j(f)]-\frac1{4\cdot \dl^2}\cdot \sum_{e\in \mcE}\underset{f\sim \sg}{\bbE}[\text{Cost}_j(f)]\\
        & = \frac{4\dl-1}{4\dl^2}\sum_{e\in \mcE}\underset{f\sim \sg}{\bbE}[\text{Cost}_j(f)]
    \end{align*}\pause

Primal is $\lt(1+\frac1{\dl}\rt)$-approximate solution to the optimal solution. So we get a bound of $\mathcolor{myr}{\lt(1+\frac1{\dl}\rt)\dfrac{4\dl^2}{4\dl-1}}$ bound on $\textsf{PoA}$.  Take $\dl=\frac{1+\sqrt{5}}{4}$ you will get a bound of $\mathcolor{myr}{1+\Phi}$ where $\Phi$ is the golden ratio.
\end{frame}
\standout{Questions?}


\section{Simultaneous Second-Price Auctions}
\begin{frame}{Definition}
    \begin{itemize}
        \item $\mcM$: Set of $m$ items
        \item $\mcN$: Set of $n$ players\pause
        
        \item For each player $j \in \mcN$, $v_j:2^{\mcM}\to \bbR_{\geq 0}$ is the valuation function of player $j$ of $T\subseteq \mcM$. $v_j$ is submodular.\pause

        \item Each player $j$ submits a bid $b_j\in\bbR^{m}_{\geq 0}$ which follows $\sum\limits_{i\in T}b_{ij}\leq v_j(T)$ for all $T\subseteq \mcM$.\pause
 
        \item Let $W_j(b)$ denote the set of items won by player $j\in\mcN$ when the bids are $b$. \pause

        \item Let $p(i,b)$ is the second highest bid for item $i$ when the bids are $b$.\pause

        \item Let $u_j(b)$ be the utility of player $j$ when the bids are $b$. Then $u_j(b)=v_j(W_j(b))-\sum\limits_{i\in W_j(b)}p(i,b)$.\pause
 
        \item Auctions of each item follows Second-Price auctions rule.
    \end{itemize}\vspace{3mm}\pause

    GOAL: Maximize the social welfare of the players $V(b)=\sum\limits_{j\in\mcN}v_j(W_j(b))$
\end{frame}

\begin{frame}
    \frametitle{Property of  Biddings}
    \begin{theorem}
        $\forall\ j\in\mcN$, $\forall\ T\subseteq \mcM$, $\forall \ b\in\bbR^{m\times n}_{\geq 0}$, $\exs\ b_j(T)\in\bbR^{m}_{\geq 0}$ such that $$u_j(b_j(T),b_{-j})\geq v_j(T)-\sum\limits_{i\in T}\max\limits_{j'\in \mcN\setminus \{j\}}\{b_{ij'}\}$$
    \end{theorem}\pause
    \vspace*{5mm}


\visible<2->{Let $T=\{1,\dots, i\}$. Take $b_{ij}^*=v_j(1,2,\dots, i)-v_j(1,2,\dots, i-1)$. Take $b_j(T)=b^*_j$}\vspace*{5mm}

\visible<3->{Observe: $\sum\limits_{i\in T'}b^*_{i,j}\leq v_j(T')$ for all $T'\subseteq T$ by submodularity and for $T=T'$ its equality.}

\end{frame}

\begin{frame}
    \frametitle{Proof of Theorem}
\begin{align*}
    u_j(b_j(T),b_{-j}) & = v_j(T^*)-\sum\limits_{i\in T^*}\max\limits_{j'\in\mcN\setminus \{j\}}\{b_{ij'}\}\\
    & \geq v_j(T^*)-\sum\limits_{i\in T^*} \max\limits_{j'\in\mcN\setminus \{j\}}\{b_{ij'}\} + \lt[\sum_{i\in T\setminus T^*}b_{i,j}^*-\max\limits_{j'\in\mcN\setminus \{j\}}\{b_{ij'}\} \rt]\\
    & \geq v_j(T)-\sum\limits_{i\in T}\max\limits_{j'\in \mcN\setminus \{j\}}\{b_{ij'}\} 
\end{align*}
\end{frame}


\begin{frame}{LP Formulation}
    \begin{itemize}
        \item $x\st_{j,T}$ $\coloneqq$ Variable for player $j$ winning item $T$.
    \end{itemize}\pause

\visible<2->{\begin{maxi*}
    {}{\sum_{T\subseteq \mcM}\sum_{j\in\mcN}x\st_{j,T}\cdot v_j(T)}{}{}
    \addConstraint{\sum_{j\in\mcN}\sum_{i\in T}x\st_{j,T}}{\leq 1}{\quad \forall\ i\in\mcM}
    \addConstraint{\sum_{T\subseteq \mcM}x\st_{j,T}}{\leq 1}{\quad \forall\ j\in\mcN}
    \addConstraint{x\st_{j,T}}{\geq 0}{\quad \forall\ j\in\mcN,\ T\subseteq \mcM}
\end{maxi*}}

\only<3>{  \begin{tikzpicture}[remember picture, overlay]
    % Coordinates for the first constraint
    \coordinate (constraint) at ($(current page.center)+(2,0.2)$);
    
    % UPWARD-FACING curly brace under the first constraint
    \draw 
        ($(constraint)+(-3.9,-1.1)$) rectangle ($(constraint)+(0.63,0.05)$);
        
    % Arrow start point (below the brace)
    \coordinate (arrowstart) at ($(constraint)+(-2,-1.1)$);
    
   
        % Explanation box
        \node[draw, fill=doc, 
              text width=10cm, align=left,line width=1pt, 
              below left=2cm and -4cm of arrowstart] (explanation) {
            This constraint makes sure no item is over-allocated i.e. each item is sold to only one player.
        };
        
        % Straight tapered arrow
        \fill[black, draw=black] 
            (arrowstart) -- 
            ($(explanation.north)-(0.2,0)$) -- 
            ($(explanation.north)+(0.2,0)$) -- 
            cycle;                
\end{tikzpicture}}  

\only<4>{  \begin{tikzpicture}[remember picture, overlay]
    % Coordinates for the first constraint
    \coordinate (constraint) at ($(current page.center)+(2,0.2)$);
    
    % UPWARD-FACING curly brace under the first constraint
    \draw 
        ($(constraint)+(-3.5,-2.2)$) rectangle ($(constraint)+(0.63,-1.1)$);
        
    % Arrow start point (below the brace)
    \coordinate (arrowstart) at ($(constraint)+(-3,-2.25)$);
    
   
        % Explanation box
        \node[draw, fill=doc, 
              text width=10cm, align=left,line width=1pt, 
              below left=1cm and -5cm of arrowstart] (explanation) {
            This constraint makes sure each agent receives exactly one set from $2^{\mcM}$.
        };
        
        % Straight tapered arrow
        \fill[black, draw=black] 
            (arrowstart) -- 
            ($(explanation.north)-(0.6,0)$) -- 
            ($(explanation.north)-(1,0)$) -- 
            cycle;                
\end{tikzpicture}}
\end{frame}


\begin{frame}{Dual Program}

    \begin{mini*}
        {}{\sum_{j\in\mcN}y_j+\sum_{i\in\mcM}z_i}{}{}
        \addConstraint{y_j+\sum_{i\in T}z_i}{\geq v_j(T)}{\quad \forall\ j\in\mcN,\ T\subseteq \mcM}
        \addConstraint{z_i}{\geq 0}{\quad \forall\ i\in\mcM}
        \addConstraint{y_j}{\geq 0}{\quad \forall\ j\in\mcN}
    \end{mini*}

    
\end{frame} 

\begin{frame}{Setting the Dual Variables}
    Given a \textsf{CCE} $\sg$ of the game, we set the dual variables as follows:
    \begin{itemize}
        \item $y_j=\underset{b\sim \sg}{\bbE}[u_j(b)]$ for all $j\in\mcN$.\pause
        
        \item $z_i=\underset{b\sim \sg}{\bbE}  \lt[\max\limits_{j\in\mcN}b_{ij}\rt]$ for all $i\in\mcM$.
    \end{itemize}\pause

    Since $\sg$ is an $\textsf{CCE}$ $$\underset{b\sim \sg}{\bbE}[u_j(b)]\geq \underset{b\sim \sg}{\bbE}\lt[u_j(b_j(T),b_{-j})\rt]\qquad \forall\ T\subseteq \mcM$$\pause

    By the theorem $$u_j(b_j(T),b_{-j})\geq v_j(T)-\sum\limits_{i\in T}\max\limits_{j'\in \mcN\setminus \{j\}}\{b_{ij'}\}\geq v_j(T)-\sum\limits_{i\in T}\max\limits_{j'\in \mcN}\{b_{ij'}\}$$ \pause

So $\underset{b\sim \sg}{\bbE}[u_j(b)]\geq v_j(T)-\sum\limits_{i\in T}\underset{b\sim \sg}{\bbE}\lt[\max\limits_{j'\in \mcN}\{b_{ij'}\}\rt]$. So it is feasible solution to the dual program.

\end{frame}
\begin{frame}{Bound on \textsf{PoA}}
    \begin{align*}
        \text{Primal-Sol} & \leq \sum_{j\in\mcN}\underset{b\sim \sg}{\bbE}[u_j(b)]+\sum_{i\in\mcM}\underset{b\sim \sg}{\bbE}\lt[\max\limits_{j\in \mcN}\{b_{ij}\}\rt]\\
        & = \underset{b\sim \sg}{\bbE}\lt[\sum_{j\in\mcN}u_j(b)\rt]+\underset{b\sim \sg}{\bbE}\lt[\sum_{i\in\mcM}\max\limits_{j\in \mcN}\{b_{ij}\}\rt]\\
        & \leq 2\cdot \underset{b\sim \sg}{\bbE}[V(b)]
    \end{align*}\pause

    So we get a bound of \textcolor{myr}{2}.

\end{frame}
\standout{Questions?}

\section{Facility Location Games}


\end{document}